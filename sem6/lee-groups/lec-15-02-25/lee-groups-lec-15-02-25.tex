\documentclass[a4paper]{article}
\usepackage[a4paper,%
    text={180mm, 260mm},%
    left=15mm, top=15mm]{geometry}
\usepackage[utf8]{inputenc}
\usepackage{cmap}
\usepackage[english, russian]{babel}
\usepackage{indentfirst}
\usepackage{amssymb}
\usepackage{amsmath}
\usepackage{amsthm}
\usepackage{mathtools}
\usepackage{tcolorbox}
\usepackage{import}
\usepackage{xifthen}
\usepackage{pdfpages}
\usepackage{transparent}
\usepackage{graphicx}
\graphicspath{ {./figures} }

\newcommand{\incfig}[1]{%
\def\svgwidth{\columnwidth}
\import{./figures/}{#1.pdf_tex}
}

\newtheorem*{theorem}{Теорема}

\theoremstyle{definition}
\newtheorem*{definition}{Определение}


\begin{document}
\title{Группы и алгебры Ли}
\author{}
\date{15 Февраля 2025 г.}
\maketitle

\begin{tcolorbox}[title=Группы Ли]
    \begin{definition}[Группа Ли]
    Пусть $ G $ - группа и гладкое многообразие над $ \mathbb{R} $ 

    Если 
    \[
        \mu: G \times G \to G
    \]
    и
    \[
        i: G \to G, \ i(x) = x^{-1}
    \]
    гладкие отображения, то $ G $ называется группой Ли
\end{definition}
\end{tcolorbox}

\[
    (U, \phi) \text{ - карта}
\]
\[
    \phi: U \to \widetilde{U} \subset \mathbb{R}^{n} \text{ - гомеоморфизм}
\]
\[
    (V, \psi) \text{ - карта}
\]
\[
    \phi: V \to \widetilde{V} \subset \mathbb{R}^{n} \text{ - гомеоморфизм}
\]
\[
    x \in U \cap V
\]
\[
    \phi: U \cap V \to \phi(U \cap V) \subset \widetilde{U} \subset \mathbb{R}^{n}
\]
\[
    \psi: U \cap V \to \psi(U \cap V) \subset \widetilde{V} \subset \mathbb{R}^{n}
\]
$ (x_1, \dots, x_n) $ - координаты $ \phi $ \\
$ (y_1, \dots, y_n) $ - координаты $ \psi $
\[
    f_{\phi, \psi}: \psi(U \cap V) \to \phi(U \cap V)
\]
\[
    z \in \psi(U \cap V)
\]
\[
    x_i(f_{\phi, \psi}(z)) = f_i(y_1(z), \dots, y_n(z))
\]
\[
    x_i = f_i(y_1, \dots, y_n), \quad i = \overline{1,n}
\]
\[
    u \in U \quad x_i(\phi(u))
\]
\[
    y_i \circ \psi = y_i
\]
\end{document}
