\documentclass[a4paper]{article}
\usepackage[a4paper,%
    text={180mm, 260mm},%
    left=15mm, top=15mm]{geometry}
\usepackage[utf8]{inputenc}
\usepackage{cmap}
\usepackage[english, russian]{babel}
\usepackage{indentfirst}
\usepackage{amssymb}
\usepackage{amsmath}
\usepackage{amsthm}
\usepackage{mathtools}
\usepackage{tcolorbox}
\usepackage{import}
\usepackage{tikz-cd}
\usepackage{xifthen}
\usepackage{pdfpages}
\usepackage{transparent}
\usepackage{graphicx}
\graphicspath{ {./figures} }

\DeclarePairedDelimiter\set\{\}

\newcommand{\incfig}[1]{%
\def\svgwidth{\columnwidth}
\import{./figures/}{#1.pdf_tex}
}

\newtheorem*{theorem}{Теорема}
\newtheorem*{statement}{Утверждение}
\newtheorem*{lemma}{Лемма}
\newtheorem*{proposal}{Предложение}
\newtheorem*{consequence}{Следствие}


\theoremstyle{definition}
\newtheorem*{definition}{Определение}

\theoremstyle{remark}
\newtheorem*{remark}{Замечание}

\renewcommand\qedsymbol{$\blacksquare$}

\begin{document}
\[
    v \in T_e(G)
\]
\begin{equation}
    \label{eq:1}
    \begin{cases}
        \frac{d\theta}{dt} = dL_{\theta(t)}(v)\\
        \theta(0) = e
    \end{cases}
\end{equation}

$ \exists! \  \theta(t) $ - решение $ (\ref{eq:1}) $ при 
$ t \in (-\varepsilon, \varepsilon) $ 
\[
    \psi(t) = \theta\left( \frac{t}{N} \right)^N, \ t \in \mathbb{R}
\]
\[
    N \in \mathbb{N} \ : \ \bigg| \frac{t}{N} \bigg| < \frac{\varepsilon}{2} 
\]
\[
    t \in (t_0 - \delta, t_0 + \delta)
\]
\[
    \bigg| \frac{t}{N} \bigg| < \frac{\varepsilon}{2} \implies
    \frac{t}{N} \text{ гладко зависит от } t \in (t_0 - \delta, t_0 + \delta)
\]

$ \theta\left( \frac{t}{N} \right) $ - гладкая \\
$ \theta\left( \frac{t}{N} \right)^N $ - гладко зависит от $ t $ \\
$ \implies \psi(t) $ - гладко зависит от $ t \in \mathbb{R} $ 

Рассмотрим:
\[
    \psi(s + t) = \theta\left( \frac{s + t}{N} \right)^N = \left(
    \theta\left( \frac{s}{N}\right) \theta\left( \frac{t}{N} \right)\right)^N
    = \theta\left(\frac{s}{N} \right)^N \cdot \theta\left(\frac{t}{N} \right)^N
    = \psi(s) \cdot \psi(t)
\]
\[
    \theta\left(\frac{s}{N}\right)\theta\left(\frac{t}{N}\right)
    = \theta\left( \frac{s+t}{N} \right) = \theta\left(\frac{t+s}{N}\right)
    = \theta\left(\frac{t}{N}\right)\theta\left(\frac{s}{N}\right)
\]

\[
    \frac{d \psi}{dt} \bigg|_{t=0} = \frac{d \theta}{dt} \bigg|_{t=0} = v
\]

\section*{\centering Экпоненциональное отображение}
$ \theta_v(t) $ - однопараметрическая подгруппа такая, что $ \frac{d\theta_v}{dt}
\big|_{v=0} = v$ 

Определим:
\[
    \exp : \ T_e(G) \to G, \ \exp(v) = \theta_v(1) \in G
\]

$ G \sim (\mathbb{R}^{\star}, \cdot) $.
\[
    T_1(G) = \mathbb{R}, \ \exp v = e^{v}
\]
\[
    \frac{d\theta}{dt} = \theta(t) \cdot v
\]

$ G = GL(n, \mathbb{R}), \quad T_e(G) = M_n(\mathbb{R}) $ 
\[
    \theta_v(t) = A(t)
\]
\[
    \frac{dA(t)}{dt} = A(t) \cdot v \quad v \in M_n(\mathbb{R})
\]
\[
    \exp(B) = E + B + \frac{B^2}{2!} + \dots + \frac{B^{m}}{m!} + \dots
\]
\[
    \bigg|\bigg|\frac{B^m}{m!}\bigg|\bigg| \leq \frac{||B||^{m}}{m!} 
\]
\[
    \exp(||B||) = 1 + ||B|| + \dots + \frac{||B||^m}{m!} + \dots
\]
\[
    \exp(At) = 1 + A \cdot t + \frac{A^2t^2}{2} + \dots
\]
\[
    \frac{d\exp(At)}{dt} = A + A^2 t + \frac{A^3 t^2}{2!} + \dots =
    \exp(At) \cdot A
\]
\[
    \exp(At)\big|_{t=0} = 1 = E
\]
\[
    ab=ba \quad \exp(a+b) = \exp a \cdot \exp b 
\]
\[
    \theta_A(t) = \exp(At)
\]

\begin{tcolorbox}
\begin{theorem}
    $ G, H $ - группы Ли. $ \phi: \ G \to H $ - гомоморфизм групп Ли
    \[
        \begin{tikzcd}
            T_e(G) \arrow{r}{(d \phi)_e } \arrow[swap]{d}{\exp} & T_e(H) \arrow{d}{\exp}\\
            G \arrow{r}{\phi} & H
        \end{tikzcd}
    \]
    коммутативна

    \begin{proof}
        \[
            \phi(\exp(v)) = \exp(d \phi_e(v))
        \]

        $ \theta_v(t) $ - однопар. подгруппа в $ G $ 
        \[
            \frac{d \phi(\theta_v(t))}{dt} \bigg|_{t=0} = (d \phi)_e \cdot 
            \frac{d\theta_v(t)}{dt} \bigg|_{t=0} = d \phi_e(v)
        \]
        \[
            w = \frac{d \tilde{\theta}_w(t)}{dt} \bigg|_{t=0} = d \phi_e(v)
        \]
        \[
            (\theta_v(t)) = \widetilde{\theta}_{(d \phi)_e(v)} \ | \ t = 0 \implies
        \]
    \end{proof}
\end{theorem}
\end{tcolorbox}

\[
    O(n) \hookrightarrow GL(n, \mathbb{R})
\]
\[
    v \in O(n)
\]
\[
    v \in T_e(G) \subset M_n(\mathbb{R}) = T_e(H)
\]
\end{document}
