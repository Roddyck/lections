\documentclass[a4paper]{article}
\usepackage[a4paper,%
    text={180mm, 260mm},%
    left=15mm, top=15mm]{geometry}
\usepackage[utf8]{inputenc}
\usepackage{cmap}
\usepackage[english, russian]{babel}
\usepackage{indentfirst}
\usepackage{amssymb}
\usepackage{amsmath}
\usepackage{amsthm}
\usepackage{mathtools}
\usepackage{tcolorbox}
\usepackage{import}
\usepackage{xifthen}
\usepackage{pdfpages}
\usepackage{transparent}
\usepackage{graphicx}
\graphicspath{ {./figures} }

\DeclarePairedDelimiter\set\{\}

\newcommand{\incfig}[1]{%
\def\svgwidth{\columnwidth}
\import{./figures/}{#1.pdf_tex}
}

\newtheorem*{theorem}{Теорема}
\newtheorem*{statement}{Утверждение}
\newtheorem*{lemma}{Лемма}
\newtheorem*{proposal}{Предложение}


\theoremstyle{definition}
\newtheorem*{definition}{Определение}

\theoremstyle{remark}
\newtheorem*{remark}{Замечание}

\renewcommand\qedsymbol{$\blacksquare$}

\begin{document}
\begin{tcolorbox}[title=Подгруппа Ли]
    \begin{definition}[Подгруппа Ли]
        $ G $ - группа Ли, $ H \subset G $ - подгруппа и подмногообразие (т.е. для любого
        $ h \in H, \ \exists $ окрестность $ h \in U \subset H $ и координатная окрестность
        $ V $ точки $ h $ в $ G $, такие что $ U $ - координатная плоскость в $ V $)  
    \end{definition}
\end{tcolorbox}

$ H \stackrel{i}{\hookrightarrow} G $. i - гладкое отображение 
\[
    v \in T_e(H)
\]
$ \widetilde{X} $ - векторное поле на $ G $.
\[
    \widetilde{X}(g) = (dL_g)_e(v), \ g \in G
\]
\[
    h \in H, \ X(h) \text{ - значение поля на } H
\]
\[
    L(H) \to L(G)
\]
\[
    \text{ левое инвариантоное }X \to X(e) = v \to \widetilde{X}(g) = (dL_g)_e(v)
\]
\[
    \widetilde{X}(h) = X(h), \ \forall h \in H
\]
\[
    L(H) \stackrel{di}{\hookrightarrow} L(G)
\]
\[
    X,Y \in L(H) \quad [X,Y] \in L(H)
\]
\[
    \widetilde{[X,Y]} = [\widetilde{X}, \widetilde{Y}]
\]

$ X $ и $ \widetilde{X} $ - $ i $ - связаны. $ (di)_h(X(h)) = \widetilde{X}(i(h)) =
\widetilde{X}(h)$ 
\[
    di: \ X \to \widetilde{X} \text{ - гомооморфизм алгебр Ли}
\]
\[
    di([X,Y]) = [di(X), di(Y)]
\]
\[
    L(H) \equiv di(L(H)) \subset L(G)
\]

$ A $ алгебра Ли ортогональной группы
\[
    O(h) \subset GL(n)
\]
\[
    \mathbb{R}^n, \ (X,Y) = x_1 y_1 + \dots + x_n y_n = 
    (x)^t (y) \quad x,y \in \mathbb{R}
\]
\[
    A(x) \text{ - действие } A \in GL(n) \text{ i } x \in \mathbb{R}^n
\]
\[
    (Ax, Ay) = (x,y) \quad (Ax)^t \cdot (Ay) = (x)^t (y)
\]
\[
    (x)^t A^t A (y) = (x)^t (y) \implies A^t \cdot A = E
\]
\[
    O(n) = \set{A \in GL(n) \ | \ A^t \cdot A = E}
\]
\[
    \sum_{k=1}^{n} x_{ki} x_{kj} = s_{ij} 
\]
\[
    A(s) \in O(n) \quad A(0) = E
\]
\[
    A(s)^t \cdot A(s) = E
\]
\[
    \frac{d}{ds} (A(s)^t \cdot A(s)) = 0
\]
\[
    \frac{dA(s)^t}{ds} * A(s) + (A(s))^t \cdot \frac{dA(s)}{ds} = 0 \bigg|_{s=0}
\]
\[
    \frac{dA(s)^t}{ds} \bigg|_{s=0} + \frac{dA(s)}{ds} \bigg|_{s=0} = 0
\]
\[
    \frac{dA(s)}{ds}\bigg|_{s=0} = B \in gl(n) \implies B^t + B = 0
\]
\begin{tcolorbox}
    \[
        B^t = -B
    \]
\end{tcolorbox}

\[
    T_e(0(n)) = \set{B \in gl(n) \ | \ B^t = -B}
\]

$ G $ - группа Ли, $ H $ - подгруппа Ли
\[
    L(H) \subset L(G) \quad \mathfrak{h} = T_e(H) \quad \mathfrak{g} = (T_e(G), \, [,])
\]
\[
    \mathfrak{h} \subset \mathfrak{g}, \ h \text{ - подалгебра Ли в } g
\]
\end{document}
