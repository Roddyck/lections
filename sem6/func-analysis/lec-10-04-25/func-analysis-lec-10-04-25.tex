\documentclass[a4paper]{article}
\usepackage[a4paper,%
    text={180mm, 260mm},%
    left=15mm, top=15mm]{geometry}
\usepackage[utf8]{inputenc}
\usepackage{cmap}
\usepackage[english, russian]{babel}
\usepackage{indentfirst}
\usepackage{amssymb}
\usepackage{amsmath}
\usepackage{amsthm}
\usepackage{mathtools}
\usepackage{tcolorbox}
\usepackage{import}
\usepackage{xifthen}
\usepackage{pdfpages}
\usepackage{transparent}
\usepackage{graphicx}
\graphicspath{ {./figures} }

\DeclarePairedDelimiter\set\{\}

\newcommand{\incfig}[1]{%
\def\svgwidth{\columnwidth}
\import{./figures/}{#1.pdf_tex}
}

\newtheorem*{theorem}{Теорема}
\newtheorem*{statement}{Утверждение}
\newtheorem*{lemma}{Лемма}
\newtheorem*{proposal}{Предложение}
\newtheorem*{consequence}{Следствие}


\theoremstyle{definition}
\newtheorem*{definition}{Определение}

\theoremstyle{remark}
\newtheorem*{note}{Замечание}

\renewcommand\qedsymbol{$\blacksquare$}

\begin{document}
\begin{tcolorbox}
    \begin{definition}[Сходимость по норме]
    \[
        x_n \to x \ ||x_n - x|| \to 0
    \]
    \end{definition}
\end{tcolorbox}
\begin{tcolorbox}[title=Непрерывность нормы]
    \[
        ||x_n - x|| \to 0 \implies ||x_n|| \to ||x||
    \]
\end{tcolorbox}
\[
    x_n \to x \quad y_n \to y \quad \lambda_n \to \lambda
\]
\[
    \lambda_n x_n + y_n \to \lambda x + y
\]
\section*{\centering Полная система элементов}
\begin{tcolorbox}
    \begin{definition}[Полная система элементов]
    Система элементов называется полной, если для любого элемента $ x $ этого пространства
    $ \forall \epsilon > 0 \ \exists L $ - линейная комбинация элементов системы такая, что
    $ ||L - x|| < \epsilon$ 

    Эквивалетно:

    система полная, если $ \forall x \in X \ E \set{L_n} \ L_n \to x \quad
    ||L_n - x|| \to 0$ 
    \end{definition}
\end{tcolorbox}

\emph{Пример} $ \set{1, t, t^2, t^3, \dots, t^n, \dots} $ - полная в $ C, L_2 $ 

$ \set{1, \cos x, \sin x, \cos 2x, \sin 2x, \dots} $ - полная в $ C_{2\pi}, L_{2\pi} $ 

\begin{tcolorbox}
    \begin{definition}[сходимость в себе]
        $ \set{x_n} $ сходится в себе по норме, если $ \forall \epsilon > 0, \ 
        \exists N(\epsilon): \ \forall m > n > N(\epsilon) \implies ||x_n - x_m|| < \epsilon$ 
    \end{definition}
\end{tcolorbox}

\begin{tcolorbox}
    \begin{definition}
        Если в ЛНП из сходимости в себе следует сходимость, то это полное ЛНП называется
        Банахово пространство (B - пространство)
    \end{definition}
\end{tcolorbox}

\begin{tcolorbox}
\begin{note}
    Подпространство в Банаховом пространстве само является Банаховым
\end{note}
\end{tcolorbox}

\section*{\centering Ряд элементов в Банаховом пространстве}

Пусть $ \set{x_n}_{n = 0\dots \infty} $. $ \set{s_n} \ s_n = \sum_{k=0}^{n} x_k $  
\[
    \lim_{n \to \infty} s_n = s = \sum_{n=0}^{\infty} x_n
\]
\[
    \underbrace{s_n}_{\to s} - \underbrace{s_{n-1}}_{\to s} = x_n
    \implies x_n \to 0
\]

\[
    \sum_{n=0}^{\infty} x_n \text{ - сходится} \iff
    \set{s_n} \text{ - сходится в себе}
\]

\begin{tcolorbox}
    \begin{theorem}[Достаточный признак сходимости]
        Если $ \forall n \ ||x_n|| \leq c_n, \quad \sum_{n=0}^{\infty}
        c_n = c$, то $ \sum_{n=0}^{\infty} x_n $ - сходится, причем $ ||s|| \leq c $   
    \end{theorem}
\end{tcolorbox}

\section*{\centering Понятие линейного ограниченного оператора}

\begin{tcolorbox}
\begin{definition}
    Пусть $ X, Y $ - ЛНП. $ A: X \to Y $ - линейный оператор

    Если $ \exists c: \ \forall x \in X \quad ||Ax|| \leq c \cdot ||x|| $, то
    оператор $ A $ называют линейным ограниченным
\end{definition}
\end{tcolorbox}

Рассмотрим
\[
    \rho(Ax, Az) = ||Ax - Az|| = ||A(x-z)|| \leq c \cdot ||x-z|| = c \cdot
    \rho(x,z)
\]

\begin{tcolorbox}
\begin{theorem}[1]
    Для того, чтобы линейный оператор был ограниченным необходимо и достаточно,
    чтобы он был непрерывным

    \begin{proof}
        $ \implies: $ A - лин. огр $ \implies A \in Lip \implies A $ - непрерывный 

        $ \impliedby: $ А - лин. непрерывный\\
        От противного. Допустим $ A $ - не ограниченный. $ \forall c \ \exists x
        ||Ax|| > c \cdot ||x||$.\\
        $ \forall n \ \exists x_n : \ ||Ax_n|| > n \cdot ||x_n|| $.

        При том $ x_n \neq 0 $.

        Построим последовательность
        \[
            z_n \coloneq \frac{x_n}{n \cdot ||x_n||}
        \]
        \[
            ||z_n|| = \frac{1}{n} \to 0, \ n \to \infty, \quad z_n \to 0
        \]

        \[
            Az_{n} = A\left(\frac{x_n}{n \cdot ||x_n||}\right) = 
            \frac{1}{n \cdot ||x_n||} \cdot A(x_n)
        \]
        \[
            ||Az_n|| = \frac{1}{n \cdot ||x_n||} ||Ax_n|| > 
            \frac{n \cdot ||x_n||}{n \cdot ||x_n||} = 1
        \]
        \[
            z_n \to 0 \quad Az_n \to A 0 = 0
        \]

        Противоречие
    \end{proof}
\end{theorem}
\end{tcolorbox}

\begin{tcolorbox}
    \begin{theorem}[2]
        Для того, чтобы линейный оператор был ограниченным необходимо и достаточно, чтобы
        он ограниченное множество отображал в ограниченное

        \begin{proof}
            Аналогично теореме 1
        \end{proof}
    \end{theorem}
\end{tcolorbox}
\begin{tcolorbox}
    \begin{statement}
        $ A $ - линейный ограниченный
        \[
            \sum_{n=0}^{\infty} \lambda_n x_n = s \implies \sum_{n=0}^{\infty} \lambda_n A x_n = As
        \]

        \begin{proof}
            \[
                A(s_n) = A\left(\sum_{k=0}^{\infty}  \lambda_k x_k\right)
                = \sum_{k=0}^{\infty} \lambda_k A(x_k)
            \]
            \[
                s_n \to s \quad A(s_n) \to As
            \]
        \end{proof}
    \end{statement}
\end{tcolorbox}

\section*{\centering Норма оператора}

\[
    M = \set{c: \ ||Ax|| \leq c ||x|| \quad \forall \, x}
\]
\[
    \alpha = \inf M \quad \exists c_n \in M, \ c_n \to \alpha
\]

Рассмотрим:
\[
    \forall x : \ ||Ax|| \leq c_n ||x|| \quad n \to \infty
\]
\[
    ||Ax|| \leq \alpha \cdot ||x|| \quad \forall x
\]
\[
    \alpha = \min M
\]
\begin{tcolorbox}
\begin{definition}
    Если $ A $ - линейный ограниченный оператор, то нормой оператора будем называть
    \[
        ||A|| = \min \set{c: \ ||Ax|| \leq c ||x||, \ \forall x}
    \]
\end{definition}
\end{tcolorbox}

\begin{tcolorbox}
1) $ ||A|| \geq 0 $\\
2) $ ||Ax|| \leq c \cdot ||x|| \ \forall x, \quad ||A|| \leq c $\\
\begin{equation}
    ||Ax|| \leq ||A|| \cdot ||x||
\end{equation}
\end{tcolorbox}

\[
    ||A|| = \min \set*{c: \frac{||Ax||}{||x||} \leq c, \ \forall x \neq 0}
\]
\[
    ||A|| = \sup \set*{\frac{||Ax||}{||x||}, \ \forall x \neq 0}
\]
\[
    ||A|| = \sup \set*{||Az||, \ \forall z: ||z|| = 1}
\]

Рассмотрим:
\[
    \beta = \sup \set{||Ax||, \ \forall x: \ ||x|| \leq 1}
\]
\[
    \beta \geq ||A||
\]
\[
    ||Ax|| \leq ||A|| \cdot ||x|| \leq ||A|| \quad \beta \leq ||A||
\]
\[
    \beta = ||A||
\]
\end{document}
