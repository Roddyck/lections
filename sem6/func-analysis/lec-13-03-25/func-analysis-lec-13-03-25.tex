\documentclass[a4paper]{article}
\usepackage[a4paper,%
    text={180mm, 260mm},%
    left=15mm, top=15mm]{geometry}
\usepackage[utf8]{inputenc}
\usepackage{cmap}
\usepackage[english, russian]{babel}
\usepackage{indentfirst}
\usepackage{amssymb}
\usepackage{amsmath}
\usepackage{amsthm}
\usepackage{mathtools}
\usepackage{tcolorbox}
\usepackage{import}
\usepackage{xifthen}
\usepackage{pdfpages}
\usepackage{transparent}
\usepackage{graphicx}
\graphicspath{ {./figures} }

\DeclarePairedDelimiter\set\{\}

\newcommand{\incfig}[1]{%
\def\svgwidth{\columnwidth}
\import{./figures/}{#1.pdf_tex}
}

\newtheorem*{theorem}{Теорема}
\newtheorem*{statement}{Утверждение}
\newtheorem*{lemma}{Лемма}
\newtheorem*{proposal}{Предложение}


\theoremstyle{definition}
\newtheorem*{definition}{Определение}

\theoremstyle{remark}
\newtheorem*{remark}{Замечание}

\renewcommand\qedsymbol{$\blacksquare$}

\begin{document}
\section*{\centering Предел в метрическом пространстве}

\begin{tcolorbox}
    \begin{definition}[Сходимость по расстоянию]
        $ x_n \to x: \ \rho(x_n) \xrightarrow[n \to +\infty]{} 0 $ 
    \end{definition}
\end{tcolorbox}

В $ \mathbb{R}^{n} $ - сх-ть покоординатная\\
$ m,C $ - сх-ть равномерная\\
$ s $ - сх-ть покоординатная\\
$ S $ - сх-ть почти всюду\\
$ L_{\infty} $ - равномерная сх-ть почти всюду\\
$ l, L $ - сх-ть в среднем\\
$ l_2, L_2 $ - среднеквадратичная сходимость

\subsection*{Свойства предела}

1) доб $ \lor $ удал 

2) единственнось

3) сходимость $ \implies $ ограниченность

\begin{tcolorbox}
    \begin{statement}[Непрерывность расстояния]
        \[
            \begin{cases}
                x_n \to x\\
                y_n \to y
            \end{cases} \implies
            \rho(x_n, y_n) \to \rho(x,y)
        \]
    \end{statement}
\end{tcolorbox}

$ x $ - точка прикосновения $ M $ 
\[
    \forall \epsilon = \frac{1}{n} \implies \exists x_n \in M: \ \rho(x_n,x) <
    \epsilon_n = \frac{1}{n} 
\]

\begin{tcolorbox}
\begin{theorem}
    Чтобы $ F $ было замкнутым $ \iff \Big((x_n \in F, \ x_n \to x) \implies (x \in
    F)\Big)$ 
\end{theorem}
\end{tcolorbox}

\section*{\centering Плотные множества}

\begin{tcolorbox}
    \begin{definition}[Плотное множество]
        $ A,B $ - мн-во в МП

        $ A $ плотно в $ B $, если $ \forall x \in B \ \forall \epsilon > 0 \ 
        \exists y \in A: \ y \in S(x,\epsilon)$ ($ \forall x \in B \ \exists
        \set{y_n}, \ y_n \in A: \ y_n \to X )$ 
    \end{definition}
\end{tcolorbox}

\emph{Пример} 1) $ A = \mathbb{Q}, \ B = \mathbb{R} $ \\
2) $ A = P, \ B = C $ 

\subsection*{Свойства}
1) $ A $ плотно в $ A $ 

2) $ A $ плотно в $ B $. $ B $ плотно в $ D $. Тогда $ A $ плотно в $ D $  

\begin{tcolorbox}
    \begin{definition}
        Если множество плотно во всем метрическом пространстве, то оно называется
        плотным всюду
    \end{definition}
\end{tcolorbox}

Рассмотрим множестве алгебраических полиномов с рациональными коэффицентами $ P_r $.
Оно плотно в $ P $, которое плотно в $ C \implies P_r$ плотно в $ C $  

$ P_r, P, C $ плотные в $ L_2 $ 

$ C_{2\pi}[-\pi, \pi] \ x(-\pi) = x(\pi) $ плотно в $ C[-\pi, \pi] $ по
метрике $ L_2[-\pi, \pi] $ 

$ T_r $ плотно в $ T $ плотно в $ C_{2\pi} $ 

\section*{\centering Сепарабельное метрическое пространство}
\begin{tcolorbox}
    \begin{definition}[Сепарабельное метрическое пространство]
        Пространство сепарабельно, если в нем существует конечное или счётное
        всюду плотное множество
    \end{definition}
\end{tcolorbox}

Известно, что $ m $ - не сепарабельное

\begin{tcolorbox}
    \begin{statement}
        Если $ X $ - сепарабельное МП, $ Y $ - подмножество, то $ Y $ - сепарабельное
        МП с метрикой, индуцированной из $ X $ 

        \begin{proof}
            $ X $ - сепарабельно, значит есть счётное $ A $ - всюду плотное в $ X $
            множество
            \[
                \forall y \in Y \ \forall n \ \epsilon = \frac{1}{n} \ \exists
                x_{ny} \in A: \ \rho(y, x_{ny}) < \frac{1}{n} 
            \]

            Рассмотрим множество
            \[
                \set*{z: \ z \in Y. \ \rho(z, x_{ny}) < \frac{1}{n}} 
            \]
            Счётное количество множеств

            Выберем $ z_n \in Y: \ \rho(z_n, x_{ny}) < \frac{1}{n} $.

            Тогда
            \[
                \rho(z_n,y) \leq \rho(z_n, x_{ny}) + \rho(x_{ny}, y) < \frac{2}{n}    
            \]

            Значит $ Y $ - сепарабельно
        \end{proof}
    \end{statement}
\end{tcolorbox}

\section*{\centering Сходимость в себе}
\begin{tcolorbox}
    \begin{definition}
        $ \set{x_n} $ сх-ся в себе, если $ \rho(x_m, x_n) \to 0, \ m > n \to +\infty $ 
    \end{definition}
\end{tcolorbox}

\begin{tcolorbox}
\begin{statement}
    Из сходимости следует сходимость в себе
    \begin{proof}
        $ x_n \to a $ 
        \[
            \rho(x_m, x_n) \leq \rho(x_m, a) + \rho(x_n, a) \to 0
        \]
    \end{proof}
\end{statement}
\end{tcolorbox}

\begin{tcolorbox}
\begin{statement}
    $ \exists \epsilon > 0 \quad \rho(x_m, x_n) \geq \epsilon $, то нет сходящейся
    подпоследовательности
\end{statement}
\end{tcolorbox}

\begin{tcolorbox}
\begin{statement}
    Если последовательность сходится в себе, то она ограничена
\end{statement}
\end{tcolorbox}

Сх-ть в себе $ \stackrel{?}{\implies} $ Сх-ть

\begin{tcolorbox}
    \begin{definition}[Полное метрическое пространство]
    Метрическое пространство называется полным, если в нём из сходимости в себе
    следует сходимость
\end{definition}
\end{tcolorbox}

\section*{\centering Компактное множество}

\begin{tcolorbox}
    \begin{definition}[Предкомпактное множество]
        Множество называется предкомпактным (относительно компактным), если
        любая последовательность его элементов содержит сходящуюся подпоследовательность
    \end{definition}
\end{tcolorbox}

\end{document}
