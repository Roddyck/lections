\documentclass[a4paper]{article}
\usepackage[a4paper,%
    text={180mm, 260mm},%
    left=15mm, top=15mm]{geometry}
\usepackage[utf8]{inputenc}
\usepackage{cmap}
\usepackage[english, russian]{babel}
\usepackage{indentfirst}
\usepackage{amssymb}
\usepackage{amsmath}
\usepackage{amsthm}
\usepackage{mathtools}
\usepackage[most]{tcolorbox}
\usepackage{import}
\usepackage{xifthen}
\usepackage{pdfpages}
\usepackage{transparent}
\usepackage{graphicx}
\graphicspath{ {./figures} }

\DeclarePairedDelimiter\set\{\}

\newcommand{\incfig}[1]{%
\def\svgwidth{\columnwidth}
\import{./figures/}{#1.pdf_tex}
}

\newtheorem*{theorem}{Теорема}
\newtheorem*{statement}{Утверждение}
\newtheorem*{lemma}{Лемма}
\newtheorem*{proposal}{Предложение}


\theoremstyle{definition}
\newtheorem*{definition}{Определение}

\theoremstyle{remark}
\newtheorem*{remark}{Замечание}

\renewcommand\qedsymbol{$\blacksquare$}

\begin{document}
\section*{\centering Компактное множество}

\begin{tcolorbox}
    \begin{statement}
        Если множество предкомпактно, то оно ограничено

        \begin{proof}
            Допустим, что это не так
            \[
                x_1 \in M \quad \exists x_2: \ \rho(x_1, x_2) > 1, \quad
                \exists x_3: \ \rho(x_1, x_3) > \rho(x_1, x_2) + 1 \dots
                \exists x_{n+1}: \ \rho(x_1, x_{n+1}) > \rho(x_1, x_n) + 1
            \]
            \[
                \rho(x_{n+1}, x_n) \geq \rho(x_{n+1}, x_1) - \rho(x_n, x_1) = 1
            \]
        \end{proof}
    \end{statement}
\end{tcolorbox}

\begin{tcolorbox}
    \begin{remark}
        Обратное верно только для конечномерных пространств
    \end{remark}
\end{tcolorbox}

\section*{\centering Критерий Хаусдорфа предкомпактности множества}

\begin{tcolorbox}
    \begin{definition}[$ \epsilon $ - сеть множества $K$]
        Множество $ M $ называется $ \epsilon $ - сетью множества $ K $, 
        если
        \[
            \forall x \in K \ \exists y \in M: \ \rho(x,y) < \epsilon
        \]
        $ M $ - конечное множество, конечная $ \epsilon $ - сеть
    \end{definition}
\end{tcolorbox}

\begin{tcolorbox}[enhanced,breakable,skin first=enhanced,skin middle=enhanced,skin last=enhanced]
    \begin{theorem}[Критерий Хаусдорфа предкомпактности множества]
        Для предкомпактности множества необходимо, а в полном метрическом
        пространстве и достаточно, чтобы
        \[
            \forall \epsilon > 0 \ \exists \text{ конечная } \epsilon \text{ - сеть
            для этого множества}
        \]

        \begin{proof}
            1. $ \implies: $ 

            Пусть дан $ \epsilon > 0 $\\
            Возьмём $ x_1 \in K $. Рассмотрим $ S(x_1, \epsilon) $\\
            1) $ K \subseteq S(x_1, \epsilon) $\\
            2) $ K \not\subseteq S(x_1, \epsilon) $\\
            Рассмотрим $ x_2 \notin S(x_1, \epsilon) \quad \rho(x_2, x_1) > \epsilon $\\
            1) $ K \subseteq S(x_1, \epsilon) \cup S(x_2, \epsilon) $ \\
            2) $ K \not\subseteq S(x_1, \epsilon) \cup S(x_2, \epsilon) $ \\
            Возьмём $ x_3, \ \rho(x_3, x_1) > \epsilon, \rho(x_3, x_2) > \epsilon $

            Пусть $ x_n \in \bigcup_{k=\overline{1, n-1}} S(x_k, \epsilon) $.
            Получаем последовательность $ \set*{x_n} $ 
            \[
                \rho(x_m, x_n) > \epsilon \quad m > n
            \]
            Противоречие с предкомпактностью. Значит процесс не бесконечный.
            Тогда существует конечная $ \epsilon $ - сеть

            2. $ \impliedby: $ предкомпактность? 

            Пусть $ \set{x_n}, \ x_n \in K $. Зададим $ \epsilon_n \to 0 $  

            $ \forall \epsilon_m \ \exists $ конечная $ \epsilon $ - сеть $ a_{p,m} $ 

            Рассмотрим шары $ S(a_{p,m}, \epsilon_1) $ 
            \[
                \bigcup S(a_{p,m}, \epsilon_1) \ni x_n
            \]
            \[
                S_1 = \bigcup S(a_{p,1}, \epsilon_1) \ni x_n 
            \] 
            \[
                T_1 \text{ - последовательность попавшая в } S_1
            \]
            \[
                S_2 = \bigcup S(a_{p,2}, \epsilon_2) \ni x_n 
            \] 
            \[
                T_2 \text{ - последовательность попавшая в } S_2
            \]
            \[
                T_{k+1} \subseteq T_k \subset S_k
            \]

            Выберем из $ T_1 $ элемент $ x_{n_1} $.\\ 
            Выберем из $ T_2 $ элемент $ x_{n_2}, \ n_2 > n_1 $.\\ 
            Выберем из $ T_k $ элемент $ x_{n_k} $. 

            Возьмём $ k > m $. Тогда $ x_{n_m} \in T_m \subset S_m $,
            $ x_{n_k} \in T_k \subset T_m \subset S_m $ 
            \[
                \rho(x_{n_k}, x_{n_m}) < 2 \epsilon_m \to 0 \quad
                \forall k>m \to +\infty
            \]
            \[
                \set*{x_{n_k}} \text{ сходится в себе}
            \]

            Полное МП $ \implies \exists \lim_{k \to \infty} x_{n_k} = x $ 
        \end{proof}
    \end{theorem}
\end{tcolorbox}

\section*{\centering Критерии предкомпактности в пространствах $ C, L_2 $} 
\begin{tcolorbox}
    $ C[a,b] $

    $ K $ - предкомпактно $ \iff $\\
    1) $ \exists c: \ |x(t)| \leq c \ (\forall x \in K, \ \forall t \in [a,b]) $\\
    2) $ \forall \epsilon > 0 \ \exists \delta(\epsilon): \ \forall h: \ |h| < 
    \delta(\epsilon), \ \forall x \in K, \ \forall t \in [a,b] \implies
    |x(t+h) - x(t)| < \epsilon$ - равностепенная непрерывность
\end{tcolorbox}

\begin{tcolorbox}
    $ L_2[a,b] $

    $ K $ - предкомпактно $ \iff $\\
    1) $ \exists c: \ \int_{a}^{b} x^2(t) dt \leq c \ \forall x \in K $ \\
    2) $ \forall \epsilon > 0 \ \exists \delta(\epsilon): \ \forall h: \ |h| < 
    \delta(\epsilon), \ \forall x \in K\implies
    \int_{a}^{b} (x(t+h) - x(t))^2 dt < \epsilon$
\end{tcolorbox}

\begin{tcolorbox}
    \begin{definition}[Компактное множество]
        Замкнутое предкомпактное множество называется компактным
    \end{definition}
\end{tcolorbox}

Отсюда следует, что в любой последовательности точек компакта есть сходящаяся
подпоследовательность, причём её предел тоже принадлежит компакту

\section*{\centering Непрерывные операции в метрических пространствах}

Пусть $ X,Y $ - метрические пространства. $ A: \ X \to Y $ - операция 

\begin{tcolorbox}
    \begin{definition}
        Если $ (x_n \to x) \implies (A(x_n) \to A(x)) $, то $ A $ непрерывная в точке $ x $ 
    \end{definition}
\end{tcolorbox}

\begin{tcolorbox}
    \[
        A: \ X \to Y = \mathbb{R} \text{ - функционал}
    \]
\end{tcolorbox}

\begin{tcolorbox}
    \begin{theorem}
        Если функционал $ f $ непрерывный на компакте $ K $, то $ f(K) $ ограниченно
        и на компакте существуют оба экстремума этого функционала
    \end{theorem}
\end{tcolorbox}
\end{document}
