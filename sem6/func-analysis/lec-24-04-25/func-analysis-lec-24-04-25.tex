\documentclass[a4paper]{article}
\usepackage[a4paper,%
    text={180mm, 260mm},%
    left=15mm, top=15mm]{geometry}
\usepackage[utf8]{inputenc}
\usepackage{cmap}
\usepackage[english, russian]{babel}
\usepackage{indentfirst}
\usepackage{amssymb}
\usepackage{amsmath}
\usepackage{amsthm}
\usepackage{mathtools}
\usepackage{tcolorbox}
\usepackage{import}
\usepackage{xifthen}
\usepackage{pdfpages}
\usepackage{transparent}
\usepackage{graphicx}
\graphicspath{ {./figures} }

\DeclarePairedDelimiter\set\{\}

\newcommand{\incfig}[1]{%
\def\svgwidth{\columnwidth}
\import{./figures/}{#1.pdf_tex}
}

\newtheorem*{theorem}{Теорема}
\newtheorem*{statement}{Утверждение}
\newtheorem*{lemma}{Лемма}
\newtheorem*{proposal}{Предложение}
\newtheorem*{consequence}{Следствие}

\theoremstyle{definition}
\newtheorem*{definition}{Определение}

\theoremstyle{remark}
\newtheorem*{remark}{Замечание}

\renewcommand\qedsymbol{$\blacksquare$}

\begin{document}
\section*{\centering B-пространство линейных ограниченных операторов}
Пусть $ X,Y $ - ЛНП\\
$ A_n: \ X \to Y $ - лин. огр.
\[
    \forall m > n \to + \infty \quad ||A_m  - A_n|| \to 0
\]

\begin{tcolorbox}
\begin{theorem}
    Если $ Y $ - Банахово пространство, то $ (X \to Y) $ - Банахово пространство

    \begin{proof}
        Рассмотрим:
        \[
            ||A_mx - A_n x|| \leq ||(A_m - A_n)(x)|| \leq 
            \underbrace{||A_m - A_n||}_{\to 0} \cdot ||x||
            \implies \set{A_n x} \text{ - сходится в себе}
        \]
        \[
            \exists \lim_{n \to \infty} A_n x = y \in Y
        \]
        Обозначим $ y \coloneq A(x) $ 
        \[
            A(\lambda x) = \lim_{n \to \infty} A_n(\lambda x) = \lambda
            \lim_{n \to \infty} A_n(x) = \lambda A(x)
        \]
        \[
            A(x+z) = \dots = A(x) + A(z)
        \]

        Дано $ \set{A_m} $ - сходится в себе $ \implies \set{A_n} $ - огр. т.е.
        $ \exists c \ | \ ||A_n|| \leq c, \ \forall n $ 
        \[
            ||A_n x|| \leq ||A_n|| \cdot ||x|| \leq c \cdot ||x||
        \]
        \[
            ||A_n x|| \leq c \cdot ||x||, \ n \to \infty
        \]
        \[
            ||Ax|| \leq c ||x||
        \]
        Следовательно $ A $ - линейный ограниченный

        Пусть $ ||x|| = 1 $. Рассмотрим:
        \[
            ||A_m x - A_n x|| \leq ||A_m - A_n|| \cdot ||x|| = ||A_m - A_n|| < 
            \varepsilon
        \]

        Фиксируем $ n $. $ m \to +\infty $. Рассмотрим:
        \[
            A_m x \to Ax
        \]
        \[
            \alpha_n = ||Ax - A_n x|| \leq \varepsilon
        \]
        \[
            ||A - A_n|| = \sup_{||x|| = 1} \alpha_n \leq \varepsilon
        \]

        Получаем, что
        \[
            A_n \to A
        \]
    \end{proof}
\end{theorem}
\end{tcolorbox}

\section*{\centering Положительно определённые операторы в линейных нормированных
пространствах}
\begin{tcolorbox}
\begin{definition}
    Пусть $ X,Y $ - ЛНП. $ A: \ X \to Y $ - линейный оператор

    Будем говорить, что этот оператор положительно определён $ A > 0 $, с нижней
    границей $ m > 0 $, если
    \[
        \forall x \quad ||Ax|| \geq m \cdot ||x||
    \]
\end{definition}
\end{tcolorbox}

\textbf{Пример}

1) $ A = I $
\[
    ||Ix|| = 1 \cdot ||x||
\]

2) $ C: X \to X, \ ||C|| \leq q < 1 $\\
$ A \coloneq I - C $ 
\[
    ||Ax|| = ||(I - C)x|| = ||Ix - Cx|| = ||x - Cx|| \geq ||x|| - ||Cx|| \geq
\]
\[
    ||Cx|| \leq ||C|| \cdot ||x|| \leq q \cdot ||x||
\]
\[
    \geq ||x|| - q \cdot ||x|| = (1 - q) \cdot ||x|| = m \cdot ||x||
\]

\section*{\centering Условия существования и ограниченности обратного оператора}
\begin{tcolorbox}
\begin{theorem}[1]
    Пусть $ X,Y $ - ЛНП, линейный оператор $ A: \ X \to Y $.

    Для того, чтобы $ A: \ X \to Y $ - сюрьекция и $ A > 0, \ m > 0 $
    необходимо и достаточно:
    \[
        \exists A^{-1} \text{ - сюрьективный линейный ограниченный,}
    \]
    причём
    \[
        ||A^{-1}|| \leq \frac{1}{m}, \ ||A^{-1} y|| \leq \frac{1}{m} ||y|| 
    \]

    \begin{proof}
        $ \implies: $ 
        \[
            A(X) = Y \quad \exists A^{-1} \quad A^{-1}(Y) = ?
        \]

        от противного. Рассмотрим $ x,y \in X. \ Ax = y, \ Az = y. $ 
        \[
            \underbrace{Ax-Az}_{=y-y=0} = A(x-z)
        \]
        \[
            0 = ||0|| = ||A(x-z)|| \geq m \cdot ||x-z||
        \]
        \[
            ||x-z|| \leq 0 \quad x - z = 0 \quad x = z
        \]

        Рассмотрим:
        \[
            ||y|| = ||Ax|| \geq m \cdot ||x|| = m \cdot ||A^{-1} y||
        \]
        \[
            ||y|| \geq m \cdot ||A^{-1} y||
        \]
        \[
            ||A^{-1} y|| \leq \frac{1}{m} ||y||
        \]

        $ \impliedby: $ 
        \[
            A^{-1}(Y) = X \quad ||x|| = ||A^{-1} y|| \leq \frac{1}{m} ||y||
            = \frac{1}{m} ||Ax||
        \]
        \[
            ||Ax|| \geq m \cdot ||x|| \quad \forall x
        \]
    \end{proof}
\end{theorem}
\end{tcolorbox}

\[
    Ax = y \quad x = A^{-1} y
\]
\[
    ||\Delta x|| = ||\Delta A^{-1} y|| = ||A^{-1} \Delta y|| \leq \frac{1}{m} 
    ||\Delta y||
\]

\begin{tcolorbox}
\begin{theorem}[2]
    Пусть $ A $ - лин.огр. $ : \ X \to X $. $ X $ - B-пр-во.\\
    $ ||A^n|| \leq a_n \ : \ \sum_{n=0}^{\infty} a_n = \sigma $. 

    Тогда
    \[
        \exists S = (I - A)^{-1} \quad S = \sum_{n=0}^{\infty} A^n, \ 
        ||S|| \leq \sigma
    \]

    \begin{proof}
        Существование оператора $ S $ следует из теоремы с прошлой лекции.

        Докажем, что $ S = (I - A)^{-1} $ 
        \[
            (I-A) \cdot S = (I - A) \cdot \sum_{n=0}^{\infty} A^n
            = \sum_{n=0}^{\infty} (A^n - A^{n+1}) = \sum_{n=0}^{\infty} A^n
            - \sum_{n=0}^{\infty} A^{n+1} = 
            \sum_{n=0}^{\infty} A^n - \sum_{m=1}^{\infty} A^m = A^0 = I
        \]
    \end{proof}
\end{theorem}
\end{tcolorbox}

Рассмотрим:
\[
    x = Ax + y
\]
\[
    Ix - Ax = y
\]
\[
    (I-A) x = y
\]
\[
    x = (I-A)^{-1} y = \sum_{n=0}^{\infty} A^n y
\]
\[
    x_n = \sum_{k=0}^{n} A^k y = y + Ay + A^2 y + \dots + A^n y
    = y + A(y + Ay + \dots + A^{n-1}y)
\]
\[
    x_n = y + Ax_{n-1}, \quad x_0 = y, \quad x_n \to x
\]

\begin{tcolorbox}
\begin{theorem}[3]
    В условиях теоремы 2 уравнение
    \[
        x = Ax + y
    \]
    имеет единственно решение и итерационный процесс
    \[
        x_n = y + Ax_{n-1}, \quad x_0 = y, \quad x_n \to x
    \]
    сходится к этому решению
\end{theorem}
\end{tcolorbox}
\end{document}
