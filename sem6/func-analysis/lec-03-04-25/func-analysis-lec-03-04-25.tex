\documentclass[a4paper]{article}
\usepackage[a4paper,%
    text={180mm, 260mm},%
    left=15mm, top=15mm]{geometry}
\usepackage[utf8]{inputenc}
\usepackage{cmap}
\usepackage[english, russian]{babel}
\usepackage{indentfirst}
\usepackage{amssymb}
\usepackage{amsmath}
\usepackage{amsthm}
\usepackage{mathtools}
\usepackage{tcolorbox}
\usepackage{import}
\usepackage{xifthen}
\usepackage{pdfpages}
\usepackage{transparent}
\usepackage{graphicx}
\graphicspath{ {./figures} }

\DeclarePairedDelimiter\set\{\}

\newcommand{\incfig}[1]{%
\def\svgwidth{\columnwidth}
\import{./figures/}{#1.pdf_tex}
}

\newtheorem*{theorem}{Теорема}
\newtheorem*{statement}{Утверждение}
\newtheorem*{lemma}{Лемма}
\newtheorem*{proposal}{Предложение}
\newtheorem*{consequence}{Следствие}

\theoremstyle{definition}
\newtheorem*{definition}{Определение}

\theoremstyle{remark}
\newtheorem*{remark}{Замечание}

\renewcommand\qedsymbol{$\blacksquare$}

\begin{document}
\section*{\centering Линейное пространство}
\begin{tcolorbox}
    \begin{definition}
        $ A \subset X $ - лин. пр-во.

        А - линейное подпространство, если 
        \[
            \forall \, x,y \in A \implies x+y \in A \quad \lambda \cdot x \in A
        \]
    \end{definition}
\end{tcolorbox}

\begin{tcolorbox}
    \begin{definition}
        $ X,Y $ - лин. пр-во. $ A: X \to Y $.

        A - линейный оператор, если 
        \[
            A(x+y) = A(x) + A(y)
        \]
        \[
            A(\lambda x) = \lambda A(x)
        \]
    \end{definition}
\end{tcolorbox}

Рассмотрим операции
\[
    (A+B)x \coloneq Ax + Bx
\]
\[
    (\lambda A)x \coloneq \lambda Ax
\]
\[
    0x \coloneq 0
\]

Таким образом, множество линейных операторов является линейным пространством

Введем в рассмотрение
\[
    (AB)x \coloneq A(Bx) \quad \forall\, x
\]
\[
    A(B-C) = AB - AC
\]

Рассмотрим
\[
    A: \ X \to X
\]
\[
    Ix \equiv x \ (\forall x) \text{ тождественный оператор}
\]
\[
    A^0 = I \quad A^1 = A \quad A^2 = A A \quad \dots \ A^n = A A^{n-1}
\]

\[
    A: \ X \to Y
\]
Рассмотрим
\[
    A(X) \subseteq Y
\]

\begin{tcolorbox}
    \begin{definition}[Обратный оператор]
        $ X,Y $ - лин. пр-ва, $ A: \ X \to Y $ - сюрьект.

        Если $ \forall y \in Y\  \exists ! x \in X: \ y = Ax $, то A - биекция

        Тогда существует обратный оператор
        \[
            A^{-1}: \ Y \to X
        \]
        \[
            Ax = y \sim x = A^{-1} y
        \]
    \end{definition}
\end{tcolorbox}

\emph{Пример}\\
$ X $ - гладкие функции $ x(t) \quad x \in [a,b], \ x(a) = 0, \ Y = C[a,b] $ 

\[
    y = Ax, \ y = x'(t), \ y = \frac{d}{dt}(x(t))
\]
\[
    x = A^{-1} y \quad x(t) = \int_{a}^{t} y(\tau) d \tau
\]

\begin{tcolorbox}
    \begin{statement}
        $ A $ - линейный, то $ A^{-1} $ - линейный
    \end{statement}
\end{tcolorbox}

\begin{tcolorbox}
    \begin{statement}
        $ A $ - линейный обратимый, то $ \set{e_1, e_2, \dots , e_n} $ - лнз
        $ \iff \set{Ae_1, Ae_2, \dots, Ae_n} $ - лнз 
    \end{statement}
\end{tcolorbox}

\section*{\centering Линейное нормированное пространство}
\begin{tcolorbox}
    \begin{definition}
        $ X $ - ЛП, называется ЛНП, если  
        \[
            \forall \, x \in X, \ \exists \ ||x||
        \]
        1) $ ||x|| \geq 0, \quad ||x|| = 0 \iff x = 0 $ 

        2) $ ||\lambda x|| = |\lambda| \cdot ||x|| $ 

        3) $ ||x+y|| \leq ||x|| + ||y|| $ 
    \end{definition}
\end{tcolorbox}

\subsection*{Свойства}
$ || -x || = ||x|| $\\
$ ||0|| = 0 $\\
$ ||x-y|| \leq ||x|| + ||y|| $ \\
$ \bigg| ||x|| - ||y|| \bigg| \leq ||x-y|| $ 

\begin{tcolorbox}
    \begin{statement}
        ЛНП является МП
    \end{statement}
\end{tcolorbox}

\begin{tcolorbox}
    \begin{definition}
        Подпространством $ Y $ в ЛНП $ X $ будем называть линейное подпространство
        линейного пространства замкнутое (относительно предельного перехода)
        \[
            \forall y_n \in Y, \ y_n \to y \text{ в ЛНП } X \implies y \in Y
        \]
    \end{definition}
\end{tcolorbox}
\end{document}
