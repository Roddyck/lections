\documentclass[a4paper]{article}
\usepackage[a4paper,%
    text={180mm, 260mm},%
    left=15mm, top=15mm]{geometry}
\usepackage[utf8]{inputenc}
\usepackage{cmap}
\usepackage[english, russian]{babel}
\usepackage{indentfirst}
\usepackage{amssymb}
\usepackage{amsmath}
\usepackage{amsthm}
\usepackage{mathtools}
\usepackage{tcolorbox}
\usepackage{import}
\usepackage{xifthen}
\usepackage{pdfpages}
\usepackage{transparent}
\usepackage{graphicx}
\graphicspath{ {./figures} }

\newcommand{\incfig}[1]{%
\def\svgwidth{\columnwidth}
\import{./figures/}{#1.pdf_tex}
}

\newtheorem*{theorem}{Теорема}
\newtheorem*{statement}{Утверждение}
\newtheorem*{lemma}{Лемма}
\newtheorem*{proposal}{Предложение}


\theoremstyle{definition}
\newtheorem*{definition}{Определение}

\theoremstyle{remark}
\newtheorem*{note}{Замечание}

\renewcommand\qedsymbol{$\blacksquare$}

\begin{document}
\section*{\centering Метрические пространства}
$ X $ - метрическое пространство, если: \\
$ \forall x,y \in X $ задано число $ \rho(x,y) $ - расстояние между $ x $ и $ y $  
(метрика)

\subsection*{Примеры}

1) $ \rho(x,y) \geq 0, \ \rho(x,y) = 0 \iff x = y $\\
2) $ \rho(x,y) = \rho(y,x) $\\
3) $ \rho(x,z) \leq \rho(x,y) + \rho(y,z) $ 

\[
    |\rho(x,z) - \rho(y,z)| \leq \rho(x,y) \text{ - следствие из 3)}
\]
\[
    \rho(x,y) = \rho(x-y, 0) \text{ - инвариантность относительно сдвига}
\]
\[
    \rho(x) \stackrel{\text{dn}}{=} \rho(x,0)
\]
\[
    \rho(x+y) \leq \rho(x) + \rho(y)
\]

1) $ \mathbb{R}^{n} $. $ \vec{x} = \{x_1, x_2, \dots, x_n\} $
\[
    \rho^2(x) = \sum_{k=1}^{n} x^2_k
\]

2) $ m_n $. $ x = (x_1, x_2, \dots, x_n) $ - упорядоченный набор 
\[
    \rho(x) = \max_{k = 1, \dots, n} |x_k|
\]

3) $ m $ - пространство ограниченных числовых последовательностей.
$ x = (x_1, x_2, \dots, x_n \dots) $ 
\[
    \rho(x) = \sup_{k=1, \dots, \infty} |x_k|
\]

4) $ l $. $ x = (x_1, \dots, x_n \dots) $  
\[
    \rho(x) = \sum_{n=1}^{\infty} |x_n| < +\infty
\]

5) $ l_2 $. $ x = (x_1, \dots, x_n \dots) $  
\[
    \rho^2(x) = \sum_{n=1}^{\infty} x^2_n < +\infty
\]

6) $ s $ - пространство всех числовых последовательностей. 
$ x = (x_1, \dots, x_n \dots) $
\[
    \rho(x) = \sum_{n=1}^{\infty} \frac{|x_n|}{2^{n}(1 + |x_n|)} 
\]

7) $ S $ - измеримые функции. $ x(t), \ t \in [a,b] $ 
\[
    \rho(x) = \int_{a}^{b} \frac{|x|}{1+|x|} dt 
\]

8) $ L $ 
\[
    \rho(x) = \int_{a}^{b} |x(t)| dt
\]

9) $ L_2 $ 
\[
    \rho^2(x) = \int_{a}^{b} x^2(t) dt
\]

10) $ C $ 
\[
    \rho(x) = \max_{t \in [a,b]} |x(t)|
\]

11) $ L_{\infty} $ - ограниченно измеримые функции 
\[
    \rho(x) = \inf \{C: C \geq |x(t)| \text{ п.в. } t \in [a,b]\}
\]

12) $ C^1_{[a,b]} \subset C_{[a,b]} $ 
\[
    \rho(x) = \max_{\forall t}|x(t)| + \max_{\forall t}|x'(t)|
\]

\section*{\centering Индуцированная метрика}
\begin{tcolorbox}
\begin{definition}
    $ X $ - метр. пр-во, $ Z \subset X $ 

    $ Z $ - метр. пр-во, с метрикой, индуцированной из $ X $ 
\end{definition}
\end{tcolorbox}

\subsection*{Виды множеств в метрических пространствах}
$ a \in X $ - м.п., $ r \geq 0 $ 
\[
    S(a,r) = \{ x: \ x \in X, \ \rho(a,x) < r \} \text{ открытый шар}
\]
\[
    \overline{S}(a,r) = \{ x: \ x \in X, \ \rho(a,x) \leq r \} \text{ замкнутый шар}
\]

\begin{tcolorbox}
    \begin{definition}
        Окрестность $ S(a,\epsilon), \ \epsilon > 0 $ 
    \end{definition}
\end{tcolorbox}

\begin{tcolorbox}
\begin{definition}
    $ M \subset X $ - метр. пр-во

    Если для $ x \in M \ \exists S(x,\epsilon) \subset M $, то $ x $ - внутренняя
    точка $ M $ 
\end{definition}
\end{tcolorbox}

\begin{tcolorbox}
\begin{definition}
    $ x \in X $ - метр. пр-во, называется точкой прикосновения множества $ M $,
    если $ \forall \epsilon > 0 \ \exists y \in M \ \rho(x,y) < \epsilon $ 
\end{definition}
\end{tcolorbox}

\begin{tcolorbox}
\begin{definition}
    Если множество содержит все свои точки прикосновения, то оно называется 
    замкнутым
\end{definition}
\end{tcolorbox}

\begin{tcolorbox}
\begin{statement}
    Замыкание множества является замкнутым множеством
\end{statement}
\end{tcolorbox}

\begin{tcolorbox}
\begin{statement}
    Для того, чтобы множество было замкнутым необходимо и достаточно
    открытость его дополнения
\end{statement}
\end{tcolorbox}

\begin{tcolorbox}
\begin{definition}
$ x_n \to x: \ \rho(x_n, x) \xrightarrow[n \to +\infty]{} 0 $ 
\end{definition}
\end{tcolorbox}

\end{document}
