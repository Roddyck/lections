\documentclass[a4paper]{article}
\usepackage[a4paper,%
    text={180mm, 260mm},%
    left=15mm, top=15mm]{geometry}
\usepackage[utf8]{inputenc}
\usepackage{cmap}
\usepackage[english, russian]{babel}
\usepackage{indentfirst}
\usepackage{amssymb}
\usepackage{amsmath}
\usepackage{amsthm}
\usepackage{mathtools}
\usepackage{tcolorbox}
\usepackage{import}
\usepackage{xifthen}
\usepackage{pdfpages}
\usepackage{transparent}
\usepackage{graphicx}
\graphicspath{ {./figures} }

\newcommand{\incfig}[1]{%
\def\svgwidth{\columnwidth}
\import{./figures/}{#1.pdf_tex}
}

\newtheorem*{theorem}{Теорема}

\theoremstyle{definition}
\newtheorem*{definition}{Определение}

\theoremstyle{remark}
\newtheorem*{remark}{Замечание}

\renewcommand\qedsymbol{$\blacksquare$}

\begin{document}
\[
    f(x) = 0 \ \forall x \in M \implies f(x) = 0 \text{ п.в.}
\]
\[
    f(x) = 0 \text{ п.в.}, \ f(x) \text{ непрерывная} \implies f(x) = 0 \text{ п.в.}
\]

\section*{\centering Ступенчатые функции}

$ [a,b] $ разобьём точками $ x_i \quad i = \overline{0,n}, \ x_0 = a, \ x_n = b $ 

\[
    \Delta x_i = x_i - x_{i-1} \quad x_{i-1} < x_i
\]
\[
    \phi(x) = c_i \quad (x_{i-1} < x < x_i)
\]

\begin{figure}[ht]
    \centering
    \incfig{ступенчатая-функция}
    \caption{ступенчатая функция}
    \label{fig:ступенчатая-функция}
\end{figure}

$ \phi, \psi $ - ступенчатые функции\\
$ \phi \leq \psi $, если $ \phi(x) \leq \psi(x) $ п.в. (во всех точках, кроме
узлов)

\section*{\centering Понятие интеграла Лебега}
\[
    \int_{a}^{b} \phi(x) dx = \sum_{i=1}^{n} c_i \Delta x_i
\]

Пусть $ f(x) \leq 0 \quad x \in [a,b] $ 

Если $ \phi_n(x) \rightarrow f(x) $ п.в. и $ 0 \leq \phi_n(x) \leq \phi_{n+1}(x) $,
то говорят, что $ \phi_n $ сходится к $ f $ снизу (обозначение $ \phi_n \uparrow f $) 

\begin{tcolorbox}[title=Интеграл Лебега]
    \begin{definition}[Интеграл Лебега]
        \[
            \phi_n \uparrow f, \ \exists \lim_{n \to \infty} \int_{a}^{b} 
            \phi_n(x) dx = I \text{ - интеграл Лебега}
        \]
        \[
            I = \int_{a}^{b} f(x) dx
        \]
    \end{definition}
\end{tcolorbox}

\begin{tcolorbox}
\begin{remark}
    Известно, что если несобственный интеграл от неограниченной функции сходится
    абсолютно, то существует интеграл Лебега с тем же значением
\end{remark}
\end{tcolorbox}

\[
    f(x) \gtrless 0 \quad x \in [a,b]
\]
\[
    f_+(x) = \begin{cases}
        f(x), &\quad f(x) \geq 0\\
        0, &\quad f(x) < 0
    \end{cases}
\]
\[
    f_-(x) = \begin{cases}
        0, &\quad f(x) > 0\\
        -f(x), &\quad f(x) \leq 0\\
    \end{cases}
\]
\[
    f = f_+ - f_-
\]
\[
    |f| = f_+ + f_-
\]
\[
    \begin{cases}
        \exists \int_{a}^{b} f_+(x) dx\\
        \exists \int_{a}^{b} f_-(x) dx\\
    \end{cases} \implies
    \exists \int_{a}^{b} f(x) dx = \int_{a}^{b} f_+(x) dx - \int_{a}^{b} f_-(x) dx
\]

Зафиксируем отрезок $ [a,b] $ и рассмотрим множество функций, для которых
существует интеграл Лебега. Такое множество называется пространством Лебега
$ L[a,b] $ 
\subsection*{Свойства}
\[
    1) \ \int_{a}^{b} c dx = c(b-a)
\]
\[
    2) \ \left| \int_{a}^{b} f(x) dx \right| \leq \int_{a}^{b} |f(x)| dx
\]
\[
    3) \ f \in L, \ g(x) = f(x) \text{ п.в} \implies g(x) \in L \quad 
    \int_{a}^{b} f(x) dx = \int_{a}^{b} g(x) dx
\]
\[
    4) \ f(x) \geq g(x) \text{ п.в.} \implies \int_{a}^{b} f dx \geq \int_{a}^{b} 
    g dx
\]
\[
    5) \ \int_{a}^{b} |f(x)| dx = 0 \iff f(x) = 0 \text{ п.в.}
\]
\[
    6) \ f,g \in L, \ \lambda \in \mathbb{R} \implies \lambda f \pm g \in L
    \quad \int_{a}^{b} (\lambda f \pm g) dx = \lambda \int_{a}^{b} fdx \pm 
    \int_{a}^{b} gdx
\]
\[
    7) \ \int_{a}^{b} |f \pm g| \leq \int_{a}^{b} |f|dx + \int_{a}^{b} |g|dx
\]

\begin{tcolorbox}[title=Теорема Лебега]
    \begin{theorem}[Лебега]
        Пусть $ \{ f_n(x) \}, \ f_n(x) \in L $. $ f_n(x) \to f(x) $ п.в.\\
        $ |f_n(x)| \leq \phi(x) $ п.в., при достаточно больших $ n \geq n_0 $.
        $ \phi(x) \in L $ 

        Тогда
        \begin{displaymath}
            f \in L \quad \int_{a}^{b} f_n(x) dx \to \int_{a}^{b} f(x) dx
        \end{displaymath}

    \end{theorem}
\end{tcolorbox}

\begin{tcolorbox}[title=Сходимость в среднем]
    \begin{definition}[Сходимость в среднем]
        $ f_n, \ f \in L. \quad n \in \mathbb{N} \cup \{ 0 \} $ 

        Будем говорить $ f_n $ сходится в среднем (сходимость в среднем 1-го порядка)
        к $ f $, если
        \[
            \int_{a}^{b} |f_n - f| dx \to 0 \quad (f_n \xrightarrow{L} f)
        \]
    \end{definition}
\end{tcolorbox}

\subsection*{Свойства}

\begin{tcolorbox}[title=Теорема о непрерывности интеграла относительно сходимости]
    \begin{theorem}[о непрерывности интеграла относительно сходимости]
        \[
            f_n \xrightarrow{L} f \implies \int_{a}^{b} f_n dx \to \int_{a}^{b} 
            f dx
        \]

        \begin{proof}
            \[
                \left| \int_{a}^{b} f_n dx - \int_{a}^{b} f dx \right| =
                \left|\int_{a}^{b} (f_n - f) dx \right| \leq \int_{a}^{b} 
                |f_n - f| dx \to 0
            \]
        \end{proof}
    \end{theorem}
\end{tcolorbox}

\begin{tcolorbox}
\begin{theorem}
    \[
        f_n \rightrightarrows f \ x \in [a,b] \implies f_n \xrightarrow{L} f
    \]

    \begin{proof}
        \[
            \forall \epsilon > 0 \ \exists N(\epsilon) : \ \forall n > N \ 
            \forall x \in [a,b] \implies |f_n - f| < \frac{\epsilon}{2(b-a)} 
        \]
        \[
            \int_{a}^{b} |f_n - f| dx < \frac{\epsilon}{2(b-a)}\int_{a}^{b} dx =
            \frac{\epsilon}{2} < \epsilon
        \]
    \end{proof}
\end{theorem}
\end{tcolorbox}
\end{document}
