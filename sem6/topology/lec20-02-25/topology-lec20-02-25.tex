\documentclass[a4paper]{article}
\usepackage[a4paper,%
    text={180mm, 260mm},%
    left=15mm, top=15mm]{geometry}
\usepackage[utf8]{inputenc}
\usepackage{cmap}
\usepackage[english, russian]{babel}
\usepackage{indentfirst}
\usepackage{amssymb}
\usepackage{amsmath}
\usepackage{amsthm}
\usepackage{mathtools}
\usepackage[most]{tcolorbox}
\usepackage{import}
\usepackage{xifthen}
\usepackage{pdfpages}
\usepackage{transparent}
\usepackage{graphicx}
\graphicspath{ {./figures} }

\newcommand{\incfig}[1]{%
\def\svgwidth{\columnwidth}
\import{./figures/}{#1.pdf_tex}
}

\newtheorem*{theorem}{Теорема}
\newtheorem*{statement}{Утверждение}
\newtheorem*{corollary}{Следствие}

\theoremstyle{definition}
\newtheorem*{definition}{Определение}

\theoremstyle{remark}
\newtheorem*{remark}{Замечание}

\renewcommand\qedsymbol{$\blacksquare$}

\begin{document}
\begin{tcolorbox}[title=Гомотопный класс отображения]
    \begin{definition}[Гомотопный класс отображения]
        Пусть $ f: X \to Y $ - непрерывное 

        Гомотопным классом отображения $ f $ называется
        \[
            [f] \coloneqq \{ g: \ X \to Y \text{ непр-на}. \ f \simeq g\}
        \]
    \end{definition}
\end{tcolorbox}

\textbf{\underline{Пример}} $ X $ - любое топологическое пространство

$ f: X \to \mathbb{R}^{n} $ - непр

$ \forall g: X \to \mathbb{R}^{n} $ - непр. $ f \simeq g $ 

\emph{Доказательство}

\begin{figure}[ht]
    \centering
    \incfig{линейная-гомотопия}
    \caption{линейная гомотопия}
    \label{fig:линейная-гомотопия}
\end{figure}

\[
    h_t(x) \coloneqq (1-t)f(x) + tg(x) \text{ - линейная гомотопия}
\]

\textbf{\underline{Пример}} $ Y \subset \mathbb{R}^{n} $ - выпуклое множество.\\
Любое топологическое пространство $ X \quad \forall f,g : \ X \to Y $ непр. 
$ f \simeq g $ 

\vspace{5mm}

Пусть $ X = \{ p \} $ - 1-элементное множество $ \implies $ гомотопия 
$ h_t: X \to Y $ - путь в $ Y $ 

\begin{figure}[ht]
    \centering
    \incfig{гомотопия-путь}
    \caption{гомотопия-путь}
    \label{fig:гомотопия-путь}
\end{figure}
\newpage

\begin{tcolorbox}[title=Связная гомотопия]
    \begin{definition}[Связная гомотопия]
        Пусть $ f,g: X \to Y $ - непр. $ A \subset X \quad f|_{A} = g|_{A} $ 

        Гомотопия $ h_t $ отображений $ f,g $ называется связной на множестве $ A $,
        если
        \[
            \forall t \quad h_t|_{A} = f|_{A} = g|_{A}
        \]
        т.е.
        \[
            \forall x \in A \quad h_t(x) = f(x) = g(x)
        \]
    \end{definition}
\end{tcolorbox}

\begin{tcolorbox}[title=Гомотопия путей]
    Пусть $ f,g : \ I \to Y $ - пути. $ f(0) = g(0), \ f(1) = g(1) $.

    Под гомотопией путей всегда понимаем гомотопию, связанную на $ \{ 0, 1 \} $ 
\end{tcolorbox}

\begin{tcolorbox}[title=Свободная гомотопия]
    \begin{remark}
        $ A = \varnothing $, то гомотопия называется свободной
    \end{remark}
\end{tcolorbox}

\section*{\centering \S 2. Определение фундаментальной группы}
\begin{tcolorbox}
    \begin{definition}
        Топологическим пространством с отмеченной точкой называется пара
        $ (X, x_0), \ x_0 \in X $ 
        
        Петлёй с началом в точке $ x_0 $ называется непрерывное отображение
        $ f: \ I \to X, \ f(0) = f(1) = x_0 $ 
    \end{definition}
\end{tcolorbox}

\begin{figure}[ht]
    \centering
    \incfig{петля}
    \caption{петля}
    \label{fig:петля}
\end{figure}
\newpage

Гомотопия петель связана на $ \{ 0, 1 \} $ 

$ F(X, x_0) $ - множество всех петель с началом в точке $ x_0 $ 

$ \pi_1(X, x_0) $ - множество гомотопических классов петель

\textbf{\underline{Пример}} $ \pi_1(\mathbb{R}^{n}, 0) $ - одноэлементное множество,
т.к. линейная гомотопия путей связана на $ \{ 0, 1 \} $ 
\[
    h_t(x) = (1-t)h_0(x) + t h_1(x)
\]
\[
    h_t(0) = (1-t)h_0(0) + t h_1(0) = (1-t+t)h_0(0) = h_0(0)
\]
аналогично $ h_t(1) = h_0(1) $ 

\subsection*{Умножение петель}
\[
    u,v \in F(X, x_0) \quad (uv)(t) \coloneqq
    \begin{cases}
        u(2t), &\quad 0 \leq t \leq \frac{1}{2} \\
        v(2t-1), &\quad \frac{1}{2} \leq t \leq 1
    \end{cases}
\]
\newpage

\begin{figure}[ht]
    \centering
    \incfig{умножение-петель}
    \caption{умножение петель}
    \label{fig:умножение-петель}
\end{figure}

\begin{tcolorbox}
    \begin{statement}
        $ f_0, g_0 $ - пути в $ X $. $ f(1) = g(0) $ \\
        \[
            (f_0 g_0)(t) = 
            \begin{cases}
                f_0(2t), &\quad 0 \leq t \leq \frac{1}{2} \\
                g_0(2t-1), &\quad \frac{1}{2} \leq t \leq 1
            \end{cases}
        \]
        \[
            \begin{cases}
                f_0 \simeq f_1 \\
                g_0 \simeq g_1
            \end{cases} \implies
            f_0 g_0 \simeq f_1 g_1
        \]
        \begin{proof}
            \begin{center}
                \incfig{схема-построения-гомотопии}
            \end{center}

            $ f_t $ - гомотопия, соединяющая $ f_0 $ с $ f_1 $

            $ g_t $ - гомотопия, соединяющая $ g_0 $ с $ g_1 $ 

            $ \forall t \quad f_t g_t $ соединяет $ f_0 g_0 $ с $ f_1 g_1 $ 

            По лемме о фундаментальном покрытии $ \{ f_t g_t \} $ непрерывное отображение
        \end{proof}
    \end{statement}

    \begin{corollary}
        $ u,v \in F(X, x_0) $ 
        \[
            [u][v] \coloneqq [uv] \text{ корректно определено}
        \]
    \end{corollary}
\end{tcolorbox}

\begin{tcolorbox}[title=Теорема о фундаментальной группе,
    enhanced, 
    breakable,
    skin first=enhanced,
    skin middle=enhanced,
    skin last=enhanced,
    ]
    \begin{theorem}[о фундаментальной группе]
        \[
            (\pi_1(X, x_0), \text{ умножение}) \text{ - группа}
        \]

        \begin{proof}
            1) ассоциативность
            \[
                ([u][v])[w] = [u]([v][w]) - ?
            \]
            \[
                [uv][w] = [u][vw] - ?
            \]
            \[
                [(uv)w] = [u(vw)] - ?
            \]
            \[
                (uv)w \simeq u(vw) - ?
            \]
            \begin{center}
                \incfig{построение}
            \end{center}
            \[
                H(s,t) =
                \begin{cases}
                    u \left(\frac{4s}{t+1} \right), &\quad 0 \leq s \leq \frac{t+1}{4}\\
                    v \left(4s-t-1 \right), &\quad \frac{t+1}{4} \leq s \leq \frac{t+2}{4}\\
                    w \left(\frac{4s-t-2}{2-t} \right), &\quad 
                    \frac{t+2}{4} \leq s \leq 1\\
                \end{cases}
            \]

            2) $ \exists \epsilon_{x_0} = [e_{x_0}]: \ [u]\epsilon_{x_0}
            = \epsilon_{x_0}[u] = [u]$ ?

            $ e_{x_0} $ - постоянная петля: $ e_{x_0}(I) = x_0 $ 
            \[
                u e_{x_0} \simeq u - ?
            \]

            \begin{center}
                \incfig{нейтральный-элемент}
            \end{center}
            \[
                H(s,t) \coloneqq
                \begin{cases}
                    u \left(\frac{2s}{t+1} \right), &\quad 0 \leq s \leq \frac{t+1}{2}\\ 
                    e_{x_0}(s), &\quad \frac{t+1}{2} \leq s \leq 1
                \end{cases}
            \]
        \end{proof}
    \end{theorem}
\end{tcolorbox}
\end{document}
