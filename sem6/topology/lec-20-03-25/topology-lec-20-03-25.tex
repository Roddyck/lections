\documentclass[a4paper]{article}
\usepackage[a4paper,%
    text={180mm, 260mm},%
    left=15mm, top=15mm]{geometry}
\usepackage[utf8]{inputenc}
\usepackage{cmap}
\usepackage[english, russian]{babel}
\usepackage{indentfirst}
\usepackage{amssymb}
\usepackage{amsmath}
\usepackage{amsthm}
\usepackage{mathtools}
\usepackage[most]{tcolorbox}
\usepackage{xfrac}
\usepackage{import}
\usepackage{xifthen}
\usepackage{pdfpages}
\usepackage{transparent}
\usepackage{graphicx}
\graphicspath{ {./figures} }

\DeclarePairedDelimiter\set\{\}

\newcommand{\incfig}[1]{%
\def\svgwidth{\columnwidth}
\import{./figures/}{#1.pdf_tex}
}

\newtheorem*{theorem}{Теорема}
\newtheorem*{statement}{Утверждение}
\newtheorem*{lemma}{Лемма}
\newtheorem*{proposal}{Предложение}


\theoremstyle{definition}
\newtheorem*{definition}{Определение}

\theoremstyle{remark}
\newtheorem*{remark}{Замечание}

\renewcommand\qedsymbol{$\blacksquare$}

\begin{document}
\[
    I \xrightarrow{f} B 
\]
\begin{figure}[ht]
    \centering
    \incfig{pic1}
    \caption{pic1}
    \label{fig:pic1}
\end{figure}
Отрезок разбит на $ t_0, \dots, t_n $ 
\[
    \widetilde{f}|_{[t_0,t_1]} \coloneq p^{-1} \circ f|_{[t_0, t_1]}
\]
\[
    f(t_i) \eqcolon b_i
\]
\[
    \widetilde{f}(t_1) \eqcolon x_1 \in V_2 \quad \widetilde{f}|_{[t_{i-1}, t_i]}
    \coloneq p^{-1} \circ f|_{[t_{i-1}, t_i]}
\]
\[
    x_i \in V_{i+1}, \ p: \ V_{i+1} \to U_{i+1}
\]

$ \widetilde{f} $ непрерывно, т.к. $ \forall i \ \widetilde{f}|_{t_{i-1}, t_i} $ 
непрерывно, а $ set{[t_{i-1}, t_i]} $ - фундаментальное покрытие отрезка $ I $.
Единственность $ \widetilde{f} $ следует из построения

\begin{tcolorbox}
    \begin{theorem}[О накрывающей гомотопии]
        Пусть $ p: \ X \to B $ - накрытие, $ b_0 \in B $, $ x_0 \in p^{-1}(b_0) $.
        Пусть пути $ f, f': I \to B $ с $ f(0) = f'(0) = b_0, \ f(1) = f'(1) $ 
        гомотопны.

        Тогда накрывающие пути $ \widetilde{f}, \widetilde{f'} $ с $ \widetilde{f}(0)
        = \widetilde{f}'(0) = x_0$ гомотопны и, в частности $ \widetilde{f}(1)
        = \widetilde{f}'(1)$ 

        \begin{proof}
            $ \set*{ U \text{ - прав. накр. окр-ть некоторой точки} \in B} $ -
            открытое покрытие базы $ B $.

            $ H $ непр $ \implies \set*{H^{-1}(U)} $ - открытое покрытие пространства
            $ I^2 $ 

            $ I^2 $ комп. метрическое пр-во $ \implies  $ по лемме Лебега
            $ \exists $ разбиение $ \set{Q_{ij}} $ пр-ва $ I $ 
            \[
                H(Q_{ij}) \subset U_{ij} \text{ - прав. накрытая окр-ть}
            \]
            \[
                Q_{ij} = [s_{i-1}, s_i] \times [t_{j-1}, t_j], \ i,j = 0 \dots, n
            \]
            \[
                \exists ! \, V_{00} \ni x_0: \ p : V_{00} \xrightarrow{\cong} U_{00}
            \]
            \[
                \widetilde{H}|_{Q_{00}} \coloneq p^{-1} \circ H|_{Q_{00}}
            \]
            \[
                \exists ! \, V_{10} \supset \widetilde{H}(s_1 \times [t_0, t_1])
                : \ p: V_{10} \xrightarrow{\cong} U_{10}
            \]
            \[
                \widetilde{H}|_{Q_{10}} \coloneq p \circ H|_{Q_{10}}: \
                Q_{10} \to V_{10}
            \]
            \[
                \widetilde{H}|_{Q_{n-1, 0}} = p^{-1} \circ H|_{n-1, 0}
            \]
            \[
                f_i = H|_{I \times t_i}, \quad i = 1, \dots, n
            \]
            \begin{center}
                \def\svgwidth{0.5\columnwidth}
                \import{./figures/}{pic2.pdf_tex}
            \end{center}

            Можно считать, что $ H|_{T \times [t_0, t_1]} $ - гомотопия между
            $ f $ и $ f_1 \implies \widetilde{H}|_{I \times [t_0, t_1]} $ -
            гомотопия между $ \widetilde{f} $ и $ \widetilde{f}_1 $  

            Аналогично строится $ \widetilde{H}|_{T \times [t_j, t_{j+1}]} \implies
            \widetilde{f} \simeq \widetilde{f_1} \simeq \widetilde{f_2} \dots
            \simeq \widetilde{f_n} = f' \implies \widetilde{f} \simeq \widetilde{f'}
            \implies \widetilde{f}(1) = \widetilde{f'}(1)$ 
        \end{proof}
    \end{theorem}
\end{tcolorbox}

\section*{\centering \S Фундаментальная группа окружности}

\begin{tcolorbox}[enhanced,breakable,skin first=enhanced,skin middle=enhanced,skin last=enhanced]
    \begin{theorem}
        \[
            \pi_1(S^1) \cong \mathbb{Z}
        \]

        \begin{proof}
            \[
                p: \ \mathbb{R} \to S^1
            \]
            \[
                p(x) = (\cos 2 \pi x, \sin 2 \pi x)
            \]
            \[
                \phi: \ \mathbb{Z} \to \pi_1(S, b_0)
            \]
            \[
                \phi(n) = [f_n], \text{ где } f_n(t) \coloneq (\cos 2\pi nt,
                \sin 2\pi nt), \quad t \in [0,1]
            \]
            \begin{center}
                \def\svgwidth{0.5\columnwidth}
                \import{./figures/}{pic3.pdf_tex}
            \end{center}

            1) $ \phi $ - гомоморфизм - ?
            \[
                \phi(k+n) = \phi(k) \phi(n) - ?
            \]
            \[
                [f_{k+n}] = [f_k f_n] - ?
            \]
            \[
                f_{n+k} \simeq f_k f_n - ?
            \]
            \begin{center}
                \def\svgwidth{\columnwidth}
                \import{./figures/}{pic4.pdf_tex}
            \end{center}

            2) $ \phi $ - мономорфизм - ?
            \[
                \ker \phi = \set{0} - ?
            \]
            \[
                \forall \, n \in \mathbb{Z} \quad n \in \ker \phi \implies \phi(n)
                = [e_{b_0}] \implies f_n \simeq e_{b_0}
            \]
            \[
                n = 0 - ?
            \]

            Пусть $ \widetilde{f}_n $ - путь, накрывающий петлю $ f_n $  

            Ясно, что $ \widetilde{f_n}(t) = nt, \ t \in [0,1] $, т.к.
            $ p(\widetilde{f_n}(t)) = (\cos 2\pi nt, \sin 2\pi nt) = f_n(t) \implies
            \widetilde{f}_n(1) = n$ 

            Ясно, что $ \widetilde{e}_{b_0} = e_0, \ O \in \mathbb{R} $
            $\implies \widetilde{e}_{b_0}(1) = 0$

            По теореме о накрывающей гомотопии $ \widetilde{f_n} \simeq e_0 \implies
            n = 0$ 

            3) $ \phi $ - эпиморфизм - ?
            \[
                \forall \, [u] \in \pi_1(S^1, b_0) \ \exists \, n \in \mathbb{Z}: \
                [u] = \phi(n) = [f_n] - ?
            \]

            Пусть $ \widetilde{u} $ - путь с началом в точке $ 0 \in \mathbb{R} $ 
            накрывающий петлю $ u \implies n = \widetilde{u}(1) \in \mathbb{Z} $ 
            \[
                u \simeq f_n - ?
            \]
            \[
                \widetilde{u} \simeq \widetilde{f_n} - ?
            \]

            $ \mathbb{R} $ односвязно, $ \widetilde{u}(0) = \widetilde{f_n}(0) = 0 $.
            $ \widetilde{u}(1) = \widetilde{f_n}(1) = n \implies \widetilde{u} \simeq
            \widetilde{f_n} \implies u = p \circ \widetilde{u} \simeq p \circ \widetilde{f_n}
            = f_n$ 
        \end{proof}
    \end{theorem}
\end{tcolorbox}

\begin{tcolorbox}
\begin{remark}
    \[
        [f_n] = [f_1]^n
    \]
\end{remark}
\end{tcolorbox}

\section*{\centering \S Фундаментальная группа вещественного проективного пространства}
\begin{tcolorbox}
\[
    \mathbb{R} P^n = \set{ l \subset R^{n+1} \ | \ 0 \in l}, \ l \text{ - прямая}
\]
\[
    S^n \subset R^{n+1} \text{ - единичная сфера с центром в точке 0}
\]

$ \exists $ отображение $ p: S^n \to \mathbb{R}P^n: p(x) \coloneq l \ni x, -x   $ 
\[
    \forall l \in \mathbb{R} P^n: \ p^{-1}(l) = \set{x, -x} \implies
    \mathbb{R} P^n = \sfrac{S^n}{x \sim -x} \implies \text{ топология в } \mathbb{R}P^n
    \text{ - фактор-топология, $ p $ - проекция на фактор-пр-во - непр. отображение}
\]
\end{tcolorbox}

\end{document}
