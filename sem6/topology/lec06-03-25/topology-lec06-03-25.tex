\documentclass[a4paper]{article}
\usepackage[a4paper,%
    text={180mm, 260mm},%
    left=15mm, top=15mm]{geometry}
\usepackage[utf8]{inputenc}
\usepackage{cmap}
\usepackage[english, russian]{babel}
\usepackage{indentfirst}
\usepackage{amssymb}
\usepackage{amsmath}
\usepackage{amsthm}
\usepackage{mathtools}
\usepackage[most]{tcolorbox}
\usepackage{import}
\usepackage{xifthen}
\usepackage{pdfpages}
\usepackage{transparent}
\usepackage{graphicx}
\graphicspath{ {./figures} }

\newcommand{\incfig}[1]{%
\def\svgwidth{\columnwidth}
\import{./figures/}{#1.pdf_tex}
}

\DeclarePairedDelimiter\set\{\}

\newtheorem*{theorem}{Теорема}
\newtheorem*{statement}{Утверждение}
\newtheorem*{lemma}{Лемма}
\newtheorem*{proposal}{Предложение}


\theoremstyle{definition}
\newtheorem*{definition}{Определение}

\theoremstyle{remark}
\newtheorem*{remark}{Замечание}

\renewcommand\qedsymbol{$\blacksquare$}

\begin{document}
\begin{tcolorbox}
    \begin{lemma}[Лебега о покрытии]
        $ X $ - компактное метрическое пространство
        
        Любые открытые покрытия $ \{ U_{\alpha} \} $ пространства $ X $ 
        \[
            \exists \delta > 0: \ \forall \, A \subset X, \ diam A < \delta
            \ \exists \alpha \ A \subset U_{\alpha}
        \]

        \begin{proof}
            $ X $ - метрическое $ \implies \forall b \in X \ \exists r_b > 0: \ 
            \exists \alpha: \ D_{r_b}(b) \subset U_{\alpha}$ 

            \[
            \set*{D_{\frac{r_b}{2}}} \text{ - окрытое покрытие пространства } X
            \]

            $ X $ - компактное $ \implies \exists b_1, \dots , b_n : \ 
            X = \bigcup_{i=1}^{n} D_{\frac{r_{b_i}}{2}}(b_i)$\\
        $ \delta \coloneq min \set[\Big]{\frac{r_{b_i}}{2}}  $ 
        \begin{center}
            \def\svgwidth{0.5\columnwidth}
            \import{./figures/}{lebesgues-lemma.pdf_tex}
        \end{center}

        Пусть $ a \in A \ \exists i : \ a \in D_{\frac{r_{b_i}}{2}}(b_i) $\\
        $ \forall \subset D_{r_{b_i}}(b_i) \ ? $ 

        \[
            \forall c \in A \ \rho(c,b_i) \leq \rho(c,a) + \rho(a,b_i) < \delta
            + \frac{r_{b_i}}{2} \leq r_{b_i} \implies c \in D_{r_{b_i}}(b_i)
        \]
        \[
            A \subset D_{r_{b_i}}(b_i) \subset U_{\alpha}
        \]
        \end{proof} 
    \end{lemma}
\end{tcolorbox}

\begin{tcolorbox}[enhanced,breakable,skin first=enhanced,skin middle=enhanced,skin last=enhanced]
    \begin{theorem}
        При $ n > 1 $ сфера $ S^{n} $ односвязна

        \begin{proof}
            Пусть $ S^{n} \subset \mathbb{R}^{n+1}:  \ \sum_{i=1}^{n} x_i^2 = 1 $ 
            \[
                x_0 = (1, 0, \dots, 0)
            \]

            Пусть $ u: \ I \to S^{n} $ - петля с началом в точке $ x_0 $ 
            \begin{center}
                \def\svgwidth{0.5\columnwidth}
                \import{./figures/}{loop.pdf_tex}
            \end{center}

            $ 1^{\circ} $. $ u(I) \neq S^n \implies \exists N \in S^n \setminus
            u(I)$  

            Пусть $ S_{p_{N}}: \ S^n \setminus N \to \mathbb{R}^{n} $ - 
            стереографическая проекция c центром в точке $ N $ 
            \begin{center}
                \def\svgwidth{0.5\columnwidth}
                \import{./figures/}{stereorgraphic-projection.pdf_tex}
            \end{center}
            \[
                \frac{|t|}{1} = \frac{|x|}{1 - x_{n+1}} 
            \]
            \[
                t = (t_1, \dots, t_n) = \frac{1}{1 - x_{n+1}} (x_1, \dots, x_n)
                \text{ - гомеморфизм}
            \]
            \[
                (S_{p_N} \circ u) \text{ - петля в } \mathbb{R}^n \text{ с началом
                в точке }(1, 0, \dots, 0) \implies
                \exists \text{ гомотопия } h_t : \ h_0 = S_{p_N} \circ u, \ 
                h_1 = e_{x_0}
            \]
            \[
                (S_{p_{N}})^{-1} \circ h_t \text{ - гомотопия, соединяющая } u 
                \text{ с } e_{x_0}
            \]

            $ 2^{\circ} $ $ u(I) = S^n $. Покроем сферу $ S^n $ открытыми полусферами
            $ \set*{U_{\alpha}} $ 
            \[
                u: I \to S^n \text{ непр } \implies \set{u^{-1}(U_{\alpha}}
                \text{ - открытое покрытие отрезка } I
            \]
            $ I $ - компактное метр-ое $ \implies \exists $ разбиение 
            $ (t_0, t_1, \dots, t_n): \ u([t_{i-1},t_i]) \subset U_{\alpha} $ 

            $ u_i \coloneq u |_{[t_{i-1}, t_i]} $ - путь в $ U_{\alpha} \neq S^n
            \implies \exists N \in S^n \setminus U_{\alpha}, N \notin u_i[t_{i-1}, t_i]$
            $ \implies S_{p_N} \circ u_i $ - путь в $ \mathbb{R}^n \implies
            $ он гомотопен  линейному пути $ \implies  S^{-1}_{p_N}$ заменяет путь
            $ u_i $ гомотопным ему путем $ \widetilde{u}_i $ \\
            $ \implies  u = u_1 u_2 \dots u_k \simeq \widetilde{u_1} \widetilde{u_2}
            \dots \widetilde{u_n} \eqcolon \widetilde{u}$ 

            $ \widetilde{u}(I) $ - объеденение дуг $ k $ окружностей $ \implies
            \widetilde{u}(I) \neq S^n $ по $ 1^{\circ} $ $ \widetilde{u} \simeq
            e_{x_0} \implies u \simeq e_{x_0}$ 
        \end{proof}
    \end{theorem}
\end{tcolorbox}

\begin{tcolorbox}
\begin{remark}
    $ S^1 $ не односвязна
\end{remark}
\end{tcolorbox}

\begin{tcolorbox}
    \begin{theorem}[Критерий односвязности]
        Пусть $ X $ линейно связано: следующие утверждения равносильны:\\
        1. $ X $ односвязно, т.е. любая петля в $ X $ с началом в отмеченной точке
        $ x_0 $ гомотопна $ e_{x_0} $\\
        2. Любая петля в $ X $ свободно гомотопна нулю\\
        3. любое непрерывное отображение $ f: S^1 \to X $ продолжается до
        непрерывного отображения $ D^2 \to X $\\
        4. Любые два пути с одинаковыми началами и концами гомотопны как пути

        \begin{proof}
            $ 1 \implies 2 \ ? $  - очевидно по определению связанной гомотопии 

            $ 2 \implies 3: $ Пусть $ u: S^1 \to X $ - петля в $X$.\\
            Существует гомотопия $ H: S^1 \times I \to X$ 
            \begin{center}
                \def\svgwidth{0.5\columnwidth}
                \import{./figures/}{homotopy-1.pdf_tex}
            \end{center}

            $ K $ - конус

            $ pr_b: \ K \to D^2 $ - проекция на основанипе - гомеоморфизм
            \[
                \implies \overline{H} \circ pr_b^{-1}: \ D^2 \to X
            \]
            \[
                (\overline{H} \circ pr_b^{-1}) |_{S^1 \times 0} = \overline{H}
                |_{S^1 \times 0} = H |_{S^{1} \times 0} = f
            \]
            \begin{center}
                \def\svgwidth{0.5\columnwidth}
                \import{./figures/}{homotopy-2.pdf_tex}
            \end{center}

            по 3 $ \exists \ \overline{H} : \ D^2 \to X: \ \overline{H}|_{s_1}
            = s_0 \overline{s_1}$ 

            $ \implies \overline{H} \circ pr $ - искомая гомотопия

            $ 4 \implies 1: $ т.к. петля $ u $ с началом в точке $ x_0 $ и 
            петля $ e_{x_0} $ - частный случай пустей с общим началом и концом
        \end{proof}
    \end{theorem}
\end{tcolorbox}

\section*{\centering Накрытия}

\begin{tcolorbox}
\begin{definition}
    Накрытием топологического пространства $ B $ называется непрерывное отображение
    $ p: \ X \ro B $, которое сюрьективно и $ \forall b \in B \ \exists $ окрестность
    $ U_b \subset B $ этой точки, для которой $ p^{-1}(U_b) = \bigcup_{\alpha}
    V_{\alpha}, \ \forall \alpha$. $ V_{\alpha} $ - открыто в $ X $,
    $ p|_{V_{\alpha}} : \ V_{\alpha} \to U_b $ - гомеоморфизм и $ V_{\alpha}
    \cap V_{\beta} = \varnothing,$ если $ \alpha \neq \beta $.

    $ X $ - называется пространством накрытия. $ B $ - база накрытия.
    $ p $ - проекция накрытия
\end{definition}
\end{tcolorbox}

\end{document}
