\documentclass[a4paper]{article}
\usepackage[a4paper,%
    text={180mm, 260mm},%
    left=15mm, top=15mm]{geometry}
\usepackage[utf8]{inputenc}
\usepackage{cmap}
\usepackage[english, russian]{babel}
\usepackage{indentfirst}
\usepackage{amssymb}
\usepackage{amsmath}
\usepackage{amsthm}
\usepackage{mathtools}
\usepackage{tcolorbox}
\usepackage{import}
\usepackage{xifthen}
\usepackage{pdfpages}
\usepackage{transparent}
\usepackage{graphicx}
\graphicspath{ {./figures} }

\newcommand{\incfig}[1]{%
\def\svgwidth{\columnwidth}
\import{./figures/}{#1.pdf_tex}
}

\newtheorem*{theorem}{Теорема}
\newtheorem*{statement}{Утверждение}
\newtheorem*{lemma}{Лемма}
\newtheorem*{proposal}{Предложение}

\theoremstyle{definition}
\newtheorem*{definition}{Определение}

\theoremstyle{remark}
\newtheorem*{remark}{Замечание}

\renewcommand\qedsymbol{$\blacksquare$}

\begin{document}
\[
    u \overline{u} \simeq e_{x_0}
\]
\begin{figure}[ht]
    \centering
    \incfig{ассоциативность}
    \caption{ассоциативность}
    \label{fig:ассоциативность}
\end{figure}

\[
    H(s,t) = 
    \begin{cases}
        u(2s), &\quad 0 \leq s \leq \frac{1-t}{2}\\
        u(1-s) = \overline{u}(s), &\quad \frac{1-t}{2} \leq s \leq \frac{1+t}{2} \\
        \overline{u}(2s-1), &\quad \frac{1+t}{2} \leq s \leq 1
    \end{cases}
\]
\[
    \overline{u} u \simeq e_{x_0} \text{ доказывается аналогично}
\]

\begin{tcolorbox}[title=Фундаментальная группа]
    \begin{definition}[Фундаментальная группа]
        $ \pi_1(X, x_0) $ называется фундаментальной группой топологического
        пространства $ X $ с началом в точке $ x_0 $ 
    \end{definition}
\end{tcolorbox}

\begin{tcolorbox}
\begin{remark}
    1) Пусть $ X_1 $ - компонента линейной связности пр-ва $ X $, $ x_0 \in X \implies
    \pi_1(X, x_0) = \pi_1(X_1, x_0)$ 

    2) Если $ \pi_1(X, x_0) = \{ [e_{x_0}] \} $, тогда пишут $ \pi_1(X, x_0) = 0 $ 
    и говорят, что фундаментальная группа тривиальна

    3) Если петля $ u \in F(X, x_0) $.\\
    $ u \simeq e_{x_0} $, то говорят, что $ u $ гомотопна нулю, $ u \simeq 0 $ 
\end{remark}
\end{tcolorbox} 

\section*{\centering Поведение фундаментальной группы при переносе начала}
Пусть $ X $ линейно связно и $ x_0, x_1 \in X $. Пусть $ s $ - путь, соединяющий
$ x_0 $ с $ x_1 $. 

\[
    F(X, x_0) \to F(X, x_1): \ u \mapsto (\overline{s} u)s
\]
\[
    A_{\sigma}([u]) \coloneqq \sigma^{-1} [u] \sigma, \text{ где } \sigma = [s]
\]

$ A_{\sigma} $ - отображение переноса

\begin{tcolorbox}
    \begin{theorem}
        $ A_{\sigma} $ - изоморфизм групп

        \begin{proof}
            1) $ A_{\sigma} $ - гомоморфизм ?\\
            $ \forall \alpha, \beta \in \pi_1(X, x_0) $ 
            \[
                A_{\sigma}(\alpha \beta) = \sigma^{-1} \alpha \beta \sigma = 
                \sigma^{-1} \alpha \sigma \sigma^{-1} \beta \sigma = A_{\sigma}
                (\alpha) \cdot A_{\sigma}(\beta)
            \]

            2) $ (A_{\sigma})^{-1} = A_{\sigma^{-1}} $\\
            $ A_{\sigma} \circ A_{\sigma^{-1}} = id_{\pi_1(X,x_1)} - ? $ 
            \[
                (A_{\sigma} \circ A_{\sigma^{-1}})(\alpha) = A_{\sigma}
                (A_{\sigma^{-1}}(\alpha)) = A_{\sigma}(\sigma \alpha \sigma^{-1})
                = \sigma^{-1} \sigma \alpha \sigma^{-1} \sigma = [e_{x_1}]
                \alpha[e_{x_1}] = \alpha
            \]
            Аналогично $ A_{\sigma^{-1}} \circ (A_{\sigma})^{-1} = id_{\pi_1(X,x_1)}
            \implies  (A_{\sigma})^{-1} = A_{\sigma^{-1}} \implies A_{\sigma}$ -
            биекция
        \end{proof}
    \end{theorem}
\end{tcolorbox}

\begin{tcolorbox}
    \begin{remark}
        \[
            A_{\tau} \circ A_{\sigma} = A_{\sigma\tau}, \text{ где } \sigma=[s], \
            \tau = [t]
        \]

        \begin{proof}
            $ \forall \alpha \in \pi_1(X, x_0) $ 
            \[
                (A_{\tau} \circ A_{\sigma})(\alpha) = A_{\tau}(A_{\sigma}(\alpha))
                = A_{\tau}(\sigma^{-1} \alpha \sigma) = \tau^{-1} \sigma^{-1}
                \alpha \sigma \tau = (\sigma \tau)^{-1} \alpha (\sigma \tau) =
                A_{\sigma \tau}(\alpha)
            \]

            2) $ A_{e_{x_0}} = id_{\pi_1(X, x_0)} $ 
        \end{proof}
    \end{remark}
\end{tcolorbox}

\begin{tcolorbox}[title=Изоморфизм переноса]
    \begin{definition}[Изоморфизм переноса]
        $ A_{\sigma} $ называется изоморфизм переноса
    \end{definition}
\end{tcolorbox}

\begin{tcolorbox}
    \begin{definition}
        $ X $ линейно связно. $ \pi_1(X) $ - фундаментальная группа пр-ва $ X $ 
        с точностью до изоморфизма
    \end{definition}
\end{tcolorbox}

\begin{tcolorbox}
    \begin{theorem}
        Изоморфизм переноса не зависит от выбора пути, по которому переносится
        начало $ \iff $ $ \pi_1(X) $ коммутативна
    \end{theorem}

    \begin{proof}
        1) $ \implies: $ 
        \[
            \forall a,b \in \pi_1(X, x_0) \quad ab = ba - ?
        \]
        \[
            aba^{-1}b^{-1} = [e_{x_0}] - ?
        \]
        \[
            \begin{cases}
                A_a(ab) = a^{-1}(ab)a = ba\\
                A_{[e_{xO}]}(ab) = id_{\pi_1(X, x_0)}(a) = ab
            \end{cases}
            \implies ab = ba
        \]

        2) $ \impliedby: $ 
        \[
            A_{\sigma} = A_{\tau} - ?
        \]
        $ \forall \alpha \in \pi_1(X, x_0) $ 
        \[
            A_{\sigma}(a) = \sigma^{-1} \alpha \sigma
        \]
        \[
            A_{\tau} = \tau^{-1} \alpha \tau
        \]
        \[
            \sigma^{-1} \alpha \sigma \stackrel{?}{=} \tau^{-1} \alpha \tau
        \]
        \[
            \alpha \sigma \stackrel{?}{=} \sigma \tau^{-1} \alpha \tau
        \]
        \[
            \alpha \stackrel{?}{=} \underbrace{\sigma\tau^{-1}}_{\in \pi_1(X,x_0)}
            \alpha \tau \sigma^{-1}
        \]
        \[
            (\sigma \tau^{-1}) \alpha \tau \sigma^{-1} = \alpha \sigma \tau^{-1}
            \tau \sigma^{-1} = \alpha
        \]
    \end{proof}
\end{tcolorbox}

\section{\centering Односвязные пространства}
\begin{tcolorbox}[title=Односвязное пространство]
    \begin{definition}[Односвязное пространство]
        Линейно связное топологическое пространство называется односвязным,
        если его фундаментальная группа тривиальна: $ \pi_1(X) = 0 $ 
    \end{definition}
\end{tcolorbox}

\begin{tcolorbox}
    \begin{theorem}
        При $ n > 1 $ сфера $ S^{n} $ односвязна
        \begin{lemma}[Лебега о покрытии]
            Пусть $ X $ - компактное метрическое пр-во. $ \{ U_{\alpha} \} $ -
            открытое покрытие пр-ва $ X $ 

            Тогда $ \exists \epsilon > 0 \ \forall A \subset X, \ diam A < \epsilon
            \implies \exists \alpha : \ A \subset U_{\alpha}$ 
        \end{lemma}
    \end{theorem}
\end{tcolorbox}
\end{document}
