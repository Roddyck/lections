\documentclass[a4paper]{article}
\usepackage[a4paper,%
    text={180mm, 260mm},%
    left=15mm, top=15mm]{geometry}
\usepackage[utf8]{inputenc}
\usepackage{cmap}
\usepackage[english, russian]{babel}
\usepackage{indentfirst}
\usepackage{amssymb}
\usepackage{amsmath}
\usepackage{amsthm}
\usepackage{mathtools}
\usepackage{tcolorbox}
\usepackage{import}
\usepackage{xifthen}
\usepackage{pdfpages}
\usepackage{transparent}
\usepackage{graphicx}
\graphicspath{ {./figures} }

\DeclarePairedDelimiter\set\{\}

\newcommand{\incfig}[1]{%
\def\svgwidth{\columnwidth}
\import{./figures/}{#1.pdf_tex}
}

\newtheorem*{theorem}{Теорема}
\newtheorem*{statement}{Утверждение}
\newtheorem*{lemma}{Лемма}
\newtheorem*{proposal}{Предложение}
\newtheorem*{consequence}{Следствие}


\theoremstyle{definition}
\newtheorem*{definition}{Определение}

\theoremstyle{remark}
\newtheorem*{remark}{Замечание}

\renewcommand\qedsymbol{$\blacksquare$}

\begin{document}
\begin{tcolorbox}
\begin{theorem}
    $ M $ - n-мерное гладкое ориентированное многообразие, то его край
    $ (n-1) $-мерное гладкое ориентированное многообразие
\end{theorem}
\end{tcolorbox}

\begin{tcolorbox}
\begin{definition}
    $ f: \ M \to N $ - отображение, $ M $ - n-мерное гладкое многообразие,
    $ N $ - k-мерное. $ f $ называется гладким, если
    \[
        \forall \phi \in Atl M, \ \forall \psi \in Atl N
    \]
    \[
        \psi \circ f \circ \phi^{-1}: \ \mathbb{R}^n \to \mathbb{R}^k
    \]
    гладкое отображение
\end{definition}
\end{tcolorbox}

\begin{tcolorbox}
\begin{definition}
    $ r \coloneq \phi ^{-1}: \ \mathbb{R}^n \to U $ - параметризация. (Пусть
    для простоты $ U \subset \mathbb{R}^m) $ 

    $ r'_{x_i}(0), \ i = 1, \dots, n $ 
    \[
        v(x_0) \coloneq \sum_{i=1}^{n} v_i r'_{x_i}(0)
    \]
    касательный вектор точки $ x_0 \in M $ 

    \[
        T_{x_0}M \coloneq \set{v(x_0) = \sum v_i r'_{x_i}(0) \ | \ v_i \in \mathbb{R}}
    \]
    называется касательным пространством в точке $ x_0 $ 

    \[
        TM \coloneq \bigcup_{x_0 \in M} T_{x_0}M
    \]
    тотальное касательное пространство или касательное расслоение многообразия
    $ M $ 
\end{definition}
\end{tcolorbox}
\begin{figure}[ht]
    \centering
    \incfig{def2}
    \caption{Касательное пространство}
    \label{fig:def2}
\end{figure}

\section*{\centering \S Объём многообразия}

\subsection*{Примеры}
1) Прямоугольник:
\[
    S = |\vec{a} \times \vec{b}| = |
    \begin{vmatrix}
    \vec{i} & \vec{j} & \vec{k}\\
    a_1 & a_2 & 0\\
    b_1 & b_2 & 0\\
    
    \end{vmatrix}|
\]

2) Параллелепипед
\[
    V(\vec{a}, \vec{b}, \vec{c}) = |\vec{a} \vec{b} \vec{c}| = 
    \begin{vmatrix}
    a_1 & a_2 & a_3\\
    b_1 & b_2 & b_3\\
    c_1 & c_2 & c_3\\
    
    \end{vmatrix}
\]

\begin{tcolorbox}
\begin{definition}
    В $ \mathbb{R}^n $ пусть $ e_1, \dots, e_n $ - ортонормированный базис,
    $ a_1, \dots, a_n \in \mathbb{R}^n $. Ориентированным объёмом параллелепипеда,
    построенного на векторах $ a_1, \dots, a_n $ называется число $ V(a_1, \dots, a_n)
    \coloneq \det A$, где $ A $ - матрица, составленная из координат векторов
    $ a_1, \dots, a_n $ в базисе $ e_1, \dots, e_n $ 
\end{definition}
\end{tcolorbox}

\begin{tcolorbox}
\begin{remark}
    1) $ a_1, \dots, a_n  $ и  $ e_1, \dots, e_n $ дают одинаковую ориентацию
    пространства $ \mathbb{R}^n \iff \det A > 0 $ 

    2) $ G = (e_i, e_j) $ - матрица грама базиса $ e_1, \dots, e_n $ 
    \[
        G = A A^T \implies |G| = |A|^2 \implies
        V(a_1, \dots, a_n) = \sqrt{\det G} 
    \]
\end{remark}
\end{tcolorbox}

\subsection*{Объем многообразия}

$ M $ n-мерное гладкое многообразие. $ \phi: \ U \to \mathbb{R}^n $ - карта,
$ r = \phi^{-1} $ 

\begin{figure}[ht]
    \centering
    \incfig{volume}
    \caption{Объем многообразия}
    \label{fig:volume}
\end{figure}

Покроем $ \mathbb{R}^n $ маленькими параллелепипедами $ V_{x_0}(\Delta x_1, \dots,
\Delta x_n) \approx \sqrt{\det (r'_{x_i} r'_{x_j}} $ 
\[
    V(U) = \int_{U} V_{x_0} dx_1 \dots dx_n
\]
\[
    V(M) = \int_{M} V(x) d x_1 \dots d x_n \text{ - объём многообразия } M
\]

\newpage
\section*{\centering \S Внешняя алгебра}
\begin{tcolorbox}
\begin{definition}
    Пусть $ L $ - вещественное линейное пространство с базисом $ e_1, \dots, e_n $.

    \[
        \Lambda L = \bigcup_{k=0}^{n} \Lambda^n L,
    \]
    где $ \Lambda^0 L = \mathbb{R}, \ \Lambda^1 L = L, \Lambda^k L $ - линейное пространство
    с базисом $ (e_{\alpha} \coloneq e_{\alpha_1} \wedge \dots \wedge e_{\alpha_k}) $,
    где $ 1 \leq \alpha_1 < \alpha_2 \dots < \alpha_k \leq n $ 
    \[
        \dim \Lambda^k = C_n^k
    \]
\end{definition}
\end{tcolorbox}

Введём операцию внешнего умножения в $ \Lambda L: $\\
1. $ \forall \lambda \in \Lambda^0 \ \forall e_{\alpha} \in \Lambda^k L $ 
\[
    \lambda \wedge e_{\alpha} \coloneq \lambda e_{\alpha}
\]
2. $ \forall e_{\alpha}, e_{\beta}, \ e_{\alpha} = e_{\alpha_1} \wedge \dots
\wedge e_{\alpha_k}, \ e_{\beta} = e_{\beta_1} \in \Lambda^1 L $ 
\[
    e_{\alpha} \wedge e_{\beta} = e_{\alpha_1} \wedge \dots \wedge e_{\alpha_i}
    \wedge e_{\beta_1} \wedge \dots \wedge e_{\alpha_k} \cdot (-1)^{n-i},
    \ \alpha_1 < \dots
    < \alpha_i < \beta_1 < \alpha_{i+1} < \dots < \alpha_k
\]
\[
    \beta_1 \notin \set{\alpha_1, \dots, \alpha_k}
\]
\[
    e_{\alpha} \wedge e_{\beta_1} = 0, \ \exists i \quad \alpha_i = \beta_1
\]

Т.е. $ \forall e_i, e_j \in \Lambda^1 L $ 
\[
    e_i \wedge e_j = 
    \begin{cases}
        0, &\quad i = j\\
        =-e_j \wedge e_i, &\quad i \neq j
    \end{cases}
\]
3. По ассоциативности внешнего умножения $ \wedge $ и по линейности операция
$ \wedge $ продолжается на все $ \Lambda L $.

$ \Lambda L $ называется внешней алгеброй, а её элементы называются внешними
формами:
\[
    \omega = \sum_{\alpha} \omega_{\alpha} e_{\alpha}
    \coloneq \sum_{1 \leq \alpha_1 < \dots < \alpha_k \leq n} 
    \omega_{\alpha_1, \dots, \alpha_k} e_{\alpha_1} \wedge \dots \wedge
    e_{\alpha_k}, \quad \omega_{\alpha_1, \dots, \alpha_k} \in \mathbb{R}
\]

\subsection*{Свойства $ \wedge $} 
1. $ \omega \in \Lambda^k $ (т.е. $ \omega $ - это к-форма), $ \eta \in \Lambda^m $  
\[
    \omega \wedge \eta = (-1)^{km} \eta \wedge \omega
\]

2. $ k+m > n \implies \omega \wedge \eta = 0 $ 

3. $ \omega \wedge \omega = 0 \quad \forall \omega \in \Lambda^k L, \ k $ нечётно

\textbf{\underline{Д-во}}
\[
    \left(\sum \omega_{\alpha} e_{\alpha}\right) \left(\sum \omega_{\beta} e_{\beta}\right)
    = \sum_{\alpha,\beta} \omega_{\alpha} \omega_{\beta} e_{\alpha} e_{\beta}
    = \sum (\omega_{\alpha} \omega_{\beta} - \omega_{\beta} \omega_{\alpha})
    e_{\alpha} e_{\beta} = 0
\]

\begin{tcolorbox}
\begin{remark}[Значение внешней формы на векторах]
    $ L $ - линейное пространство с базисом $ e_1, \dots, e_n \implies L^{\star} $ -
    двойственное линейное пространство с базисом $ e^{\star}_1, \dots, e^{\star}_n $
    \[
        \implies e_i^{\star}(e_j) = \delta_{ij} = 
        \begin{cases}
            1, \ i = j\\
            0, \ i \neq j
        \end{cases}
    \]
    \[
        \omega \in \Lambda^k L^{\star}, \ \omega = \sum_{\alpha}
        \omega_{\alpha} e_{\alpha}, \ v_1, \dots, v_k \in L
    \]
\end{remark}
\end{tcolorbox}
\end{document}
