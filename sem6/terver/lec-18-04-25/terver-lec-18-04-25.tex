\documentclass[a4paper]{article}
\usepackage[a4paper,%
    text={180mm, 260mm},%
    left=15mm, top=15mm]{geometry}
\usepackage[utf8]{inputenc}
\usepackage{cmap}
\usepackage[english, russian]{babel}
\usepackage{indentfirst}
\usepackage{amssymb}
\usepackage{amsmath}
\usepackage{amsthm}
\usepackage{mathtools}
\usepackage{tcolorbox}
\usepackage{import}
\usepackage{xifthen}
\usepackage{pdfpages}
\usepackage{transparent}
\usepackage{graphicx}
\graphicspath{ {./figures} }

\DeclarePairedDelimiter\set\{\}

\newcommand{\incfig}[1]{%
\def\svgwidth{\columnwidth}
\import{./figures/}{#1.pdf_tex}
}

\newtheorem*{theorem}{Теорема}
\newtheorem*{statement}{Утверждение}
\newtheorem*{lemma}{Лемма}
\newtheorem*{proposal}{Предложение}
\newtheorem*{consequence}{Следствие}

\theoremstyle{definition}
\newtheorem*{definition}{Определение}

\theoremstyle{remark}
\newtheorem*{remark}{Замечание}

\renewcommand\qedsymbol{$\blacksquare$}

\begin{document}
\section*{\centering Пуассоновский процесс}
$ \set{X_t} , \ t \geq 0 $ \\
1) С целочисленными значениями\\
2) $ X_t $ - процесс с независимыми приращениями\\
3) $ P(X_t = k) = \frac{(\lambda t)^{k}}{k!} e^{-\lambda t} $ 

Предположение\\
1) $p(h)$ - вероятность того, что ... $ \geq 1 $\\
$ p(h) = ah + o(h), \ h \to 0$ 
\[
    P_m(t) = P(X(t) = m)
\]
\[
    p(h) = P_1(h) + P_2(h) + \dots
\]
2) $ \geq 2: \ P_2(h) + P_3(h) + \dots = o(h) $.

Формула полной вероятности
\[
    P_0(t+h) = P_0(t) P_0(h)
\]
\[
    P_0(h) + \sum_{j=1}^{\infty} P_j(h) = 1
\]
\[
    P_0(h) = 1 - p(h)
\]
\[
    P_0(t+h) = P_0(t) - P_0(t) p(h)
\]
\[
    \frac{\partial P_0(t)}{\partial t} \leftarrow \frac{P_0(t+h) - P_0(t)}{h} =
    -P_0(t) \frac{p(h)}{h} \to - a P_0(t)
\]
\[
    \frac{\partial P_0(t)}{\partial t}  = -P_0(t) \cdot a
\]
\[
    \dots\dots\dots\dots\dots\dots \dots\dots\dots\dots\dots\dots
\]

\[
    P_m(t), \ m = 1, 2, \dots
\]
\[
    P_m(t+h) = P_m(t) \cdot P_0(h) + P_{m-1}(t) P_1(h) + 
    \sum_{j=2}^{\infty} P_{m-j}(t) P_j(h)
\]
\[
    \sum_{j=2}^{\infty} P_{m-j}(t) P_j(h)
    \leq \sum_{j=2}^{\infty} P_j(h) = o(h)
\]
\[
    P_m(t+h) = P_m(t) (1- p(h)) + P_{m-1}(t)ah + o(h)
\]
\[
    \frac{P_m(t+h) - P(t)}{h} = -P_m(t)a + P_{m-1}(t) a + o(h)
\]

При $ h \to 0 $ 
\[
    \frac{d P_m(t)}{dt} = -P_m(t)a + aP_{m-1}(t)
\]

\[
    Q_m(t) = P_m(t) e^{-at}
\]
\[
    P_m(0) = 0
\]
\[
    Q'_m(t) = P'_m(t) \cdot e^{-at} + aP_m(t) \cdot e^{-at}
\]
\[
    Q'_m(t) = aQ_{m-1}(t)
\]
\[
    Q_1'(t) = -a Q_0(t)
\]
\[
    Q_1(t) = at + c = at
\]

\[
    Q_2'(t) = a Q_1(t) \quad Q(0) = 0
\]
\[
    Q_2(t) = \frac{a^2 t^2}{2} + c \implies c = 0
\]

\section*{\centering Процессы рождения и гибели}
\[
    \frac{dP_k(t)}{dt} = \underbrace{\lambda_{k-1}}_{\text{рождение}} P_{k-1}(t) - 
    (\lambda_k + \mu_k) P_k(t) + \underbrace{\mu_{k+1}P_{k+1}(t)}_{\text{гибель}}
\]
\[
    P_k(t+h) = \lambda_{k-1} h P_{k-1}(t) + (1 - \lambda_kh - \mu_k h)
    P_k(t) + \mu_{k+1} h P_{k+1}(t)
\]


\end{document}
