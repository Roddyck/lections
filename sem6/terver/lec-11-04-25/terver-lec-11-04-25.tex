\documentclass[a4paper]{article}
\usepackage[a4paper,%
    text={180mm, 260mm},%
    left=15mm, top=15mm]{geometry}
\usepackage[utf8]{inputenc}
\usepackage{cmap}
\usepackage[english, russian]{babel}
\usepackage{indentfirst}
\usepackage{amssymb}
\usepackage{amsmath}
\usepackage{amsthm}
\usepackage{mathtools}
\usepackage{tcolorbox}
\usepackage{import}
\usepackage{xifthen}
\usepackage{pdfpages}
\usepackage{transparent}
\usepackage{graphicx}
\graphicspath{ {./figures} }

\newcommand{\incfig}[1]{%
\def\svgwidth{\columnwidth}
\import{./figures/}{#1.pdf_tex}
}

\DeclarePairedDelimiter\set\{\}

\newtheorem*{theorem}{Теорема}
\newtheorem*{statement}{Утверждение}
\newtheorem*{lemma}{Лемма}
\newtheorem*{proposal}{Предложение}
\newtheorem*{consequence}{Следствие}


\theoremstyle{definition}
\newtheorem*{definition}{Определение}

\theoremstyle{remark}
\newtheorem*{remark}{Замечание}

\renewcommand\qedsymbol{$\blacksquare$}

\begin{document}
\section*{\centering Тождество А. Вальда}
\[
    \xi_1, \xi_2, \dots, \xi_n, \dots \quad \mathcal{F}_n = \sigma(\xi_1, \dots, \xi_n)
\]
$ \tau $ - момент остановки
\[
    (\tau = n) \in \mathcal{F}_n
\]
\[
    s_n = \xi_1 + \xi_2 + \dots + \xi_n, \ s_{\tau} = s_n, \ (\tau = n)
\]
\[
    E(|\xi_i|) < \infty
\]
\[
    E(s_{\tau}) = E(\tau) E(\xi_i) = a \cdot E(\tau)
\]
\[
    a = E(\xi_i)
\]
\[
    E(\xi_i) = a_i = a, \ i = 1, 2, \dots
\]

\section*{Равенство Абеля}
\[
    B_k = b_k + b_{k+1} + \dots + b_{n-1}, \ b_n = 0
\]
\[
    A_k = a_1 + a_2 + \dots + a_{n}, \ a_0 = 0
\]
\begin{tcolorbox}
    \[
        \sum_{k=1}^{n} B_k a_k = \sum_{k=1}^{n-1} A_k b_k
    \]
\end{tcolorbox}
\[
    \sum_{k=1}^{n} B_k(A_k - A_{k-1}) = \sum_{k=1}^{n} B_k A_k - 
    \sum_{k=1}^{n} B_k A_{k-1} = \sum_{k=1}^{n-1} (B_k - B_{k+1}A_k = 
    \sum_{k=1}^{n-1} b_k A_k
\]
\[
    E(\tau) \underbrace{E(\xi_i)}_{a} = \sum_{k=1}^{\infty} KP(\tau = k)a = 
    \sum_{k=1}^{\infty} \underbrace{P(\tau = k)}_{b_k} (a_1 + \dots + a_k) = 
    \sum_{k=1}^{\infty} b_k A_k = \sum_{k=1}^{\infty} B_k a_k
    = \sum_{k=1}^{\infty} P(\tau \geq k) E(\xi_k)
\]
\[
    \overline{(\tau = k)} = (\tau < k) = (\tau = k-1) \in \mathcal{F}_{k-1}
\]
\[
    (\tau \geq k) \in \mathcal{F}_{k-1}
\]
\[
    \sum_{k=1}^{\infty} P(\tau \geq k) E(\xi_k | \tau \geq k)
    = \sum_{k=1}^{\infty} \int_{(\tau \geq k)}\xi_k dP
    \sum_{k=1}^{\infty} \sum_{m=k}^{\infty} \int_{(\tau = m)} \xi_m dP
    \sum_{k=1}^{\infty} \sum_{m=1}^{k} \int_{(\tau = m)} \xi_m dP
    = \int_{k=1}^{\infty} \int_{(\tau = m)} s_{\tau}dP = E(s_{\tau})
\]

\begin{tcolorbox}
    \begin{theorem}[1]
        $ \set{S_n, \mathcal{F}_n}_{n \geq 1} $ - субмартингал
        \[
            \lambda P(\max_{k \leq n} S_k > \lambda) \leq E(S_n I(\max_{k \leq n}
            s_k > \lambda))
        \]
        \begin{proof}
            \[
                B = (\max_{k \leq n} s_k > \lambda) = \bigcup_{i=1}^{n} (s_i > \lambda, \ 
                \max_{1 \leq j \leq i-1} s_j \leq \lambda)            
                = \bigcup_{i=1}^{n}B_i
            \]
            \[
                P(B) \leq \sum_{k=1}^{n} P(B_k) = \sum_{k=1}^{n} E(I(B_i))
            \]
            \[
                \lambda P(B) \leq \sum_{k=1}^{n} E(s_k I(B_k)) \leq 
                \sum_{k=1}^{n} E(E(s_n | \mathcal{F}_k I(B_k)) = 
                \sum_{k=1}^{n} E(E(S_n I(B_k) | \mathcal{F})) = 
                E(s_n I(b))
            \]
        \end{proof}
    \end{theorem}
\end{tcolorbox}

$ \set{S_n, \mathcal{F}_n}_{n \geq 1} $ - мартингал

$ p \geq 1: $ 
\[
    \lambda^{p} P(\max_{1 \leq k \leq n} S_n) \leq E(|S_n|^{p})
\]

\begin{tcolorbox}
    \begin{theorem}[закон больших чисел]
        \[
            S_n = \sum_{k=1}^{n} X_k \text{ - мартингал}
        \]
        \[
            X_{nk} = X_k \cdot I(|X_k| \leq b_n) \quad b_n \to \infty, \ n \to \infty
        \]

        Условия:

        1) \[
            \sum_{k=1}^{n} P(|X_k| \geq b_n | \mathcal{F}_{k-1})
            \xrightarrow[n\to \infty]{} 0
        \]

        2) \[
            \frac{1}{b_n} \sum_{k=1}^{n} E(X_{nk}|\mathcal{F}_{k-1} \xrightarrow{p} 0
        \]

        3) \[
            \frac{1}{b_n^2} E(X_{nk}^2|\mathcal{F}_{k-1}) - 
            E(E(X_{nk}|\mathcal{F}_{k-1})) \xrightarrow[n\to \infty]{} 0
        \]
    \end{theorem}
\end{tcolorbox}

\begin{tcolorbox}
    \begin{theorem}[центральная предельная теорема]
        \[
            S_n = \sum_{k=1}^{n} X_k \text{ - мартингал}
        \]
        \[
            X_{nk} = X_k \cdot I(|X_k| \leq b_n) \quad b_n \to \infty, \ n \to \infty
        \]

        Условия:

        1) \[
            \sum_{k=1}^{n} P(|X_k| \geq b_n | \mathcal{F}_{k-1})
            \xrightarrow[n\to \infty]{} 0
        \]

        2) \[
            \frac{1}{b_n} \sum_{k=1}^{n} E(X_{nk}|\mathcal{F}_{k-1} \xrightarrow{p} 0
        \]

        3) \[
        \frac{S_n - \sum E(X_{nk}|\mathcal{F}_{k-1})}{\sqrt{b_n} } 
        \xrightarrow[n\to \infty]{} 0
        \]
    \end{theorem}
\end{tcolorbox}

\section*{\centering Задача о разорении}

Рассмотрим "монету".

1 выпадает с вероятность $ p $. 0 - с вероятностью $ q = 1 - p $.  

1 $ \sim $ второй игрок платит первому 1 рубль\\
0 $ \sim $ первый платит второму 1 рубль

1 игрока капитал $ A $, второго - $ B $. $ N = A + B $  
\[
    f(n) = P(\text{выигрыш, если на руках капитал }n), \quad 0 \leq n \leq N
\]
\[
    f(n) = pf(n+1) + qf(n-1)
\]
\[
    f(0) = 0 \quad f(N) = 1
\]
\[
    pf(n+1) - f(n) + qf(n-1) = 0 \ | : p
\]
\[
    f(n+1) - \frac{1}{p} f(n) + \frac{q}{p} f(n-1) = 0
\]
\begin{equation}
    f(n+1) + af(n) + bf(n-1) = 0
\end{equation}
\[
    f(n) = \lambda^{n}, \ 0 < \lambda \leq 1
\]
\[
    \lambda^{n+1} + a\lambda^{n} + b \lambda^{n-1} = 0
\]
\[
    \lambda^2 + a \lambda + b = 0
\]
\begin{align*}
    1) \lambda_1, \lambda_2\\
    2) D = 0\\
    3) D < 0
\end{align*}
\[
    f(n) = C_1 \lambda_1^{n} + C_2 \lambda_2^{n}
\]

\[
    f(n+1) - \frac{1}{p} f(n) + \frac{q}{p} f(n-1) = 0
\]
\[
    1 - \frac{1}{p} + \frac{q}{p} = 0, \ \lambda_1 = 1
\]
\[
    \lambda_2 = \frac{q}{p} 
\]
\[
    f(n) = C_1 + C_2 \left( \frac{q}{p} \right)^{n}
\]
\[
    \begin{cases}
        C_1 + C_2 = 1\\
        C_1 + C_2 \left( \frac{q}{p} \right)^{N} = 0 
    \end{cases}
\]
\[
    C_2 \left( 1 - \left(\frac{q}{p}\right)^{N}\right) = 1
\]
\[
    C_1 = \frac{\left(\frac{q}{p}\right)^{N}}{1-\left(\frac{q}{p}\right)^{N}} 
\]
\[
    f(n) = \frac{\left(\frac{q}{p}\right)^{n} - {\left(\frac{q}{p}\right)^{N}}}
    {1 - {\left(\frac{q}{p}\right)^{N}}} 
\]
 \[
     m(n) \text{ - средняя продолжительность игры}
 \]
 \[
     m(n) = 1 + pm(n+1) + qm(n-1)
 \]
\end{document}
