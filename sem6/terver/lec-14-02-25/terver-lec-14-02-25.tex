\documentclass[a4paper]{article}
\usepackage[a4paper,%
    text={180mm, 260mm},%
    left=15mm, top=15mm]{geometry}
\usepackage[utf8]{inputenc}
\usepackage{cmap}
\usepackage[english, russian]{babel}
\usepackage{indentfirst}
\usepackage{amssymb}
\usepackage{amsmath}
\usepackage{mathtools}
\usepackage{tcolorbox}
\usepackage{import}
\usepackage{xifthen}
\usepackage{pdfpages}
\usepackage{transparent}
\usepackage{graphicx}
\graphicspath{ {./figures} }

\newcommand{\incfig}[1]{%
\def\svgwidth{\columnwidth}
\import{./figures/}{#1.pdf_tex}
}

\begin{document}
\title{ТВиМС. Лекция}
\maketitle

\[
    X(t) = X(\omega, t), t - \text{ фикс}, \ X_t(\omega) - \text{ с.в.}
\]
\[
    \{ X(t_1), \dots, X(t_n) \}, \ t_1 < t_2 \dots < t_n, \ \forall n
\]

Числовые характеристики  \\
1) Матем. ожидание $ m(t) = E(X(t)) $ \\
2) Дисперсия $ \sigma^2(t) = D(X(t)) $ 

Корреляционная функция
\[
    K(t_1, t_2) = E((X(t_1) - m(t_1))(X(t_2) - m(t_2))) = E(X(t_1)X(t_2))
    - m(t_1) m(t_2)
\]
\[
    E(X(\omega,t_1)X(\omega,t_2))
\]

Стационарный процесс:
\[
    \{ X(t_1), \dots, X(t_n) \} \land \{ X(t_1 + h), \dots, X(t_n+h) \}
    \text{ - одинаково распределены } \forall h \land \forall t_1 < \dots < t_n
\]

Если процесс стационарный, то $ m(t) = const, \ \sigma^2 = const, \ K(t_1, t_2) =
K(t_2 - t_1) = K(|t_2 - t_1|)$ 
\[
    K(t_1, t_2) = K(t_1 - t_1, t_2 - t_1) = K_1(t_2 - t_1)
\]

\[
    X(t) = \xi_1 \sin t + \xi_2 \cos t
\]
$ \xi_1, \xi_2 $ - независимые случайные величины\\
$ P(\xi_i = \pm 1) = \frac{1}{2} $ 
\[
    E(X(t)) = m(t) = E(\xi_1(\omega) \sin t + \xi_2(\omega) \cos t) = 
    E(\xi_1(\omega) \sin t) + E(\xi_2(\omega) \cos t) =
    \sin t E(\xi_1) + \cos t E(\xi_2)
\]
\[
    E(\xi_1) = 1 \cdot \frac{1}{2} + (-1) \cdot \frac{1}{2} = 0
\]
\[
    m(t) = 0
\]
\[
    D(\xi_1) = E(\xi_1^2) = 1^2 \cdot \frac{1}{2} + (-1)^2 \cdot \frac{1}{2} = 1
\]
\[
    \sigma^2(t) = D(X(t)) = D(\xi_1 \sin t) + D(\xi_2 \cos t) + 2 cov(\xi_1 \sin t,
    \xi_2 \cos t)
\]
\[
    \sigma^2(t) = \sin^2 t D(\xi_1) + \cos^2 t D(\xi_2) = \sin^2 t + \cos^2 t = 1
\]
\[
    K(t_1, t_2) = E(X(t_1) \cdot X(t_2)) = E((\xi_1 \sin t_1 + \xi_2 \cos t_1)
    \cdot (\xi_1 \sin t_2 + \xi_2 \cos t_2)) = E(\xi_1^2 \sin t_1 \sin t_2 +
\]
\[
    + \xi_1 \xi_2 (\sin t_1 \cos t_2 + \cos t_1 \sin t_2) + \xi_2^2 \cos t_1
    \cos t_2)
\]
\[
    = \sin t_1 \sin t_2 + \cos t_1 \cos t_2 = \cos(t_2 - t_1)
\]

\begin{tcolorbox}[title=Определение]
    X(t) называется стационарным в широком смысле, если
    \[
        E(X(t)) = const \quad K(t_1, t_2) = K(t_2 - t_1)
    \]
\end{tcolorbox}
\[
    X(t) = \xi_1 \sin t + \xi_2 \cos t
\]
\[
    m(t) = 0, \quad K(t_1, t_2) = \cos(t_2 - t_1)
\]
1) $ t = 0 $\\
$ X(0) = \xi_2 $\\
$ t = \frac{\pi}{4} $ \\
$ X(\frac{\pi}{4}) = \xi_1 \frac{\sqrt{2} }{2} + \xi_2\frac{\sqrt{2} }{2} $ 

\[
    \xi_1 = 1, \ \xi_2 = 1 \implies
\]

\emph{Пример}
\[
    E(X(t)) = m_x(t) = 0
\]
\[
    K_x(t_1, t_2) = 2 \sin(t_1) \sin(t_2)
\]
\[
    z(t) = \int_{0}^{t} X(\tau) d \tau = \int_{0}^{t} X(\omega, \tau) d \tau
\]
\[
    E(z(t)) = ? \quad D(z(t)) = ? \quad K(t_1, t_2) = ?
\]
\[
    m_z(t) = E \left( \int_{0}^{t} X(\omega, \tau) d \tau \right) =
    \int_{0}^{t} E(X(\omega, \tau)) d \tau = \int_{0}^{t} 0 \cdot d \tau = 0
\]
\[
    K(t_1, t_2) = E(z(t_1) z(t_2)) = E \left( \int_{0}^{t_1} X(\omega, \tau_1)
    d \tau_1 \cdot \int_{0}^{t_2} X(\omega, \tau_2) d \tau_2 \right) =
    E\left( \int_{0}^{t_1} \int_{0}^{t_2} X(\omega, \tau_1) X(\omega, \tau_2) d
    \tau_1 d \tau_2 \right)
\]
\[
    = \int_{0}^{t_1} \int_{0}^{t_2} E(X(\omega, \tau_1) X(\omega, \tau_2))
    d \tau_1 d \tau_2 = \int_{0}^{t_1} \int_{0}^{t_2} 2 \sin(\tau_1)\sin(\tau_2)
    d \tau_1 d \tau_2 = 2(-\cos(\tau_1) |_{0}^{t_1})(-\cos(\tau_2) |_{0}^{t_2})
\]
\[
    = 2(1 - \cos(t_1))(1 - \cos(t_2))
\]
\[
    \sigma^2(t) = 2(1 - \cos(t))^2
\]

\section*{Марковские цепи}
\[
    X_1, X_2, \dots, X_n, \dots
\]
\[
    P(X_1 = k_1, X_2 = k_2, X_3 = k_3) = P(X_3 = k_3 | X_1 = k_1, X_2 = k_2)
    \cdot P(X_2 = k_2, X_1 = k_1) \stackrel{*}{=}
\]

Если процесс Марковский
\[
    P(X_3 = k_3 | X_1 = k_1, X_2 = k_2) = P(X_3 = k_3 | X_2 = k_2)
\]
\[
    \stackrel{*}{=} \underbrace{P(X_3 = k_3 | X_2 = k_2)}_{= P_{k_2, k_3}}
    \underbrace{P(X_2 = k_2 | X_1 = k_1)}_{= P_{k_1, k_2}} = 
    a_{k_1}p_{k_1, k_2}p_{k_2, k_3}
\]

\[
    \mathbb{P} = \begin{pmatrix}
    p_{11} & p_{12} & \dots & p_{1n}\\
    p_{21} & p_{22} & \dots & p_{2n}\\
    \dots & \dots & \dots & \dots\\
    p_{n1} & p_{n 2} & \dots & p_{nn}\\
    
    \end{pmatrix}
\]

По формуле полной вероятности:
\[
    p_{k_1, k_2} = \sum_{j=1}^{n} p_{k_1 j} p_{j k_2}
\]
\[
    \mathbb{P}^2 = \mathbb{P} \cdot \mathbb{P}
\]
\[
    \mathbb{P}^{m+n} = \mathbb{P}^{m} \cdot \mathbb{P}^{n}
\]
\[
    \mathbb{P}^{n+1} = \mathbb{P}^{n} \mathbb{P}
\]
\begin{equation}
    p_{ij}^{(n+1)} = \sum_{k=1}^{n} p_{ik}^{(n)} \cdot p_{kj}
\end{equation}
\[
    p_{kj} > 0, \ \forall k, j \implies p_{ij}^{n} \to \pi_j
\]
\begin{equation}
    \pi_j = \sum_{k=1}^{n} \pi_k p_{kj}
\end{equation}

\emph{Задача}

Матрица перехода цепи Маркова имеет вид
\[
    \mathbb{P} = \begin{pmatrix}
    0.2 & 0.5 & 0.3\\
    0.4 & 0.1 & 0.5\\
    0.2 & 0.7 & 0.1\\
    
    \end{pmatrix}
\]
\[
    a^{T} = (0.7, 0.2, 0.1)
\]

a) Найти распределения ... при $ t = 2 $\\
b) $ t = 0, 1, 2, 3 $ цепь находилась в состоянии $ 3 \ 2 \ 1 \ 2 $ \\
Найти стационарное распределение
\[
    a) \ p_{ij}^{2}
\]
\[
    \mathbb{P}^2 = \mathbb{P}\mathbb{P} = \begin{pmatrix}
    0.3 & 0.36 & 0.24\\
    0.22 & 0.56 & 0.22\\
    0.34 & 0.24 & 0.42\\
    
    \end{pmatrix}
\]
\[
    P(X = 3, X_1 = 2, X_2 = 1, X_3 = 2) = a_3 p_{32} p_{21} p_{12} = 
    0.7 \cdot 0.4 \cdot 0.5 \cdot 0.4 = 0.056
\]
\[
    \begin{cases}
        \pi_2 = 0.5 \pi 1 + 0.1 \pi_2 + 0.7 \pi_3\\
        \pi_3 = 0.3 \pi 1 + 0.5 \pi_2 + 0.1 \pi_2\\
        \pi_1 = 0.2 \pi_1 + 0.4 \pi_2 + 0.2 \pi_3\\
    \end{cases}
\]
\[
    \pi_3 = 1 \quad \pi_1 = \frac{23}{26} \quad \pi_2 = \frac{33}{26} 
\]
\end{document}
