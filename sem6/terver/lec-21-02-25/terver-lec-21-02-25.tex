\documentclass[a4paper]{article}
\usepackage[a4paper,%
    text={180mm, 260mm},%
    left=15mm, top=15mm]{geometry}
\usepackage[utf8]{inputenc}
\usepackage{cmap}
\usepackage[english, russian]{babel}
\usepackage{indentfirst}
\usepackage{amssymb}
\usepackage{amsmath}
\usepackage{amsthm}
\usepackage{mathtools}
\usepackage{tcolorbox}
\usepackage{import}
\usepackage{xifthen}
\usepackage{pdfpages}
\usepackage{transparent}
\usepackage{graphicx}
\graphicspath{ {./figures} }

\newcommand{\incfig}[1]{%
\def\svgwidth{\columnwidth}
\import{./figures/}{#1.pdf_tex}
}

\newtheorem*{theorem}{Теорема}
\newtheorem*{lemma}{Лемма}

\theoremstyle{definition}
\newtheorem*{definition}{Определение}

\theoremstyle{remark}
\newtheorem*{remark}{Замечание}

\renewcommand
\qedsymbol{$\blacksquare$}

\begin{document}
\section*{\centering Типы случайных процессов}

1) $ t = 0, 1, 2, \dots \ (t = -\infty, \dots, -1, 0, 1, \dots) $ - временные
ряды

Процессы с независ. значениями
\[
    X_0, X_1, X_2, \dots \text{ - послед-ть с.в.}
\]
\[
    X_t \land X_s, \ t \neq s \text{ - независимы}
\]

2) Процессы с независимыми приращениями
\[
     t \in [0, T] \ ([0, +\infty]
\]

$ X_t, \ t \in T $ процесс с независ. приращениями, если $ \forall t_1 < t_2 <
\dots < t_n, \ n \in \mathbb{N}$ 
\[
    X_{t_1}, X_{t_2}, \dots , X_n, \dots \text{ - нез с.в.}
\]

\[
    S_n = X_1 + X_2 + \dots + X_n \implies \{ S_n \}_{n \geq 1}
    \text{ - процесс с незав приращ}
\]

$ t_1 < t_2 $ $ X_t, \ t \in T $ процесс с независимыми приращениями
\[
    X(t_2 + t_1) - X(t_2) \text{ - не зависит от } t_2
\]
\[
    m_X(t_2 + t_1) = E(X(t_2 + t_1)) = E(X(t_2 + t_1) - X(t_2)) + E(X(t_2)) =
    m(t_1) + m(t_2)
\]
\[
    t > s \quad E(X(t)) = 0
\]
\[
    K(s,t) = E(X(t)X(s))
\]
\[
    X(t) \cdot X(s) = (X(t) - X(s) + X(s))X(s) = (X(t) - X(s))X(s) + 
    X^2(s)
\]
\[
    K(s,t) = E((X(t) - X(s))X(s)) + E(X^2(s))
\]
\[
    K(s,t) = \min (\sigma^2(s), \sigma^2(t))
\]
\[
    E((X(t) - X(s))^2) = \sigma^2(t) + \sigma^2(s) - 2 \sigma^2(s) = 
    \sigma^2(t) - \sigma^2(s)
\]
\[
    \sigma^2(t) = \sigma^2(s) + \sigma^2(t-s)
\]
\[
    \sigma^2(t_1 + t_2) = \sigma^2(t_1) + \sigma^2(t_2)
\]

Рассмотрим функциональное уравнение
\[
    y(t_1 + t_2) = y(t_1) + y(t_2)
\]
\[
    t_1 = t_2 = 1
\]
\[
    y(2) = y(1) + y(1) = 2y(1)
\]
\[
    y(m+1) = y(m) + y(1) = (m+1)y(1)
\]
\[
    y(n) = ny(1) = Bnm(t)
\]
\[
    y\left(\frac{m}{n}\right) = m 
\]

Марковские процессы
\begin{tcolorbox}
    \begin{definition}
        Процесс называется Марковским, если
        \[
            P(X_{t_{n+1}} \in B | X_{t_n}= x_n, \dots, X_{t_1}= x_1) =
            P(X_{t_{n+1}} \in B | X_t = x_n) \text{ - переходная функция}
        \]
    \end{definition}
\end{tcolorbox}

Показать:
1) Для процесса с незав. приращ определить конечномерные распределения\\
2) Процесс с нез. приращениями $ \equiv $ марковский процесс

\section*{Стационарные процессы}

\begin{tcolorbox}
\begin{definition}
$ X_t, t \in T $ - стационарный, если
\[
    \forall t_1 < t_2 < \dots < t_n \ \forall n, \ \forall h > 0
\]
\[
    (X_{t_1}, \dots, X_{t_n}) \land (X_{t_1 + h}, \dots, X_{t_n + h})
    \text{ одинаково распределены}
\]
\end{definition}
\end{tcolorbox}

\[
    E(X(t_1)) = m(t_1)
\]
\[
    E(X(t_2)) = m(t_2)
\]
\[
    X(t_1) \land X(t_2) = X(t_1 + h) \text{ - одинаково расп} \implies
\]
\[
    1) \ m(t_1) = m(t_1 + h) = const
\]
\[
    2) \ \sigma^2(t) = const
\]
\[
    E(X(t)) = 0
\]
\[
    K(s,t) = E(X(t) \cdot X(s)) = E(X(s+h)X(s)) \quad h = t-s
    \text{ возьмем } s = 0 \implies
\]
\[
    K(s+h,s) = K(h) = K(t-s) = K(|t-s|)
\]
\begin{tcolorbox}
    \begin{remark}
        1-ое определение стационарного процесса называется стационарным процессом в узком
        смысле
    \end{remark}
\end{tcolorbox}

\begin{tcolorbox}
    \begin{definition}
        Пусть $ X_t, t \in T $ - процесс:\\
        $ \{ m(t) = const, K(t,s) = K(|t-s|)$ 

            Тогда $ X_t, \ t \in T $ называется стационарным процессом в широком смысле
    \end{definition}
\end{tcolorbox}

\[
    X(t) = A \sin(\omega t + \phi) \quad A, \omega \text{ - постоянные (неслучайные)}
\]
\[
    \phi \in \mathcal{R} [0, 2\pi]
\]

\section*{\centering Гауссовские процессы}
\begin{tcolorbox}
\begin{theorem}
    $ X_t $ - стационарен в узком смысле $ \iff $ он стационарен в широком смысле

    \begin{proof}
        
    \end{proof}
\end{theorem}
\end{tcolorbox}

\[
    X_t = f(\xi + t)
\]
\[
    \xi \in \mathcal{R}[0,T), \ f \text{ - периодическая ф-ия с периодом T.}
\]
Показать, что $ X_t $ - стационарный в узком смысле

\underline{Д-во}
\[
    t_1 < t_2 < \dots < t_n
\]
\[
    E\left(\exp\left(in \sum_{j=1}^{n} \lambda_j X_{n_j} + h\right)\right) = \phi_{n_1 + h, \dots ,
    n_n + h}
\]
\[
    = \frac{1}{T} \int_{0}^{T} \exp\left(i \sum_{j=1}^{n} f(u_j + y)\right)dy = 
    \frac{1}{T} \int_{0}^{T} \exp\left(i\sum_{j=1}^{n} f(u_1 + h, y)\right) dy
\]
\[
    \frac{1}{T} \int_{h}^{T+h} \exp\left(i\sum_{j=1}^{n} f(u_1, y)\right) dy
\]
\[
    \Phi_k \text{ - компл. число}
\]
\[
    X_t = \sum_{k=-n}^{n} e^{i \lambda_k + \Phi_k}
\]
\[
    E(\Phi_k) = 0 \implies m(t) = 0
\]
\[
    E(\Phi_k \overline{\Phi}_k) = F_k
\]
\[
    K(t,s) = \sum_{k=-n}^{n} e^{i \lambda_k(t-s) + \Phi_k} = K(t-s)
\]

\section*{\centering Броуновское движение}
1) $ X_t $ - имеет непрерывные траектории

2) $ P(X_0 = 0) = 1 $ 

3) Процесс с независимыми приращениями

4) Приращение $ X_t - X_s \in \mathcal{N}(0, \sigma^2(t-s)), \ t > s $ 
\[
    m(t) = E(X_t) = 0
\]
\[
    K(t,s) = E(X_t X_s) = E((X_t - X_s)X_s) + E(X^2_s) = E(X_s) \cdot
    E(X_t - X_s) = 0
\]

Д/з. Найти конечномерные распределения Броуновского движения

\[
    w(t) \text{ - Броуновское движение}
\]
\[
    \tau_x = \inf \{ t : \ w(t) = x \}
\]
\[
    \widehat{w}(t) = \max_{u \in [0,t]} \{w(u) \leq x\}
\]
\[
    P(\widehat{w}(t) \leq x) = 2 P(w(t) \geq x)
\]
\[
    P(w(t) \geq x) = \int_{0}^{t} P(w(t) \geq x | \tau(x) = y) f(y) dy = 
    \int_{0}^{t} P(w(t) - w(y) \geq 0 | \tau(x) = y) f(y) dy
\]
\[
    = \int_{0}^{t} P(w(t) - w(y) \geq 0) f(y) dy = \frac{1}{2} \int_{0}^{t} 
    f(y) dy
\]
\[
    P(\tau(x) \leq t) = 2P(w(t) \geq x) = 2 \int_{x}^{\infty} \frac{1}{\sqrt{2\pi t}} 
    e^{\frac{-u^2}{2t}} du = 2 \int_{x}^{\infty} \frac{1}{\sqrt{2\pi}}
    e^{\frac{-v^2}{2t}} dv
\]
\[
    P(\tau(t)
\]

\end{document}
