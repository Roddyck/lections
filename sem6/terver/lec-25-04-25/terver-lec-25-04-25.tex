\documentclass[a4paper]{article}
\usepackage[a4paper,%
    text={180mm, 260mm},%
    left=15mm, top=15mm]{geometry}
\usepackage[utf8]{inputenc}
\usepackage{cmap}
\usepackage[english, russian]{babel}
\usepackage{indentfirst}
\usepackage{amssymb}
\usepackage{amsmath}
\usepackage{amsthm}
\usepackage{mathtools}
\usepackage{tcolorbox}
\usepackage{import}
\usepackage{xifthen}
\usepackage{pdfpages}
\usepackage{transparent}
\usepackage{graphicx}
\graphicspath{ {./figures} }

\DeclarePairedDelimiter\set\{\}

\newcommand{\incfig}[1]{%
\def\svgwidth{\columnwidth}
\import{./figures/}{#1.pdf_tex}
}

\newtheorem*{theorem}{Теорема}
\newtheorem*{statement}{Утверждение}
\newtheorem*{lemma}{Лемма}
\newtheorem*{proposal}{Предложение}
\newtheorem*{consequence}{Следствие}


\theoremstyle{definition}
\newtheorem*{definition}{Определение}

\theoremstyle{remark}
\newtheorem*{remark}{Замечание}

\renewcommand\qedsymbol{$\blacksquare$}

\begin{document}
\section*{\centering Броуновское движение}

$ \set{x_t}, \ t \geq 0 $. Р.Броун 1827

\underline{Н. Винер}

\begin{enumerate}
    \item $ x_0 = 0 $ п.н.
    \item $ t_0 = 0 < t_1 < t_2 < \dots < t_n $\\
        $ x_{t_n} - x_{t_{n-1}}, \dots, x_{t_2} - x_{t_1} $ - независимы 
    \item $ (x_t - x_s) \in \mathcal{N}(0, \sigma^2(t-s)) $ 
    \item $ \set{x_t} $ - траектория непрерывна
\end{enumerate}

\[
    E(x_t) = 0 \quad E(x_tx_s) = \sigma^2 = \min(t,s)
\]
$ \set{\psi_n(t)} $ - полная ортогональная система функций
\[
    x_t(\omega) = \sum_{n=1}^{\infty} a_n(t) \psi(t)
\]
\[
    E(a_m) = 0
\]
\[
    E(a_ma_n) = \int_{0}^{1} \int_{0}^{1} E(x_tx_s) \psi_n(t) \psi_n(s) dt ds
\]
\[
    \set{\psi_n} = \set*{\sin\left(n + \frac{1}{2}\right) \pi t}
\]

$ \set{W(t)}_{t \geq 0} \quad (\sigma^2 = 1) $ 
\begin{enumerate}
    \item $ D(W_t - W_s) = t-s $ 
    \item $ \xi_n = \sum_{i=0}^{n} (W_{t_{i+1}} - W_{t_i})^2 
        \xrightarrow[n\to \infty]{p} $ 
\end{enumerate}

\textbf{Задача}

Есть два города, между которые ездят поезда.

Есть две конкурирующие компании. Цель перевести 1000 пассажиров

$ \nu $ - имеет отрицательное биномиальное распределение
\[
    NB(p, m)
\]
\[
    P(\nu = m) = C_{n-1}^{m-1} p^m q ^{n-m}
\]
\end{document}
