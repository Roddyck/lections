\documentclass[a4paper]{article}
\usepackage[a4paper,%
    text={180mm, 260mm},%
    left=15mm, top=15mm]{geometry}
\usepackage[utf8]{inputenc}
\usepackage{cmap}
\usepackage[english, russian]{babel}
\usepackage{hyperref}
\usepackage{indentfirst}
\usepackage{amssymb}
\usepackage{amsmath}
\usepackage{amsthm}
\usepackage{mathtools}
\usepackage{tcolorbox}
\usepackage{import}
\usepackage{xifthen}
\usepackage{pdfpages}
\usepackage{transparent}
\usepackage{graphicx}
\graphicspath{ {./figures} }

\DeclarePairedDelimiter\set\{\}

\newcommand{\incfig}[1]{%
\def\svgwidth{\columnwidth}
\import{./figures/}{#1.pdf_tex}
}

\newtheorem*{theorem}{Теорема}
\newtheorem*{statement}{Утверждение}
\newtheorem*{lemma}{Лемма}
\newtheorem*{proposal}{Предложение}


\theoremstyle{definition}
\newtheorem*{definition}{Определение}

\theoremstyle{remark}
\newtheorem*{remark}{Замечание}

\renewcommand\qedsymbol{$\blacksquare$}

\begin{document}
\section*{\centering Частная и множественная корреляция}
\[
    \begin{pmatrix}
    X_1\\
    X_2\\
    
    \end{pmatrix} \in N \left( \begin{pmatrix}
    0\\
    0\\
    
    \end{pmatrix},
    \begin{pmatrix}
    1 & \rho_{12}\\
    \rho_{12} & 1\\
    
    \end{pmatrix}
    \right)
\]
\[
    \begin{pmatrix}
    X_1\\
    X_2\\
    X_3
    
    \end{pmatrix}
    \rightarrow
    \begin{pmatrix}
    1 & \rho_{12} & \rho_{13}\\
    \rho_{12} & 1 & \rho_{23}\\
    \rho_{13} & \rho_{23} & 1\\
    
    \end{pmatrix}
\]

Корреляция $ X_1, X_2 $ при условии $ X_3 = x_3 $ - фикс 
\[
    \rho_{12 \cdot 3}
    = \frac{\rho_{12}-\rho_{13}\rho_{23}}{\sqrt{1-\rho_{13}^2}\sqrt{1-\rho_{23}^2}} 
\]
\[
    \mathbb{X} \in N(0, \Sigma)
\]
\[
    \mathbb{X} = \begin{bmatrix}
    \mathbb{X}_1\\
    \mathbb{X}_2\\
    
    \end{bmatrix}
\]
\[
    \Sigma = \begin{bmatrix}
    \Sigma_{11} & \Sigma_{12}\\
    \Sigma_{21} & \Sigma_{22}\\
    
    \end{bmatrix}
\]
\[
    \Sigma_{11} = E(\mathbb{X}_1 \mathbb{X}_1^T)
\]
\[
    \Sigma_{12} = E(\mathbb{X}_1 \mathbb{X}_2^T)
\]
\[
    \Sigma_{22} = E(\mathbb{X}_2 \mathbb{X}_2^T)
\]
\[
    \mathbb{Y}_1 = \mathbb{X}_1 - A \cdot \mathbb{X}_2
\]
\[
    \mathbb{Y}_2 = \mathbb{X}_2
\]
\[
    \begin{bmatrix}
    \mathbb{Y}_1\\
    \mathbb{Y}_2\\
    
    \end{bmatrix}
    \in N(0, \dots)
\]
\[
    E(\mathbb{Y}_1 \cdot \mathbb{Y}_2^T) = 0
\]
\[
    E(\mathbb{Y}_1) = 0 \quad E(\mathbb{Y}_2) = 0
\]
\[
    E((\mathbb{X}_1 - A \mathbb{X}_2) \mathbb{X}_2^T) = 0 = \Sigma_{12} - A
    \Sigma_{22} = 0
\]
\[
    \Sigma_{12} = A \Sigma_{22} \ \big| \ \cdot \Sigma_{22}^{-1}
\]
\[
    A = \Sigma_{12} \Sigma_{22}^{-1}
\]
\[
    \mathbb{Y}_1 = \mathbb{X}_1 - \Sigma_{12} \Sigma_{22}^{-1} \mathbb{X}_2
\]
\[
    E(\mathbb{Y}_1 | \mathbb{X}_2) = E(\mathbb{X}_1 | \mathbb{X}_2) -
    \Sigma_{12}\Sigma_{22}^{-1} \mathbb{X}_2 = 0
\]
\[
    E(\mathbb{X}_1 | \mathbb{X}_2) = \Sigma_{12}\Sigma_{22}^{-1} \mathbb{X}_2
\]
\[
    E(\mathbb{X}_1) = \mu_1 \quad E(\mathbb{X}_1 - \mu_1) = 0
\]
\[
    E(\mathbb{X}_2) = \mu_2 \quad E(\mathbb{X}_2 - \mu_2) = 0
\]
\[
    E(\mathbb{X}_1 | \mathbb{X}_2) = \mu_1 + \Sigma_{12}\Sigma_{22}^{-1}
    (\mathbb{X}_2 - \mu_2)
\]
\[
    \mathbb{X} = \begin{pmatrix}
    X_1\\
    X_2\\
    
    \end{pmatrix} \in N \left( \begin{pmatrix}
    0\\
    0\\
    
    \end{pmatrix},
    \begin{pmatrix}
    \sigma_1^2 & \rho \sigma_{12}\\
    \rho \sigma_{12} & \sigma_{2}^2\\
    
    \end{pmatrix}
    \right)
\]
\[
    E(X_1 | X_2) = \mu_1 + \rho \sigma_1 \sigma_2 \cdot \frac{1}{\sigma_2^2} 
    (x_2 - \mu_2) = \mu_1 + \rho \frac{\sigma_1}{\sigma_2} (x_2 - \mu_2)
\]

\[
    \mathbb{X} = \begin{pmatrix}
    \mathbb{X}_1\\
    \mathbb{X}_2\\
    
    \end{pmatrix},
    \quad
    \mathbb{X}_1 = \begin{pmatrix}
    X_1\\
    X_2\\
    
    \end{pmatrix}, \quad \mathbb{X}_2 = X_3
\]
\[
    \Sigma = \begin{pmatrix}
    1 & \rho_{12} & \vdots & \rho_{13}\\
    \rho_{12} & 1 & \vdots & \rho_{23}\\
    \dots & \dots & \dots & \dots\\
    \rho_{13} & \rho_{23} & \vdots & 1\\
    
    \end{pmatrix}
\]
\[
    \mathbb{Y}_1 = \mathbb{X}_1 - \Sigma_{12}\Sigma_{22}^{-1} \mathbb{X}_2
\]
\[ 
    E(\mathbb{Y}_1) = 0
\]
\[
    E(\mathbb{Y}_1 \cdot \mathbb{Y}_1^T) = \Sigma_{12 \cdot 3} = \dots =
    \Sigma_{11} - \Sigma_{12}\Sigma_{22}^{-1}\Sigma_{21}
\]
\begin{tcolorbox}
    \[
        \mathbf{\Sigma \ balls}
    \]
\end{tcolorbox}
\[
    (X_1, X_2, X_3, X_4)^T
\]
\[
    \rho_{12 \cdot 34} = \frac{\rho_{12 \cdot 4} - \rho_{13 \cdot 4}\rho_{23 \cdot 4}}
    {\sqrt{1-\rho_{13 \cdot 4}^2} \sqrt{1 - \rho_{23 \cdot 4}^2}} 
\]
\[
    \rho_{1j \cdot q_j} = \frac{-\mathbb{R}_{1j}}{\sqrt{\mathbb{R}_{11}} \sqrt{\mathbb{R}_{jj}}} 
\]
\[
    q_j = \set{1, 2 \dots, m} \setminus \set{1, j}
\]
\[
    \mathbb{R} \text{ - матрица корреляции}
\]
\[
    \mathbb{R}_{1j}, \ \mathbb{R}_{11}, \mathbb{R}_{jj} \text{ - алгебраческие
    дополнения}
\]
\[
    E(X_1 | \mathbb{X}_2 = \mathbf{x}_2)
\]
\begin{tcolorbox}
    \begin{center}
        \url{https://source.unn.ru/}
    \end{center}
\end{tcolorbox}

\begin{tcolorbox}[title=Множественный коэффициент корреляции]
    \[
        1 - R_{1(2 \dots n)}^2 = \frac{\sigma_{1 \cdot 2 \dots m}^2}{\sigma_1^2} 
    \]
    \[
        R_{1(2 \dots n)}^2
        \text{ - множественный коэффициент корреляции}
    \]
\end{tcolorbox}
\end{document}
