\documentclass[a4paper]{article}
\usepackage[a4paper,%
    text={180mm, 260mm},%
    left=15mm, top=15mm]{geometry}
\usepackage[utf8]{inputenc}
\usepackage{cmap}
\usepackage[english, russian]{babel}
\usepackage{indentfirst}
\usepackage{amssymb}
\usepackage{amsmath}
\usepackage{amsthm}
\usepackage{mathtools}
\usepackage{tcolorbox}
\usepackage{import}
\usepackage{xifthen}
\usepackage{pdfpages}
\usepackage{transparent}
\usepackage{graphicx}
\graphicspath{ {./figures} }

\newcommand{\incfig}[1]{%
\def\svgwidth{\columnwidth}
\import{./figures/}{#1.pdf_tex}
}

\newtheorem{theorem}{Теорема}

\theoremstyle{definition}
\newtheorem*{definition}{Определение}

\renewcommand\qedsymbol{$\blacksquare$}

\begin{document}
\section*{\centering Пуассоновские процессы}
\[
    X_t, \ t> 0 \quad X_t \in \{ 0 \} \cup \mathbb{N}
\]

1) $ X_t $ - процесс с независимыми приращениями\\
2) $ \underbrace{P(X_{t+h} - X_t \geq 2)}_{\text{ не зависит от } t} = o(h) \implies
P(X_{t+h} - X_t = 1) = \lambda h + o(h) \quad P(X_{t+h} - X_t = 0) = 1 - 
h \lambda + o(h)$ 

\emph{Задание} Выписать конечномерные распределения
\[
    P(X_t = k) = \frac{(\lambda t)^{k}}{k!} e^{-\lambda t}
\]

\emph{Перенос занятий 12.05.2025 и 16.05.2025 19:00 zoom}

\textbf{\underline{Кр к экзамену}}\\
1) временные ряды, метод сезонной декомпозиции\\
2) числовые характеристики случайных процессов\\
3) SDE (стохастические дифференциальные уравнения). Формула замены переменных
Ито

\vspace{5mm}
\textbf{Успешно решённая кр - 3 за экзамен}
\vspace{5mm}

\[
    X_t \quad E(X^2_t) < \infty
\]
\[
    ||X|| = (E(X^2_t))^{\frac{1}{2} }
\]

\begin{tcolorbox}
\begin{definition}
    С.п. $ X_t $ называется непрерывным в средне квадратичным, если $ ||X_t - X_s||
    \xrightarrow{t - s \to 0} 0$ 
\end{definition}
\end{tcolorbox}

\begin{tcolorbox}
\begin{theorem}
    $ X_t $ непр. в ср. кв., если $ m(t) = E(X_t) $ - непр и корреляционная ф-я
    непрерывна по $ (t,s) $ 
\end{theorem}
\end{tcolorbox}

Пусть $ X_t $ - стационарный. $ K(t,s) = R(t-s)$ \\
R(t) = K(t,0)\\
$K(t,0) = E(X_t \cdot X_0) $

\begin{tcolorbox}
    \begin{theorem}
        $ R(t) $ - непр в нуле (необходимо и достатчно для непр в ср.кв)

        \begin{proof}
            1) Необходимость
            \[
                K(t,s) = E(X_t X_s)
            \]
            \[
                |K(t,0) - K(0,0)| = R(t) - R(0)| = |E(X_t X_0 - X_0^2)|
                = |E((X_t - X_0)X_0)| \leq E(|X_t - X_0|^2)^{\frac{1}{2} }
                (E(X_0^2))^{\frac{1}{2}}
            \]
            2) Достаточность
            \[
                t > s, \ t = s+h
            \]
            \[
                ||X_t - X_s||^2 = ||X_{s+h} - X_s||^2 = ||X_h - X_0||^2 =
                E((X_h - X_0)^2) = E(X_h^2) - 2E(X_h X_0) + E(X_0^2) =
            \]
            \[
                R(0) - 2R(h) + R(0) \xrightarrow{h \to 0} 0
            \]
        \end{proof}
    \end{theorem}
\end{tcolorbox}

\begin{tcolorbox}
    \begin{theorem}[Колмогоров А.Н.]
        \[
            E(|X_t - X_s|^{a}) \leq C |t-s|^{1+r} \quad a \geq 0, \ r > 0
        \]
        Тогда $ X_t $ имеет непрерывные траектории
    \end{theorem}
\end{tcolorbox}

\begin{tcolorbox}
    \begin{theorem}[Ченцов Н.Н.]
        $ t_1 < t_2 < t_3 $ 
        \[
            E(|X_{t_1} - X_{t_2}|^{a} \cdot |X_{t_2} - X_{t_3}|^{b}) \leq C
            |t_1 - t_3|^{1+r}
        \]
    \end{theorem}
\end{tcolorbox}

\underline{Пример} Броун. движение
\[
    E((W_t - W_s)^2) = |t-s|
\]
\[
    E((W_t - W_s)^4) = 3|t-s|^2
\]

$ W_t $ - имеет непрерывные траектории

\begin{figure}[ht]
    \centering
    \incfig{пуассоновский-процесс}
    \caption{пуассоновский процесс}
    \label{fig:пуассоновский-процесс}
\end{figure}
\[
    m(t) = \lambda t
\]
\[
    K(t,s) = \min(t,s)
\]

\begin{tcolorbox}
\begin{definition}
    $ E(X_t^2) < \infty $ 
    \[
        \left|\left| \frac{X_{t+h}- X_t}{h} - X'_t \right|\right| \xrightarrow{h\to 0} 0
    \]

    $ X'_t $ - производная в среднеквадратичном
\end{definition}
\end{tcolorbox}

\begin{tcolorbox}
\begin{theorem}
    Необходимо и достаточно (для существования производной в ср.кв?)
    \[
        \left| \frac{\partial^{2} K(t,s)}{\partial t \partial s} \leq c < \infty 
        \right|
    \]
    диф $ m(t) $ 
\end{theorem}
\end{tcolorbox}

\section*{\centering Винеровский процесс}
\[
    K(t,s) = 
    \begin{cases}
        s, \quad s < t\\
        t, \quad s \geq s
    \end{cases}
\]
\[
    \frac{\partial K}{\partial s} = 
    \begin{cases}
        1, \quad s < t\\
        0, \quad s \geq t
    \end{cases}
\]

\underline{Пример}
\[
    K = K(t,s) = 2 e^{- \alpha (t-s)^2}
\]
\[
    \frac{\partial K}{\partial s} = 4 \alpha(t-s) e^{- \alpha (t-s)^2 }
\]
\[
    \frac{\partial^2 K}{\partial s \partial t} = 4 \alpha(1 - 2 \alpha(t-s))^2
    e^{- \alpha (t-s)^2 }
\]

\underline{Пример}
\[
    z(t) = a(t) X(t) + b(t) \frac{dx(t)}{dt} 
\]
\[
    m(t) = E(X(t) \quad E(z(t)) = a(t) m(t) + b(t) \frac{dm(t)}{dt} = m_z(t)
\]
\[
    K_z(t_1, t_2) = E\left(\left(ax(t_1) + \frac{dx(t_1)}{dt} -m_z(t_1)\right) \cdot
    \left(ax(t_2) + \frac{dx(t_2)}{dt} -m_z(t_2)\right)\right) = \dots
\]

\[
    z(t) = X(t) + \frac{d^2 X(t)}{dt^2} 
\]
\[
    D_z(t) = E\left( \left( X(t) + \frac{d^2 X(t)}{dt^2} \right)^2 \right)
    = D_x(t) + 2 \frac{\partial^{2} D_x(t)}{\partial t^2} +
    \frac{\partial^{4} D_x(t)}{\partial t^4}
\]

\[
    X_t = (-1)^{N_t} - \text{ телеграфный процесс}
\]
\[
    N_t  \text{- пауссоновский процесс}
\]

\begin{equation}
    \langle X \rangle_T = \frac{1}{T} \int_{0}^{T} X(t) dt
\end{equation}
\[
    \text{траектория } x(t)
\]
\[
    \langle x \rangle_t = \frac{1}{T} \int_{0}^{T} x(t) dt
\]

\[
    E(\langle X \rangle_T) = E\left(\frac{1}{T} \int_{0}^{T} X(t) dt \right) =
    \frac{1}{T} \int_{0}^{T} E(X(t)) dt = \frac{1}{T} \int_{0}^{T} m(t)dt 
    \xrightarrow{T \to \infty} m
\]

Назовём процесс эргодическим, если 
\[
    \langle x \rangle_t \text{ - сх в ср.кв к } m
\]

\begin{tcolorbox}
\begin{theorem}
    $ X_t $ - эргодичен
    \[
        \frac{1}{T^2} \int_{0}^{T} \int_{0}^{T} K(t_1, t_2) d t_1 d t_2
        \xrightarrow{T \to \infty} 0
    \]
\end{theorem}
\end{tcolorbox}

\begin{tcolorbox}
    \begin{theorem}[Слуцкий]
        $ X_t $ - стационарен
        1)
        \[
            J_t = \frac{1}{T^2} \int_{0}^{T} \int_{0}^{T} K(t_1, t_2) d t_1 d t_2
            = \frac{2}{T} \int_{0}^{T} \left(1 - \frac{\tau}{T} \right) R(\tau)
            d \tau
        \]

        2) $ J_t \xrightarrow{T \to \infty} 0 $, то $ X_t $ - эргодичен 
    \end{theorem}
\end{tcolorbox}
\end{document}
