\documentclass[a4paper]{article}
\usepackage[a4paper,%
    text={180mm, 260mm},%
    left=15mm, top=15mm]{geometry}
\usepackage[utf8]{inputenc}
\usepackage{cmap}
\usepackage[english, russian]{babel}
\usepackage{indentfirst}
\usepackage{amssymb}
\usepackage{amsmath}
\usepackage{amsthm}
\usepackage{mathtools}
\usepackage{tcolorbox}
\usepackage{import}
\usepackage{xifthen}
\usepackage{pdfpages}
\usepackage{transparent}
\usepackage{graphicx}
\graphicspath{ {./figures} }

\newcommand{\incfig}[1]{%
\def\svgwidth{\columnwidth}
\import{./figures/}{#1.pdf_tex}
}

\newtheorem*{theorem}{Теорема}
\newtheorem*{statement}{Утверждение}
\newtheorem*{lemma}{Лемма}
\newtheorem*{proposal}{Предложение}

\theoremstyle{definition}
\newtheorem*{definition}{Определение}

\theoremstyle{remark}
\newtheorem*{note}{Замечание}

\renewcommand\qedsymbol{$\blacksquare$}

\begin{document}
\section*{\centering Временные ряды}

Было $ X_1, X_2, \dots, X_n \dots $ - незавсимые один расп случайные величины

Теперь $ X_1, X_2, \dots, X_n \dots $ - посл-ть случайных величин (зависимые) 
\[
    X(t) = m(t) + \epsilon(t)
\]
\[
    t = 1, 2, \dots
\]
\[
    t = 0, 1, 2, \dots
\]
\[
    t = -\infty, \dots 0, 1, 2, \dots
\]

Мат ожидание
\[
    m(t) = E(X(t)) \quad E(\epsilon(t)) = 0
\]

\begin{tcolorbox}
\begin{proposal}
    $ \epsilon(t) $ - стационарная пос-ть
\end{proposal}
\end{tcolorbox}

\begin{tcolorbox}
    \begin{proposal}[2]
        a) $ m(t) $ - медленное изменение\\
        b) $ m(t) $ - небольшой период\\
        c) $ m(t) $ - большой период

        \begin{proof}
            a) $ m(t) = \theta_0 \theta_1 t + \dots + \theta_m t^{m} $

            $ x(t) = \theta_0 + \theta_1 t + \epsilon(t) $ 
            \[
                \{ x(t) \}_{-m}^{m} \quad t = -m, -m + 1, \dots , -1, 0, 1, \dots, m
            \]

            Оценка по МНК $ \theta_0, \theta_1 $ 
            \[
                \begin{cases}
                    (2m+1)\theta_0 + \theta_1 \sum_{t=-m}^{m} t = \sum_{t=-m}^{m} x(t)\\
                    \theta_0\sum_{t=-m}^{m} t + \theta_1 
                    \sum_{t=-m}^{m} t^2 = \sum_{t=-m}^{m} t x(t)
                \end{cases}
            \]
            \[
                \sum_{t=-m}^{m} t = -m + (-m + 1) + \dots + (m-1) + m = 0
            \]
            \[
                \hat{\theta}_0 = \frac{\sum_{t=-m}^{m}x(t)}{2m+1} = \hat{m}(0) 
            \]
            \[
                \hat{\theta}_1 = \frac{\sum_{t=-m}^{m}t^2x(t)}{\sum_{t=-m}^{m}t^2} 
            \]
            \[
                \sum_{i=1}^{n} t^2 = \frac{n(n+1)(2n+1)}{6} 
            \]
            \[
                \hat{m}(t) = \frac{1}{2m+1} \sum_{k=-m}^{m} x(t+k)
            \]
            \[
                Q(m)  =\sum_{k=0}^{t-1} \lambda^{k} (x(t-k) - m)^2
            \]
            \[
                \hat{m}(t) = \lambda \hat{m}(t-1) + (1 - \lambda) x(t)
            \]
        \end{proof}
\end{proposal}
\end{tcolorbox}

\section*{\centering Метод сезонной декомпозиции}

В таблице преведены квартальные данные о продажах фирмы за период 1990-1993
(млн долларов)

тут воображаемая (не очень) табличка
и 100500 перемножений, складываний, делений чиселок

% TODO: надо чекнуть его лекцию
\end{document}
