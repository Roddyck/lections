\documentclass[a4paper]{article}
\usepackage[a4paper,%
    text={180mm, 260mm},%
    left=15mm, top=15mm]{geometry}
\usepackage[utf8]{inputenc}
\usepackage{cmap}
\usepackage[english, russian]{babel}
\usepackage{indentfirst}
\usepackage{amssymb}
\usepackage{amsmath}
\usepackage{amsthm}
\usepackage{mathtools}
\usepackage{tcolorbox}
\usepackage{import}
\usepackage{xifthen}
\usepackage{pdfpages}
\usepackage{transparent}
\usepackage{graphicx}
\graphicspath{ {./figures} }

\DeclarePairedDelimiter\set\{\}

\newcommand{\incfig}[1]{%
\def\svgwidth{\columnwidth}
\import{./figures/}{#1.pdf_tex}
}

\newtheorem*{theorem}{Теорема}
\newtheorem*{statement}{Утверждение}
\newtheorem*{lemma}{Лемма}
\newtheorem*{proposal}{Предложение}
\newtheorem*{consequence}{Следствие}


\theoremstyle{definition}
\newtheorem*{definition}{Определение}

\theoremstyle{remark}
\newtheorem*{remark}{Замечание}

\renewcommand\qedsymbol{$\blacksquare$}

\begin{document}
\textbf{\underline{Крайный срок кр - 25 апреля 2025 23:59}}

\[
    x_t = B\theta + \epsilon t, \quad t = 1, \dots, n
\]
$ \set{\epsilon_t} $ - независимые одномер. распр. случайные величины
\[
    E(\epsilon_t) = 0
\]
\[
    E(\epsilon \epsilon^T) = \sigma^2
\]

Предположение
\[
    \epsilon_t = \rho \epsilon_{t-1} + \delta_t
\]
\[
    E(\delta_t) = 0, \ E(\delta_t \delta_s) = \begin{cases}
        \sigma_0^2, \ s = t\\
        0,\  s \neq t
    \end{cases}
\]
\[
    \epsilon_t  = \rho(\rho \epsilon_{t-2} + \delta_{t-1}) + \delta_t = 
    \rho^2 \epsilon_{t-2} + \rho \delta_{t-1} + \delta_t = \rho^2(\rho \epsilon_{t-3}
    + \delta_{t-2}) + \rho\delta_{t-1} + \delta_t = \rho^3
    \epsilon_{t-3} + \rho^2\delta_{t-2} + \rho \delta_{t-1}
\]
\[
    + \delta_t = \delta_t + \rho \delta_{t-1} + \rho^2 \delta_{t-2} + \rho^3
    \delta_{t-3} + \dots
\]
\[
    D(\epsilon_t) = D(\delta_t) + \rho^2 D(\delta_{t-1}) + \rho^4 D(\delta_{t-2}) + \dots
\]
\[
    E(\epsilon_t) = 0 \quad D(\epsilon_t) = \sigma^2 = \frac{\sigma_0^2}{1-\rho^2} 
\]
\[
    \epsilon_t \epsilon_{t-k} = ((\delta_t + \rho \delta_{t-1} + \dots +
    \rho^{k-1}\delta_{t-k+1} + \rho^k(\delta_{t-k} + \rho \delta_{t-k-1} + \dots))
\]
\[
    E(\delta_t \delta_{t-k}) = \rho^k \sigma^2 = \frac{\sigma_0^2}{1-\rho^2} 
    \rho^k
\]
\[
    \begin{cases}
        X = B\theta + \epsilon\\
        E(\epsilon) = 0\\
        \Sigma = E(\epsilon \epsilon^T) = \sigma^2 \Sigma_0
    \end{cases}
\]
\[
    \Sigma_0 = \begin{pmatrix}
    1 & \rho & \rho^2 & \dots & \rho^{n-1}\\
    \rho & 1 & \rho & \dots & \rho^{n-2}\\
    \rho^2 & \rho & \dots & \dots & \dots\\
    \dots & \dots & \dots & \dots & \dots\\
    \rho^{n-1} & \rho^{n-2} & \dots & \dots & 1\\
    
    \end{pmatrix}
\]
\[
    \Sigma_{03} = \begin{pmatrix}
    1 & \rho & \rho^2\\
    \rho & 1 & \rho\\
    \rho^2 & \rho & 1\\
    
    \end{pmatrix}
\]
\[
    \Sigma_{03}^{-1} = \frac{1}{1-\rho^2} \begin{pmatrix}
    1 & -\rho & 0\\
    -\rho & 1+\rho & -\rho\\
    0 & -\rho & 1\\
    
    \end{pmatrix}
\]
\[
    A^{-1}_3 = 
    \begin{pmatrix}
    1 & -\rho & 0\\
    -\rho & 1+\rho & -\rho\\
    0 & -\rho & 1\\
    
    \end{pmatrix}
\]
\[
    A_{3}^{-1/2} = 
    \begin{pmatrix}
    \sqrt{1-\rho}  & 0 & 0\\
    -\rho & 1 & 0\\
    0 & -\rho & 1\\
    
    \end{pmatrix}
\]
\[
    C^{-1} = A_n^{-1/2} = 
    \begin{pmatrix}
    \sqrt{1-\rho^2}  & 0 & 0 & \dots & 0\\
    -\rho & 1 & 0 & \dots & 0\\
    0 & -\rho & 1 & \dots & 0\\
    \dots & \dots & \dots & -\rho & 1\\
    
    \end{pmatrix}
\]
\[
    Y = C^{-1} X = (\sqrt{1-\rho^2} x_1, x_2- \rho x_1, \dots, x_n - \rho x_{n-1})
\]
\[
    U = C^{-1} B = \begin{pmatrix}
    \sqrt{1-\rho^2}  & \sqrt{1-\rho^2} b_{11} & \dots & \sqrt{1-\rho^2} b_{1p} \\
    1- \rho& b_{21}-\rho b_{11}  & \dots & b_{2p} - \rho b_{1p} \\
     &  &  & \\
    
    \end{pmatrix}
\]
\[
    x_t = \theta b_t + \epsilon_t
\]

Считаем, что $ \set{\epsilon_t} $ - независимые случайные величины
\[
    \tilde{\theta}_{\text{об}} = \frac{\sum_{t} b_t x_t}{\sum_{t} b^2_t} 
\]
\[
    D(\tilde{\theta}_{\text{a}}) = \sigma^2(B^TB)^{-1} \frac{\sigma^2}{\sum_{t} b^2_t} 
\]

\[
    DW = \frac{\sum_{t=2}^{n} (e_t - e_{t-1})^2}{\sum_{t=2}^{n} e_t^2} =
    \frac{\sum_{t=2}^{n} e_t^2 - 2 \sum_{t=2}^{n} e_t e_{t-1} + \sum_{t=1}^{n-1}e_t^2 +
    e_n^2}{\sum_{t=2}^{n} e_t^2} = 
    2 \left( 1 - \frac{\sum_{t=2}^{n}e_te_{t-1}}{\sum_{t=2}^{n} e_t^2}\right) 
\]

\emph{Пример}

доходность $ A \sim Y_t, \quad B \sim X_t $ 
\[
    y_t = \beta_1 + \beta_2 x_t + \epsilon_t
\]
\begin{center}
    \begin{tabular}{ |c|c|c| }
        \hline
        $t$ & $y_t$ & $x_t$\\
        \hline
        1 & -2.825 & -5.31 \\
        2 & 26.02 & 16.84\\
        \dots & \dots & \dots\\
        15 & 18.01 & 10.73\\
        \hline
    \end{tabular}
\end{center}

$ \epsilon_t $ - нез. с.в. Оценим по МНК $ \beta_1, \ \beta_2 $ 
\[
    \hat{y} = 4.06 + 1.3 x_t \quad DW = 0.503
\]
\[
    e_t = y_t - \hat{y}_t \quad r = 0.748
\]

\section*{\centering Мартингал}
\begin{tcolorbox}
\begin{definition}
    1) $ \set{X_n}, \ n = 1, 2, \dots $ \\
    2) Фильтрация $ \sigma\text{-алгебра - }\mathcal{F} \subset \mathcal{F}_{n+1} $ \\
    Естественная фильтрация
    \[
        \mathcal{F}_n = \mathcal{F}(X_1, \dots, X_n) \quad \forall\, n
    \]

    Пусть $ E(|X_n|) < \infty $ 

    a) Мартингал
    \[
        E(X_m | \mathcal{F}_n) = X_n
    \]
    b) субмартингал
    \[
        X_n \leq E(X_m | \mathcal{F}_n)
    \]
    c) супермартингал
    \[
        X_n \geq E(X_m | \mathcal{F}_n)
    \]
\end{definition}
\end{tcolorbox}
\end{document}
