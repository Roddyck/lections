\documentclass[a4paper]{article}
\usepackage[a4paper,%
    text={180mm, 260mm},%
    left=15mm, top=15mm]{geometry}
\usepackage[utf8]{inputenc}
\usepackage{cmap}
\usepackage[english, russian]{babel}
\usepackage{indentfirst}
\usepackage{amssymb}
\usepackage{amsmath}
\usepackage{amsthm}
\usepackage{mathtools}
\usepackage[most]{tcolorbox}
\usepackage{import}
\usepackage{xifthen}
\usepackage{pdfpages}
\usepackage{transparent}
\usepackage{graphicx}
\graphicspath{ {./figures} }

\DeclarePairedDelimiter\set\{\}

\newcommand{\incfig}[1]{%
\def\svgwidth{\columnwidth}
\import{./figures/}{#1.pdf_tex}
}

\newtheorem*{theorem}{Теорема}
\newtheorem*{statement}{Утверждение}
\newtheorem*{lemma}{Лемма}
\newtheorem*{proposal}{Предложение}


\theoremstyle{definition}
\newtheorem*{definition}{Определение}

\theoremstyle{remark}
\newtheorem*{remark}{Замечание}

\renewcommand\qedsymbol{$\blacksquare$}

\begin{document}
\begin{equation}
    u_t(x,y,z,t) - a^2 \Delta u(x,y,z,t) = f(x,y,z,t) \quad 
    (x,y,z) \in \mathbb{R}^3, \ t > 0
    \label{eq:1}
\end{equation}
\begin{equation}
    u|_{t = 0} = u_0(x,y,z)
    \label{eq:2}
\end{equation}
\begin{tcolorbox}[title=Теорема о единственности решения задачи Коши,%
    enhanced,breakable,skin first=enhanced,skin middle=enhanced,skin last=enhanced]
\begin{theorem}[Единственность решения задачи Коши]
    Задача Коши имеет не больлее одного ограниченного решения.

    $ \exists M > 0 \ |u(x,y,z,t)| < M, \ (x,y,z) \in \mathbb{R}^3, t \leq 0 $ 

    \begin{proof}
        Предположим, что есть два решения задачи $ (\ref{eq:1}) - (\ref{eq:2}) $ 

        Рассмотрим $ u = u^{1} - u^{2} $ 

        Получим
        \begin{equation}
            u_t(x,y,z,t) - a^2 \Delta u(x,y,z,t) = 0 \quad 
            (x,y,z) \in \mathbb{R}^3, \ t > 0
            \label{eq:3}
        \end{equation}
        \begin{equation}
            u|_{t = 0} = 0
            \label{eq:4}
        \end{equation}
        \[
            |u| < 2M
        \]

        Определим в $ \mathbb{R}^3 \quad B_L(0) $ - шар радиуса $ L $ с центром
        в нуле
        \[
            B_L(0) = \set*{(x,y,z) \in \mathbb{R}^3: \ x^2+y^2+z^2 < L^2}
        \]
        \[
            \Gamma_L(0) = \set*{(x,y,z) \in \mathbb{R}^3: \ x^2+y^2+z^2 = L^2}
        \]

        \[
            Q_T = B_L(0) \times (0; T) \subset \mathbb{R}^4
        \]
        \[
            \overline{Q_T} = \overline{B_L(0)} \times [0;T]
        \]
        \[
            \Sigma = \Gamma_L(0) \times [0;T]
        \]
        \[
            \Sigma_0 = \overline{B_L(0)}
        \]

        Рассмотрим функцию
        \begin{equation}
            v(x,y,z,t) = \frac{12M}{L^2} \left(\frac{x^2+y^2+z^2}{6} + a^2t \right)
            \label{eq:5}
        \end{equation}

        Легко убедиться в том, что
        \[
            v_t - a^2 \Delta v = 0
        \]

        Также $ v - u $ и $ v+u $ тоже удовлетворяют уравнению теплопроводности

        Тогда можем применить к этим функция теорему о максимуме и минимуме
        уравнения теплопроводности

        На нижнем основании $ \Sigma_0 $
        \[
            v-u \geq 0, \quad v+u \geq 0
        \]

        На $ \Sigma $:
        \[
            v = \frac{12M}{L^2} \left(\frac{L^2}{6} + a^2 t \right) \geq 
            \frac{12M}{6} \cdot \frac{L^2}{6} = 2M
        \]
        Т.е. $ v \geq 2M $ \\
        Т.к. $ |u| < 2M $, то на $ \Sigma $ 
        \[
            v-u \geq 0, \quad v+u \geq 0
        \]

        По теореме о макс и мин ур-я теплопроводности функции $ v-u, \ v+u $ 
        достигают наименьшего значения либо на основании, либо на боковой
        поверхности, а это наименьшее значение не отрицательно. А значит
        и во всем цилиндре $ \overline{Q_T} $:
        \[
            v-u \geq 0, \quad v+u \geq 0
        \]

        Из этого следует, что в $ \overline{Q_T} $ 
        \[
            |u| \leq v
        \]

        Рассмотрим произвольную точку $ (x_0, y_0, z_0, t_0) \in \mathbb{R}^3
        \times (0;T)$ 

        Можно выбрать достаточно большие $ T, L $, чтобы точка попала в цилиндр
        $ Q_T $

        Тогда
        \[
            |u(x_0,y_0,z_0,t_0)| \leq v(x_0,y_0,z_0,t_0)
        \]

        Рассмотрим
        \[
            v(x_0,y_0,z_0,t_0) = \frac{12M}{L^2} 
            \left(\frac{x_0^2+y_0^2+z_0^2}{6} + a^2 t_0 \right)
        \]

        $ v $ в этой точке может быть сколь угодно малым, в зависимости от $ L $ 

        Тогда $ u(x_0, y_0,z_0,t_0) = 0 $.

        Значит, что $ u^{1} = u^2 $ 

    \end{proof}
\end{theorem}
\end{tcolorbox}

\section*{Теорема о единственности решения смешанной задачи для уравнения
теплопроводности}

Решения будем искать в области $ \Omega \subset \mathbb{R}^3, \ \Gamma = 
\partial \Omega$ 

Пусть $ \Gamma = \Gamma_1 \cup \Gamma_2 \cup \Gamma_3, \ \Gamma_i \cap \Gamma_j
= \varnothing, \ \text{if } i \neq j$ 

Начальные и граничные условия:
\[
    u|_{t=0} = u_0(x,y,z), \ (x,y,z) \in \Omega
\]
\begin{subequations}
    \begin{align}
        &u|_{\Gamma_1} = u_{\Gamma}(x,y,z,t), \ (x,y,z) \in \Gamma_1, \ t \geq 0\\
        &\lambda \frac{\partial u}{\partial v} = -q_{\Gamma}(x,y,z,t)\\
        &\lambda \frac{\partial u}{\partial v} + hu = h_t...
    \end{align}
\end{subequations}

\begin{tcolorbox}[enhanced,breakable,skin first=enhanced,skin middle=enhanced,skin last=enhanced]
    \begin{theorem}
        Решение смешанной задачи единственно
        \begin{proof}
            Предположим, что есть два решения. $ u = u^1 - u^2 $. Получим
            \begin{equation}
                u_t(x,y,z,t) - a^2 \Delta u(x,y,z,t) = 0
                \label{eq:7}
            \end{equation}
            \begin{equation}
                u|_{t=0} = 0 \ (x,y,z) \in \Omega
            \end{equation}
            \begin{subequations}
                \begin{align}
                &u|_{\Gamma_1} = 0 \ (x,y,z) \in \Gamma_1, \ t \geq 0\\
                &\lambda \frac{\partial u}{\partial v} = 0\\
                &\lambda \frac{\partial u}{\partial v} + hu = 0
                \end{align}
            \end{subequations}
            \[
                (\ref{eq:7}) \cdot u, \ \int_{\Omega} dx dy dz
            \]

            Получим следующее
            \begin{equation}
                \int_\Omega u u_t dx dy dz - a^2 \int_\Omega u \Delta u dxdydz = 0
                \label{eq:10}
            \end{equation}
            \[
                u u_t = \frac{1}{2} \frac{\partial}{\partial t} (u^2)
            \]
            \[
                \int_\Omega u u_t dxdydz = \frac{1}{2} \frac{d}{dt} \int_{\Omega}
                u^2 dxdydz
            \]
            \[
                u \Delta u = \frac{\partial}{\partial x} \left( u \frac{\partial u}{\partial x} 
                \right) + \frac{\partial}{\partial y} \left( u \frac{\partial u}{\partial y} 
                \right) + \frac{\partial}{\partial z} \left( u \frac{\partial u}{\partial z} 
                \right) + \left[ \left( \frac{\partial u}{\partial x} \right)^2
                    + \left( \frac{\partial u}{\partial y} \right)^2 +
                \left( \frac{\partial u}{\partial z} \right)^2 \right]
            \]
            \[
                \int_{\Omega} u \Delta u dxdydz
                = \int_{\Omega}\frac{\partial}{\partial x} \left( u \frac{\partial u}{\partial x} 
                \right) + \frac{\partial}{\partial y} \left( u \frac{\partial u}{\partial y} 
                \right)dxdydz
                + \frac{\partial}{\partial z} \left( u \frac{\partial u}{\partial z} 
                \right) + \int_{\Omega}\left[ \left( \frac{\partial u}{\partial x} \right)^2
                    + \left( \frac{\partial u}{\partial y} \right)^2 +
                \left( \frac{\partial u}{\partial z} \right)^2 \right] dxdydz
            \]

            Применим теорему Гаусса-Остроградского
            \begin{equation}
                \int_\Omega u \Delta u dxdydz = \int_{\Gamma} u \frac{\partial u}{\partial v} 
                d \Gamma - \int_{\Omega}(u_x^2 + u_y^2 + u_z^2) dxdydz
                \label{eq:11}
            \end{equation}

            Подставим в $ (\ref{eq:10}) $ 
            \begin{equation}
                \frac{1}{2} \frac{d}{dt} \int_{\Omega}u^2 dxdydz - a^2
                \int_{\Gamma}u \frac{\partial u}{\partial \nu} d \Gamma
                + a^2 \int_\Omega (u_{x}^2 + u_y^2 + u_z^2) dxdydz = 0
                \label{eq:12}
            \end{equation}
            \[
                \frac{1}{2} \frac{d}{dt} \int_{\Omega}u^2 dxdydz + a^2
                \int_{\Gamma} \frac{h}{\lambda} u^2 d \Gamma
                + a^2 \int_\Omega (u_{x}^2 + u_y^2 + u_z^2) dxdydz = 0
            \]
            \[
                \frac{1}{2} \frac{d}{dt} \int_{\Omega}u^2 dxdydz = - a^2
                \int_{\Gamma} \frac{h}{\lambda} u^2 d \Gamma
                - a^2 \int_\Omega (u_{x}^2 + u_y^2 + u_z^2) dxdydz
            \]

            Рассмотрим функцию
            \[
                \Phi(t) = \int_\Omega u^2(x,y,z,t) dxdydz
            \]

            Из последнего равенства следует
            \[
                \frac{1}{2} \frac{d\Phi(t)}{dt} \leq 0
            \]
            \[
                \Phi(0) = 0, \ \Phi(t) \geq 0
            \]
            \[
                \Phi(t) \equiv 0
            \]

            Отсюда
            \[
                u^2(x,y,z,t) \equiv 0
            \]
            \[
                u \equiv 0
            \]
        \end{proof}
    \end{theorem}
\end{tcolorbox}

\section*{Применение метода разделения переменных для уравнения теплопроводности
стержня}
\begin{equation}
    u_t(x,t) - a^2 u_{xx}(x,t) = f(x,t) \quad x \in (0,l), \ t > 0
\end{equation}
\begin{equation}
    u|_{t=0} = u_0(x)
\end{equation}
\begin{equation}
    u(0,t) = u(l,t) = 0
\end{equation}

Рассматриваем
\[
    u_t - a^2 u_{xx} = 0
\]

Ищем решение вида
\[
    u = T(t) X(x)
\]

\[
    T'(t) X(x) - a^2 T(t) X''(x) = 0
\]
\[
    \frac{T'(t)}{a^2 T(t)} = \frac{X''(x)}{X(x)} = -\lambda
\]
\[
    \begin{cases}
        X''(x) + \lambda X(x) = 0\\
        X(0) = X(l) = 0
    \end{cases}
\]
\[
    \lambda_n = \left( \frac{\pi n}{l} \right)^2, \ n \in \mathbb{N}
\]
\[
    X_n(x) = \sin \frac{\pi n x}{l} 
\]

Будем искать решение в виде
\[
    u(x,t) = \sum_{n=1}^{\infty} T_n(t) X_n(x) = \sum_{n=1}^{\infty} T_n(t)
    \sin \frac{\pi n x}{l}
\]
\end{document}
