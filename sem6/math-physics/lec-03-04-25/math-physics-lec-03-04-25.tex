\documentclass[a4paper]{article}
\usepackage[a4paper,%
    text={180mm, 260mm},%
    left=15mm, top=15mm]{geometry}
\usepackage[utf8]{inputenc}
\usepackage{cmap}
\usepackage[english, russian]{babel}
\usepackage{indentfirst}
\usepackage{amssymb}
\usepackage{amsmath}
\usepackage{amsthm}
\usepackage{mathtools}
\usepackage{tcolorbox}
\usepackage{import}
\usepackage{xifthen}
\usepackage{pdfpages}
\usepackage{transparent}
\usepackage{graphicx}
\graphicspath{ {./figures} }

\DeclarePairedDelimiter\set\{\}

\newcommand{\incfig}[1]{%
\def\svgwidth{\columnwidth}
\import{./figures/}{#1.pdf_tex}
}

\newtheorem*{theorem}{Теорема}
\newtheorem*{statement}{Утверждение}
\newtheorem*{lemma}{Лемма}
\newtheorem*{proposal}{Предложение}
\newtheorem*{consequence}{Следствие}


\theoremstyle{definition}
\newtheorem*{definition}{Определение}

\theoremstyle{remark}
\newtheorem*{remark}{Замечание}

\renewcommand\qedsymbol{$\blacksquare$}

\begin{document}
\begin{tcolorbox}
\begin{theorem}
    
    \begin{proof}
        $ \Omega \subset \mathbb{R}^3 $ 

          $ \Omega \to \Omega_{\epsilon} $ 

          $ \Gamma \to \Gamma \cup S_{\epsilon} $
          \[
              v = \frac{1}{r_{M M_0}} = \frac{1}{\sqrt{(x-x_0)^2 + (y-y_0)^2 +
              (z-z_0)^2}} = \frac{1}{r}
          \]
          \begin{equation}
              \iiint_{\Omega_{\epsilon}} \frac{\Delta u}{r} dxdydz 
              = \iint_{\Gamma} \left(\frac{1}{r} \frac{\partial u}{\partial \nu} - u 
              \frac{\partial \frac{1}{r} }{\partial \nu} \right) d \Gamma
              + \iint_{S_{\epsilon}} \left(\frac{1}{r} \frac{\partial u}{\partial \nu} - u 
              \frac{\partial \frac{1}{r} }{\partial \nu} \right) d \Gamma
          \end{equation}

          Предпологаем $ u \in C^n (\overline{\Omega}) $\\
          $ \overline{\Omega} = \Omega \cup \Gamma $ 

          Тогда по теореме Вейерштрасса
          \[
              \exists C > 0: \ \forall (x,y,z) \in \overline{\Omega} \  
              |u(x,y,z)| \leq C, \ |\text{grad}\, u| = 
              \sqrt{\left(\frac{\partial u}{\partial x}\right)^2 +
              \left(\frac{\partial u}{\partial y}\right)^2 +
              \left(\frac{\partial u}{\partial z}\right)^2} \leq C
          \]

          Рассмотрим:
          \[
              \iiint_{\overline{B_{\epsilon}(M_0)}} \frac{\Delta u}{r} dxdydz
          \]

          Если перейти в сферическую систему координат
          \[
              \begin{cases}
                  x = r \cos \phi \sin \theta\\
                  y = r \sin \phi \sin \theta\\
                  z = r \cos \theta
              \end{cases} \quad \theta \in [0;\pi], \ \phi \in [0; 2\pi]
          \]

          То элемент объёма
          \[
              dV = dxdydz = r^2 \sin \theta d\theta d\phi dr
          \]
          \[
              \iiint_{\overline{B_{\epsilon}(M_0)}} \frac{\Delta u}{r} dxdydz
              = \int_{0}^{2\pi} d \phi \int_{0}^{\pi} \sin \theta d \theta
              \int_{0}^{\epsilon} \Delta u \cdot r dr
          \]
          \[
              \Bigg|\iiint_{\overline{B_{\epsilon}(M_0)}} \frac{\Delta u}{r} dxdydz\Bigg|
              = \int_{0}^{2\pi} d \phi \int_{0}^{\pi} \sin \theta d \theta
              \int_{0}^{\epsilon} |\Delta u| \cdot r dr \leq C \frac{\epsilon^2}{2} 
              \cdot 2 \cdot 2 \pi \xrightarrow[\epsilon \to 0]{} 0
          \]
          \[
              \iiint_{\Omega_{\epsilon}} \frac{\Delta u}{r} dxdydz = 
              \iiint_{\Omega} \frac{\Delta u}{r} dxdydz - 
              \iiint_{B_{\epsilon}} \frac{\Delta u}{r} dxdydz
          \]
          \[
              \iiint_{\Omega_{\epsilon}} \frac{\Delta u}{r} dxdydz
              \xrightarrow[\epsilon \to 0]{} \iiint_{\Omega} \frac{\Delta u}{r} dxdydz
          \]
          \[
              \iint_{S_{\epsilon}} \frac{1}{r} \frac{\partial u}{\partial \nu} d\Gamma
              = \epsilon^2 \int_{0}^{2\pi} d \phi \int_{0}^{\pi} \sin\theta d\theta
              \cdot \frac{1}{\epsilon} \bigg| \frac{\partial u}{\partial \nu} 
              \bigg| \leq \epsilon \cdot C \cdot 4\pi
              \xrightarrow[\epsilon \to 0]{} 0
          \]
          \[
              \bigg| \frac{\partial u}{\partial \nu} \bigg|
              = | (grad \, u, \vec{\nu}) \leq 
              \sqrt{\left(\frac{\partial u}{\partial x}\right)^2 +
              \left(\frac{\partial u}{\partial y}\right)^2 +
              \left(\frac{\partial u}{\partial z}\right)^2} \leq C
          \]

          На $ S_{\epsilon} $:
          \[
              \frac{\partial}{\partial \nu} = -\frac{\partial}{\partial r}, \
              \frac{\partial \frac{1}{r} }{\partial \nu} = -
              \frac{\partial }{\partial r}\left(\frac{1}{r}\right) = \frac{1}{\epsilon^2}  
          \]
          \[
              \iint_{S_{\epsilon}} u \frac{\partial \frac{1}{r} }{\partial \nu} 
              d \Gamma = \frac{1}{\epsilon^2} \iint_{S_{\epsilon}}u d\Gamma
              = \frac{1}{\epsilon^2} u(M^{\epsilon}) \cdot 4\pi \epsilon^2
              \xrightarrow[\epsilon \to 0]{} 4\pi u(M_0)
          \]
          \[
              \frac{1}{4\pi} \iiint_{\Omega} \frac{\Delta u}{r_{M M_0}} dxdydz = 
              \iint_{\Gamma}\left(\frac{1}{r} \frac{\partial u}{\partial \nu} -
              u \frac{\partial \frac{1}{r} }{\partial \nu} \right) d\Gamma - 4\pi
              u(M_0)
          \]
          \[
              u(M_0) = -\frac{1}{4\pi} \iiint_{\Omega} \frac{\Delta u}{r_{M M_0}} dxdydz = 
              + \frac{1}{4\pi} 
              \iint_{\Gamma}\left(\frac{1}{r} \frac{\partial u}{\partial \nu} -
              u \frac{\partial \frac{1}{r} }{\partial \nu} \right) d\Gamma
          \]
    \end{proof}
\end{theorem}
\end{tcolorbox}

\begin{tcolorbox}
    \begin{consequence}
        Если $ u $ - гармоническая, т.е. $ \Delta u = 0 $ в $ \Omega $. Тогда
        \begin{equation}
            u(M_0) = \frac{1}{4\pi} \iint_{\Gamma}\left(\frac{1}{r} \frac{\partial u}{\partial \nu} -
              u \frac{\partial \frac{1}{r} }{\partial \nu} \right) d\Gamma 
        \end{equation}
    \end{consequence}
\end{tcolorbox}

\section*{\centering Свойства гармонических функций}

\begin{tcolorbox}
    \underline{Вопросы из билета}\\
    1. Свойства гармонических функций. Теорема о среднем арифметическом\\
    2. Свойства гармонических функций. Теорема о максимуме и минимуме для
    гармонической функции
\end{tcolorbox}
\begin{equation}
    \Delta u(x,y,z) = 0 \quad u \in C^2(\overline{\Omega})
\end{equation}
\[
    \iint_{\Omega} \left(\frac{\partial}{\partial x} \left(\frac{\partial u}{\partial x}\right) 
    +\frac{\partial}{\partial y} \left(\frac{\partial u}{\partial y}\right)
    +\frac{\partial}{\partial z} \left(\frac{\partial u}{\partial z}\right)\right)
    dxdydz
    = 0
\]
\[
    \iint_{\Omega} \left(\frac{\partial u}{\partial x} \cos \widehat{\nu x} 
    +\frac{\partial u}{\partial y} \cos \widehat{\nu y}
    +\frac{\partial u}{\partial z} \cos \widehat{\nu z}\right)
    d \Gamma = \iint_{\Gamma}\frac{\partial u}{\partial \nu} d\Gamma = 0
\]
\begin{tcolorbox}
    \begin{theorem}[О среднем арифмитическом]
        Пусть $ u \in C^2(\overline{B_R(M_0)}) $ гармоническая в $ B_R(M_0) $ 

        $ \Gamma = S_{R}(M_0) $ 

        По следствию:
        \[
            u(M_0) = \frac{1}{4\pi} \iint_{S_R(M_0)} \frac{1}{R} \frac{\partial u}{\partial \nu} 
            d\Gamma - \frac{1}{4\pi} \iint_{S_R(M_0)}u \left(-\frac{1}{r^2} \right)
            \bigg|_{r=R} 
        \]
        \begin{equation}
            u(M_0) = \frac{1}{4 \pi R^2}  \iint_{S_R(M_0)} u d\Gamma
        \end{equation}
    \end{theorem}
\end{tcolorbox}

\begin{tcolorbox}
    \begin{theorem}[о максимуме и минимуме для гармонической функции]

        Пусть $ \Omega \subset \mathbb{R}^3 $ - открытое ограниченное, с границей
        $ \Gamma $, $ u \in C^2(\Omega) \cap C(\overline{\Omega}) $ - 
        гармоническая.

        Тогда наибольшее и наименьшее значения достигаются на границе $ \Gamma $ 
        и функция может принимать наибольшше или наименьшее значение во внутренней точке
        только в том случае, когда $ u \equiv const $ 

        \begin{proof}
            Предположим, что $ u $ приняла своё наибольшее значение во внутренней точке
            $ M_0 $. Докажем, что $ u \equiv const $.

            Тогда $ \exists\, r_0 > 0: \ \overline{B_{r_0}(M_0)} \subset \Omega $.

            По теореме о среднем:
            \[
                u(M_0) = \frac{1}{4\pi r_0^2} \iint_{S_{r_0}(M_0)} u(M) d \Gamma
            \]

            Также справедливо
            \[
                u(M_0) = \frac{1}{4\pi r_0^2} \iint_{S_{r_0}(M_0)} u(M_0) d \Gamma
            \]
            \[
                \frac{1}{4\pi r_0^2} \iint_{S_r(M_0)}\underbrace{(u(M_0) - u(M))}_{\geq 0}
                d \Gamma = 0
            \]
            \[
                u(M_0) \equiv u(M)
            \]

            Те же рассуждения справедливы для любых $ r: \ 0 < r \leq r_0 $ 

            Тогда в $ \overline{B_{r_0}(M_0)} \ u(M) \equiv u(M_0)$ 

            Если рассмотрим точку $ N \in \Omega $. Можем соединить $ M_0 $ с
            $ N $ линией конечное длины. Множество точек этой линий $ l $ замкнуто.

            Если взять $ R_0 = dist \set{l; \Gamma} > 0 $ 

            И если в предыдущих рассуждениях взять $ r_0 = R_0 $, то можно построить
            конечную цепь шаров $ \set{B_{R_0}(M_i)}_{i=0}^{m}, \ M_{i+1} \in
            B_{R_0}(M_i), \ i = 0, \dots, m-1$. $ N \in \overline{B_{R_0}(M_m)} $  
        \end{proof}
    \end{theorem}
\end{tcolorbox}
\end{document}
