\documentclass[a4paper]{article}
\usepackage[a4paper,%
    text={180mm, 260mm},%
    left=15mm, top=15mm]{geometry}
\usepackage[utf8]{inputenc}
\usepackage{cmap}
\usepackage[english, russian]{babel}
\usepackage{indentfirst}
\usepackage{amssymb}
\usepackage{amsmath}
\usepackage{amsthm}
\usepackage{mathtools}
\usepackage{tcolorbox}
\usepackage{import}
\usepackage{xifthen}
\usepackage{pdfpages}
\usepackage{transparent}
\usepackage{graphicx}
\graphicspath{ {./figures} }

\newcommand{\incfig}[1]{%
\def\svgwidth{\columnwidth}
\import{./figures/}{#1.pdf_tex}
}

\newtheorem*{theorem}{Теорема}

\theoremstyle{definition}
\newtheorem*{definition}{Определение}

\begin{document}
\title{УМФ. Лекция}
\author{}
\date{20 Февраля 2025 г}
\maketitle

\begin{equation}
    v(r,t) = \sum_{i=1}^{\infty} v_i(r,t) = \sum_{i=1}^{\infty} J_0\left(\frac{\mu_i^{(0)}}{L} 
    \right) \left( A_i \cos \frac{a\mu_i^{(0)} t}{L} + B_i 
    \sin\frac{a\mu_i^{(0)} t}{L}\right)
\end{equation}

\begin{tcolorbox}
    \emph{Условие ортогональности функции Бесселя}
    \begin{equation}
        \int_{0}^{L} x J_0 \left( \frac{\mu_i^{(0)}x}{L} \right)\cdot
        J_0\left(\frac{\mu_j^{(0)}x}{L}\right) dx =
        \begin{cases}
            0, &\quad i \neq j \\
            \frac{L^2}{2} (J'_0 (\mu_i^{(0)}))^2, &\quad i = j
        \end{cases}
    \end{equation}
\end{tcolorbox}

\[
    v(r,t) |_{r=0} = \phi(r) \implies \sum_{i=1}^{\infty} J_0 \left(\frac{\mu_i^{(0)}x}{L}\right)
    \cdot A_i = \phi(r) \ | \ \cdot r J_0 \left(\frac{\mu_0 r}{L} \right), \
    \int_{0}^{L} 
\]
\[
    v_t(r,t) |_{r=0} = \psi(r) \implies \sum_{i=1}^{\infty} J_0 \left(\frac{\mu_i^{(0)}x}{L}\right)
    \frac{a \mu_i^{(0)}}{L} B_i = \psi(r)
\]

\[
    \frac{L^2}{2}(J'_0(\mu_j))^2 A_j = \phi(r) r J_0 \left(\frac{\mu_j r}{L} \right)
\]
\begin{equation}
    A_j = \frac{1}{\frac{L^2}{2}(J_0'(\mu_j))^2 } \cdot \int_{0}^{L} 
    \phi(r) \cdot r \cdot J_0 \left(\frac{\mu_j r}{L} \right)
\end{equation}
\begin{equation}
    B_j = \frac{1}{\frac{a \mu_j^{(0)}}{L}\frac{L^2}{2}(J_0'(\mu_j))^2  } 
    \cdot \int_{0}^{L} \psi(r) \cdot r \cdot J_0 \left(\frac{\mu_j r}{L} \right)
\end{equation}

\section*{\centering Параболические уравнения. Уравнения теплопроводности}

\textbf{Вопрос из билета: Вывод уравнения теплопроводности}

\[
    u(x,y,z,t) \text{ - температура в точке наблюдения в момент времени } t
\]
\[
    V \subset \mathbb{R}^3, \ t \in [t_1; t_2]
\]

\[
    dV = dxdydz
\]
\[
    \iiint_V C \rho u(x,y,z,t) dV
\]
\[
    C \text{ - удельная теплоёмкость} \quad \rho \text{ - объёмная плотность 
    вещества}
\]
\[
    \iiint_V \rho(x,y,z,t) dV = m_V
\]
\[
    \iiint_V C \rho u(x,y,z,t_2) dV - \iiint_V C \rho u(x,y,z,t_1) dV
\]

Пусть существует функция, называемая объёмная ... 
\[
    \iiint_V F(x,y,z,t) dx dy dz \text{ - количество тепла, выделенное внутренними
    источниками в еденицу времени}
\]
\[
    \int_{t_1}^{t_2} \iiint_V F(x,y,z,t) dx dy dz dt 
    \text{ + приход тепла через границу}
\]
\[
    \vec{J}(x,y,z,t) \text{ - плотность потока тепла}
\]
\[
    (\vec{J} \cdot \vec{n}) dS \text{ - количество тепла, проходящего через
    площадку } dS \text{ в направлении n за еденицу времени}
\]

Пусть $ S $ - поверхность, ограничивающия объём $ V $ 
\[
    \int_{t_1}^{t_2} \iint_S (\vec{J} \cdot \vec{n}) dS dt
\]


\[
    \iiint_V C \rho u(x,y,z,t_2) dV - \iiint_V C \rho u(x,y,z,t_1) dV =
    \int_{t_1}^{t_2} \iiint_V F(x,y,z,t) dx dy dz dt +
    \int_{t_1}^{t_2} \iint_S (\vec{J} \cdot \vec{n}) dS dt
\]

\begin{tcolorbox}[title=Закон Фурье]
\[
    \vec{J}(x,y,z,t) = -\lambda \, \text{grad} \, u(x,y,z,t)
\]
\end{tcolorbox}

\[
    (\vec{J}, \vec{n}) = -\lambda \frac{\partial u}{\partial n} 
\]

\[
    \iiint_V C \rho u(x,y,z,t_2) dV - \iiint_V C \rho u(x,y,z,t_1) dV =
    \int_{t_1}^{t_2} \iiint_V F(x,y,z,t) dx dy dz dt -
    \int_{t_1}^{t_2} \iint_S \left(-\lambda \frac{\partial u}{\partial n} \right) dS dt
\]
\[
    \iiint_V \left( \frac{\partial}{\partial x} 
        \left(\lambda \frac{\partial u}{\partial x}\right) +
    \frac{\partial}{\partial y} 
            \left(\lambda \frac{\partial u}{\partial y}\right) +
    \frac{\partial}{\partial z} 
            \left(\lambda \frac{\partial u}{\partial z}\right) 
    \right)
    = \iint_S \lambda \frac{\partial u}{\partial n} 
    = \iint_D (P \cos \hat{nx} + Q \cos \hat{ny} + R \cos \hat{nz})
\]
\[
    = \iint_s \lambda (\text{grad}(u \cdot \vec{n})) ds
\]
\[
    \iiint_V C \rho u(x,y,z,t_2) dV - \iiint_V C \rho u(x,y,z,t_1) dV =
    \int_{t_1}^{t_2} \iiint_V F(x,y,z,t) dx dy dz dt -
\]
    \[
    \int_{t_1}^{t_2} 
    \iiint_V \left( \frac{\partial}{\partial x} 
        \left(\lambda \frac{\partial u}{\partial x}\right) +
    \frac{\partial}{\partial y} 
            \left(\lambda \frac{\partial u}{\partial y}\right) +
    \frac{\partial}{\partial z} 
            \left(\lambda \frac{\partial u}{\partial z}\right) 
    \right)
\]
\[
    \int_{t_1}^{t_2} \iiint_V C \rho \frac{\partial u}{\partial t} dV dt
\]
\[
    C \rho \frac{\partial u}{\partial t} = 
    \frac{\partial}{\partial x} 
            \left(\lambda \frac{\partial u}{\partial x}\right) +
        \frac{\partial}{\partial y} 
                \left(\lambda \frac{\partial u}{\partial y}\right) +
        \frac{\partial}{\partial z} 
                \left(\lambda \frac{\partial u}{\partial z}\right) 
    + F(x,y,z,t)
\]

\end{document}
