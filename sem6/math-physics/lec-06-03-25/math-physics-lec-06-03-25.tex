\documentclass[a4paper]{article}
\usepackage[a4paper,%
    text={180mm, 260mm},%
    left=15mm, top=15mm]{geometry}
\usepackage[utf8]{inputenc}
\usepackage{cmap}
\usepackage[english, russian]{babel}
\usepackage{indentfirst}
\usepackage{amssymb}
\usepackage{amsmath}
\usepackage{amsthm}
\usepackage{mathtools}
\usepackage{tcolorbox}
\usepackage{import}
\usepackage{xifthen}
\usepackage{pdfpages}
\usepackage{transparent}
\usepackage{graphicx}
\graphicspath{ {./figures} }

\newcommand{\incfig}[1]{%
\def\svgwidth{\columnwidth}
\import{./figures/}{#1.pdf_tex}
}

\newtheorem*{theorem}{Теорема}
\newtheorem*{statement}{Утверждение}
\newtheorem*{lemma}{Лемма}
\newtheorem*{proposal}{Предложение}


\theoremstyle{definition}
\newtheorem*{definition}{Определение}

\theoremstyle{remark}
\newtheorem*{remark}{Замечание}

\renewcommand\qedsymbol{$\blacksquare$}

\begin{document}
\section*{Задача Коши для уравнения теплопроводности}

\[
    (x,y,z) \in \mathbb{R}^3, \ t \geq 0
\]
\begin{equation}
    u_t(x,y,z,t) - a^2 \Delta u(x,y,z,t) = 0
\end{equation}
\begin{equation}
    u|_{t=0} = \phi(x,y,z), \ (x,y,z) \in \mathbb{R}^3
\end{equation}
\[
    u \in C^2(\mathbb{R}^3 \times (O, T))
\]
\[
    u \in C(\mathbb{R}^3 \times [O, T])
\]
\[
    \phi \in C(\mathbb{R}^3) \text{ и ограничена}
\]

Рассмотрим:
\begin{equation}
    F(x,y,z; \xi, \eta, \zeta, t) = \frac{1}{(2a \sqrt{\pi t})^3} 
    e^{-\frac{(x-\xi)^2 + (y-\eta)^2 + (z-\zeta)^2}{4a^2 t} }
\end{equation}

Эта функция как функция от переменных $ (x,y,z,t) $ является решением уравнения
теплопроводности

\begin{equation}
    \iiint_{-\infty}^{\infty} F(x,y,z; \xi, \eta, \zeta, t)d\xi d\eta d\zeta = 
    \iiint_{-\infty}^{\infty} F(x,y,z; \xi, \eta, \zeta, t) dx dy dz = 1
\end{equation}

Если $ (\xi, \eta, \zeta) \neq (x,y,z) $, то
\[
    \lim_{t \to +0} F(x,y,z,\xi,\eta,\zeta) = 0
\]

Если $ (\xi, \eta, \zeta) \equiv (x,y,z) $, то
\[
    \lim_{t \to +0} F(x,y,z,\xi,\eta,\zeta) = \infty
\]

\begin{equation}
    u(x,y,z,t) = \iiint_{-\infty}^{\infty} F(x,y,z; \xi, \eta, \zeta, t)
    \phi(\xi,\eta,\zeta) d\xi d\eta d\zeta
    \label{eq:5}
\end{equation}
$ (\ref{eq:5}) $ Является решением задачи Коши 
\[
    F(x,y,z; \xi, \eta, \zeta, t) \phi(\xi,\eta,\zeta) d\xi d\eta d\zeta 
    \xrightarrow{t\to +0} \phi(x,y,z)
\]

$ \mathbb{R}^2 $:
\[
    u_t(x,y,t) - a^2 \Delta_{x,y} u(x,y,t) = 0
\]
\[
    F(x,y,\xi,\eta,t) = \frac{1}{(2a \sqrt{\pi t})^2}
    e^{-\frac{(x-\xi)^2 + (y-\eta)^2}{4a^2t} } 
\]

$ \mathbb{R}^{1} $:
\[
    F(x,\xi,t) = \frac{1}{(2a \sqrt{\pi t})^2} e^{-\frac{(x-\xi)^2}{4a^2t}}
\]
\[
    u_t - a^2 \frac{\partial^{2} u}{\partial x^2} 
\]

$ F $ называется фундаментальным решением уравнения теплопроводности

\section*{Теорема о максимуме и минимуме для уравнения теплопроводности}
\[
    (x,y,z) \in \Omega \subset \mathbb{R}^3 \quad \Omega \text{ - открытое и ограниченное}
\]
\[
    t \in (0; T)
\]
\[
    \overline{\Omega} = \Omega \cup \Gamma
\]
\[
    Q_T = \Omega \times (O;T)
\]
\[
    \overline{Q_T} = \overline{\Omega} \times [O;T]
\]

Нижнее основание $ \Sigma_0 = \overline{\Omega} \times \{ 0 \} $, верхнее основание
$ \Sigma_T = \overline{\Omega} \times \{ T \} $.

Боковая поверхность $ \Gamma_{\text{бок}} = \Gamma \times [0,T] $ 

Классическим решением для уравнения теплопроводности будем называть функцию
\[
    u \in C^2(\Omega \times (0;T)) \cap C^{1}(\overline{Q}_T)
\]

\newpage

\begin{tcolorbox}
    \begin{theorem}
        Пусть $ u $ - классическое решение уравнения теплопроводности. Тогда $ u $ принимает
        свои наибольшие и наименьшие значения либо на нижнем основании цилиндра, либо на
        боковой поверхности

        \begin{proof}
            Предположим, что $ u $ своё наибольшее значение $ M $ на $ \overline{Q}_T $ 
            принимает в некоторой точке $ (x_0, y_0, z_0, t_0), \  0 < t_0 \leq T,
            \ (x_0,y_0,z_0) \in \Omega$.

            Пусть
            \[
                m = \max_{\Sigma_0 \cup \Gamma_{\text{бок}}} u
            \]
            
            Тогда $ m < M $ 

            Рассмотрим вспомогательную функцию:
            \[
                v(x,y,z,t) = u(x,y,z,t) +  \frac{M - m}{6 d^2} [(x-x_0)^2 
                + (y-y_0)^2 + (z-z_0)^2]
            \]

            Заметим, что если $ (x,y,z,t) \in \Sigma_0 \cup \Gamma_{\text{бок}} $:
            \[
                v(x,y,z,t) \leq m + \frac{M - m}{6 d^2} \cdot d^2 = \frac{M+5m}{6} <
                \frac{M+5M}{6} < M
            \]
            
            Также заметим, что
            \[
                v(x_0, y_0, z_0, t_0) = u(x_0, y_0, z_0, t_0) = M
            \]

            Построенная функция $ v $ так же не может принимать своё наибольшее
            значение на нижнем основании или боковой поверхности.

            Тогда $ v $ принимает наибольшее значение в точке $ (x_1, y_1, z_1,t_1), \ 
            0 < t_1 \leq T, \ (x_1,y_1,z_1) \in \Omega$ 

            В точке $ (x_1, y_1, z_1,t_1) $
            \[
                \frac{\partial^2 v}{\partial x^2} \leq 0, \
                \frac{\partial^2 v}{\partial y^2} \leq 0, \
                \frac{\partial^2 v}{\partial z^2} \leq 0, \
            \]
            \[
                \frac{\partial v}{\partial t} = 0, \text{ если } t_1 < T
            \]

            Если $ t_1 = T $:
            \[
                \frac{\partial v}{\partial t} \geq 0
            \]

            Получаем, что в точке $ (x_1, y_1, z_1,t_1) $
            \[
                \frac{\partial v}{\partial t} - a^2 \Delta v \geq 0
            \]

            С другой стороны:
            \[
                \frac{\partial v}{\partial t} - a^2 \Delta v = \underbrace{
                \frac{\partial u}{\partial t} - a^2 \Delta u}_{=0} - \frac{M-m}{d^2} 
                < 0
            \]

            Получили противоречие
        \end{proof}
    \end{theorem}
\end{tcolorbox}

\begin{tcolorbox}
\begin{remark}
    Док-во теоремы о наименьшем значении получается при замене $ u $ на $ -u $ 
\end{remark}
\end{tcolorbox}

\section*{Теорема о единственности решения задачи Коши для уравнения теплопроводности}
\begin{equation}
    u^{i}_t(x,y,z,t) - a^2 \Delta u^{i}(x,y,z,t) = f(x,y,z,t)
\end{equation}
\begin{equation}
    u^{i}(x,y,z,t) |_{t=0} = \phi (x,y,z), \ (x,y,z) \in \mathbb{R}^3
\end{equation}
\[
    u^{i}, \ i = 1, 2
\]
\[
    u = u^{1} - u^2
\]
\begin{equation}
    u_t(x,y,z,t) - a^2 \Delta u(x,y,z,t) = 0
\end{equation}
\begin{equation}
    u(x,y,z,t) |_{t=0} = 0
\end{equation}

Докажем:
\[
    u(x,y,z,t) \equiv 0, \ (x,y,z) \in \mathbb{R}^3, \ t > 0
\]

\end{document}
