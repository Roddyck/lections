\documentclass[a4paper]{article}
\usepackage[a4paper,%
    text={180mm, 260mm},%
    left=15mm, top=15mm]{geometry}
\usepackage[utf8]{inputenc}
\usepackage{cmap}
\usepackage[english, russian]{babel}
\usepackage{indentfirst}
\usepackage{amssymb}
\usepackage{amsmath}
\usepackage{amsthm}
\usepackage{mathtools}
\usepackage{tcolorbox}
\usepackage{import}
\usepackage{xifthen}
\usepackage{pdfpages}
\usepackage{transparent}
\usepackage{graphicx}
\graphicspath{ {./figures} }

\DeclarePairedDelimiter\set\{\}

\newcommand{\incfig}[1]{%
\def\svgwidth{\columnwidth}
\import{./figures/}{#1.pdf_tex}
}

\newtheorem*{theorem}{Теорема}
\newtheorem*{statement}{Утверждение}
\newtheorem*{lemma}{Лемма}
\newtheorem*{proposal}{Предложение}
\newtheorem*{consequence}{Следствие}


\theoremstyle{definition}
\newtheorem*{definition}{Определение}

\theoremstyle{remark}
\newtheorem*{remark}{Замечание}

\renewcommand\qedsymbol{$\blacksquare$}

\begin{document}
\section*{Применение теоремы о максимуме и минимуме}
\[
u \in C^2(\Omega) \cap C(\overline{\Omega})
\]
\begin{equation}
    \Delta u(x,y,z) = f(x,y,z)
\end{equation}
\begin{equation}
    u |_{\Gamma} = U|_{\Gamma}(x,y,z), \ (x,y,z) \in \Gamma
\end{equation}

$ \Omega \subset \mathbb{R}^3 $ - открытая ограниченная область

Предположим, что существует два решения $ u^i, \ i = 1,2 $ 

Рассмотрим $ u = u^1 - u^2 $ 
\[
    \Delta u(x,y,z) = 0
\]
\[
    u(x,y,z)|_{\Gamma} = 0
\]
\[
    0 \leq u(x,y,z) \leq 0 \implies u \equiv 0
\]

\section*{Единственность решения основных краевых задач для уравнения Пуассона}

$ \Omega \subset \mathbb{R}^3 $ - открытая ограниченная область. 
$ \Gamma = \partial \Omega = \Gamma_1 \cup \Gamma_2 \cup \Gamma_3, \ 
\Gamma_i \cap \Gamma_j = \varnothing, \ i \neq j$ 
\begin{subequations}
    \begin{align}
            &u(x,y,z) |_{\Gamma_1} = u_{\Gamma}(x,y,z), \ (x,y,z) \in \Gamma_1\\
            &\frac{\partial u(x,y,z)}{\partial \nu} \bigg|_{\Gamma_2} = q_{\Gamma}(x,y,z)
            \ (x,y,z) \in \Gamma_2\\
            &\left( \frac{\partial u}{\partial \nu} + hu \right)\bigg|_{\Gamma_3} = T_{\Gamma}(x,y,z)
            \ (x,y,z) \in \Gamma_3
    \end{align}
\end{subequations}

Предположим, что существует два решения $ u^i, \ i = 1,2 $ 

Рассмотрим $ u = u^1 - u^2 $ 
\begin{equation}
    \Delta u(x,y,z) = 0
\end{equation}
\begin{subequations}
    \begin{align}
            &u(x,y,z) |_{\Gamma_1} = 0\\
            &\frac{\partial u(x,y,z)}{\partial \nu} \bigg|_{\Gamma_2} = 0\\
            &\left( \frac{\partial u}{\partial \nu} + hu \right)\bigg|_{\Gamma_3} = 0
    \end{align}
\end{subequations}
\[
    \underbrace{\iiint_{\Omega} u \cdot \Delta u dxdydz}_{=0} = \iint_{\Gamma}u \frac{\partial u}{\partial \nu} 
    d\Gamma - \iiint_{\Omega} \left( \left( \frac{\partial u}{\partial x} \right)^2
    + \left( \frac{\partial u}{\partial x} \right)^2 + 
\left( \frac{\partial u}{\partial x} \right)^2\right)
\]
\[
    \underbrace{\iiint_{\Omega} \left( \left( \frac{\partial u}{\partial x} \right)^2
    + \left( \frac{\partial u}{\partial x} \right)^2 + 
\left( \frac{\partial u}{\partial x} \right)^2\right)}_{\geq 0} = 
\underbrace{-\iint_{\Gamma_3}h u^2 d\Gamma}_{\leq 0}
\]
\[
    \frac{\partial u}{\partial x} = 0, \quad 
    \frac{\partial u}{\partial y} = 0, \quad
    \frac{\partial u}{\partial z} = 0
\]
\[
    u(x,y,z) \equiv const \text{ в } \Omega 
\]

Пусть $ \Gamma_1 \neq \varnothing \implies u(x,y,z) \equiv 0 $ 

Пусть $ \Gamma_3 \neq \varnothing $. Тогда: 
\[
    \iint_{\Gamma_3} h u^2 d\Gamma = 0 \implies u|_{\Gamma_3} = 0 \implies
    u(x,y,z) \equiv 0
\]

Пусть $ \Gamma_1 \neq \varnothing, \ \Gamma_3 \neq \varnothing, \ \Gamma_2 = \Gamma $ 
\[
    \Delta u = 0 \quad \frac{\partial u}{\partial \nu} = 0
\]

В случае, когда на всей границе задаётся граничное условие Неймана, если
решение существует, то оно не единственно. Все другие решения будет отличаться
на константу.

\begin{tcolorbox}
    \begin{remark}[по поводу задачи Неймана]
        \[
            \Delta u(x,y,z) = f(x,y,z)
        \]
        \[
            \frac{\partial u}{\partial \nu} \bigg|_{\Gamma} = q_{\Gamma}(x,y,z)
        \]

        \[
            \iiint_{\Omega} \Delta u dxdydz = \iiint_{\Omega} f dxdydz
        \]
        \[
            \iint_{\Gamma} \frac{\partial u}{\partial \nu} d\Gamma = 
            \iiint_{\Omega} f dxdydz
        \]
        \[
            \iint_{\Gamma} q_{\Gamma} d\Gamma = 
            \iiint_{\Omega} f dxdydz
        \]

        Получили необходимое условие существования решения\\
        Оно же является и достаточным условием (без доказательства)
    \end{remark}
\end{tcolorbox}

\section*{Задача Дирихле для уравнений Пуассона в круге. Формула Пуассона}
\[
    x^2 + y^2 < L^2
\]
\[
    \frac{\partial^{2} u(x,y)}{\partial x^2} + \frac{\partial^{2} u(x,y)}{\partial y^2}
    = 0
\]
\[
    u\big|_{x^2+y^2=L^2} = u_{\Gamma}(x,y)
\]
\[
    \begin{cases}
        x = r\cos \phi\\
        y = r \sin \phi
    \end{cases} \quad
    0 < r \leq L, \ \phi \in [0; 2\pi]
\]
\[
    u(x,y) = \widetilde{u}(r,\phi) = u(r\cos \phi, r\sin \phi)
\]
\[
    \Delta u = \frac{1}{r} \frac{\partial}{\partial r} \left( r
    \frac{\partial \widetilde{u}(r,\phi)}{\partial r} \right)
    + \frac{1}{r^2} \frac{\partial^{2} \widetilde{u}(r,\phi)}{\partial \phi^2} 
    = 0
\]
\begin{equation}
    \frac{1}{r} \frac{\partial}{\partial r} \left( r
    \frac{\partial \widetilde{u}(r,\phi)}{\partial r} \right)
    + \frac{1}{r^2} \frac{\partial^{2} \widetilde{u}(r,\phi)}{\partial \phi^2} 
    = 0
\end{equation}
\begin{equation}
    \widetilde{u}(r,\phi) \big|_{r = L} = f(\phi)
\end{equation}

Будем считать $ \phi \in (-\infty, \infty) $. $ f(\phi) = f(\phi + 2\pi),
\ \forall \phi \in \mathbb{R} $  

Добавим условие
\begin{equation}
    \widetilde{u}(r,\phi) = \widetilde{u}(r,\phi + 2\pi)
\end{equation}

Так как преобразование вырожденно в точке $ r = 0 $. Потребуем
\begin{equation}
    |\widetilde{u}(r;\phi)| \bigg|_{r = +0} < \infty
\end{equation}
\end{document}
