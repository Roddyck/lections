\documentclass[a4paper]{article}
\usepackage[a4paper,%
    text={180mm, 260mm},%
    left=15mm, top=15mm]{geometry}
\usepackage[utf8]{inputenc}
\usepackage{cmap}
\usepackage[english, russian]{babel}
\usepackage{indentfirst}
\usepackage{amssymb}
\usepackage{amsmath}
\usepackage{amsthm}
\usepackage{mathtools}
\usepackage{tcolorbox}
\usepackage{import}
\usepackage{xifthen}
\usepackage{pdfpages}
\usepackage{transparent}
\usepackage{graphicx}
\graphicspath{ {./figures} }

\DeclarePairedDelimiter\set\{\}

\newcommand{\incfig}[1]{%
\def\svgwidth{\columnwidth}
\import{./figures/}{#1.pdf_tex}
}

\newtheorem*{theorem}{Теорема}
\newtheorem*{statement}{Утверждение}
\newtheorem*{lemma}{Лемма}
\newtheorem*{proposal}{Предложение}
\newtheorem*{consequence}{Следствие}

\theoremstyle{definition}
\newtheorem*{definition}{Определение}

\theoremstyle{remark}
\newtheorem*{remark}{Замечание}

\renewcommand\qedsymbol{$\blacksquare$}

\begin{document}
\title{УМФ. Лекция}
\date{29 Мая 2025 г.}
\maketitle

\section*{\centering Проекционные методы}
\subsection*{\centering Метод Ритца(-Бубнова-Галёркина)}
\begin{equation}
    \Delta u(x,y,z) = f(x,y,z)
\end{equation}
\begin{equation}
    u_{\Gamma} = 0
\end{equation}
\begin{equation}
  \begin{cases}
    I(u) \to \min \quad u \in K \\
    I(u) = \frac{1}{2} \iiint\limits_{\Omega} (u_x^2 + u_y^2 + u_z^2) dx dy dz -
    \iiint f u dxdydz
  \end{cases}
\end{equation}
\textbf{Задача 1}\\
Найти $ u \in K $ 
\[
    a(u,v) = l(v) \ : \ \forall v \in K
\]
\textbf{Задача 2}
\[
  \begin{cases}
    I(u) \to \min \quad u \in K \\
    I(u) = \frac{1}{2} a(u,u) - l(u)
  \end{cases}
\]
Задачи 1 и 2 эквивалентны

 
Выделим подпространство
\[
    K_N \subset K
\]
\[
  \begin{cases}
    I(\widetilde{u}) \to \min \quad \widetilde{u} \in K_N \\
    I(\widetilde{u}) = \frac{1}{2} a(\widetilde{u},\widetilde{u}) - l(\widetilde{u})
  \end{cases}
\]

Пусть $ \set{\varphi}_{i=1}^{N} $ - базис

Решение будем искать в виде:
\[
  \sum_{i=1}^{N} U_i \varphi_i(x), \quad \set{U_1, \dots, U_N}
\]
\[
    I(\widetilde{u}, \widetilde{u}) = \frac{1}{2} a\left(\sum_{i=1}^{N} U_i \varphi_i(x),
  \sum_{j=1}^{N} U_j \varphi_j(x)\right) - l\left(\sum_{k=1}^{N} U_k \varphi_k(x)\right)
  \to \min
\]
\[
    I(\widetilde{u}, \widetilde{u}) = \sum_{i=1}^{N} \sum_{j=1}^{N}
    \frac{a(\varphi_i, \varphi_j}{2} U_i U_j - \sum_{k=1}^{N} l(\varphi_k)
    U_k \to \min
\]

\[
    \mathcal{F}(U_1, \dots, U_N) \to \min
\]

\textbf{Необходимые условия минимума}
\[
  \frac{\partial \mathcal{F}}{\partial U_n} = 0, \quad n = 1, \dots, N
\]
\[
    \sum_{j=1}^{N} \frac{a(\varphi_n, \varphi_j)}{2} U_j + \sum_{i=1}^{N} 
  \frac{a(\varphi_i, \varphi_n)}{2} U_i - l(\varphi_n) = 0
\]
\[
    \sum_{i=1}^{N} a(\varphi_n, \varphi_i) U_i = l(\varphi_n), \quad n = 1, \dots, N
\]

\[
    a(u,u) = 0 \iff u \equiv 0
\]

\[
  (a(\varphi_n, \varphi_i)) \geq 0 \quad = 0 \iff a(\varphi_n, \varphi_i) \equiv 0
\]

Тогда система имеет единственное решение
\end{document}
