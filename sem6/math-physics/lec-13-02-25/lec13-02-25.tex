\documentclass[a4paper]{article}
\usepackage[a4paper,%
    text={180mm, 260mm},%
    left=15mm, top=15mm]{geometry}
\usepackage[utf8]{inputenc}
\usepackage{cmap}
\usepackage[english, russian]{babel}
\usepackage{indentfirst}
\usepackage{amssymb}
\usepackage{amsmath}
\usepackage{mathtools}
\usepackage{tcolorbox}
\usepackage{import}
\usepackage{xifthen}
\usepackage{pdfpages}
\usepackage{transparent}
\usepackage{graphicx}
\graphicspath{ {./figures} }

\newcommand{\incfig}[1]{%
\def\svgwidth{\columnwidth}
\import{./figures/}{#1.pdf_tex}
}

\begin{document}
\section*{\centering Колебания грубой мембраны. Функция Бесселя}

\begin{equation}
    u_{tt}(x,y,t) - \Delta u(x,y,t) = 0
\end{equation}
\[
    \Delta = \frac{\partial^{2}}{\partial x^2} + \frac{\partial^{2}}{\partial y^2}
\]
\begin{equation}
    u(x,y,t) |_{x^2 + y^2 = l^2} = 0
\end{equation}
\begin{equation}
    u |_{t=0} = u_0(x,y)
\end{equation}
\begin{equation}
    u_t |_{t=0} = u_1(x,y)
\end{equation}
\begin{equation}
    \begin{cases}
        x = r \cos \phi\\
        y = r\sin \phi
    \end{cases}
\end{equation}
\[
    0 < r \leq L \quad \phi \in [0; 2\pi]
\]
\[
    \Delta u(x,y,t) = \frac{1}{r} \frac{\partial}{\partial r} \left( r 
        \frac{\partial \widetilde{u}}{\partial r}\right) + \frac{1}{r^2} 
         \frac{\partial^2 \widetilde{u}}{\partial \phi^2}
\]

\subsection*{Задача о радиальных колебаниях}
\[
    u_0(x,y,t) \to \widetilde{u}_0(\rho, \phi, t)
\]
\[
    u_1(x,y,t) \to \widetilde{u}_1(\rho, \phi, t)
\]
\[
    \widetilde{u} |_{r=L} = 0
\]

Рассмотрим радиальные колебания.\\
Предположим, что начальные данные $ \widetilde{u_0}, \widetilde{u_1} $
не зависят от $ \phi $ 
\[
    \widetilde{u} \text{ не зависит от } \phi
\]

Обозначим:
\[
    \widetilde{u}(r, \phi, t) = v(r,t)
\]

\begin{equation}
    v_{t t}(r,t) - a^2 \frac{1}{r}\frac{\partial}{\partial r} ( r v_r(r,t)) = 0
\end{equation}
\begin{equation}
    v |_{t=0} = \phi(r)
\end{equation}
\begin{equation}
    v_t |_{t=0} = \psi(r)
\end{equation}
\begin{equation}
    v |_{r = L} = 0
\end{equation}

\begin{equation}
    |v|_{r=0} < \infty
\end{equation}

\begin{equation}
    v(r,t) = R(r) \cdot T(t)
\end{equation}

На первом этапе ищются все нетривиальные решения вида (11), удовлетворяющее
уравнению (6) и граничным условиям

\[
    T''(t)R(r) - a^2 T(t) \cdot \frac{1}{r} (r R'(r))' = 0 \ | \ : a^2 T(t) R(t)
\]
\[
    \frac{T''(t)}{a^2T(t)}  = \frac{\frac{1}{r} (r \cdot R'(r))'}{R(r)} 
    = - \lambda^2
\]

\begin{equation}
    T''(t) + a^2 \lambda^2 T(t) = 0
\end{equation}
\begin{equation}
    \frac{1}{r} (r R'(r))' + \lambda^2 R(r) = 0
\end{equation}

Подставим (9) и (10) в (11):
\[
    T(t)R(L) = 0
\]
\begin{equation}
    R(L) = 0
\end{equation}
\begin{equation}
    |R(L)| < \infty
\end{equation}

(13) - (15) - Задача Штурма-Луивилля

\[
    R''(r) + \frac{1}{r} R'(r) + \lambda^2 R(r) = 0
\]

Замена:
\[
    x = \lambda r
\]
\[
    R(r) = R \left(\frac{x}{\lambda} \right) = y(x) = y(\lambda r)
\]
\[
    R'(r) = y'(x) \cdot \lambda
\]
\[
    R''(r) = y''(x) \cdot \lambda^2
\]
\[
    y''(x) \lambda^2 + \frac{\lambda}{x} \lambda y'(x) + \lambda^2 y(x) = 0
\]
\begin{equation}
    y''(x) + \frac{1}{x} y'(x) + \left(1 - \frac{n^2}{x^2}\right)y(x) = 0
\end{equation}

Уравнение вида (16) называется уравнения Бесселя порядка n

Общее решение:
\[
    y(x) = C J_n(x) + D Y_n(x)
\]

$ J_n, Y_n $ - лнз решения уравнения (16)

$ J_n $ - функция Бесселя первого рода порядка n. $ Y_n $ - функция Бесселя второго рода порядка n

$ J_n, Y_n $ - образуют ФСР

\[
    |Y_n(0)| = \infty
\]
Функция Бесселя первого рода имеет счётное количество корней
\[
    J_n(\mu) = 0 \quad 0 < \mu_1^{(n)} < \mu_2^{(n)} < \dots
\]
\[
    \mu_k \xrightarrow[k \to \infty]{} + \infty
\]
\begin{equation}
    \int_{0}^{L} x J_n \left(\frac{\mu_i^{(n)} x}{4} \right) \cdot J_n
    \left(\frac{\mu_j^{(n)} x}{L} \right) dx = 
    \begin{cases}
        0, &\quad i \neq j\\
        \frac{L^2}{2} (J'_n(\mu_i^{(n)}))^2, &\quad i = j
    \end{cases}
\end{equation}
\[
    R(r) = y(\lambda r) = C J_0(\lambda r) + D Y_0(\lambda r)
\]

Из (15) следует $ D Y_0(\lambda r) = 0 $ 

Из (14):
\[
    C J_0(\lambda L) = 0
\]
\[
    J_0(\mu) = 0
\]
\[
    0 < \mu_1 < \mu_2 < \dots
\]
\begin{equation}
    \lambda_k = \frac{\mu_k}{L}, \quad k \in \mathbb{N}
\end{equation}

\begin{equation}
    R_k(r) = C_k J_0\left(\frac{\mu_k}{L} \right)
\end{equation}

\[
    T''_k(t) + \left( \frac{a\mu_k}{L} \right) T_k(t) = 0
\]
\[
    T_k(t) = A_k \cos \frac{a\mu_k}{L} t + B_k \sin \frac{a\mu_k}{L} t
\]
\[
    v_k(r,t) = T_k(t)R_k(r) = \left(A_k \cos \frac{a\mu_k}{L} t + B_k \sin \frac{a\mu_k}{L} t\right)
    J_0\left( \frac{\mu_k r}{L} \right)
\]

\end{document}
