\documentclass[a4paper]{article}
\usepackage[a4paper,%
    text={180mm, 260mm},%
    left=15mm, top=15mm]{geometry}
\usepackage[utf8]{inputenc}
\usepackage{cmap}
\usepackage[english, russian]{babel}
\usepackage{indentfirst}
\usepackage{amssymb}
\usepackage{amsmath}
\usepackage{mathtools}
\usepackage{tcolorbox}
\usepackage{graphicx}
\graphicspath{ {./figures} }

\begin{document}
\title{ТФКП. Лекция}
\author{Mr. Robot}
\maketitle

\[
    z = x +iy, \ \overline{z} = x - iy, \ z \overline{z} = x^2 + y^2
\]
\[
    x = \frac{z + \overline{z}}{2} , \ y = \frac{z - \overline{z}}{2i} 
\]
\[
    y = -i \frac{z - \overline{z}}{2} 
\]
\[
    A z \cdot \overline{z} + B \frac{z + \overline{z}}{2} -i C \frac{z - \overline{z}}{2} 
    + D = 0
\]
\[
    A \cdot z \cdot \overline{z} + \left(\frac{B}{2} -i \frac{C}{2}\right)z + 
    \left(\frac{B}{2} + i \frac{C}{2} \right) \overline{z} + D = 0
\]
\[
    A z \overline{z} + Mz + \overline{M} \overline{z} + D = 0
\]
\[
    A,D \in \mathbb{R} \quad M,\overline{M} \in \mathbb{C}
\]

\begin{tcolorbox}
\textbf{\underline{Th2}}(Круговое свойство дробно-линейного преобразования)\\
Всякое дробно-линейное преобразование $ w = \frac{az + b}{cz + d}  $ отображает
окружность в широком смысле на окружность в широком смысле(при этом обычная 
окружность может перейти в прямую и наоборот)

\textbf{\underline{Proof}} 
\[
    w = \frac{a}{c} + \frac{bc - ad}{c(cz+d)} 
\]
\[
\begin{aligned}
    &1) \  w_1 = cz + d\\
    &2) \  w_2 = \frac{1}{w_1} \\
    &3) \  w_3 = \frac{a}{c} + \frac{bc - ad}{c} \cdot w_2 \equiv w
\end{aligned}
\]
\[
    w = \frac{1}{z} \quad \overline{w} = \frac{1}{\overline{z}} 
\]
\[
    A \cdot \frac{1}{w} \cdot \frac{1}{\overline{w}} + M \frac{1}{w} + \overline{M}
    \frac{1}{\overline{w}} + D = 0
\]
\[
    D w \overline{w} + \overline{M} w + M \overline{w} + A = 0
\]
\end{tcolorbox}

\textbf{\underline{Note}} $ w = L(z) $ \\
$ \Gamma $ - окружность или прямая на плоскости (z)\\
$ C = L(\Gamma) $ 
\[
    |z - z_0| = R \not\ni z = \infty
\]
\[
    Bx + Cy + D = 0 \ni z = \infty
\]
\[
    z = \frac{d}{c} \rightarrow w = \infty
\]
\[
    z = -\frac{d}{c} \in \Gamma \implies w = L(-\frac{d}{c} ) = \infty \in C \implies
    C \text{ - прямая на (w)}
\]
\[
    z = - \frac{d}{c} \notin \Gamma \implies w = \infty \notin C \implies
    C \text{ - окружность на (w)}
\]

\textbf{\underline{Example}} 
\[
     w = \frac{z - 5}{z + 2i} \implies z = -2i 
\]
\[
    \Gamma_1: \ |z + 5| = 2
\]
\[
    \Gamma_2: \ |z - i| = 3
\]
\[
    |-2i + 5| \stackrel{?}{=} 2
\]
\[
    \sqrt{4 + 25} \neq 2
\]
\[
    z = -2i \notin \Gamma_1 \implies L(\Gamma_1) \text{ - окружность}
\]
\[
    |-2i - i| \stackrel{?}{=} 3
\]
\[
    3 = 3
\]
\[
    z = -2i \in \Gamma_2 \implies L(\Gamma_2) \text{ - прямая}
\]
\[
    z_1 = 4i \quad w_1 = \frac{4i -5}{6i} = \frac{-i(4i - 5)}{6} =
    \frac{2}{3} + \frac{5}{6} i
\]
\[
    z_2 = 3 + i
\]
\[
    w_2 = \frac{3 + i - 5}{3 + i + 2i} = \frac{-2 + i}{3 + 3i} 
    = \frac{1}{3} \frac{(-2+i)(1-i)}{2} = \frac{-2+1+i+2i}{6} =
    \frac{-1}{6} + \frac{i}{2} 
\]
\[
\begin{aligned}
    &z_1 = -3\\
    &z_2 = -7\\
    &z_3 = -5 + 2i
\end{aligned}
\]

\textbf{\underline{Note.2}} (Отображение круговых областей)
\[
    w = L(z)
\]
\[
    \Gamma - \text{ окр или прямая}
\]
\[
    D_1, D_2 - \text{ круговые области}
\]
\[
    C = L(\Gamma)
\]
\[
    G_1, G_2
\]
\[
    D_1 \to G_1 \lor G_2
\]
\[
    1. \text{ По внутренней точке}
\]
\[
    z_0 \in D_1
\]
\[
    L(z_0) = w_0
\]
\[
    w_0 \in G_1 \implies D_1 \to G_1
\]
\[
    w_0 \in G_2 \implies D_1 \to G_2
\]
\[
    w = \frac{z-5}{z+2i} = \frac{i -5}{3i} = \frac{-i(i-5)}{3} = \frac{1}{3} +
    \frac{5}{3} i
\]
\[
    2. \text{ По направлению обхода границы}
\]
\[
    z_1, z_2, z_3 - \text{ точки на границе области} \in \Gamma = \partial D_1
\]
\[
    w_1 = L(z_1) \subset \partial(L(D_1))
\]
\[
    w_2 = L(z_2) \subset \partial(L(D_1))
\]
\[
    w_3 = L(z_3) \subset \partial(L(D_1))
\]
Пусть для определённости $ D_1 $ остаётся справо\\
$ D_1 $ перейдёт в ту из областей $ G_1 \lor G_2 $, которая будет лежать справа
при движении по границе $ \partial(L(D_1)) $ 

\begin{tcolorbox}
\textbf{\underline{Th.3}}(Единственность дробно-линейного преобразования)\\
$ w = L(z) $ однозначно определяется заданием 3-х различных точек
\[
    L(z_k) = w_k, \ k = 1,2,3
\]

\textbf{\underline{Proof}}\\
1.
\[
    w_1 = \frac{\alpha z + \beta}{z + \gamma} 
\]
\[
    L(z_k) = w_k, \ k= 1,2,3
\]
2.
\[
    z = z_0, \ w = w_0 \quad w_0 = L(z_0)
\]
\[
    w_k = L(z_k), \ k= \overline{0,3}
\]
\[
    w_k - w_l = \frac{(ad - bc)(z_k - z_l)}{(cz_k + d)(cz_l +d}  \quad k,l = \overline{0,3}
\]
Отбросив индекс 0 получим:
\begin{equation}
    \frac{w - w_1}{w - w_2} : \frac{w_3 - w_1}{w_3-w_2} = 
    \frac{z - z_1}{z - z_2} : \frac{z_3 - z_1}{z_3-z_2} \implies w = \dots
    \label{eq:1}
\end{equation}
\end{tcolorbox}

\begin{tcolorbox}
\textbf{\underline{Def}} Выражение стоящее справа (слева) в формуле $ (\ref{eq:1}) $,
называется двойным или ангармоническим отношением четырёх точек

Это отношение - инвариант любого др-линейного отображения
\end{tcolorbox}

\underline{Note} 
\[
    z_k, w_k - \text{конечны}, \ k = 1,2,3
\]
Если какая-либо из $ z_k = \infty $ или $ w_l = \infty $, то в формуле $ (\ref{eq:1}) $ 
соответ. разницу надо заменить на 1

\underline{Ex}
\[
    z_1 = i, z_2 = 1, z_3 = \infty
\]
\[
    w_1 = \infty, w_2 = -1, w_3 = 2i
\]
\[
    \frac{w - \infty}{w + 1} : \frac{2i - \infty}{zi + 1} = 
    \frac{z - i}{z - 1} : \frac{\infty - i}{\infty - 1} 
\]
\[
    \frac{2i + 1}{w + 1} = \frac{z - i}{z - 1} 
\]

\textbf{\underline{Def}} Неподвижными точками отображения $ w = \Phi(z) $ называются
точки, переходящие сами в себя, т.е. $ z = \Phi(z) $ 

\begin{tcolorbox}
\underline{Th.4} $ w = L(z)\ (w \not\equiv z) $ имеет две неподвижные точки (в
частности кратную неподвижную точку)

\underline{Proof}
\[
    z = \frac{az + b}{cz + d} 
\]
\[
    cz^2 + (d-a)\cdot z - b = 0
\]
\[
    1) \text{ корней} > 2 \implies \exists \text{ бесконечно много решений}
\]
\[
    c = 0, \ d-a = 0, \ b = 0
\]
\[
    w = \frac{az}{a} = z
\]
\[
    2) \ c \neq 0 \implies \text{2 корня или корень кратности 2}
\]
\[
    3) \ c = 0, \ d - a \neq 0 \implies \text{1 корень}
\]
\end{tcolorbox}

\end{document}
