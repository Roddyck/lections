\documentclass[a4paper]{article}
\usepackage[a4paper,%
    text={180mm, 260mm},%
    left=15mm, top=15mm]{geometry}
\usepackage[utf8]{inputenc}
\usepackage{cmap}
\usepackage[english, russian]{babel}
\usepackage{indentfirst}
\usepackage{amssymb}
\usepackage{amsmath}
\usepackage{mathtools}
\usepackage{tcolorbox}
\usepackage{graphicx}
\graphicspath{ {./figures} }

\begin{document}
\title{Комплан. Лекция}
\author{Хуй}
\maketitle
\section*{\centering Линейная функция}
\[
    w = az + b, \; a,b \in \mathbb{C}
    \text{ - линейная ф-ия}
\]
\[
    w' = a \neq 0 \text{ конф на всей комп пл-ти}
\]
1. Преобразование переноса (сдвига)
\[
    w = z + c, \; c = c_1 + i c_2
\]
\[
    z = x + iy, \; w = u + iv
\]
\[
    u + iv = x + iy + c_1 + i c_2
\]
\[
    \begin{cases}
        u  = x + c_1\\
        v = y + c_2
    \end{cases}
\]
2. Преобразование поворота (вращения)
\[
    w = z \cdot e^{i\alpha}, \; \alpha \in \mathbb{R}
\]
\[
    \alpha > 0 \text{ против}
\]
\[
    \alpha < 0 \text{ по}
\]
\[
    z = r e^{i\phi}
\]
\[
    w = r e^{i(\phi + \alpha)}
\]
\[
    |w| = r = |z| \quad \arg w = \phi + \alpha = \arg z + \alpha
\]
\[
    e^{i(\phi + \alpha)} = r \cos(\phi  +\alpha)  + i \sin (\phi + \alpha)
\]
\[
    \begin{cases}
        u = r \cos(\phi + \alpha) \\
        v = r \sin(\phi + \alpha) \\
    \end{cases}
\]
3. Преобразование подобия (гомотетия)
\[
    w = k z, \; k > 0
\]
\[
    w = k \cdot r e^{i\phi}, \; |w| = k |z|, \; \arg w = \phi = \arg z
\]
\[
    k > 1 \text{ - растяжение}, \; k < 1 \text{ - сжатие}
\]
\[
    w = az + b, \; a = k e^{i\alpha}, \; k = |a| > 0,\; \alpha = \arg a
\]
\[
    w = k e^{i\alpha}z + b
\]
\[
    1) \; w_1 = z e^{i\alpha}
\]
\[
    2) \; w_2 = w_1 \cdot k
\]
\[
    3) \; w_3 \equiv w = w_2 + b
\]
\subsection*{\centering Дробно-линейное преобразование}
\[
    w = \frac{az + b}{cz + d} \equiv L(z)
\]
\[
    \delta = \begin{vmatrix}
    a & b\\
    c & d\\
    
    \end{vmatrix}
    \neq 0    
\]
\[
    \delta = 0 = ad - bc \implies ad = bc
\]
\[
    \frac{a}{c} = \frac{b}{d} = \lambda \implies a = \lambda b,\; b = \lambda d
\]
\[
    w = \frac{\lambda cz + \lambda d}{c z + d} \equiv \lambda
\]
1 Пусть c = 0, $ a \neq 0, \; d \neq 0 $ 
\[
    w = \frac{a}{d} z  + \frac{b}{d} \equiv a_1 z + b_1
\]

\underline{Def} Точки $ z_1, \, z_2 $ называются симметричными относительно 
$ C: |z - z_0| = R $, если \\
1) Лежат на одном луче, исходящем из центра окружности\\
2) Произведение их расстояний от центра равно квадрату радиуса. $ |z_1 - z_0| \cdot
|z_2 - z_0| = R^2$ \\
$ |z_1 - z_0| < R \implies |z_2 - z_0| > R $ \\
$ z_1 \rightarrow z_0 \implies z_2 \rightarrow \infty $ \\
$ z_1 = z_0, \; z_2 = \infty $ 

\underline{Def} Преобразование переводящие точки $ z $ в симметричные с ними
относительно окружности C точки $ \xi $, называется симметрией (инверсией)
относительно окружности

4.
\[
     w  = \frac{R^2}{z} 
\]
\[
    z = |z| e^{i\phi}, \; w = |w| e^{i\theta}
\]
\[
    |w| e^{i\theta} = \frac{R^2}{|z|} e^{i\phi}
\]
\[
    |w| = \frac{R^2}{|z|} \implies |z||w| = R^2
\]
\[
    \theta = -\phi
\]
\[
    1) \; w_1 = \frac{R^2}{\overline{z}} \quad \overline{z} = |z| e^{-i\phi}
\]
\[
    w_1 = \frac{R^2}{|z|} e^{i\phi} \iff |w_1| |z| = R^2, \; \arg w_1 = \phi = \arg z
\]
$ w_1 $ - симметрия. $ |z| = R $ 
\[
    2) \; w = \overline{w_1}
\]
\[
    w = \frac{R^2}{z} \text{ - симметрия отн} \; |z| = R \text{ с последующем
    зеркальным отражением относительно действ. оси}
\]

\begin{tcolorbox}
\underline{Th} (Конформность дробно-линейного преобразования)\\
Дробно-линейное преобразование $ w = \frac{az + b}{cz + d}\; (\delta \neq 0) $ 
отображает конформно расширенную компл. пл-ть z на расш. комп. пл-ть w
(др. линейное преобр. конформно на всей расш компл. пл-ти)

\underline{Proof}

1. Взаимнооднозначность\\
1) $ c \neq 0 $ 
\[
    \forall \, z = -\frac{d}{c} \text{ соотв } !\; w \neq \infty
\]
\[
    w(cz+d) = az + b
\]
\[
    wcz - az = b -wd
\]
\[
    z = \frac{b - wd}{wc-a} 
\]
\[
    \forall \; w \neq \frac{a}{c} \text{ соотв ! } z \neq \infty
\]
\[
    L(-\frac{d}{c}) = \infty, \; L(\infty) = \frac{a}{c} 
\]
\[
    \mathbb{\overline{C}}_{z} \stackrel{\text{вз-одн}}{\leftrightarrow}
    \mathbb{\overline{C}}_{w}
\]

2) $ c = 0 $ 
\[
    w = a_1 z + b_1, \; a_1 \neq 0
\]
\[
    \forall z \neq \infty \text{ соотв ! } w \neq \infty
\]
\[
    z = \frac{w}{a_1} - \frac{b_1}{a_1} 
\]
\[
    \forall w \neq \infty \text{ соотв !} z \neq \infty
\]
\[
    L(\infty) = \infty
\]

2. Конформность\\
1) $ c \neq 0 $ 
\[
    \frac{\partial w}{\partial z} = \frac{a(cz+d) - c(az+b)}{(cz+d)^2} 
    \frac{ab - bc}{(cz+d)^2} 
\]
\begin{align*}
    \left[
    \begin{array}{ll}
        \neq 0, &\quad z \neq \infty \\
        \neq \infty, &\quad z \neq -\frac{d}{c} 
    \end{array}
    \right .
    \end{align*}

a) $ z = \infty $ 
\[
    z = \frac{1}{t}, \; z = \infty \rightarrow t = 0
\]
\[
    w(t) = \frac{a \frac{1}{t} + b}{c \frac{1}{t} + d} 
\]
\[
    w(t) = \frac{a + bt}{c + ^{}} 
\]
\[
    \frac{\partial w}{\partial t} |_{t=0} \frac{a}{c} \neq 0
\]

b) $ z = \frac{-d}{c} \rightarrow w = \infty $ 
\[
    w = \frac{1}{\zeta} 
\]
\[
    \frac{1}{\zeta} = \frac{az + b}{cz + d} \implies \zeta
\]
\[
    \frac{\partial \zeta}{\partial z} |_{z = \frac{-d}{c} } \neq 0
\]
\end{tcolorbox}

\begin{tcolorbox}
2) $ c = 0 $ 
\[
    w = a_1 z + b_1 \quad \frac{dw}{dz} = a_1 \neq 0
\]
\[
    L(\infty) = \infty
\]
\[
    z = \frac{1}{t}, \; w = \frac{1}{\zeta} 
\]
\[
    z = \infty \rightarrow t = 0, w = \infty \rightarrow \zeta = 0
\]
\[
    \frac{d\zeta}{dt} |_{t=0} \neq 0
\]
\end{tcolorbox}

\underline{Def} Окружность на $ \mathbb{\overline{C}} $ или окружность в широком
смысле называется всякая кривая определяемая уравнением:
\[
    A(x^2 + y^2) + Bx + Cy + D = 0,\; A,B,C,D \in \mathbb{R}
\]
\[
    A = 0 \text{ - прямая }, \; A \neq 0 \text{ - окружность}
\]
\end{document}
