\documentclass[a4paper]{article}
\usepackage[a4paper,%
    text={180mm, 260mm},%
    left=15mm, top=15mm]{geometry}
\usepackage[utf8]{inputenc}
\usepackage{cmap}
\usepackage[english, russian]{babel}
\usepackage{indentfirst}
\usepackage{amssymb}
\usepackage{amsmath}
\usepackage{mathtools}
\usepackage{tcolorbox}
\usepackage{import}
\usepackage{xifthen}
\usepackage{pdfpages}
\usepackage{transparent}
\usepackage{graphicx}
\graphicspath{ {./figures} }

\newcommand{\incfig}[1]{%
\def\svgwidth{\columnwidth}
\import{./figures/}{#1.pdf_tex}
}

\begin{document}
\title{Комплан. Лекция}
\maketitle

\[
    0 < \phi < \frac{2\pi}{n} 
\]

\begin{figure}[ht]
    \centering
    \incfig{pic1}
    \caption{pic1}
    \label{fig:pic1}
\end{figure}

\[
    \forall \gamma_k: \quad k \cdot \frac{2\pi}{n} < \phi < (k+1)\frac{2\pi}{n} 
    , \ k = \overline{1, n-1}
\]
Плоскость (w), в которой каждому $ \theta = Arg w $ об-ся\\
$ T_k $ (лист, копия комплекской плоскости)\\
\[
    T = T_0 \cup T_1 \cup \dots \cup T_{n-1} - \text{ поверхность Римана}
\]

Для $ w = z^{n} $ - n - листная в (z) каждой из $ \gamma_k $ - область однолистонсти
$ z = \sqrt[n]{n}  $ 
\end{document}
