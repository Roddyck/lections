\documentclass[a4paper]{article}
\usepackage[a4paper,%
    text={180mm, 260mm},%
    left=15mm, top=15mm]{geometry}
\usepackage[utf8]{inputenc}
\usepackage{cmap}
\usepackage[english, russian]{babel}
\usepackage{indentfirst}
\usepackage{amssymb}
\usepackage{amsmath}
\usepackage{mathtools}
\usepackage{tcolorbox}
\usepackage{import}
\usepackage{xifthen}
\usepackage{pdfpages}
\usepackage{transparent}
\usepackage{graphicx}
\graphicspath{ {./figures} }

\newcommand{\incfig}[1]{%
\def\svgwidth{\columnwidth}
\import{./figures/}{#1.pdf_tex}
}

\begin{document}
\title{Комплан. Лекция}
\author{Родион Иващенко - ХУЙ}
\maketitle

\[
    \int_{C} f(z)dz = \lim_{\lambda \to 0} \sum_{k=1}^{n} f(\zeta_k) \Delta z_k
    \quad \zeta_k = \xi_k + i \eta_k
\]
f(z) непр на C
\[
    f(\zeta_k) = u(\xi_k, \eta_k) + iv(\xi_k, \eta_k) = u_k + iv_k
\]
\[
    \sigma = \sum_{k=1}^{n} (u_k + i v_k) \Delta z_k = \sum_{k=1}^{n} (u_k \Delta x_k - v_k 
    \Delta y_k) + i \sum_{k=1}^{n} (u_k \Delta y_k + v_k \Delta x_k)
\]
\[
    \int_{C}f(z)dz = \int_{C} u(x,y)dx - v(x,y)dy + i\left(\int_{C}u(x,y)dy + v(x,y)dx\right)
\]
\[
    f(z) = u + iv \quad dz = dx + idy
\]
Св-ва:\\
1. Ориентированность
\[
    \int_{C^{-}} f(z)dz = -\int_{C^{-}} f(z)dz
\]
2. Линейность
\[
    \int_{C} \left( \sum_{k=1}^{n} A_k f_k(z) \right) dz = \sum_{k=1}^{n} A_k
    \int_{C} f_k(z) dz
\]
3. Аддитивность по контуру
\[
    \int_{C_1 \cup C_2} = \int_{C_1}f(z)dz + \int_{C_2}f(z)dz
\]
4. Оценка модуля 
\[
    \Delta s_k \text{ - длина дуги}
\]
\[
    |\Delta z_k| \leq \Delta s_k
\]
\[
    |\sum_{k=1}^{n} f(\zeta_k) \Delta z_k| \leq \sum_{k=1}^{n} |f(\zeta_k)| 
    \Delta s_k
\]
\[
    \left| \int_{C} f(z)dz \right| \leq\int_{C} |f(z)|ds
\]
\underline{След}
\[
    |f(z)| \leq M = const\  \forall z \in C \implies \left| \int_{C} f(z) dz \right| \leq
    M \cdot \text{дл С}
\]
\[
    C: z = z(t) = x(t) + iy(t), \quad t \in [\alpha, \beta]
\]
C - непрерывная дифф. (в част, гладкая) кривая. $ x(t), \ y(t) $ - непр диф
\[
    \int_C P(x,y)dx = \int_{\alpha}^{\beta} P(x(t), y(t)) x'(t)dt
\]
\[
    \int_C f(z) dz = \int_{\alpha}^{\beta} ( u(x(t), y(t)) + iv(x(t), y(t)) )
    \cdot (x'(t) + iy'(t)) dt = \int_{\alpha}^{\beta} f(z(t)) \cdot z'(t) dt
\]
\[
    C: z = z(t) \quad t \in [\alpha, \beta]
\]
\[
    \int_C f(z) dz = \int_{\alpha}^{\beta} f(z(t)) \cdot z'(t) dt
\]

\underline{\textbf{Example}}
\[
    C = [0, 2\pi] \quad \int_C = e^{iz}dz = \int_{0}^{2\pi} e^{ix}dx = 
    \int_{0}^{2\pi} ( \cos x + i \sin x) dx = 0
\]
\[
    z = x \quad x \in [0, 2\pi]
\]
\underline{Note} $ \int_{a}^{b} f(z) dz = f(c) \cdot (b-a) \quad e^{ic} \neq 0 $ 

\section*{\centering Два важных примера}
\underline{Задача 1}
\[
    \int_C 1 \cdot dz = \lim_{\lambda \to 0} \sum_{k=1}^{n} 1 \cdot (z_k - z_{k-1})
    = \lim_{\lambda \to 0} (z_n - z_0) = b-a
\]
\[
    \int_C 1 \cdot dz = b - a
\]
\[
    \int_a^{b} 1 \cdot dz = b - a
\]
В частности $ a = b $
\[
    \oint_C 1 \cdot dz = 0
\]
\underline{Задача 2}
\[
    C: \ |z - z_0| = \rho
\]
\[
    \int_C (z - z_0)^{n} dz = \int_{0}^{2\pi} \rho^{n} e^{i n \phi} \cdot \rho
    i e^{i\phi}d\phi = i \rho^{n+1} \int_{0}^{2\pi} e^{i(n+1)\phi} d\phi
\]
\[
    \begin{aligned}
        &1) \ n + 1 = 0 \\
        &\int_C \frac{1}{z - z_0} dz = i \int_{0}^{2\pi} d\phi = 2\pi i\\
        &2) \ n + 1 \neq 0\\
        &\int_{0}^{2\pi} e^{i(n+1)\phi} d\phi = \int_{0}^{2\pi} ( \cos(n+1)\phi +i
        \sin(n+1)\phi) d\phi = 0
    \end{aligned}
\]
\[
    z - z_0 = \rho \cdot e^{i\phi}, \ 0 \leq \phi \leq 2\pi
\]
\[
    dz = \rho i e^{i\phi}d\phi
\]
\[
    \oint_{|z - z_0| = \rho} (z - z_0)^{n}dz = 
    \begin{cases}
        2\pi i, &\quad n = -1\\
        0, &\quad n \neq -1
    \end{cases}
\]

\section*{\centering Интегральная теорема Коши}

1) Простую замкнутую гладкую (кусочно-гладкую) кривую L - замкнутый контур
\[
    Int L \equiv I(L)
\]
\[
    Ext L \equiv E(L)
\]
\[
    \overline{I(L)} = I(L) \cup L
\]

2) Односвязность области D $ \equiv $ односвязность отн $ \mathbb{C} $ 
\[
    \forall \ L \subset D \implies \overline{I(L)} \subset D
\]
\[
    L \text{ - замкнутый контур}
\]

\begin{tcolorbox}
\underline{\textbf{Th Коши}} Пусть D - односвязная область $ D \subset \mathbb{C} $ и 
$ f(z) $ - однозначная аналит ф-ия\\
Тогда интеграл от неё по любой замкнутой спрямляемой кривой $ L \subset D $ 
равен 0: $ \oint_L f(z) dz = 0 $ ( Интеграл от регулярной в односвязной области D 
функциии f(z) не зависит от пути интегрирования, а зависит только от начальной и 
конечной точки)

\underline{Proof} Допустим предположение $f(z)$ и $ f'(z) $ непр. в D
\[
    \oint_L f(z) dz = \oint_L (u + iv)(dx + idy) = \oint_L udx - vdy + i 
    \oint vdx + udy =  
\]
\[
    \iint_{\overline{I(L)}} \left(- \frac{\partial v}{\partial x} -
    \frac{\partial u}{\partial y} \right) dxdy + i
    \iint_{\overline{I(L)}} \left( \frac{\partial u}{\partial x} - \frac{\partial v}{\partial y} 
        \right) dxdy = 0
\]
\end{tcolorbox}

\begin{tcolorbox}
\underline{Th Коши} (для многосвязной области)\\
Пусть $ f(z) \in H(\overline{D}) $ $ \overline{D} $ ограничена n+1 замкнутым
контуром: внешним контуром $ \Gamma_0 $ и внутр $ \Gamma_1, \dots, \Gamma_n $ \\
Тогда $ \int_L f(z)dz = 0 $ 
\end{tcolorbox}
\begin{figure}[ht]
    \centering
    \incfig{pic2}
    \caption{pic2}
    \label{fig:pic2}
\end{figure}
\end{document}
