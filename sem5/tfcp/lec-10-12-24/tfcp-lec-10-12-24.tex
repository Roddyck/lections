\documentclass[a4paper]{article}
\usepackage[a4paper,%
    text={180mm, 260mm},%
    left=15mm, top=15mm]{geometry}
\usepackage[utf8]{inputenc}
\usepackage{cmap}
\usepackage[english, russian]{babel}
\usepackage{indentfirst}
\usepackage{amssymb}
\usepackage{amsmath}
\usepackage{mathtools}
\usepackage{tcolorbox}
\usepackage{import}
\usepackage{xifthen}
\usepackage{pdfpages}
\usepackage{transparent}
\usepackage{graphicx}
\graphicspath{ {./figures} }

\newcommand{\incfig}[1]{%
\def\svgwidth{\columnwidth}
\import{./figures/}{#1.pdf_tex}
}

\begin{document}
\title{Комплан. Лекция}
\author{dajflsadf}
\maketitle

\section*{\centering Ряды Лорана}

\begin{tcolorbox}
\underline{Th} (о !)\\
Ф-ия $ f(z) \in H(z_0) $ ! образом разлагается в ряд Тейлора по $ (z - z_0) $  

\underline{Proof}
\[
    f(z) = \sum_{n=0}^{\infty} c_n (z - z_0)^{n}, \ c_n = \frac{f^{(n)(z_0)}}{n!} 
\]
\[
    f(z) = \sum_{n=0}^{\infty} d_n (z - z_0)^{n}, \ d_n = \frac{f^{(n)}(z_0)}{n!} 
\]
\[
    \implies d_n = c_n \implies !
\]
\end{tcolorbox}
\[
    e^{z} = \sum_{n=0}^{\infty} \frac{z^{n}}{n!} 
\]
\[
    \sin z = 
\]
\[
    \cos z = 
\]
\[
    \ln (1 + z) = \sum_{n=0}^{\infty} (-1)^{n-1} \frac{z^{n}}{n} 
\]
\[
    (1 + z)^{\mu} = 
\]

\begin{tcolorbox}
\underline{Def} Рядом Лорана по степеням $ (z - z_0) $ называется ряд вида
\begin{equation}
    f(z) = \sum_{n=-\infty}^{+\infty} c_n \cdot (z - z_0)^{n}
    = \sum_{n=-\infty}^{-1} c_n \cdot (z - z_0)^{n} + \sum_{n=0}^{\infty} 
    c_n \cdot (z - z_0)^{n}
    = \sum_{k=1}^{+\infty} c_{-k} \cdot \frac{1}{(z - z_0)^{k}} + \sum_{n=0}^{\infty} 
    c_n \cdot (z - z_0)^{n}
    \label{eq:1}
\end{equation}
\[
    f(z) = f_1(z) + f_2(z)
\]
\[
    f_1(z) \text{ - главная часть ряда Лорана (сингулярная)}
\]
\[
    f_2(z) \text{ - правильная часть ряда Лорана (регулярная)}
\]
\end{tcolorbox}

Выясним область сходимости ряда Лорана \\
$1) \ f_2(z)$ - регуляр
\[
    |z - z_0| < R = \frac{1}{\overline{\lim}_{n \to \infty} \sqrt[n]{|c_n|}  } 
\]
2) $ f_1(z) $ - регулярна
\[
    \frac{1}{z - z_0}  = \xi
\]
\[
    |\xi| = \left| \frac{1}{z - z_0} \right| < r_1 = \frac{1}{\overline{\lim}
    _{k \to \infty} \sqrt[k]{|c_{-k}|}  } 
\]
\[
    |z - z_0| > r = \overline{\lim}_{k \to \infty} \sqrt[k]{|c_{-k}|}   
\]
\[
    r \geq R \implies (\ref{eq:1}) \text{ не сходится ни в какой области}
\]

Будем считать, что $ r < R $ 

\begin{tcolorbox}
1) Области сходимости $ (\ref{eq:1}) $ есть круговое кольцо $ 0 \leq r < |z - z_0| < R
\leq \infty$  
\[
    R = \frac{1}{\overline{\lim}_{n \to \infty} \sqrt[n]{|c_n|}  }
\]
\[
    r = \overline{\lim}_{k \to \infty} \sqrt[k]{|c_{-k}|}
\]

2) В этом кольце ряд $ (\ref{eq:1}) $ сходится абсолютно, а внутри этого кольца
равномерно

3) В этом кольце ряд Лорана $ (\ref{eq:1}) $ определяет однозначную аналитическую
функцию
\end{tcolorbox}

\subsection*{\centering Частные случаи}

\begin{tcolorbox}
1) $ c_{-k} = 0, \ k = 1, 2, \dots $. $ f(z) = f_2(z) $\\
Ряд Тейлора частный случай ряда Лорана

2) $ r = 0 \quad \exists c_{-k} \neq 0 $\\
Ряд Лорана сходится в кольце $ 0 < |z-z_0| < R $ \\
$ (\ref{eq:1}) $ - ряд Лорана в окр-ти т $ z_0 $ 

3) $ z_0 = 0, \ R = \infty $. $ r < |z| < \infty $  
\[
    \sum_{n=-\infty}^{\infty} c_n \cdot z^{n}
\]
Ряд $ (\ref{eq:1}) $ - ряд Лорана в окрестности бесконечности

Если $ r = 0 $. $ 0 < |z| < \infty $\\
Ряд Лорана сходится в окрестности нуля и бесконечности
\end{tcolorbox}

\subsection*{\centering Вычесление коэф-тов}

\[
    \gamma_{\rho}: \ |z - z_0| = \rho, \ r < \rho < R
\]
\[
    (\ref{eq:1}) \cdot \frac{1}{2 \pi i} \frac{1}{(z - z_0)^{k+1}}, \ k \in \mathbb{Z}
\]

На $ \gamma_\rho $ ряд $ (\ref{eq:1}) $ сходится равномерно
\[
    \frac{1}{2 \pi i} \int_{\gamma_\rho} \frac{f(z)}{(z - z_0)^{k+1}} dz =
    \sum_{n=-\infty}^{\infty} \frac{c_n}{2 \pi i} \int_{\gamma_\rho}
    (z - z_0)^{n-(k+1)} dz = \begin{cases}
        0, \ n \neq k\\
        2 \pi i, \ n - k - 1 = -1
    \end{cases}
\]
\[
    c_k = \frac{1}{2 \pi i} \int_{\gamma_\rho} \frac{f(z)}{(z - z_0)^{k+1}} dz
\]
\[
    c_n = \frac{1}{2 \pi i} \int_{\gamma_\rho} \frac{f(t)}{(t - z_0)^{n+1}} dt, \
    n = 0, \pm 1, \pm 2, \dots
\]

\begin{tcolorbox}
\underline{Th} (Лорана)

1) Ф-ия $ f(z) $ регулярная в кольце $ r < |z - z_0| < R $ представима в этом кольце
сходящимся рядом Лорана $ (\ref{eq:1}) $ по степеням $ (z - z_0) $ 

2) В данном кольце ряд $ (\ref{eq:1}) $ единственный

\underline{Proof}

1) В данном кольце фиксируем точку z
\[
    c_1: \ |t - z_0| = \rho_1
\]
\[
    c_2: \ |t - z_0| = \rho_2
\]
\[
    r < \rho_1 < |z - z_0| < \rho_2 < R
\]

По фор-ле Коши для многосвязной области:
\[
    f(z) = \frac{1}{2 \pi i} \int_{c_2 \cup c_1^{-}} \frac{f(t)dt}{t - z} = 
    \underbrace{\frac{1}{2 \pi i} \int_{c_2}\frac{f(t)dt}{t - z}}_{\phi_2(z)}
    \underbrace{- \frac{1}{2 \pi i} \int_{c_1}\frac{f(t)dt}{t - z}}_{\phi_1(z)}
    = \phi_2(z) + \phi_1(z)
\]

a) $ \phi_2(z) $ 
\[
    \frac{1}{t - z} \text{ - разложим по пол степеням }(z - z_0)
\]
\[
    \frac{1}{t - z} = \frac{1}{t - z + z_0 - z_0} = 
    \frac{1}{(t - z_0) - (z - z_0)} = \frac{1}{(t - z_0)} \cdot 
    \frac{1}{1 - \frac{z - z_0}{t - z_0}} 
\]
\[
    \frac{1}{1 - z} = \sum_{n=0}^{\infty} z^{n}
\]
\[
    \phi_2(z) = \frac{1}{2 \pi i} \int_{c_2} \frac{f(z)dt}{t - z} =
    \sum_{n=0}^{\infty} \frac{1}{2 \pi i} \int_{c_2}\frac{f(t)\cdot(z -z_0)^{n}}
    {(t-z_0)^{n+1}} dt
\]
\[
    \phi_2(z) = \sum_{n=0}^{\infty} \frac{1}{2 \pi i} \int_{c_2} \frac{f(t)dt}{(t - z_0)^{n+1}} 
    \cdot (z - z_0)^{n}
\]

b) $ \phi_1(z) $ разложим по отрицательным степеням $ (z - z_0) $ 
\[
    -\frac{1}{t-z} = \frac{1}{z-t} = \frac{1}{(z - z_0)-(t-z_0)} =
    \frac{1}{z - z_0} \cdot \frac{1}{1 - \frac{t-z_0}{z-z_0}} =
    \sum_{n=0}^{\infty} \frac{(t-z_0)^{n}}{(z-z_0)^{n+1}} = 
    \sum_{k=1}^{\infty} \frac{(t-z_0)^{k-1}}{(z-z_0)^{k}} 
\]

По признаку Вейерштрасса полученный ряд сходится равномерно
\[
    \left| \frac{(t-z_0)^{n}}{(z-z_0)^{n+1}} \right|  =\frac{1}{|z-z_0|} \cdot
    \left| \frac{t - z_0}{z - z_0} \right| = \frac{1}{|z - z_0|} \cdot
    \frac{\rho_1^{n}}{|z-z_0|^{n}} = \frac{1}{|z-z_0|} \cdot q^{n} \quad q < 1
\]
\[
    \times \frac{1}{2 \pi i} \cdot f(t) \land \int_{c_1}
\]
\[
    \phi_1(z) = -\frac{1}{2 \pi i} \int_{c_1}f(t)dt = \sum_{k=1}^{\infty} 
    \frac{1}{2 \pi i} \int_{c_1} \frac{f(t)dt}{(t-z_0)^{-k+1}} \cdot \frac{1}
    {(z - z_0)^{k}}  
\]
\[
    f(z) = \sum_{k=1}^{\infty} \frac{c_{-k}}{(z-z_0)^{k}} + \sum_{n=0}^{\infty} 
    c_n \cdot (z - z_0)^{n} = \sum_{n=-\infty}^{\infty} c_n (z - z_0)^{n}
\]
\[
    \forall z: \ r < |z-z_0| < R
\]

На $ |z-z_0| = r, \ |z-z_0| = R $ имеется хотя бы по одной особой точки функции
$ f(z) $ 

2) Т.к. коэф-ты ряда Лорана однозначно определяются по соотв-им формулам, то
разложение в ряд Лорана единственно
\end{tcolorbox}

\begin{tcolorbox}
\underline{Note} $ f(z) $ регулярна в кольце $ r < |z - z_0| < R $ 
\[
    f(z) = f_1(z) + f_2(z)
\]
\[
    f(z) = f_1(z) \cdot f_2(z)
\]

Где $ f_1(z) $ регулярна в $ |z-z_0| > r $, $ f_2(z) $ регулярна в $ |z-z_0| < R $ \\
$ f_1(z)$ разлагают по отрицательным степеням, а $ f_2(z) $ по положительным
\end{tcolorbox}

\section*{\centering Классификация изолированных особых точек однозначных аналитических
функций}

\begin{tcolorbox}
\underline{Def} Точка $ z = z_0 $ называется изолированной особой точкой
однозначного характера для функции $ f(z) $, если $ f(z) $ является однозначной
аналитической в некоторой окрестности точки $ z_0 $ т.е. в $ 0 < |z - z_0| < R $ 
\end{tcolorbox}

\end{document}
