\documentclass[a4paper]{article}
\usepackage[a4paper,%
    text={180mm, 260mm},%
    left=15mm, top=15mm]{geometry}
\usepackage[utf8]{inputenc}
\usepackage{cmap}
\usepackage[english, russian]{babel}
\usepackage{indentfirst}
\usepackage{amssymb}
\usepackage{amsmath}
\usepackage{mathtools}
\usepackage{tcolorbox}
\usepackage{import}
\usepackage{xifthen}
\usepackage{pdfpages}
\usepackage{transparent}
\usepackage{graphicx}
\graphicspath{ {./figures} }

\newcommand{\incfig}[1]{%
\def\svgwidth{\columnwidth}
\import{./figures/}{#1.pdf_tex}
}

\begin{document}
\title{ТФКП. Лекция}
\author{i dont need this}
\maketitle

В основу классификации изолированных точек однозначного характера
ложится разложения ф-ии в ряд Лорана
\[
    f(z) = \sum_{n=-\infty}^{\infty} c_n(z-z_0)^{n}
\]

1. Ряд Лорана не содержит членов с отрицательными степенями.\\
$ z_0 $ - устранимая особая точка (правильная)

2. Ряд Лорана содержит конечное число членов с отрицательными степенями\\
$ z_0 $ - полюс

3. Ряд Лорана содержит бесконечно много членов с отрицательными степенями\\
$ z_0 $ - существенно особая точка

\[
    1. \ f(z) = c_0 + c_1 (z - z_0) + \dots + c_n (z - z_0)^{n} + \dots
\]
\[
    |z-z_0| < R
\]
\[
    f(z_0) = \lim_{z \to z_0} f(z) = c_0
\]

\underline{Вывод.} Если $ z_0 $ - устранимая, то $ \lim_{z \to z_0} f(z) = const $ 

\[
    2. \ f(z) = \frac{c_{-m}}{(z-z_0)^{n}} + \dots + \frac{c_{-1}}{(z-z_0)} 
    + c_0 + c_1(z-z_0) + \dots + c_n(z-z_0)^{n} \ | \ \cdot (z-z_0)^{n}
\]
\[
    z_0 \text{ - полюс порядка m. При } m = 1 \ z_0 \text{ - простой полюс}
\]
\[
    \psi(z) \equiv (z-z_0)^{n} \cdot f(z) = c_{-m} + c_{-m + 1}(z-z_0) + \dots
    + c_{-1}(z-z_0)^{n-1} + c_0(z-z_0)^{n} + \dots
\]
\[
    \psi(z_0) = \lim_{z \to z_0a} \psi(z) = c_{-m} \neq 0 \implies
    |f(z)| = \frac{|\psi(z)|}{|z-z_0|^{m}} > \frac{q}{|z-z_0|^{m}} \xrightarrow[z \to z_0]{}
    \infty
\]

\underline{Вывод.} Если $ z_0 $ - полюс, то $ \lim_{z \to z_0} f(z) = \infty $ 

Если $ z_0 $ - полюс порядка m
\[
    f(z) = \frac{\psi(z)}{(z-z_0)^{m}}, \ \psi(z) \in H(z_0), \ \psi(z_0) \neq 0
    \text{ - критерий полюса}
\]
\[
    f(z) = \frac{(z-1)(z^2 - 5)}{z-3} \quad z_0 = 3, \ \psi(z) = (z-1)(z^2 - 5) 
\]
\[
    \psi(z) \in H(z_0 =3), \ \psi(3) \neq 0
\]
\[
    \lim_{z \to 3} f(z) = \infty \implies z_0 = 3 \text{ - полюс 1-го порядка}
\]

3. $ z_0 $ - существенно особая точка
\begin{tcolorbox}
\underline{Th Сохоцкого} Если $ z_0 $ - существенно особая точка функции $ f(z) $,
то для любога числа А, включая $ A = \infty $ $ \exists \, z_n=z_n(A), \ 
z_n \xrightarrow[n \to \infty]{} z_0$, по которой $ \lim_{z_n \to z_0} f(z_n) = A $ 
\end{tcolorbox}

\underline{Вывод} $ z_0 $ - существенно особая точка функии $ f(z) $ 
\[
    \lim_{z \to z_0} f(z) \, \nexists
\]

$ z = z_0 $ - нуль $ f(z) $ 
\[
    f(z) = 0 \quad f(z_0) \equiv 0
\]
\[
    f(z) = c_m \cdot (z-z_0)^{m} + c_{m+1}(z-z_0)^{m+1} + \dots \quad c_m \neq 0
\]
m - кратность корня $ z = z_0 $ 
\[
    f(z) = (z - z_0) \cdot \underbrace{(c_m + c_{m+1}(z - z_0) + \dots )}_{\psi(z)}
\]
\[
    \psi(z) \in H(z_0), \ \psi(z_0) \neq 0 \quad f(z) = (z - z_0)^{m}\cdot \psi(z)
\]
\[
    c_m = \frac{f^{(m)}(z_0)}{m!} \quad f^{(m)}(z_0) \neq 0
\]

\begin{tcolorbox}
    \underline{Признак кратности нуля}\\
    $ z_0 $ - корень кратности m
    \[
        f(z_0) = f'(z_0) = \dots = f^{(m-1)}(z_0) = 0
    \]
    \[
        f^{(m)}(z_0) \neq 0
    \]
\end{tcolorbox}

\begin{tcolorbox}
    \underline{Th.1 (О связи полюса и нуля)}
    
    Если $ z_0 $ - нуль ф-ции $ f(z) $ кратности m (полюсом пор m), то для
    $ \frac{1}{f(z)}  $ точка $ z_0 $ является полюсом порядка m (нулем кратности m,
    если положить $ \frac{1}{f(z)} = 0 $)

    \underline{Proof}
    \[
        1) \  z_0 \text{ - ноль кратности m для f(z)}
    \]
    \[
        f(z) = (z - z_0)^{m} \cdot \psi(z), \ \psi(z) \in H(z_0), \ \psi(z_0) \neq 0
    \]
    \[
        \frac{1}{f(z)}  = \frac{1}{(z - z_0)^{m}} \cdot \frac{1}{\psi(z)} 
        = \frac{\phi(z)}{(z-z_0)^{m}}, \ \phi(z_0) \neq 0, \ \phi(z) \in H(z_0)
    \]
    \[
        2) \  z_0 \text{ - полюс порядка m для f(z)}
    \]
    \[
        f(z) = \frac{\phi(z)}{(z - z_0)^{m}}, \ \phi(z) \in H(z_0), \ \phi(z_0) \neq 0
    \]
    \[
        \frac{1}{f(z)}  = \frac{1}{\phi(z)} \cdot (z-  z_0)^{m} = 
        (z- z_0)^{m} \cdot \psi(z), \ \psi(z_0) \neq 0, \ \psi(z) \in H(z_0)
    \]
\end{tcolorbox}

\section*{\centering Ряд Лорана в окрестности $ z = \infty $}
\begin{tcolorbox}
\underline{Def}
Точка $ z = \infty $ - изолированная особая точка однозначного характера для $ f(z) $,
если эта функция является однозначной аналитической функцией в некоторой окрестности
этой точки $ R < |z| < \infty $ 
\end{tcolorbox}

$ z = \frac{1}{t} \implies $ характер $ z = \infty $ для $ f(z) $ определяется
по характеру $ t = 0 $ для $ \phi(t) \equiv f\left(\frac{1}{t}\right) = f(z) $ 
\[
    f(z) = \underbrace{\dots + \frac{c_{-n}}{z^{n}} + \dots + \frac{c_{-1}}{z} + c_0}_{\text{правильная часть}} + 
    \underbrace{c_1 \cdot z + \dots + c_n \cdot z^{n} + \dots}_{\text{главная часть}}
\]
1. $ z = \infty $ - устранимая особая точка $ f(z) $, ряд Лорана не содержит положительных
степеней
\[
    \lim_{z \to z_0} f(z) = const \neq \infty
\]

2. $ z = \infty $ - полюс порядка m для $ f(z) $, если ряд Лорана имеет вид:
\[
    f(z) = \dots + \frac{c_{-n}}{z} + \dots + \frac{c_{-1}}{z} + c_0 + c_1
    z + \dots + c_m \cdot z^{m}
\]
\[
    \lim_{z \to \infty} f(z) = \infty, \ f(\infty) = \infty
\]

3. $ z = \infty $ - существенно особая точка, если ряд Лорана содержит
бесконечно много положительных степеней
\[
    \nexists \ \lim_{z \to \infty} f(z)
\]

\section*{\centering Теория вычетов}
\begin{tcolorbox}
\underline{Def} Вычет (residu) ф-ии $ f(z) $ в изолированной особой точке
$ z_0 $ однозначного характера - число:
\[
    \text{res}_{z = z_0}f(z) = \frac{1}{2 \pi i} \oint_{C} f(z)dz,
\]
\[
    \text{ где } C: \ |z-z_0| = \rho, \text{ окружность достаточно малого радиуса,}
\]
\[
    \text{т.е. на самой окр-ти и внутри неё нет других особых точек}
\]
\end{tcolorbox}
\[
    f(z) = \sum_{n=-\infty}^{\infty} c_n \cdot (z- z_0)^{n}, \ 0 < |z-z_0| < R, \
    \rho < R
\]
\[
    \oint_{C} f(z) dz = c_{-1} \cdot 2 \pi i
\]

\underline{Вывод} Вычет ф-и равен $ c_{-1} $ при $ (z-z_0)^{-1} $ в Лорановском
разложении $ f(z) $ в окр-ти $ z_0 $ 

\end{document}
