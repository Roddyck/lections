\documentclass[12pt]{article}
\usepackage[a4paper,%
    text={180mm, 260mm},%
    left=15mm, top=15mm]{geometry}
\usepackage[utf8]{inputenc}
\usepackage{cmap}
\usepackage[english, russian]{babel}
\usepackage{indentfirst}
\usepackage{amssymb}
\usepackage{amsmath}
\usepackage{mathtools}
\usepackage{graphicx}
\graphicspath{ {./images/} }

\begin{document}

\begin{center}
\section*{Комплан. Лекция(10.09.24)}
\end{center}

\underline{Зам.} $| z_1 z_2 | = |z_1| |z_2|$ \\
$ |\frac{z_1}{z_2}| = \frac{|z_1|}{|z_2|} $ \\
$ Arg(z_1 \cdot z_2) = Argz_1 + Argz_2 $ \\
$ arg(z_1 \cdot z_2) \neq argz_1 + argz_2 $ \\
$ Arg(\frac{z_1}{z_2}) = Argz_1 - Argz_2 $ \\
\begin{center}
    Стереографическая проекция Сфера Римана
\end{center}

"идеальное" число $z = \inf $
$ \mathbb{C} \cap \{z=\inf\} = \bar{\mathbb{C}} $ \\ 
$ \mathbb{C} $ - отрытая ( конечная) компл. пл-ть \\
$ \bar{\mathbb{C}} $ - замкнутая компл. пл-ть \\
Рассмотрим сферу един диаметра расп в пр-ве. И касающуюся плоскости Oxy в 
начале коор-т \\

где-то тут рисунок\\

Таким образом устаналивается вз/одноз соответсвие м/у всеми точками сферы и
всеми точками открытой комплексной плоскости $ \mathbb{C} $ \\
$ \{ \text{N} \} \leftrightarrow \{ z = \inf\} $ \\
$ \text{Z}(\xi, \eta, \zeta) $ - стереограф. проекция т-ки z(x,y) \\

Св-ва стереограф. проекции №44, 45, 53, 56 \\
1.$ \begin{cases} 
    x =  \\ 
    y = \\ 
\end{cases}
$
2. Окружсность на S $\leftrightarrow $ окр-ть или прямая на $\bar{\mathbb{C}}$ \\

$ N \in \text{окружности на S} \leftrightarrow \text{прямая} $ \\
$ N \notin \text{окружности на S} \leftrightarrow \text{окружность} $ \\

$ \xi^2 + \eta^2 + (\zeta - \frac{1}{2})^2 = \frac{1}{4} $\\
$ \xi^2 + \eta^2 + \zeta^2 - \zeta = 0 $\\
3. $r(z_1, z_2) = |z_1 - z_2|$  \\
$ k(z_1, z_2) = r(Z_1, Z_2) $ k - хордальное(?) расстояние \\

\underline{\textbf{Ф-ии комплексного переменного}} \\
\underline{Опр.} Правило (закон) по которому каждому числу z $\in E=\{z\} \subset
\mathbb{C} $ ставится в соответствие одно или несколько значений w, называется
функцией комплексного переменного  и $w = f(z) $ \\
$ f : E \rightarrow \mathbb{C} $ \\
E - область определения \\
$M = \{ f(z) \} $ - множество значений \\
$ w = u + iv $
$ w = f(z) \Leftrightarrow \begin{cases}
    u = u(x,y) \\
    v = v(x,y)
\end{cases}
$

\underline{Опр.} Число A, если оно $ \exists $, наз-ся пределом ф-ии f(z) в точке
$z =a $ $\lim\limits_{z\rightarrow a} f(z) = A $ \\
$ \forall \epsilon > 0 \; \exists \delta=\delta(\epsilon) > 0 \; \forall z \in E: \;
0 < |z-a| < \delta \Rightarrow |f(z) - A | < \epsilon $ \\
$f(z) = u(z) + iv(z) \quad A = A_1 + iA_2 $ \\
1. 
$ 
\lim\limits_{z\rightarrow a}f(z) = A \Leftrightarrow 
\begin{cases}
    \lim\limits_{z\rightarrow a} u(z) = A_1 \\
    \lim\limits_{z\rightarrow a} v(z) = A_2 \\
\end{cases}
$ \\
2. $ A \neq 0 \Rightarrow можно подобрать argf(z) $, что

\lim\limits_{z\rightarrow a}f(z) = A \Leftrightarrow 
\begin{cases}
    \lim\limits_{z\rightarrow a} |f(z)| = |A| \\
    \lim\limits_{z\rightarrow a} argf(z) = argA \\
\end{cases}
    
\end{document}
