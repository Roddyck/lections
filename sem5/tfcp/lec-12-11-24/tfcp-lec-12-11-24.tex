\documentclass[a4paper]{article}
\usepackage[a4paper,%
text={180mm, 260mm},%
    left=15mm, top=15mm]{geometry}
\usepackage[utf8]{inputenc}
\usepackage{cmap}
\usepackage[english, russian]{babel}
\usepackage{indentfirst}
\usepackage{amssymb}
\usepackage{amsmath}
\usepackage{mathtools}
\usepackage{tcolorbox}
\usepackage{import}
\usepackage{xifthen}
\usepackage{pdfpages}
\usepackage{transparent}
\usepackage{graphicx}
\graphicspath{ {./figures} }

\newcommand{\incfig}[1]{%
\def\svgwidth{\columnwidth}
\import{./figures/}{#1.pdf_tex}
}

\begin{document}
\title{Комплан. Лекция}
\author{what am i?}
\maketitle

\begin{tcolorbox}
\textbf{\underline{Th}} (Коши) 
\[
    \int_{L} f(z) dz = 0
\]
\[
    \int_{\Gamma_0} f(z) dz + \int_{\Gamma_1^{-}} f(z) dz  + \dots + 
    \int_{\Gamma_n^{-}} f(z) dz 
\]
\[
    \int_{\Gamma_0} f(z) dz = \int_{\Gamma_1} f(z) dz  + \dots + 
    \int_{\Gamma_n} f(z) dz 
\]

\textbf{\underline{Proof}}\\
Из многосвязной области сделаем односвязную. Проведём разрезы $ l_k = l_k[a_k, b_k] $ 
$ a_k \in \Gamma_k, \ b_k \in \Gamma_0 $ 
\[
    \Gamma = \Gamma_0 \cup \Gamma_1^{-} \cup \dots \cup \Gamma_n^{-} \cup
    \sum_{k=1}^{n} l_k \cup \sum_{k=1}^{n} l_k^{-}
\]

Попали в условия теоремы Коши для односвязной области
\[
    \int_{\Gamma_0} + \int_{\Gamma_1^{-}} + \dots + \int_{\Gamma_n^{-}} + 
    \sum_{k=1}^{n} \int_{l_k} + \sum_{k=1}^{n} \int_{l_k^{-}} = 0 
\]
\end{tcolorbox}

\begin{tcolorbox}
\textbf{\underline{Частный случай}}\\ 
\textbf{\underline{Th}} (О деформации контура) Пусть функция $ f(z) $ регулярна в области 
D ограниченной контурами $ \Gamma_0, \ \Gamma_1 $ 
\[
    \int_{\Gamma_0}f(z) dz = \int_{\Gamma_1} f(z) dz
\]
\end{tcolorbox}

\begin{tcolorbox}
\textbf{\underline{Th.4}} (Интегральная формула Коши для односвязной области)\\
Если функция f(z) регулярна в односвязной области D, какой бы не был замкнутый контур
$ L \subset D $, $ \forall z \in Int L $,
\begin{equation}
    f(z) = \frac{1}{2\pi i} \oint \frac{f(t)dt}
    {t - z} \quad t \in L, \ z \in Int L
    \label{eq:1}
\end{equation}

\textbf{\underline{Proof}}\\
Фиксируем $ z \in Int L $ 
\[
    \gamma_\rho: \ |t-z| = \rho
\]
Построим ф-ию:
\[
    \phi(t) = \frac{f(t)}{t-z}
\]
Она регулярна в области D за исключением точки $ t=z $\\
По частному случаю:
\begin{equation}
    \frac{1}{2\pi i} \oint_L \frac{f(t)dt}{t-z} = \frac{1}{2\pi i} \oint_{\gamma_\rho}
    \frac{f(t)dt}{t-z} 
\end{equation}
Покажем, что
\begin{equation}
    \lim_{\rho \to 0} \oint_{\gamma_\rho} \frac{f(t)dt}{t-z} = f(z)
\end{equation}
\[
    \int_{\gamma_\rho} \frac{1}{t-z} dt = 2\pi i \ | \cdot \frac{f(z)}{2\pi i} 
\]
\[
    \frac{1}{2\pi i} \int_{\gamma_\rho} \frac{f(z)}{t-z} dt = f(z)
\]
\[
    \left|\frac{1}{2\pi i} \int_{\gamma_\rho} \frac{f(z)}{t-z} - f(z) \right| <
    \left| \frac{1}{2\pi i} \oint_{\gamma_\rho} \frac{f(t) - f(z)}{t-z} dt
    \right|
    \leq \frac{1}{2\pi} \int \frac{|f(t) - f(z)|}{|t-z|} dt
    < \frac{\epsilon}{2\pi} \int_{\gamma_\rho}\frac{dt}{|t-z|} = \frac{\epsilon}
    {2\pi} |2\pi| = \epsilon
\]
f(t) непрерывна в $ t = z \implies \forall \epsilon \ \exists \delta \ 
\rho = |t-z| < \delta \implies |f(t) - f(z)| < \epsilon$  \\ 
Интеграл F(z) в формуле $ (\ref{eq:1}) $ называют интеграл Коши \\
$\frac{1}{t-z} \text{ - ядро интеграла Коши}$ 
\[
    F(z) = \frac{1}{2\pi i} \oint \frac{f(t)dt}{t-z} = 
    \begin{cases}
        f(z), &\quad z \in Int(L)\\
        0, &\quad z \in Ext(L)
    \end{cases}
\]
\end{tcolorbox}

\begin{tcolorbox}
\textbf{\underline{Th.5}} (Интегральная формула Коши для многосвязной области)\\
Пусть $ f(z) \in H(D) $ D - ограниченая n+1 замкнутым контурами $ \Gamma_0,
\Gamma_1, \dots, \Gamma_n$ \\
Тогда $ \forall z \in Int(D) $ 
\[
    f(z) = \frac{1}{2\pi i} \int_L \frac{f(t)dt}{t-z} 
\]
\[
    L \equiv \partial \overline{D} = \Gamma_0 \cup \Gamma_1^{-} \cup \dots \cup
    \Gamma_n^{-}
\]

\textbf{\underline{Proof}}
\[
    \gamma_{\rho}: \ |t-z| = \rho
\]
\[
    \overline{Int(\gamma_{\rho})} \subset D
\]
Получим n+2 связную область. Граница $ L \cup \gamma_\rho^{-} $ 
\[
    \phi(t) = \frac{f(t)}{t-z} \text{ регулярная в (n+2) связной области}
\]
\[
    \int_{L \cup \gamma_\rho^{-}} \phi(t) = 0
\]
\[
    \int_L \frac{f(t)}{t-z} dt = \int_{\gamma_\rho} \frac{f(t)}{t-z} dt
\]
\[
    \frac{1}{2\pi i} \int_L \frac{f(t)}{t-z} dt = \frac{1}{2\pi i}
    \int_{\gamma_\rho} \frac{f(t)}{t-z} dt = f(z)
\]
\end{tcolorbox}

\begin{tcolorbox}
\textbf{\underline{Note}} Интегральную формулу Коши можно переписать в виде
\[
    \oint_L \frac{f(z)}{z-z_0} = 2\pi i \cdot f(z_0)
\]
\[
    z \in L \quad z_0 \in Int(L) 
\]
\[
    f(z) \in H(\overline{Int(L)})
\]
\end{tcolorbox}

\textbf{\underline{Example}}
\[
    \oint_L \frac{e^{z}}{z(z-2i)} dz
\]
1) $ L \equiv \Gamma_1 $ 
\[
    \oint\frac{e^{z}}{z(z-2i)} dz = 0
\]
\[
    z = 0 \notin Int \Gamma_1
\]
\[
    z = 2\pi \notin Int \Gamma_1
\]
2) $ L \equiv \Gamma_2  $ Охватывает $ z = 0 $ 
\[
    z = 2i \notin Int \Gamma_1
\]
\[
    z = 0 \in Int \Gamma_1
\]
\[
    \oint \frac{e^{t}}{z(z-2i)} dz = \oint \frac{\frac{e^{z}}{z - 2i} }{z} =
    2\pi i \cdot f(z)|_{z = z_0} = 2\pi i \frac{e^{z}}{z-2i} |_{z=0} = -\pi
\]
\[
    \frac{e^{z}}{z - 2i} \text{ - аналит в } Int \Gamma_1
\]
\[
    f(z) = \frac{e^{z}}{z - 2i} 
\]
\[
    z_0 = 0
\]
3) Охватывает $ z = 2i $\\
4) Охватывает обе\\
Первый способ:
\[
    \frac{1}{z(z-2i)} = \frac{1}{2i} \left(\frac{}{z-2i} - \frac{1}{z} \right)
\]
\[
    \oint_L \frac{e^{z}}{z(z-2i)} = \frac{1}{2i} \left( \oint_l \frac{e^{z}}{z - 2i} 
        dz - \oint_L \frac{e^{z}}{z} dz \right)
\]
Второй способ:\\
Делим область дважды проходимой кривой l
\[
    \int_{L_1=C_1 \cup l} + \int_{L_2 = C_2 \cup l^{-}}
\]

\end{document}
