\documentclass[a4paper]{article}
\usepackage[a4paper,%
    text={180mm, 260mm},%
    left=15mm, top=15mm]{geometry}
\usepackage[utf8]{inputenc}
\usepackage{cmap}
\usepackage[english, russian]{babel}
\usepackage{indentfirst}
\usepackage{amssymb}
\usepackage{amsmath}
\usepackage{mathtools}
\usepackage{tcolorbox}
\usepackage{import}
\usepackage{xifthen}
\usepackage{pdfpages}
\usepackage{transparent}
\usepackage{graphicx}
\graphicspath{ {./figures} }

\newcommand{\incfig}[1]{%
\def\svgwidth{\columnwidth}
\import{./figures/}{#1.pdf_tex}
}

\begin{document}
\title{Комплан. Лекция}
\author{wtf}
\maketitle

\begin{tcolorbox}
\underline{Th.5}(Св-ва симметрии)\\
Точки $ z_1, z_2 $ симметричные относительно окружности в широком смысле при др.
линейном преоб-и отображаются в точки $ w_1, w_2 $ симметричные относительно 
образа $ C = L(\Gamma) $ 
\end{tcolorbox}

\begin{tcolorbox}
\underline{Задача 1} (Полупл-ть на круг)\\
найти все др-лин преобр, отображающие $ Im z > 0 $ на единичный круг так, чтобы
некоторая точка $ z_1 = \beta \in \{ z: \ Im z > 0 \} $ перешла в $ w_1 = 0 $  
\end{tcolorbox}

\begin{figure}[!ht]
    \centering
    \incfig{task1}
    \caption{task1}
\end{figure}

$ \beta,\  \bar{\beta} $ симм отн $ Im z = 0 $ \\
$ L(\beta) = 0, \  L(\bar{\beta} = \infty $ \\
$ L(\{ Im z = 0 \} ) = \{ |w| = 1 $ 
\[
    w= \frac{a(z-\beta}{c(z-\bar{\beta})}  \ A = \frac{a}{c} = ?
\]
\[
    z = x \to |w|=1
\]
\[
    |w|  = |A| \frac{|x-\beta|}{|x - \bar{\beta}|}  = 1
\]
\[
    \beta = \delta + i\gamma
\]
\[
    \bar{\beta} = \delta - i\gamma
\]
\[
    |x-\beta| = \sqrt{(x - \delta)^2 + \gamma^2} 
\]
\[
    |x-\bar{\beta}| = \sqrt{(x - \delta)^2 + \gamma^2} 
\]
\[
    A = 1 \cdot e^{i\alpha} \quad \alpha = Arg A \in \mathbb{R}
\]
\[
    w = e^{i\alpha} \cdot \frac{z -\beta}{z - \bar{\beta}} \quad Im \beta > 0 
    , \ \forall \alpha \in \mathbb{R}
\]

\underline{Note.} Разрешая равенство относительно z, получим все дробно-линейные
преобразования отображающие единичный круг на верхную полуплоскость

\begin{tcolorbox}
\underline{Task2} (Круг в себя)\\
Найти все др-лин преоб, отображающие $ |z| < 1 $ в $ |w| < 1 $ так, чтобы
$ z_1 = \beta $ перешла в $ w_1 = L(\beta) = 0 $ 
\end{tcolorbox}

\begin{figure}[!ht]
    \centering
    \incfig{task2}
    \caption{task2}
\end{figure}

\[
    z_1 = \beta \to w_1 = 0
\]
\[
    z_2 = \frac{1}{\bar{\beta}} \to w_2 = \infty
\]
\[
    w = A \cdot \frac{z-\beta}{z-\frac{1}{\bar{\beta}} } = A \cdot 
    \frac{\bar{\beta}(z -\beta)}{z\bar{\beta} - 1} 
\]
\[
    w = B \frac{z-b\eta}{1 - \bar{\beta}\cdot z} 
\]
\[
    z = 1 \to |w| = 1
\]
\[
    |w| = |B| \frac{|1 - \beta|}{|1 - \bar{\beta}|} = 1 \implies
    |B| = 1
\]
\[
    B = 1 \cdot e^{i\alpha} \quad \alpha = Arg B \in \mathbb{R}
\]
\[
    w = e^{i\alpha}\frac{z - \beta}{1 - \bar{\beta} \cdot z} \quad \forall
    \alpha \in \mathbb{R}, \ |\beta| < 1
\]

\begin{tcolorbox}
\underline{Task3} (Полупл-ть в себя)\\
Найти все др-лин преоб отображающие верхнюю полуплость в себя так, чтобы 3 граничные
точки перешли в $ 0, 1, \infty $ 
\end{tcolorbox}

\begin{figure}[!ht]
    \centering
    \incfig{task3}
    \caption{task3}
    \label{fig:task3}
\end{figure}

\[
    \frac{w - 0}{w - 1} : \frac{\infty - 0}{\infty - 1} = 
    \frac{z - \alpha}{z - \beta} : \frac{\gamma - \alpha}{\gamma - \beta} 
\]
\[
    \implies \frac{w}{w - 1} = \frac{z - \alpha}{z - \beta} \cdot 
    \frac{\gamma - \beta}{\gamma - \alpha} 
\]
\[
    w = \frac{z - \alpha}{z - \beta} \cdot \delta(w - 1)
\]
\[
    w = \frac{az + b}{cz + d} , \ a,b,c,d \in \mathbb{R}
\]

\underline{Example} Отобразить единичный круг $ |z| < 1 $ на $ Im w > 0 $ 
\begin{figure}[!ht]
    \centering
    \incfig{ex1}
    \caption{Ex1}
    \label{fig:ex1}
\end{figure}

\[
    w_1 = \frac{z + 1}{z - 1} 
\]
\[
    z = -1 \to w_1 = 0
\]

\[
    w_2 \equiv w = w_1 + e^{-i \frac{\pi}{2}}
\]
\[
    w = -i \cdot \frac{z + 1}{z - 1} 
\]

\section*{\centering Отображение с помощью степенной функции}

\[
    w = z^{\lambda}, \ \lambda > 0, \lambda \neq 1
\]
\[
    z > 0 \implies z^{\lambda} > 0
\]
\[
    \frac{dw}{dz} = \lambda \cdot z^{\lambda - 1} \neq 
    \begin{cases}
       0 \\
       \infty
    \end{cases}, \ \text{кроме } z =0,
    \ z = \infty
    \implies w = z^{\lambda} \text{ - конмформно всюду, кроме } z = 0, z = \infty
\]
\[
    z = r e^{i \phi}, \ w = \rho \cdot e^{i\theta} \quad \rho e^{i\theta} = 
    (r e^{i \phi})^{\lambda}
\]
\[
    |w| = \rho = r^{\lambda}
\]
\[
    \theta \equiv \arg w = \lambda \phi 
\]
\[
    \begin{cases}
        |w| &= |z|^{\lambda} \\
        \arg w &= \lambda \arg z
    \end{cases}
    \implies \text{ луч }\phi \text{ отобразится вз/одн в луч } \theta
\]

\underline{Вывод 1} Ф-ия $ w = z^{\lambda}, \ \lambda > 0, \lambda \neq 1 $ угол
$ \phi_1 \leq \phi < \phi_2 $ в (z) с вершиной в т $ z = 0 $ раствора $ \phi -
\phi_1 \leq 2 \pi $ отобр в угол $ \lambda \phi_1 \leq \theta < \lambda \phi_2 $ 
в (w) с вершиной в $ w = 0 $ раствора $ \lambda (\phi_2 - \phi_1) $ вз-одн, если
$ \lambda(\phi_2 - \phi_1) \leq 2 \pi $, т.е. в т. $ z = 0 $ нарушается конф 
степенной функции

\[
    w = \sqrt{z} \quad w = z^{n}, \ w = \sqrt[n]{z}, \ n \geq 2  
\]

\underline{Вывод 2} Ф-ия $ w = z^{n} $ угол $ 0 < \phi < \frac{\pi}{2} $ 
конф отображает в угол $ 0 < \theta < \pi $, т.е. в $ Im w > 0 $ 

\begin{figure}[!ht]
    \centering
    \incfig{pic4}
    \caption{pic4}
    \label{fig:pic4}
\end{figure}

\[
    w = z^{n}
\]
\[
    z = r e^{i\phi}, \ w = \rho e^{i\theta}
\]
\[
    \rho e^{i\theta}= r^{n} e^{i n \phi}
\]
\[
    \begin{cases}
        \rho = r^{n}\\
        \theta = n\phi
    \end{cases}
\]
\[
    \begin{cases}
        0 \leq r < + \infty\\
        \phi = 9
    \end{cases}
    \implies L_1 : 
    \begin{cases}
        0 \leq \rho < \infty\\
        \theta = 0
    \end{cases}
\]
\[
    l_2: \begin{cases}
        0 \leq r < \infty\\
        \phi = \frac{\pi}{2} 
    \end{cases}
\]
\[
    L_2: \begin{cases}
        0 \leq \rho < +\infty\\
        \theta = \pi
    \end{cases}
\]
\end{document}
