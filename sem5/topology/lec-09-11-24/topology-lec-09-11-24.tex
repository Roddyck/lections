\documentclass[a4paper]{article}
\usepackage[a4paper,%
    text={180mm, 260mm},%
    left=15mm, top=15mm]{geometry}
\usepackage[utf8]{inputenc}
\usepackage{cmap}
\usepackage[english, russian]{babel}
\usepackage{indentfirst}
\usepackage{amssymb}
\usepackage{amsmath}
\usepackage{mathtools}
\usepackage{tcolorbox}
\usepackage{xfrac}
\usepackage{import}
\usepackage{xifthen}
\usepackage{pdfpages}
\usepackage{transparent}
\usepackage{graphicx}
\graphicspath{ {./figures} }

\newcommand{\incfig}[1]{%
\def\svgwidth{\columnwidth}
\import{./figures/}{#1.pdf_tex}
}

\begin{document}
\title{Топология. Лекция}
\author{whoo}
\maketitle

\begin{tcolorbox}
    \textbf{\underline{Th.5}} $ ([a,b], \tau_0) $ компактен 

    \textbf{\underline{Proof}} Let $ [a,b] = \bigcup_{\alpha \in A} U_{\alpha}, \ 
    U_{\alpha} \in \tau_{0[a,b]}$\\
    Let $ X = \{ x \in (a, b] \ | \ [a,x] \text{ покрыв конечным числом } U_\alpha \} $ \\
    1. $ X \neq \varnothing $. Let $ a \in U_{\alpha_1} $  
    \[
        \implies \exists [a,a+\epsilon) \subset U_{\alpha_1} \implies 
        [a, a + \frac{\epsilon}{2}] \subset U_{\alpha_1}, \ a + \frac{\epsilon}{2} \in X 
    \]
    2. Let $ v = \sup X $. Докажем, что $ v \in X $\\
    Let $ V \subset U_{\beta} $
    \[
        \implies \exists \ \epsilon \ | \ (v - \epsilon, v + \epsilon) \subset
        U_\beta
    \]
    \[
        c \in (v - \epsilon, v + \epsilon) \cap X
    \]
    \[
        [a,c] \subset U_{\alpha_1} cup \dots \cup U_{\alpha_n}
    \]
    \[
        [a,v] \subset U_{\alpha_1} cup \dots \cup U_{\alpha_n} cup U_\beta
    \]

    Докажем, что $ v = b $ 
    \[
        \text{Let } v < b
    \]
    \[
        v \in U_\beta \implies v \in (v - \epsilon, v + \epsilon) \implies
        [a, v + \frac{\epsilon}{2}] = U_{\alpha_1} \cup \dots \cap U_{\alpha_n} cup U_\beta
    \]
\end{tcolorbox}

\begin{tcolorbox}
\textbf{\underline{Th6}} Компактное подмн-во в Хауздорфовом пр-ве замкнуто

\textbf{\underline{Proof}} $ (X, \tau) \in T_2 $\\
А - комп подмн-во.\\
Докажем, что $ CA \in \tau $ 
\[
    \forall p \in CA, \ \forall \, x \in A \implies \exists \, U_p, U_x \ | \ 
    U_p \cap U_x = \varnothing
\]
\end{tcolorbox}

\begin{figure}[!ht]
    \centering
    \incfig{pic1}
    \caption{pic1}
    \label{fig:pic1}
\end{figure}
\begin{tcolorbox}
    См. pic1
\[
    \forall x \in A \ U_p^{x} \ | \ U_{p}^{x} \cap U_x = \varnothing
\]
\[
    \exists \text{ Конечное подпокр А окр-тями }U_{x_1}, \dots U_{x_n}
\]
\[
    U_p = U_p^{x_1} \cap \dots \cap U_{p}^{x_n}
\]

Проверим, что $ U_p \subset CA $ 
\[
    U_p \cap (\underbrace{U_{x_1}\cup \dots \cup U_{x_n}}) = 
    (U_p \cap U_{x_1}) \cup (U_p \cap U_{x_n}) = \varnothing
\]
\end{tcolorbox}

\begin{tcolorbox}
\textbf{\underline{Th7}} непрерывный образ комп пр-ва компактен

\textbf{\underline{Proof}} $ f: (X,\tau) \to (Y, \omega) $ - непр отображение, f - сюрьекция 

Let $ V_{\alpha, \alpha \in A} $ - открытое покрытие Y
\[
    f^{-1}(V_\alpha) \in \tau
\]
\[
    f^{-1}(V_\alpha) \text{ - покрытие X}
\]
\[
    \exists \text{ конечное подпокрытие X мн-вами} \ f^{-1}(V_{\alpha_1}), \dots
    , f^{-1}(V_{\alpha_k})
\]
\[
    \implies V_{\alpha_1}, \dots, V_{\alpha_k} \text{ - конечное подпокр Y}
\]
\end{tcolorbox}

\begin{tcolorbox}
\textbf{\underline{Th.8}} Произведение двух компактных пр-в компактно

\textbf{\underline{Proof}} Let $ (X, \tau), \ (Y, \omega) $ 
\[
    \forall Z \in \tau \times \omega
\]
\[
    \forall(x,y) \in Z \implies \exists U_\alpha \in \tau, \ V_i \in \omega \ | \ 
    (x,y) \in U_\alpha \times V_i \subset Z, \ U_\alpha \text{ - элемент базы }\tau
    \ \alpha \in A
\]
\[
    V_i \text{ - эл-т базы }\omega, \ i \in I
\]

Достаточно проверить, что из любого покрытия $ X \times Y $ множествами вида 
$ U_\alpha \times V_i $ можно выбрать конечное подпокрытие\\
Фиксируем $ x_0 \in X $. $ \{ x_0 \} \times Y $ \\
Let $ f: (Y, \omega) \to \{x_0 \} \times Y: Y \to (x_0, Y) $ \\
Тогда f непрерывна

Тогда по Th 7:
\[
    \{ x_0 \} \times Y \text{ - компактно}
\]

Выберем конечное подпокрытие множества $ \{ x_0 \} \times Y $ множествами
$ U_{\alpha_1} \times V_{i_1}, \dots, U_{\alpha_n}\times V_{i_n} $ 
\[
    U_{x_0} = U_{\alpha_1} \cap \dots \cap U_{\alpha_n} \text{ - окр-ть т. } x_0
    \implies  U_{\alpha_1} \times V_{i_1}, \dots, U_{\alpha_n}\times V_{i_n} 
    \text{ образуют конечное покрытие } U_{x_0} \times Y
\]
\[
    \forall \, x \in X \text{ построим аналогично } U_x \text{ и конечное подпокрытие }
    U_x \times Y
\]
\[
    X \times Y = \bigcup_{x \in X} U_x \times Y \implies \text{из компактности X}
    \implies \exists \text{ конечное подпокрытие } X \times Y
\]

\textbf{\underline{Сл1}} $ I^{n} = [a,b] \times \dots \times [a_n, b_n] \subset (R^{n}, \tau_0) $ 
- компактен
\end{tcolorbox}

\begin{tcolorbox}
\textbf{\underline{Def}} Подмножество в $ \mathbb{R}^{n} $ называется ограниченным, если
$ \exists \ I^{n} $, содержащий это мн-во
\end{tcolorbox}

\begin{tcolorbox}
\textbf{\underline{Th.9}} (Критерий компактности в $ (\mathbb{R}^{n}, \tau_0) $)\\
Множество в $ \mathbb{R}^{n} $ компактно $ \iff $ оно замкнуто и ограничено

\textbf{\underline{Proof}} 1. $ \impliedby: $ А замкнуто и ограничено
\[
    \implies \exists \, I^{n} \supset A
\]
\[
    \text{В пр-ве } (I^{n}, \tau_{I^{n}}) \text{ А замкнуто, т.к. } A = A \cap I^{n}
    \stackrel{\text{Th.2}}{\implies} \text{ A компактно}
\]

2. $ \implies: $ А компактно
\[
    U_n = \underbrace{(-m, m) \times \dots \times (-m, m)}_{n} \quad m \in \mathbb{N}
\]
$ U_m $ - покрытие A. Тогда:
\[
    \exists \text{ конечное подпокрытие }
\]
\[
    \text{Let } m_0 = \max_{i = \overline{1,k}} \{ m_i \}
\]
\[
    U_{m_i}, \ i = \overline{1, k} \text{ - образует конечное подпокрытие}
\]
\[
    A \subset [-m_0, m_0] \times \dots \times [-m_0, m_0] \implies \text{A огр}
\]
\[
    A \subset I^{n}, \ \text{A компактно}
\]
\[
    (I^{n}, \ \tau_{0, I^{n}}) \in T_2 \implies \text{A замкнуто}
\]
\end{tcolorbox}

\begin{tcolorbox}
\textbf{\underline{Def}} Отображение одного т.п. в другое называется открытым(замкнутым),
если образ каждого отрытого(замкнутого) множества открыт(замкнут)
\end{tcolorbox}

\begin{tcolorbox}
\textbf{\underline{Th.10}} Непрерывное отображение компактного простванства в Хауздорфово
пр-во замкнуто

\textbf{\underline{Proof}} $ f: (X,\tau) \to (Y, \omega) $\\ 
Let F замкнуто в $ (X, \tau) $  
\[
    \stackrel{\text{Th.2}}{\implies} \text{F компактно}
\]
\[
    f(F) \text{ компактно} \stackrel{Th.6}{\implies} f(F) \text{ замкнуто}
\]
\end{tcolorbox}

\begin{tcolorbox}
\textbf{\underline{Th.11}} Непрерыное биективное отображение $ f: (X,\tau) \to (Y, \omega) $ 
компактного пр-ва в Хауздорфово пр-во гомеоморфизм

\textbf{\underline{Proof}} Достаточно доказать, что $ f^{-1} $ непрерывно\\
Let $ f^{-1} = g $. $ g: (Y, \omega) \to (X,\tau) $  
\[
    \forall \, F \text{ замкнутое в } \tau
\]
\[
    g^{-1}(F) = (f^{-1})^{-1}(F) = f(F) \implies f(F) \text{ замкнуто}
\]
\end{tcolorbox}

\section*{\centering \S 8. Фактор-топология}

\begin{tcolorbox}
\textbf{\underline{Def}} Let X мн-во на котором задано отношение эквив. S\\
Обозначим $ \sfrac{X}{S} $  - мн-во классов эквивалентности\\
Отображение $ \pi: X \to \sfrac{X}{S}: x \mapsto [x] $ - каноническая проекция
\end{tcolorbox}

\begin{tcolorbox}
\textbf{\underline{Def}} Let $ (X,\tau) $ - т.п. и S - отношение экв. на X.\\
Фактор-топологией $ \sfrac{X}{S} $ называется семейство $ \tau_S $ опред. условием:\\
$ A \in \tau_S \iff \pi^{-1}(A) \in \tau $ 
\end{tcolorbox}

\textbf{\underline{Example}} Пусть на $ (\mathbb{R}, \tau_0) $ задано отношение экв-ти:
    $ x \sim y \iff $ \begin{align*}
    \left[
    \begin{array}{ll}
        x > 0, y > 0 \\
        x < 0, y < 0
    \end{array}
    \right .
    \end{align*}

\[
    \sfrac{\mathbb{R}}{S} = \{ +1, -1, 0 \}  
\]
\[
    \tau_S = \{ \varnothing, \sfrac{\mathbb{R}}{S}
\]

\end{document}
