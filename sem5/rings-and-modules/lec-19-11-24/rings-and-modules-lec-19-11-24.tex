\documentclass[a4paper]{article}
\usepackage[a4paper,%
    text={180mm, 260mm},%
    left=15mm, top=15mm]{geometry}
\usepackage[utf8]{inputenc}
\usepackage{cmap}
\usepackage[english, russian]{babel}
\usepackage{indentfirst}
\usepackage{amssymb}
\usepackage{amsmath}
\usepackage{mathtools}
\usepackage{tcolorbox}
\usepackage{import}
\usepackage{xifthen}
\usepackage{pdfpages}
\usepackage{transparent}
\usepackage{graphicx}
\graphicspath{ {./figures} }

\newcommand{\incfig}[1]{%
\def\svgwidth{\columnwidth}
\import{./figures/}{#1.pdf_tex}
}

\begin{document}
\title{КиМ. Лекция}
\author{Dead guy}
\maketitle

\begin{tcolorbox}
\underline{def} Кольцо R наз-ся полупервичным, если оно не содержит ненулевых
нильпотентных идеалов ($ \nexists \ 0 \neq I \lhd R \ | \ I^{n} = 0) $ 
\end{tcolorbox}

\underline{Note} В полупервичном кольцо нет ненулевых нильпотентных правых идеалов\\
если $ J^{n} = 0 $ для нек-го $ 0 \neq J $ - правый идеал, то $ 0 \neq RJR \lhd R $ \\
$ RJR = I, \ I^{n+1} = RJR \cdot RJR \dots \cdot RJR \subset J^{n} = 0 $ 

\begin{tcolorbox}
\underline{Def} Кольцо R называется классически полупростым справа, если оно полупервично
и артиново справа
\end{tcolorbox}

\begin{tcolorbox}
\underline{Th} Для кольца R эквивалентно:\\
1) $ \forall \ R  $ - модуль вполне приводим\\
2) $ R_R $ - вп. пирводим\\
3) R - классически п/пр

\underline{Proof} $ 1) \implies 2) $ Ясно

$ 2) \implies 1)  \ R_R $ - вп. приводим $ \iff R_R = $ прямая сумма простых модулей \\
любой свободный R - модуль $ F \cong \bigoplus R_R \implies F =  $ прямая сумма
простых R - модулей $ \implies F $ - вп. приводим. Если М - R - модуль, то М - 
эпиморфный образ некоторого свободного R - модуля F (доказывали)

Ранее доказывали, что гомо образ вполне пирводимого модуля вполне приводим
$ \implies M $ - вп. приводим

$ 2) \implies 3) \ R_R  $ - вполне приводим $ \implies R_R = \bigoplus_{i \in I}
M_i, $ где $ M_i $ - простые $ \implies 1 \in M_1 \oplus \dots \oplus M_n \implies
1 \cdot R \subset (M_1 \oplus \dots \oplus M_n) R \subset M_1 \oplus \dots \oplus
M_n \implies R = M_1 \oplus \dots \oplus M_n \implies R_R $ - артинов $\implies
 R $ - артиново справа

Пусть $ J \neq 0, \ J \lhd R \land J^{n} = 0. $ Т.к. $ R_R $ - вп. приводим,
то $ R = J \oplus J' \iff J = eR \ e^2 = e \implies (eR)^{n} = 0 \implies e^{n}
= e = 0$ 

$ 3) \implies 2) $ Покажем, что любой правый идеал кольца R вполне приводим
$ \iff \forall  $ правый идеал $ =  $ прямая сумма $
\underbrace{\text{минимальных правых идеалов}}_{\text{простые подмодули в} R_R} $\\
Против: существует правые идеалы $ \neq \bigoplus $ min правых идеалов \\
$ \implies \exists \ \min \text{ правый идеал J с этим св-вом}$ R - артиново справа
$ \implies J  $ содержит мин идеал I $ \implies I^2 = 0 \lor I = eR \implies R =
I \oplus I'$\\
$ J = J \cap R \implies J \cap (I \oplus I') \stackrel{I \subset J}{=}
I \oplus (J \cap I')$ \\
$ J \cap I' \subsetneqq J \stackrel{\text{J - min}}{\implies} J \cap I' = 
\oplus \min \text{ правых идеалов} \implies J = \oplus \min \text{ правых идеалов}$.
Противоречие $ \implies \forall $ правый идеал - вп. приводим
\end{tcolorbox}

\begin{tcolorbox}
\underline{Th} M, N - R - модули\\
1) если $ M = \bigoplus_{i = 1}^{n} M_i $, то $ \hom_R (\bigoplus M_i, N) \cong
\bigoplus \hom_R(M_i, N)$ \\
2) если $ N = \bigoplus_{j = 1}^{m} N_j \implies \hom_R(M, \bigoplus N_j) \cong
\bigoplus \hom_R(M, N_j)$ 

\underline{Proof} 1) $ \forall \alpha \in \hom_R(\bigoplus M_i, N) $ определяет
послед-ть $ (\alpha_1, \dots, \alpha_n) $, где $ \alpha_i = \alpha|_{M_i} $ \\
Обратно, $ \forall $ посл-ть определяет $ \alpha $ 
\[
    \forall m = m_1 + \dots + m_n \in M
\]
\[
    \alpha(m) = \alpha(m_1 + \dots + m_n) = \alpha(m_1) + \dots + \alpha(m_n) = 
    \alpha_1(m_1) + \dots + \alpha_n(m_n)
\]

Значит, $ f: \hom_R(\bigoplus M_i, N) \to \bigoplus\hom_R(M_i, N): \alpha
\mapsto (\alpha_1, \dots , \alpha_n)$ - биекция 
\[
    f(\alpha + \beta) = (\alpha_1 + \beta_1, \dots \alpha_n + \beta_n) = 
    (\alpha_1, \dots, \alpha_n) + (\beta_1, \dots, \beta_n) = f(\alpha) + f(\beta)
\]
Т.о. f - изо-зм абелевых групп
\end{tcolorbox}

\end{document}
