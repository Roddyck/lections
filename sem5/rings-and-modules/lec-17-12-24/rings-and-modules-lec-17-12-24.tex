\documentclass[a4paper]{article}
\usepackage[a4paper,%
    text={180mm, 260mm},%
    left=15mm, top=15mm]{geometry}
\usepackage[utf8]{inputenc}
\usepackage{cmap}
\usepackage[english, russian]{babel}
\usepackage{indentfirst}
\usepackage{amssymb}
\usepackage{amsmath}
\usepackage{mathtools}
\usepackage{tcolorbox}
\usepackage{import}
\usepackage{xifthen}
\usepackage{pdfpages}
\usepackage{transparent}
\usepackage{graphicx}
\graphicspath{ {./figures} }

\newcommand{\incfig}[1]{%
\def\svgwidth{\columnwidth}
\import{./figures/}{#1.pdf_tex}
}

\begin{document}
\title{КиМ. Лекция}
\maketitle

\begin{tcolorbox}
\underline{Lemma 4}\\
\underline{Упр} M - простой R - модуль $ \iff \forall \, 0 \neq m_1, m_2 \in M, 
\ \exists \, r \in R \ | \ m_2 = r m_1$ 
\[
    0 \neq A = (a_{ij}) \in I, \ B \in I
\]
\[
    \begin{pmatrix}
    0 & 0 & 0\\
    a_{i 1} \dots & a_{ij}\dots & a_{in}\\
    0 & 0 & 0\\
    
    \end{pmatrix}
    \begin{pmatrix}
    0 \dots & 0\dots & 0\\
    a_{ij}^{-1}b_{i 1}\dots & \dots & a_{ij}^{-1}b_{in}\\
    0 \dots &  0 & \dots 0\\
    
    \end{pmatrix}
    = 
    \begin{pmatrix}
    0 & \dots & 0\\
    b_{i 1} & \dots & b_{in}\\
    0 & \dots & 0\\
    
    \end{pmatrix}
\]
\end{tcolorbox}

\begin{tcolorbox}
\underline{Th (Веддерберн-Артин, 1950)}

1) R - артиново справа, если $ R_R $ - артинов модуль, т.е. $ I_1 \supset I_2 
\supset \dots \supset I_n = I_{n+1} = \dots$ 

2) R- полупервичное $ \nexists \, 0 \neq I \lhd R \ | \ I^{n} = 0 $ 

3) R - кл/пр справа, если R - полупервичное и артиново справа

R - кл/пр справа $ \iff R \cong \bigoplus_{k=1}^{n} M_{n_K}(D_k) $, $ D_k $ - тело\\
При этом кольцо R имеет ровно m простых модулей (с точностью до изом-ма)

\underline{Следствие} R - простое артиново справа кольцо $ \iff R \cong M_n(D) $ 

\underline{Proof} $ \implies: $ R - кл/пр $ \iff R_R $ - вполне приводим
$ \iff R_R $ - прямая сумма простых модулей (т.к. $ R_R $ - артинов справа - сумма
конечна). Пусть $ U_1, \dots, U_m $ - классы изоморфных модулей в этой сумме.
$ \implies R_R = U_1 \oplus \dots \oplus U_m, \ U_k = N_{k_1} \oplus \dots \oplus
N_{k_t}; \ N_{k_i} \cong N_{k_j} \cong N \ \forall i, j$ 
\[
    End_R(R_R) \cong R = \hom_R (R_R, R_R) = \hom_R(U_1 \oplus \dots \oplus U_m,U_1 \oplus \dots \oplus U_m)
\]
\[
    \cong \bigoplus_{i,j} \hom(U_i, U_j)
\]

\underline{$i \neq j$}
\[
    \hom_R(U_i, U_j) = \hom_R\left(\bigoplus_{p=1}^{n_i} N_{i_{p}}, \bigoplus_{q=1}^{n_j}
N_{j_q}\right) \cong \bigoplus_{p,q} \hom_R(N_{i_p}, N_{j_q}) = 0 \text{ по лемме Шура}
\]

\underline{$i = j$}
\[
    \hom_R (U_i, U_i) = End_R(U_i) \cong End_R(N \oplus \dots \oplus N) = M_n(End_R N)
\]
\[
    End_R N = D \text{ - тело по лемма Шура }
\]
\[
    R \cong \bigoplus_{k=1}^{m}\hom_R(U_k, U_k) \cong\bigoplus_{k=1}^{m} M_{n_K}(D_k)
\]

$ \impliedby: $ 
\[
    R \cong
    \begin{pmatrix}
    M_{n_1}(D_1) & 0\dots & 0\\
     & \ddots & 0 \\
     0&  & M_{n_k}(D_k) \\
    
    \end{pmatrix}
\]

\end{tcolorbox}
\begin{tcolorbox}
$ e_{ij}^{k} $ - матричная еденица в $ M_{n_k}(D_k) $, дополним нулями до большой
матрицы
\[
    e_{ii}^{k} \cdot R = N_i^k \text{ - i - ая строка в R}
\]
\[
    \implies R = \bigoplus_{k=1}^{m}\bigoplus_{i=1}^{n_k} N_i^k \text{ - прямая сумма простых
    подмодулей} \implies R_R \text{ вп. приводим } \iff R \text{ - кл/пр}
\]
\[
    N_i^k \text{ - простой R - модуль}
\]

Пусть М - простой R - модуль
\[
    \hom_R(R_R, M) \cong M \stackrel{\star}{=} \ \forall R \text{ - модуля М}
\]
\underline{Упр} Указание: $ \phi: M \to \hom_R(R_R, M): m \mapsto \phi_m $,
где $ \phi_m(r) = mr \ \forall r \in R $ 
\[
    \stackrel{\star}{=} \hom_R\left(\bigoplus_{k,i} N_i^k, M\right)
    = \bigoplus_{k,i}\hom_R(N_i^k, M) \neq 0 \implies \exists \, N_s^k \ | \ 
    N_s^k \cong M
\]

\underline{Proof} (Следствия) $ \implies: $ R - кл/пр $ \implies R \cong
\bigoplus_{k=1}^{m} M_{n_k}(D_k) \stackrel{\text{R - простое}}{\implies} R \cong
M_n(D)$ 
\[
    R \cong \begin{pmatrix}
    M_{n_1}(D_1) & 0 & 0 \\
     & \ddots  & \\
     0&  & 0 \\
    
    \end{pmatrix}
    \begin{pmatrix}
    M_{n_1}(D_k) & 0 & 0 \\
     & \ddots  & \\
     0&  & M_{n_k}(D_k) \\
    
    \end{pmatrix}
    =
    \begin{pmatrix}
    M_{n_1}(D_k) & 0 & 0 \\
     & \ddots  & \\
     0&  & 0 \\
    
    \end{pmatrix}
\]
\end{tcolorbox}
\end{document}
