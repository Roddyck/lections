\documentclass[a4paper]{article}
\usepackage[a4paper,%
    text={180mm, 260mm},%
    left=15mm, top=15mm]{geometry}
\usepackage[utf8]{inputenc}
\usepackage{cmap}
\usepackage[english, russian]{babel}
\usepackage{indentfirst}
\usepackage{amssymb}
\usepackage{amsmath}
\usepackage{mathtools}
\usepackage{tcolorbox}
\usepackage{import}
\usepackage{xifthen}
\usepackage{pdfpages}
\usepackage{transparent}
\usepackage{graphicx}
\graphicspath{ {./figures} }

\newcommand{\incfig}[1]{%
\def\svgwidth{\columnwidth}
\import{./figures/}{#1.pdf_tex}
}

\begin{document}
\title{КиМ. Лекция}
\maketitle

\begin{tcolorbox}
    2)
    \[
        \forall \ \alpha \in \hom_{R}\left(M_i \bigoplus_{j=1}^{m} N_j\right)
    \]
    $ \alpha $ определяет последовательность $ \alpha_1, \dots , \alpha_m $,
    где $ \alpha_j = \pi_j \alpha: \ M \to \bigoplus_{j=1}^{m}N_j \to N_j $ 

    Обратно $ \forall ( \alpha_1, \dots , \alpha_m) $ определяет $ \alpha $:
    \[
        \forall m \in M: \ \alpha(m) = n = (n_1, \dots, n_m) = (\pi_1(n), \dots,
        \pi_m(n)) = (\pi_1 \alpha(m), \dots, \pi_m \alpha(m)) = (\alpha_1, \dots
        \alpha_m)(m)
    \]
    \[
        \phi: \hom_R\left(M_i \bigoplus N_j\right) \to \bigoplus_{j=1}^{m} \hom_R(M, N_j): 
        \alpha \mapsto (\alpha_1, \dots, \alpha_m)
    \]
    \[
        \phi(\alpha + \beta) = \phi(\alpha) + \phi(\beta)
    \]
\end{tcolorbox}

\begin{tcolorbox}
\underline{Lemma.1}

$ End_R R_R \cong R $ 

\underline{Proof} \underline{упр} $ \phi: R \to End_R R_R: r \mapsto \phi_r $,
где $ \forall x \in R_R: \ \phi_r(x) = rx $ - изом-зм колец 
\end{tcolorbox}

\begin{tcolorbox}
\underline{Lemma 2}
\[
    M_R = M_1 \oplus \dots \oplus M_n
\]
Тогда
\[
    End_R M \cong 
    \begin{bmatrix}
        End_R M_1 & \hom_R(M_2, M_1) & \hdots & \hom_R(M_n, M_1)\\
        \hom_R(M_1, M_2) & End_R M_2 & \hdots & \hom_R(M_n, M_2)\\
        \dots\\
        \hom_R(M_1, M_n) & \hom_R(M_2, M_n) & \dots & End_R M_n
    \end{bmatrix}
    = S
\]
В частности, если $ M_i \cong M_j \cong N \quad \forall i, j $ 
\[
    End_R M \cong M_n(End_R N)
\]

\underline{Proof} $ \pi_i: M \to M_i $ - проекция\\
$ \eta_j: M_j \hookrightarrow M $ - вложение
\[
    \pi_i \eta_j = 0, \ i \neq j
\]
\[
    \pi_i \eta_i = 1_{M_i}
\]
\[
    \sum_{i=1}^{n} \eta_i \pi_i = 1_M = 1
\]
\[
    \forall \, \alpha \in End_R M \implies \alpha = 1 \cdot \alpha \cdot 1 =
    \sum_{i=1}^{n} \eta_i \pi_i \alpha \sum_{i=1}^{n} \eta_j \pi_j =
    \sum_{i,j} \eta_i(\pi_i \alpha \eta_j) \pi_j = \sum_{i,j} \eta_i \alpha_{ij}
    \pi_j
\]
\[
    \alpha_{ij} = \pi_i \alpha \eta_j \in \hom_R(M_j, M_i)
\]
\[
    \phi: End_R M \to S: \ \alpha \mapsto (\alpha_{ij}) \text{ - изом-зм колец}
\]
\[
    1) \ \phi(\alpha + \beta) \stackrel{?}{=} \phi(\alpha) + \phi(\beta) 
    \iff (\alpha + \beta)_{ij} \stackrel{?}{=} \alpha_{ij} + \beta_{ij}
\]
\[
    (\alpha + \beta)_{ij} = \pi_i (\alpha + \beta) \eta_j = \pi_i \cdot \alpha
    \cdot \eta_j + \pi_i \beta \eta_j = \alpha_{ij} + \beta_{ij}
\]
\[
    2) \phi(\alpha \beta) \stackrel{?}{=} \phi(\alpha) \phi(\beta)
\]
\[
    (\alpha \beta)_{ij} = \pi_i \cdot \alpha \beta \cdot \eta_j = \pi_i \cdot
    \alpha \cdot 1 \cdot \beta \cdot \eta_j = \pi_i \cdot \alpha \sum_{k} \eta_k
    \pi_k \cdot \beta \eta_j = \sum_{k} (\pi_i \alpha \eta_k)(\pi_k \beta \eta_j)
    = \sum_{k} \alpha_{ik} \beta_{ki}
\]
\[
    c_{ij} = \sum_{k} \alpha_{ik} \beta_{ki} = (\alpha \beta)_{ij}
\]

3) иньективность
\[
    \forall \alpha \in \ker \phi: \ \phi(\alpha) = 0 = (\alpha_{ij}) \implies 
    \alpha_{ij} = 0 \ \forall i,j
\]
\[
    \alpha = \sum_{i,j} \eta_i \cdot 0 \cdot \pi_j = 0
\]

4) сюрьективность
\[
    \forall (c_{ij}) = C \in S
\]
\[
    \sum_{a,b} \eta_a \cdot c_{ab} \pi_b = \alpha
\]
\[
    \alpha_{ij} = \pi_i \left(\sum_{a,b} \eta_a c_{ab}\pi_b \right) \eta_j
    = \pi_i \eta_i c_{ij} \pi_j \eta_j = c_{ij}
\]
\end{tcolorbox}

\begin{tcolorbox}
\underline{Lemma 3} (Лемма Шура)

Пусть M, N - простые R - модули. Тогда
\[
    \forall 0 \neq \phi \in \hom_R(M,N) \text{ - изоморфизм}
\]

В частности, $ End_R M $ - тело

\underline{Proof} 
\[
    \forall 0 \neq \phi \in \hom_R(M,N) \implies im \phi \leq N \implies
    im \phi = 0 \lor im \phi = N \implies \phi \text{ - сюрьективно}
\]
\[
    \ker \phi \leq M \implies \ker \phi = 0 \lor \ker \phi = M \implies
    \phi \text{ - иньективно}
\]
\end{tcolorbox}

\begin{tcolorbox}
\underline{Lemma 4}\\
1) $ M_n(D) $ - простое кольцо, где D - тело\\
2) \[
    S = \begin{bmatrix}
         \begin{pmatrix}
            0 & \dots & 0\\
            a_{i 1} & \dots & a_{in}\\
            0 & \dots & 0
        \end{pmatrix}
    \end{bmatrix} 
    \text{ - простое R - модуль}
\]
\[
    E_{ij} =
    \begin{pmatrix}
        0 &\vdots &0\\
        \dots & 1 & \dots\\
        0 &\vdots & 0
    \end{pmatrix}
\]
\[
    E_{ij}E_{sk} = 
    \begin{cases}
        0, &\quad j \neq s\\
        E_{ik}, &\quad j = s
    \end{cases}
\]

1) Пусть $ 0 \neq I \lhd R, \ 0 \neq A (a_{ij}) \in I $, пусть $ a_{ij} \neq 0 $  
\[
    a_{ij}^{-1} E_{ii} A E_{jj} = E_{ij} \in I
\]
\[
    \forall \, E_{kl}: \ E_{kl} = E_{ki}E_{ij}E_{jl} \in I
\]
\[
    R = < E_{ij} >_R \implies I = R
\]

2) \underline{Упр} R -  модуль М - простой $ \iff \forall 0 \neq m_1 \in M, \ 
\forall m_2 \in M \ \exists r \in R \ | \ m_2 = m_1 \cdot r$ 

$ \impliedby $: 
\[
    0 \neq N \leq M, \text{ пусть } 0 \neq n \in N
\]
Берём $ \forall m \in M $ 
\[
    \implies m = \underbrace{n \cdot r}_{\in N} \text{ for some } r \in R
    \implies M \subset N \implies M = N
\]
\end{tcolorbox}

\end{document}
