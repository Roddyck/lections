\documentclass[a4paper]{article}
\usepackage[a4paper,%
    text={180mm, 260mm},%
    left=15mm, top=15mm]{geometry}
\usepackage[utf8]{inputenc}
\usepackage{cmap}
\usepackage[english, russian]{babel}
\usepackage{indentfirst}
\usepackage{amssymb}
\usepackage{amsmath}
\usepackage{mathtools}
\usepackage{tcolorbox}
\usepackage{import}
\usepackage{xifthen}
\usepackage{pdfpages}
\usepackage{transparent}
\usepackage{graphicx}
\graphicspath{ {./figures} }

\newcommand{\incfig}[1]{%
\def\svgwidth{\columnwidth}
\import{./figures/}{#1.pdf_tex}
}

\begin{document}
\title{УМФ. Лекция}
\author{haha}
\maketitle

\begin{equation}
    \frac{\partial^2 u(x,y,z,t)}{\partial t^2} - a^2 \Delta u(x,y,z,t) = f(x,y,z,t)
\end{equation}
\[
    \Delta u = \frac{\partial^2 u}{\partial x^2} 
    + \frac{\partial^2 u}{\partial y^2} + \frac{\partial^2 u}{\partial z^2}
\]

\underline{Задача Коши}
\[
    (x,y,z) \in \mathbb{R}^3, \ t \geq 0   
\]
\begin{equation}
    u(x,y,z) |_{t=0} = \phi(x,y,z)
\end{equation}
\begin{equation}
    u_t(x,y,z) |_{t=0} = \psi(x,y,z)
\end{equation}

\underline{Смешанные задачи}
\[
    \Omega \subset \mathbb{R}^3 \quad \Gamma = \partial \Omega
\]
\[
    (x,y,z) \in \Omega, \ t \geq 0
\]
\[
    \phi(x,y,z), \ \psi(x,y,z) \text{ заданы в } \Omega
\]
\[
    \Gamma = \Gamma_1 \cup \Gamma_2 \cup \Gamma_3, \quad \Gamma_i \cap \Gamma_j
    = \varnothing, \ i \neq j
\]

Граничные условия:
\begin{subequations}
    \begin{align}
        &u(x,y,z,t) |_{\Gamma_1} = u_{\Gamma}(x,y,z,t), \ (x,y,z) \in \Gamma_1, \ t \geq 0\\
        &\frac{\partial u(x,y,z,t)}{\partial n} |_{\Gamma_2} = q_{\Gamma}(x,y,z,t),
        \ (x,y,z) \in \Gamma_2, \ t > 0 \\
        &\frac{\partial u(x,y,z,t)}{\partial n} + hu(x,y,z,t)|_{(x,y,z) \in \Gamma_3}
        = p_{\Gamma}(x,y,z,t) , \ (x,y,z) \in \Gamma_3, \ t \geq 0
    \end{align}
\end{subequations}

\begin{equation}
    \frac{\partial^2 u(x,y,z,t)}{\partial t^2} - a^2 \Delta u(x,y,z,t) = 0
\end{equation}

Формула Пуассона
\[
    u(x,y,z,t) = \frac{1}{4 \pi a} \cdot \frac{\partial }{\partial t} 
    \iint_{S_{at}} \frac{\phi(\xi, \eta, \zeta}{at} dS + \frac{1}{4 \pi a} 
    \iint_{S_{at}} \frac{\psi(\xi, \eta, \zeta}{at} dS
\]

\begin{figure}[!ht]
    \centering
    \incfig{puasson-formula}
    \caption{Puasson formula}
    \label{fig:puasson-formula}
\end{figure}

\section*{Принцип Гюйкенса}

Локализованные в пространстве начальные возмущения порождают возмущения локализованные
во времени

\[
    D \subset \mathbb{R}^3
\]
\[
    \phi(x,y,z) = 0 \text{ in } \mathbb{R}^3 \setminus D
\]
\[
    \psi(x,y,z) = 0 \text{ in } \mathbb{R}^3 \setminus D
\]
\[
    d_{\star} = dist(M;D)
\]

\begin{figure}[!ht]
    \centering
    \incfig{gukens-principal}
    \caption{Gukens principal}
    \label{fig:gukens-principal}
\end{figure}

\[
    d^{\star} = \sup_{N \in D} |M \ N|
\]
\[
    t > d^{\star}
\]
\[
    u(x,y,z,t) \equiv 0
\]
\[
    t \in \left( \frac{d_{\star}}{a}, \ \frac{d^{\star}}{a} \right)
\]
\[
    at < d_{\star}, \ t < \frac{d_{\star}}{a} \quad t = \frac{d_{\star}}{a} 
\]
\[
    u(x,y,z,t) \equiv 0
\]

\section*{\centering Цилиндрические волны}

\[
    u = u(x,y,t) \quad \frac{\partial}{\partial z} \equiv 0 
\]

\[
    \Delta u = u_{xx} + u_{y y}
\]
\[
    \phi = \phi(x,y), \ \psi = \psi(x,y)
\]
\[
    u(x,y,t) = \frac{1}{2 \pi a} \frac{\partial}{\partial t} \iint_{C_{at}}
    \frac{\phi(\xi, \eta) d\xi d\eta}{\sqrt{(at)^2 (x - \xi)^2 (y - \eta)^2} } 
    +
    \frac{1}{2 \pi a} \frac{\partial}{\partial t} \iint_{C_{at}}
    \frac{\psi(\xi, \eta) d\xi d\eta}{\sqrt{(at)^2 (x - \xi)^2 (y - \eta)^2} }
\]
\[
    at < d_{\star}
\]
\[
    u(x,y,t) \equiv 0
\]
\[
    at > d_{\star}
\]
\[
    t > \frac{d_{\star}}{\xi} 
\]

При $ t \to \infty $ знаменатель стремится к 0

Для цилиндрических волн принцип Гюйгенса не выполняется

\begin{figure}[!ht]
    \centering
    \incfig{cylinder-waves}
    \caption{Cylinder waves}
    \label{fig:cylinder-waves}
\end{figure}

\[
    \frac{\partial^2 u(x,y,z,t)}{\partial t^2} - a^2 \Delta u(x,y,z,t) = f(x,y,z,t)
\]

Формула Кирхгофа:
\[
    u(x,y,z,t) = \frac{1}{4 \pi a} \cdot \frac{\partial }{\partial t} 
    \iint_{S_{at}} \frac{\phi(\xi, \eta, \zeta)}{at} dS + \frac{1}{4 \pi a} 
    \iint_{S_{at}} \frac{\psi(\xi, \eta, \zeta)}{at} dS
    + \frac{1}{4 \pi a^2} \iiint_{D_{at}} \frac{f(\xi, \eta, \zeta, t - \frac{r}{a}) }
    {r} d\xi d\eta d\zeta 
\]
\[
    \frac{1}{4 \pi a^2} \iiint_{D_{at}} \frac{f(\xi, \eta, \zeta, t - \frac{r}{a}) }
    {r} d\xi d\eta d\zeta \text{ - запаздывающий потенциал}
\]
\end{document}
