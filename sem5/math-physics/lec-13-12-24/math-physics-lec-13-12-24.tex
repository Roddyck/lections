\documentclass[a4paper]{article}
\usepackage[a4paper,%
    text={180mm, 260mm},%
    left=15mm, top=15mm]{geometry}
\usepackage[utf8]{inputenc}
\usepackage{cmap}
\usepackage[english, russian]{babel}
\usepackage{indentfirst}
\usepackage{amssymb}
\usepackage{amsmath}
\usepackage{mathtools}
\usepackage{tcolorbox}
\usepackage{import}
\usepackage{xifthen}
\usepackage{pdfpages}
\usepackage{transparent}
\usepackage{graphicx}
\graphicspath{ {./figures} }

\newcommand{\incfig}[1]{%
\def\svgwidth{\columnwidth}
\import{./figures/}{#1.pdf_tex}
}

\begin{document}
\title{УМФ. Лекция}
\author{Roddyk}
\maketitle

\section*{\centering Задача Дирихле}
\[
    \Omega \subset \mathbb{R}^3 \quad \Gamma = \partial \Omega
\]
\begin{tcolorbox}
    \begin{equation}
        \frac{\partial^{2} u(x,y,z,t)}{\partial t^2} - a^2 \Delta u(x,y,z,t) = 0 
    \end{equation}

    \begin{equation}
        u(x,y,z,t) |_{\Gamma} = 0
    \end{equation}
    \begin{equation}
        u(x,y,z,t) |_{t=0} = \phi(x,y,z)
    \end{equation}
    \begin{equation}
        u_t(x,y,z,t) |_{t=0} = \psi(x,y,z)
    \end{equation}
    \begin{equation}
        u(x,y,z,t) = T(t) v(x,y,z)
    \end{equation}
\end{tcolorbox}
\[
    T''(t) v(x,y,z) - a^2 T(t) \Delta v(x,y,z) = 0 \ | \ : a^2T(t) v(x,y,z)
\]
\[
    \frac{T''(t)}{a^2T(t)} = \frac{\Delta v(x,y,z)}{v(x,y,z)} = -\lambda
\]
\begin{equation}
    T''(t) + a^2 \lambda T(t) = 0
\end{equation}
\begin{equation}
    \Delta v(x,y,z) + \lambda v(x,y,z) = 0
    \label{eq:7}
\end{equation}
\begin{equation}
    v(x,y,z) |_{\Gamma} = 0
\end{equation}

Если $ A = -\Delta $, $ Av = \lambda v $  

\subsection*{Свойства решения:}
1) Все собственные значения вещественные
\[
    \lambda = Re \lambda + i Im \lambda
\]
\[
    v = Re v + i Im v
\]
\[
    (\ref{eq:7}) \cdot \overline{v}
\]
\begin{equation}
    \Delta \overline{v} + \lambda \overline{v} = 0
    \label{eq:9}
\end{equation}
\begin{equation}
    \overline{v} |_{\Gamma} = 0
    \label{eq:10}
\end{equation}
\[
    (\ref{eq:9}) \cdot v
\]
\[
    \begin{cases}
        \Delta v \cdot \overline{v} + \lambda v \overline{v} = 0\\
        \Delta \overline{v} \cdot v + \overline{\lambda} \overline{v} v = 0
    \end{cases}
\]
\[
    \Delta v \cdot \overline{v} - \Delta \overline{v} \cdot v + (\lambda
    - \overline{\lambda}) |v|^2
\]
\[
    \int_{\Omega} (\Delta v \cdot \overline{v} - \Delta \overline{v} \cdot v) dxdydz
    + (\lambda - \overline{\lambda}) \int_{\Omega}|v|^2 dxdydz = 0
\]
\[
    \Delta v \cdot \overline{v} = \frac{\partial}{\partial x} \left(\frac{\partial v}{\partial x} 
    \overline{v}\right) + \frac{\partial}{\partial y} \left( \frac{\partial v}{\partial y} 
    \overline{v} \right) + \frac{\partial}{\partial z} \left(\frac{\partial v}{\partial z} 
    \overline{v} \right) - \left( \frac{\partial v}{\partial x} \frac{\partial \overline{v}}{\partial x} 
    + \frac{\partial v}{\partial y} \frac{\partial \overline{v}}{\partial y} 
    + \frac{\partial v}{\partial z} \frac{\partial \overline{v}}{\partial z} 
    \right)
\]
\[
    \int \left(\frac{\partial P}{\partial x} + \frac{\partial Q}{\partial y} 
    + \frac{\partial R}{\partial z} \right) dxdydz = 
    \int_{\Gamma} (P\cos nx + Q \cos ny + R \cos nz) d\Gamma
\]
\[
    \int_{\Omega} \Delta v \cdot \overline{v} dxdydz = 
    \underbrace{\int_{\Gamma} \overline{v} \frac{\partial v}{\partial n} d \Gamma}_{=0}
    - \int_{\Omega} (v_x \overline{v}_x + v_y \overline{v}_y + v_z \overline{v}_z)
    dxdydz
\]
\[
    \int_{\Omega} \Delta \overline{v} \cdot v dxdydz = 
    - \int_{\Omega} (v_x \overline{v}_x + v_y \overline{v}_y + v_z \overline{v}_z)
    dxdydz
\]
\[
    (\lambda - \overline{\lambda}) \int_{\Omega} |v|^2 dxdydz = 0  
\]
\[
    \lambda - \overline{\lambda} = 0
\]
\[
    \lambda = \overline{\lambda} \implies \lambda \in \mathbb{R}
\]

В задаче (7)-(8) все собственные значения собственные
\[
    (\ref{eq:7}) \cdot v, \ \int_{\Omega}
\]
\[
    \int_{\Omega} \Delta v \cdot v dx dy dz + \lambda \int_{\Omega} v^2 dxdydz
    = 0
\]
\[
    \int_{\Gamma} v \frac{\partial v}{\partial n} d\Gamma - 
    \int_{\Omega}(v_x^2 + v_y^2 + v_z^2) dxdydz = 0
\]
\[
    \underbrace{\int_{\Omega}(v_x^2 + v_y^2 + v_z^2) dxdydz}_{\geq 0} =
    \underbrace{\lambda \int_{\Omega}v^2 dxdydz}_{> 0}
\]
\[
    \lambda \geq 0
\]
\[
    \lambda = 0 \implies \begin{cases}
        v_x \equiv 0\\
        v_y \equiv 0\\
        v_z \equiv 0
    \end{cases}
\]

Тогда $ \lambda > 0 $ 

В то время как в задаче Неймана $ \lambda = 0 $ собственное значение, а собственная ф-я $ = const $ 
\[
    0 < \lambda_1 < \dots < \lambda_n < \dots, \quad \lambda_n \xrightarrow[n \to \infty]{}
    +\infty
\]

\begin{tcolorbox}
    Можно док-ть, что каждому собст. зн-ю соответствует конечный набор собственных ф-ий
\end{tcolorbox}

\begin{tcolorbox}
    \underline{Th} Собственные ф-ии соответсвующие различным собственным значениям
    ортогональны

    \underline{План действий}
    \[
        \lambda_1 \to v_1^{1}, \dots, v_{k_1}^{1}
    \]
    \[
        \dots\dots\dots\dots
    \]
    \[
        \lambda_n \to v_1^{n}, \dots, v_{k_n}^{n}
    \]
    \[
        \Delta v_i^{n}(x,y,z) + \lambda v_i^{n}(x,y,z) = 0
    \]
    \[
        v_i^{n}(x,y,z) |_{\Gamma} = 0
    \]
    \[
        T_n(t) v_i^n(x,y,z) = 0
    \]
    \[
        u(x,y,z,t) = \sum_{n=1}^{\infty} T_n(t)
        \sum_{i=1}^{k_n} C_i^n v_i^n(x,y,z)
    \]
    \[
        C_i^n,\quad n = 1, \dots , \infty, \quad i = 1, \dots, k_n
    \]
    \[
        T_n(t) = A_n \cos \omega_n t + B_n \sin \omega_n t
    \]
    \[
        \omega_n = a \sqrt{\lambda_n} 
    \]
    \[
        u(x,y,z,t) = \sum_{n=1}^{\infty} \sum_{i=1}^{k_n} \left( 
        A_n^i \cos \omega_n t + B_n^{i} \sin \omega_n t \right) v_i^n(x,y,z)
    \]
    \[
        u |_{t=0} = \sum_{n=1}^{\infty} \sum_{i=1}^{k_n} A_n^i v_n^i(x,y,z) 
        = \phi(x,y,z) \ | \ v_j^m, \ \int_{\Omega}
    \]
    \[
        u_t |_{t=0} = \sum_{n=1}^{\infty}\omega_n \sum_{i=1}^{k_n} B_n^i v_n^i(x,y,z) 
        = \psi(x,y,z)\ | \ v_j^m, \ \int_{\Omega}
    \]
    \[
        \int_{\Omega} v_i^n \cdot v_j^m dxdydz = 0 \quad \text{ если } n \neq m
        \text{, а если } n = m \text{ то } i \neq j
    \]
    \[
        A_m^j \int_{\Omega} (v_j^m)^2 dxdydz = \int_{\Omega}\phi(x,y,z)
        v_{j}^{m}(x,y,z) dxdydz
    \]
    \[
        \omega_m B_m^j \int_{\Omega} (v_j^m)^2 dxdydz = \int_{\Omega}\psi(x,y,z)
        v_{j}^{m}(x,y,z) dxdydz
    \]
\end{tcolorbox}
\end{document}
