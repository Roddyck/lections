\documentclass[a4paper]{article}
\usepackage[a4paper,%
    text={180mm, 260mm},%
    left=15mm, top=15mm]{geometry}
\usepackage[utf8]{inputenc}
\usepackage{cmap}
\usepackage[english, russian]{babel}
\usepackage{indentfirst}
\usepackage{amssymb}
\usepackage{amsmath}
\usepackage{mathtools}
\usepackage{tcolorbox}

\begin{document}
\begin{center}
    \textbf{УМФ. Лекция (13.09.24)}\\ 
\end{center}

\[ 
    \frac{\partial U}{\partial x} = U_x, \; \frac{\partial^2U}{\partial x \partial y} 
    = u_{xy}, \dots
\] 

\begin{equation}
    au_{xx} + 2bu_{xy} + cu_{yy} + du_x + eu_y + hu = f
\end{equation}

\[ 
    b^2 - ac \begin{cases}
        > 0, \quad & \text{гиперболический тип} \\
        < 0, \quad & \text{эллиптический тип} \\
        = 0, \quad & \text{параболический тип} \\
        
    \end{cases}
\]

Замена: 
$
\begin{cases}
    \xi = \xi(x,y) \\
    \eta = \eta(x, y) \\
\end{cases}
$
\hspace{1cm}
$ D(\xi, \eta) = \det
\begin{pmatrix}
    \xi_x & \xi_y \\
    \eta_x & \eta_y
\end{pmatrix}
\neq 0
$
\begin{equation}
    2\tilde{b}\, \frac{\tilde{U}(\xi, \eta)}{\partial\xi\partial\eta}
    + \tilde{d} \tilde{u}_\xi + \tilde{e} \tilde{u}_\eta + h
    \tilde{u} = \tilde{f}
    \quad\text{(гиперболический тип КВ)}
\end{equation}

\begin{equation}
    \tilde{u}(\xi, \eta) = u\left(\xi(x,y), \eta(x,y)\right) = u(x,y)
\end{equation}

\begin{align*}
    \tilde{u}_x &= \frac{\partial}{\partial x} \tilde{u}(\xi, \eta) =
    \tilde{u}_\xi \xi_x + \tilde{u}_\eta \eta_x \\
    \tilde{u}_{xx} &= \frac{\partial}{\partial x} u_x = 
    \frac{\partial}{\partial x} (\tilde{u}_\xi \xi_x + \tilde{u}_\eta \eta_x) =  
    (\frac{\partial\tilde{u}_\xi}{\partial \xi} \xi_x + \frac{\partial\tilde{u}_\xi}
    {\partial \eta} \eta_x) \xi_x + \tilde{u}_\xi \xi_{xx} + \dots
\end{align*}

\[
    \tilde{a}\tilde{u}_{\xi\xi} + 2\tilde{b}\tilde{u}_{\xi\eta} + \tilde{c}
    \tilde{u}_{\eta\eta} + \tilde{d}\tilde{u}_{\xi} + \tilde{e}\tilde{u}_\eta
    + \tilde{h}\tilde{u} = \tilde{f}
\]
\begin{tcolorbox}
    \begin{equation}
        \begin{aligned}
            \tilde{a} &= a(\xi_x)^2 + 2b\xi_x \xi_y + c(\xi_y)^2\\
            \tilde{c} &= a(\eta_x)^2 + 2b\eta_x \eta_y + c(\eta_y)^2\\
            \tilde{b} &= a \xi_x\eta_x + b(\xi_x\eta_y + \xi_y\eta_x) + c\xi_x\eta_y \\ 
            \tilde{d} &= a\xi_{xx} + 2b \xi_{xy} + c \xi_{yy} + d \xi_x + e \xi_y \\
            \tilde{e} &= a \eta_{xx} + 2b \eta_{xy} + c \eta_{yy} + d \eta_x + e \eta_y\\
            \tilde{h} &= h \\
            \tilde{f} &= f \\
        \end{aligned}
    \end{equation}
\end{tcolorbox}

\section*{\centering Гиперболический тип}
гиперболическое ур-е:
$
\begin{cases}
    \tilde{a} &= 0 \\
    \tilde{c} &= 0
\end{cases}
$
\begin{equation}
    a(\phi_x)^2 + 2b(\phi_x\phi_y) + c(\phi_y)^2 = 0 \quad
    \text{- характеристическое уравнение}
\end{equation}
\[
    a \neq 0 \quad (\phi_x)_{1,2} = \frac{-b \pm \sqrt{b^2-ac}}{a} \cdot \phi_y
\]

\begin{equation}
   \left[
       \begin{array}{ll}
           a\phi_x + (b - \sqrt{b^2 - ac})\phi_y &= 0 \\
           a\phi_x + (b + \sqrt{b^2 - ac})\phi_y &= 0 \\
       \end{array}
   \right .
\end{equation}
\[
    \alpha(x,y)\phi_x(x,y) + \beta(x,y)\phi_y(x,y) = 0
\]

\[
    \tilde{b}^2 - \tilde{a}\tilde{b} = (b^2-ac)\left(\det
        \begin{pmatrix}
            \xi_x & \xi_y \\
            \eta_x & \eta_y
        \end{pmatrix}
    \right)^2
\]

\begin{equation}
   \left[
       \begin{array}{ll}
           \frac{dx}{a} &= \frac{dy}{b-\sqrt{b^2-ac}} \to \Phi \equiv \xi \\
           \frac{dx}{a} &= \frac{dy}{b+\sqrt{b^2-ac}} \to \Phi \equiv \eta \\
       \end{array}
   \right .
\end{equation}

\section*{\centering Параболический тип}
\[
    \begin{cases}
        \tilde{b} &= 0 \\
        \tilde{a} &= 0
    \end{cases}
    \quad\text{или} \quad
    \begin{cases}
        \tilde{b} &= 0 \\
        \tilde{c} &= 0
    \end{cases}
\]
\[
    \begin{aligned}
        &a\phi_x + b\phi_y = 0 | \cdot b \\
        &ab\phi_x + b^2\phi_y = 0 \\
        &ab\phi_x + ac\phi_y = 0 \\
        &b\phi_x + c\phi_y = 0
    \end{aligned}
\]
Замена:
$
\begin{cases}
    \phi \to \xi &= \xi(x,y) \\
    \eta &= \eta(x,y)
\end{cases}
$ \\
$ \tilde{b} = \eta_x(a\xi_x + b\xi_y) + \eta_y(b\xi_x + c\xi_y) = 0 $

\section*{\centering Эллиптический тип}
\[ b^2 - ac < 0 \quad \tilde{a}=\tilde{c} \quad \tilde{b} = 0 \]

\begin{equation}
   \left[
       \begin{array}{ll}
           a\phi_x + (b - i\sqrt{ac - b^2})\phi_y &= 0 \\
           a\phi_x + (b + i\sqrt{ac - b^2})\phi_y &= 0 \\
       \end{array}
   \right .
\end{equation}
$ \Phi(x,y) $
\[ \frac{dx}{a} = \frac{dy}{b-i\sqrt{ac-b^2}} \]
\[ \Phi(x,y) = Re\Phi(x,y) + iIm\Phi(x,y) \quad \Phi = \xi + i \eta \]
$
\begin{cases}
    \xi &= Re\Phi \\
    \eta &= Im\Phi \\
\end{cases}
$
\[
    a(\xi_x^2 - \eta_x^2) + 2b(\xi_x\xi_y - \eta_x\eta_y) + c(\xi_y^2 - \eta_y^2)
    + 2i(a\xi_x\eta_x + b(\xi_x\eta_y + \xi_y\eta_x) + c\xi_y\eta_y) = 0
\]
В получившимся уравении действительная часть равна ($\tilde{a} - \tilde{c}$)
и действительная часть равна $\tilde{b}$. И притом обе равны 0.
\end{document}
