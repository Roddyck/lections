\documentclass[a4paper]{article}
\usepackage[a4paper,%
    text={180mm, 260mm},%
    left=15mm, top=15mm]{geometry}
\usepackage[utf8]{inputenc}
\usepackage{cmap}
\usepackage[english, russian]{babel}
\usepackage{indentfirst}
\usepackage{amssymb}
\usepackage{amsmath}
\usepackage{mathtools}
\usepackage{tcolorbox}
\usepackage{import}
\usepackage{xifthen}
\usepackage{pdfpages}
\usepackage{transparent}
\usepackage{graphicx}
\graphicspath{ {./figures} }

\newcommand{\incfig}[1]{%
\def\svgwidth{\columnwidth}
\import{./figures/}{#1.pdf_tex}
}

\begin{document}
\title{УМФ. Лекция}
\author{who tf is this?}
\maketitle

\[
    u_{tt}(x,t) - a^2 u_{xx}(x,t) = A \sin\omega t
\]
\[
    u(0,t) = u(l,t) = 0
\]
\[
    u|_{t=0} = 0
\]
\[
    u_t|_{t=0} = 0
\]
\[
    \lambda_n = \left(\frac{\pi n}{l} \right)^2, \ n \in \mathbb{N}
\]
\[
    X_n(x) = \sin \frac{\pi n x}{l} 
\]
\[
    u(x,t) = \sum_{n=1}^{\infty} T_n(t) X_n(x)
\]
\[
    \omega_{n} = \frac{a \pi n}{l}, \ n \in \mathbb{N}
\]
\[
    \sum_{n=1}^{\infty} (T''_n(t) + \omega_n^2 T_n(t)) X_n(t) = A\sin\omega t\ 
    | \cdot \sin \frac{\pi m x}{l},\ \int_{0}^{l} dx 
\]
\[
    \int_{0}^{l} \sin \frac{\pi n x}{l} \cdot \sin \frac{\pi m x}{l} dx = 
    \begin{cases}
        0, &\quad n \neq m\\
        \frac{l}{2}, &\quad n = m
    \end{cases}
\]
\[
    T''_m + \omega_m T_m(t) = \frac{2}{l} \int_{0}^{l} \sin \frac{\pi m x}{l} dx
    \cdot \sin \omega t
\]
\[
    \frac{2}{l} \int_{0}^{l} \sin \frac{\pi m x}{l} dx = \alpha_m
\]
\[
    \begin{cases}
        T''_m(t) + \omega^2 T_m(t) = \alpha_m \sin \omega t\\
        T_m(0) = 0\\
        T'_m(0) = 0
    \end{cases}
\]

Пусть:
\[
    \omega \neq \omega_m, \ m \in \mathbb{N}
\]

Общее решение однородного уравнения:
\[
    T_m^{O} = A_m \cos \omega_m t + B_m \sin \omega_m t 
\]
\[
    T_m^{\text{Н}} = C_m \sin \omega t
\]
\[
    \underbrace{(\omega_m^2 - \omega^2)}_{\neq 0} C_m \sin \omega t = \alpha_m 
    \sin \omega t
\]
\[
    C_m = \frac{\alpha_m}{\omega_m^2 - \omega^2} 
\]
\[
    T_m(t) = A_m \cos \omega_m t + B_m \sin \omega_m t + \frac{\alpha_m}
    {\omega_m^2 - \omega^2} \sin \omega t
\]
\[
    T_m(0) = A_m = 0
\]
\[
    T'_m(0) = \omega_m B_m \cos \omega_m t + \frac{\alpha_m \omega \cos \omega t}
    {\omega_m^2 - \omega^2} 
\]
\[
    \omega_m B_m + \frac{\alpha_m \omega} {\omega_m^2 - \omega^2} 
    = 0
\]
\[
    B_m = - \frac{\alpha_m \omega}{(\omega_m^2 - \omega^2)\omega_m} 
\]
\[
    T_m(t) = - \frac{\alpha_m \omega \sin \omega_m t}{(\omega_m^2 - \omega^2)\omega_m}
    + \frac{\alpha_m \sin \omega t}{(\omega_m^2 - \omega^2)} 
\]
\[
    T_m(t) = \frac{\alpha_m (\omega_m \sin \omega t - \omega \sin \omega_m t}
    {(\omega_m^2 - \omega^2)\omega_m}
\]

Пусть:
\[
    \omega \to \omega_{m_0}
\]
\[
    \lim_{\omega \to \omega_{m_0}} T_{m_0}(t) = \lim_{\omega \to \omega_{m_0}} 
    \frac{\alpha_m (\omega_m \sin \omega t - \omega \sin \omega_m t}
    {(\omega_m^2 - \omega^2)\omega_m} = \lim_{\omega \to \omega_{m_0}}
    \frac{\alpha_{m_0}(\omega_{m_0} t \cos\omega t - \sin \omega_{m_0} t)}
    {\omega_{m_0}(-2 \omega)} 
\]
\[
    \lim_{\omega \to \omega_{m_0}} T_m(t) = 
    \frac{\alpha_{m_0}(t \omega_{m_0}\cos\omega_{m_0} t - \sin \omega_{m_0} t)}
    {-2 \omega_{m_0}^2} 
\]
\section*{Задача Штурма-Лиувилля}

\[
    x \in [0;l]
\]
\begin{equation}
    (p(x)X'(x))' + (\lambda\rho(x) - q(x)) X(x) = 0
\end{equation}
\begin{equation}
    \alpha X'(0) - \beta X(0) = 0
\end{equation}
\begin{equation}
    \gamma X'(l) - \delta X(l) = 0
\end{equation}
\begin{equation}
    \begin{aligned}
        &p \in C^{1}([0;l])\\
        &\rho \in C([0;l])\\
        &q \in C([0;l])\\
        &\alpha \geq 0, \ \beta \geq 0, \ \gamma \geq 0, \ \delta \geq 0\\
        &\alpha^2 + \beta^2 > 0\\
        &\gamma^2 + \delta^2 > 0\\
        &p(x) > 0, \ x \in [0;l], \ \rho(x) > 0, \ x \in [0;l]
    \end{aligned}
    \label{eq:4}
\end{equation}
Задача Штурма-Луивилля (1) - (3) заключается в нахождении собственных значений
и собственных функций. Собственными значениями называеются те значения параметра
$ \lambda $ при которых уравение имеет хотя бы одно не тривиальное решение $ X(x) \neq 0 $ 
. Сами эти нетривиальные решения называются собст. ф-ями задачи Ш-Л
\[
    \begin{aligned}
        &p \equiv 1\\
        &q \equiv 0\\
        &\rho \equiv 1\\
        &\alpha = 0 \ (\beta \neq 0)\\
        &\gamma = 0 \ (\delta \neq 0)
    \end{aligned}
\]
\[
    X''(x) + \lambda X(x) = 0
\]
\[
    X(0) = X(l) = 0
\]
\[
    X_1(x), \ X_2(x), \dots , X_m(x) \text{ - с. ф-ии} \implies c_1 X_1(x) +
    \dots + c_m X_m(x) \text{ - с. ф-я}
\]

\underline{\textbf{Вопрос из билета: Вещественность собственных значения задачи Ш-Л}}

При условиях ($ \ref{eq:4} $) все собст. значения задачи Ш-Л вещественны и неограничивая общ-ти
можно считать вещ-ми собственные функции
\[
    \lambda = Re\lambda + i Im \lambda
\]
\[
    X(x) = Re X(x) + i Im X(x)
\]
\begin{equation}
    (p(x) \cdot \overline{X}'(x))' + (\overline{\lambda} - \rho(x) - q(x)) 
    \overline{X}(x) = 0
\end{equation}
\begin{equation}
    \alpha \overline{X}'(0) - \beta \overline{X}(0) = 0
\end{equation}
\begin{equation}
    \gamma \overline{X}'(l) - \delta \overline{X}(l) = 0
\end{equation}
\[
    -
    \begin{cases}
        (1) \cdot \overline{X}\\
        (5) \cdot X
    \end{cases}
    , \ \int_{0}^{l} 
\]
\[
    \int_{0}^{l} (p(x) X'(x))' \cdot \overline{X} dx - \int_{0}^{l} (p \overline{X}'
    Xdx + \lambda \int_{0}^{l} \rho X \overline{X} dx - \overline{\lambda}
    \int_{0}^{l} \rho \overline{X} X dx - \int_{0}^{l} q X \overline{X} dx +
    \int_{0}^{l} q \overline{X} X dx = 0
\]
\[
    \int_{0}^{l} (p(x) X'(x))' \cdot \overline{X} dx - \int_{0}^{l} 
    (p \overline{X}') X dx = (\overline{\lambda}- \lambda) \int_{0}^{l} \rho |X|^2 dx
\]
\[
    p(x)X'(x) \overline{X}(x) \big|_{0}^{l} - \int_{0}^{l} p X' \overline{X}'dx
    - p \overline{X}' X \big|_{0}^{l} + \int_{0}^{l} p \overline{X}' X' dx
\]
\[
    p(l) (\underbrace{X'(l) \overline{X}(l) - \overline{X}'(l) X(l)}_{\det \begin{cases}
            (3)\\
            (7)
    \end{cases}})
\]
\[
    p(0) (X'(0) \overline{X}(0) - \overline{X}'(0) X(0))
\]
Рассмотрим
\[
    \begin{cases}
        (2)\\
        (6)
    \end{cases}
    \quad
    \begin{cases}
        (3)\\
        (7)
    \end{cases}
\]
\[
    0 = (\overline{\lambda} - \lambda) \int_{0}^{l} \rho |X|^2 dx 
\]
\[
    \implies \overline{\lambda} - \lambda = 0
\]
\[
    X(x) = Re X(x) + i Im X(x)
\]

Любую комплексную собственную функцию можно представить как лин комб вещественных
собственных функций

\underline{Вопрос билета: Линейная зависимость собс ф-ий соотв. одному собств значению}\\
Любые две собст ф-ии соотв. одному собств зн-ю линейно зависимы
\end{document}
