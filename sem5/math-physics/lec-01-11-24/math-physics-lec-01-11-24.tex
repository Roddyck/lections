\documentclass[a4paper]{article}
\usepackage[a4paper,%
    text={180mm, 260mm},%
    left=15mm, top=15mm]{geometry}
\usepackage[utf8]{inputenc}
\usepackage{cmap}
\usepackage[english, russian]{babel}
\usepackage{indentfirst}
\usepackage{amssymb}
\usepackage{amsmath}
\usepackage{mathtools}
\usepackage{tcolorbox}
\usepackage{romannum}
\usepackage{import}
\usepackage{xifthen}
\usepackage{pdfpages}
\usepackage{transparent}
\usepackage{graphicx}
\graphicspath{ {./figures} }

\newcommand{\incfig}[1]{%
\def\svgwidth{\columnwidth}
\import{./figures/}{#1.pdf_tex}
}

\begin{document}

\title{УМФ. Лекция}
\author{djkasadfhsaygfatfiask}
\maketitle
\section*{\centering Единственность решения}

\begin{equation}
    u_{tt}(x,t) - a^2 u_{xx}(x,t) = f(x,t) \quad x \in [0; l], \ t \geq 0
\end{equation}
\begin{equation}
    u |_{t=0} = \phi(x)
\end{equation}
\begin{equation}
    u_t |_{t=0} = \psi(x)
\end{equation}
\begin{equation}
\begin{aligned}
    x = 0 &\qquad x = l\\
    u(0, t) = \mu_0(t) &\qquad u(l,t) = \mu_l(t)\\
    u_x(0, t) = \nu_0(t) &\qquad u_x(l,t) = \nu_l(t)\\
    u(x,t) - h u(x,t) |_{x=0} = q_0(t) &\qquad u_x(x,t) + h u(x,t) |_{x=l} = q_l(t)\\
\end{aligned}
\end{equation}

\[
    u^{1}, \ u^2, \ u = u^{1} - u^2
\]
\[
    u_{t t}(x,t) - a^2u_{xx}(x,t) = 0
\]
\[
    u|_{t=0} = 0
\]
\[
    u_{t}|_{t=0} = 0
\]
\[
    \begin{aligned}
        u(0,t) = 0 \qquad u(l,t) = 0 \\
        u_x(0,t) = 0 &\qquad u_x(l,t) = 0 \\
        u_x - hu|_{x=0} = 0 &\qquad u_x + hu|_{x=l} = 0
    \end{aligned}
\]
\[
    u_t(u_{t t} - a^2 u_{xx}) = \frac{1}{2} \frac{\partial}{\partial t}[a^2 u^2
_x + u_t^2] - a^2 \frac{\partial}{\partial x} (u_t u_x) - a^2 u_tx \cdot u_x 
\]
\[
    \int_{0}^{l} dx
\]
\[
    \frac{1}{2} \frac{\partial}{\partial t} \int_{0}^{l} [a^2 u_x^2 + u_t^2]dx =
    a^2 u_t u_x |_{0}^{l}
\]
\[
    a^2 u_t u_x |_{x=l} = 
    \begin{cases}
        0, \romannumeral 1, \ \romannumeral 2 \\
        -ha^2 u_t(l,t) u(l,t), \ \romannumeral 3
    \end{cases}
\]
\[
    a^2 u_t u_x |_{x=0} = 
    \begin{cases}
        0, \romannumeral 1, \ \romannumeral 2 \\
        a^2h u_t(0,t) u(0,t), \ \romannumeral 3
    \end{cases}
\]
Если нет граничных условий третьего рода:
\[
    \frac{\partial}{\partial t} \int_{0}^{l} [a^2u^2_x + u_t^2]dx = 0
\]
\[
    \int_{0}^{l} [a^2u^2_x + u_t^2]dx = \Phi(t)
\]
\[
    \frac{d\Phi(t)}{dt} = 0
\]
\[
    \Phi(t) = const
\]
\[
    \Phi(t) \equiv 0
\]
\[
    [a^2u^2_x + u_t^2] \equiv 0
\]
\[
    u_x \equiv 0, \ u_t \equiv 0
\]
\[
    u = const = 0
\]
\[
    u(x,t) \equiv 0
\]

Если есть граничные условия третьего рода:
\[
    \frac{1}{2} \frac{\partial}{\partial t} \int_{0}^{l} [a^2u^2_x + u_t^2]dx =
    -ha^2 [u_t(l,t)u(l,t) + u_t(0,t)u(0,t)] = \frac{-ha^2}{2} \left[ \frac{\partial 
    u^2(l,t)}{\partial t} + \frac{\partial u^2(0,t)}{\partial t} \right]
\]
\[
    \frac{1}{2} \frac{d}{dt} \left[ \frac{ha^2}{2} u^2(l,t) + u^2(0,t) + \int_{
    0}^{l} [ a^2 u_x^2 + u_t^2] dx \right] = 0
\]
\[
    \left[ \frac{ha^2}{2} (u^2(l,t) + u^2(0,t)) + \int_{
    0}^{l} [ a^2 u_x^2 + u_t^2] dx \right] = const
\]
\[
    \int_{ 0}^{l} [ a^2 u_x^2 + u_t^2] = 0
\]
А это уже рассматривалось выше

При всех классических граничных условиях решение смешанной задачи 1-4 единственно

\section*{\centering Резонанс}
\[
    \omega_n = \frac{a\pi n}{l}, \ n \in \mathbb{Z}
\]
\[
    \sum_{n=1}^{\infty} (A_n \cos\omega_n t + B_n\sin \omega_n t) \sin \frac{\pi
    n x}{l} 
\]
\end{document}
