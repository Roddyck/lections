\documentclass[a4paper]{article}
\usepackage[a4paper,%
    text={180mm, 260mm},%
    left=15mm, top=15mm]{geometry}
\usepackage[utf8]{inputenc}
\usepackage{cmap}
\usepackage[english, russian]{babel}
\usepackage{indentfirst}
\usepackage{amssymb}
\usepackage{amsmath}
\usepackage{mathtools}
\usepackage{tcolorbox}
\usepackage{import}
\usepackage{xifthen}
\usepackage{pdfpages}
\usepackage{transparent}
\usepackage{graphicx}
\graphicspath{ {./figures} }

\newcommand{\incfig}[1]{%
\def\svgwidth{\columnwidth}
\import{./figures/}{#1.pdf_tex}
}

\begin{document}
\title{УМФ. лекция}
\maketitle

\begin{equation}
    u_{t t} - a^2u_{xx}(x,t) = 0
\end{equation}
\begin{equation}
    u(x,t) |_{t=0} =\phi(x) \quad x \in [O; l]
\end{equation}
\begin{equation}
    u_t(x,t) |_{t=0} = \psi(x) \quad x \in [O; l]
\end{equation}
\begin{equation}
    u(0,t) = u(l,t)=0
\end{equation}

\begin{equation}
    \begin{aligned}
        & u(0,0) &= 0 &= \phi(0)\\
        & u_t(0,0) &= 0 &= \psi(0) \\
        & u(l,0) &= 0 &= \phi(l) \\
        & u_t(l,0) &= 0 &= \psi(l) 
    \end{aligned}
\end{equation}

\begin{equation}
    u(x,t) = T(t)X(x)
\end{equation}

\[
    T''(t)X(x) - a^2 T(t) X''(x) = 0 \ | \ : a^2 T(t) X(x)
\]
\[
    \frac{T''(t)}{a^2T(t)} = \frac{X''(x)}{X(x)}  = - \lambda
\]
\begin{equation}
    T''(t) + a^2 \lambda T(t) = 0
\end{equation}

\begin{equation}
    X''(t) + \lambda X(t) = 0
\end{equation}
\begin{equation}
    X(0) = X(l) = 0
\end{equation}

\[
    T(t) X(0) = 0
\]
\[
    T(t) X(l) = 0
\]

(8), (9) - задача Штурма-Лиувилля.\\
Она заключается в нахождении собственных значений и собственных функций\\
Собственными значениями называются те значения $ \lambda $ при которых есть хотя
бы одно нетривиальное решение задачи. Сами эти нетривиальные решения $ X(x) $ 
называются собственными функциями соответсвующие собственным значениям

\[
    p^2 + \lambda = 0
\]
\[
    p^2 = -\lambda
\]
\[
    1) \ \lambda < 0
\]
\[
    p = \pm \sqrt{-\lambda} 
\]
\[
    X(x) = C e^{\sqrt{-\lambda \cdot x}} + D e^{-\sqrt{-\lambda } }
\]
\[
    \begin{cases}
        X(0) = C + D = 0\\
        X(l) = C(e^{\sqrt{-\lambda}l } - e^{-\sqrt{-\lambda} l})
    \end{cases}
\]
\[
    X(x) = Ax + B
\]
\[
    2) \ \lambda = 0
\]
\[
    X(0) = A \cdot 0 + B = 0 \implies B = 0
\]
\[
    3) \ \lambda \ge 0
\]
\[
    p = \pm \sqrt{\lambda} i
\]
\[
    X(x) = C \cos(\sqrt{\lambda} x) + D \sin(\sqrt{\lambda} x)  
\]
\[
    X(0) = C = 0
\]
\[
    X(l) = D\sin(\sqrt{\lambda} l) = 0
\]
\[
    \sin(\sqrt{\lambda} l) = 0
\]
\[
    \sqrt{\lambda} l = \pi n, \ n \in \mathbb{N}
\]
\begin{equation}
    \begin{cases}
        \lambda_n = (\frac{\pi n}{l})^2, \ n \in \mathbb{N}\\
        X(x) = D_n \cdot \sin \frac{\pi n x}{l} 
    \end{cases}
\end{equation}

\[
    T''_n(t) + \left(\frac{a \pi n}{l}\right)^2 T_n(t) = 0
\]

\[
    T_n(t) = A_n \cos\left(\frac{a \pi n}{l} t\right) + B_n \sin\left(\frac{a \pi n}{l} t\right)
\]
\begin{equation}
    u_n(x,t) = \left(A_n \cos\left(\frac{a \pi n}{l} t\right) + B_n \sin\left(\frac{a \pi n}{l} t\right)
    \right) \sin \frac{\pi n x}{l} 
\end{equation}

2 этап завершающий\\
Решение будет искать в виде:
\[
    u(x,t) = \sum_{n=1}^{\infty} u_n(x,t) = \sum_{n=1}^{\infty} \left(A_n 
    \cos\left(\frac{a \pi n}{l} t\right) + B_n \sin\left(\frac{a \pi n}{l} 
    t\right) \right) \sin \frac{\pi n x}{l}
\]

\begin{equation}
   u |_{t=0} = \sum_{n=1}^{\infty} A_n \sin \frac{\pi n x}{l} = \phi(x)
\end{equation}

\begin{equation}
    u_t |_{t=0} = \sum_{n=1}^{\infty} \frac{\pi n x}{l}
    B_n \sin \frac{\pi n x}{l} = \psi(x) \  \big| \cdot \sin \frac{\pi m x}{l}, \int_{0}^{l} 
\end{equation}

\[
    \int_{0}^{l} \sin \frac{\pi n x}{l} \cdot \sin \frac{\pi m x}{l} dx =
    \begin{cases}
        \frac{l}{2}, &\quad n = m\\
        0, &\quad n \neq m
    \end{cases}
\]
\[
    \sum_{n=1}^{\infty} A_n \int_{0}^{l} \sin \frac{\pi n x}{l} \sin \frac{\pi m x}{l} 
    dx = \int_{0}^{l} \phi(x) \sin \frac{\pi m x}{l} dx
\]
\begin{equation}
    \begin{cases}
        A_m &= \frac{2}{l}\int_{0}^{l} \phi(x) \sin \frac{\pi m x}{l} dx\\
        B_m &= \frac{2}{l} \frac{l}{a \pi m} \int_{0}^{l} \psi(x) \sin \frac{\pi m x}{l} dx
    \end{cases}
\end{equation}
\end{document}
