\documentclass[a4paper]{article}
\usepackage[a4paper,%
    text={180mm, 260mm},%
    left=15mm, top=15mm]{geometry}
\usepackage[utf8]{inputenc}
\usepackage{cmap}
\usepackage[english, russian]{babel}
\usepackage{indentfirst}
\usepackage{amssymb}
\usepackage{amsmath}
\usepackage{mathtools}
\usepackage{tcolorbox}
\usepackage{xfrac}
\usepackage{import}
\usepackage{xifthen}
\usepackage{pdfpages}
\usepackage{transparent}
\usepackage{graphicx}
\graphicspath{ {./figures} }

\newcommand{\incfig}[1]{%
\def\svgwidth{\columnwidth}
\import{./figures/}{#1.pdf_tex}
}

\begin{document}
\title{КиМ. Лекция}
\author{daskf}
\maketitle

\underline{Свободные модуль}
\[
    _RF \rightarrow \{ f_i \}_{i \in I} \text{ - базис в F} \iff 
    \forall \ m \in F \ \exists ! \ r_{i_1}, \dots , r_{i_k} \in R \ | \ 
    m = r_{i_1}f_{i_1}+ \dots + r_{i_k}f_{i_k} 
\]
\[
    \iff F = \bigoplus_{i \in I} Rf_i \iff
\]
\[
    1) \ \{ f_i\}_{i \in I} \text{ - лнз (над R)}
\]
\[
    2) F = \sum_{i \in I}Rf_i \iff F = _R < f_i > _{i \in I}
\]

\begin{tcolorbox}
\underline{Th} R - модуль F свободен $ \iff F \cong \bigoplus_{i \in I}\  _R R $ 

\underline{Proof}
\[
    F = \bigoplus_{i \in I} Rf_i
\]
\[
    \phi: R \to R f_i: r \mapsto r f_i \text{ - гомо левых R - модулей} 
\]
\[
    r \in \ker \phi \implies \phi(r) = rf_i = 0 \implies f_i \text{ - лнз } r = 0
    \implies \ker \phi = \{0\}
\]
$ \impliedby: $ 
\[
    F \cong \bigoplus_{i \in I}\, _RR = \{ (\dots, r_k, \dots )\ | \ r_k \in R
    \text{ и почти все }r_k = 0 \}
\]
Базис:
\[
    \{ (\dots, 0, 1, 0, \dots \} \implies F \text{ - свободен}
\]
\end{tcolorbox}

\begin{tcolorbox}
\underline{Examples}\\
1) Векторные пр-ва - свободные модули\\
2) $ \mathbb{Z} $ - модули $ (\equiv $ абелевы группы)\\
Аб. гр А - свободна $ \iff A \cong \bigoplus Z $ 
\end{tcolorbox}

\begin{tcolorbox}
\underline{Предл} любой R - модуль М является эпиморфным образом некоторого свободного
R - модуля

\underline{Proof} Пусть $ \{ m_i \}_{i \in I} $ система образующих\\
Рассмотрим своб. модуль $ F = \bigoplus_{i \in I}\, _RR $ c базисом $ 
\{f_i\}_{ i \in I} $  \\

Рассмотрим
\[
    \phi: F \to M: f = r_{i_1}f_{i_1}+ \dots + r_{i_k}f_{i_k} \mapsto m = r_{i_1}m_{i_1}+
    \dots + r_{i_k}m_{i_k} \text{ - эпиморфизм}
\]

\underline{Сл} $ \forall $ R - модуль М $ \cong $ фактормодулю своб модуля

\end{tcolorbox}

\section*{\centering Вполне приводимые модули}

\begin{tcolorbox}
\underline{Def} Модуль М наз-ся простым, если он ненулевой и имеет только два 
подмодуля $ \{0\}, \ M $ 
\end{tcolorbox}

\underline{Упр} Найти все простые $ \mathbb{Z} $ - модули ($ \cong $ абелевы группы)

\begin{tcolorbox}
\underline{Def} Модуль М называется вполне приводимым, если любой подмодуль в нем
выделяется прямым слагаемым, т.е. 
\[
    \forall N \leq M: \ M = N \oplus K \text{ some K}\leq M
\]

\underline{Note} любой простой модуль вполне приводим. Обратное неверно
\end{tcolorbox}

\begin{tcolorbox}
\underline{Lemma1} Подмодули и гомоморфные образы вполне приводимых модулей 
вполне приводимы

\underline{Proof} M - вполне приводим, $ N \leq M $. N - вп. приводим? 
\[
    \forall K \leq N \implies M = K \oplus X
\]
\[
    N = N \cap M = N \cap (K \oplus X) = K + (N \cap X) = K \oplus (N \cap X)
\]

Рассмотрим:
\[
    f: M \to f(M) \implies f(M) \cong \sfrac{M}{\ker f}  
\]
С другой стороны
\[
    \ker f \leq M \implies M = \ker f \oplus Y \implies Y \cong \sfrac{M}{\ker f} 
    \cong f(M) \implies
\]
\[
    \text{ т.к Y - вп. приводим, то } f(M) - \text{ вполне приводим}
\]
\end{tcolorbox}

\begin{tcolorbox}
    \underline{Lemma2} Пусть M - R -модуль, $ \{ M_i \}_{i \in I} $ - семейство простых
    подмодулей в М, порождающих М ($ M = \sum_{i \in I} M_i $). Тогда для любого
    подмодуля $ N \leq M\ \exists J \subset I \ | \ M = N \oplus (\bigoplus
    _{j \in J} M_j)$ 
\end{tcolorbox}

\subsection*{Сведения из теории множеств}
Уже было видимо в ОСА

\begin{tcolorbox}
\underline{Proof} Рассмотрим чум $ X = \{ K \subset I \ | \ N + (\bigoplus_{k \in K}
M_k) = N \oplus (\bigoplus_{k \in K}M_k) \}$ \\
1) $ X \neq \varnothing $, т.к. $ \varnothing \in K $ \\
2) $ \{ Y_s \}_{s \in S} $ - лу подмн-во в Х $ \implies $ верхняя грань 
$ \bigcup_{s \in S} Y_{i_s} \implies $ в Х $ \exists \max\  J \implies
N \oplus (\bigoplus_{j \in J}M_j) = N'$ \\
$ N' \stackrel{?}{=} M $. Противное: $ M \neq N' $  
\[
    \implies \text{ т.к. }М = \sum_{i \in I} M_i, \text{ then } \exists M_t \ | \ 
    M_t \not\subseteq N' \implies M_t \cap N' = \{0\}
\]
Тогда $ N' + M_t = N' \oplus M_t = N \oplus (\bigoplus_{j \in J}) M_j)
\oplus M_t = N \oplus (\bigoplus_{q \in J \cup \{t\}}M_q)$. Противоречие с макс J 
\end{tcolorbox}

\begin{tcolorbox}
\underline{Th} Для R - модуля М эквивалентно:\\
1) M - сумма простых подмодулей $ M = \sum_{i \in I} M_i $ \\
2) $ M = \bigoplus_{j \in J} M_j $, $ M_j $ -прост\\
3) M - вполне приводимый модуль

\underline{Proof} $ 1) \implies 2) $ следует из леммы 2 при $ N = 0 $\\
$ 2) \implies 3) $ 
\end{tcolorbox} следует из леммы 2
\end{document}
