\documentclass[a4paper]{article}
\usepackage[a4paper,%
    text={180mm, 260mm},%
    left=15mm, top=15mm]{geometry}
\usepackage[utf8]{inputenc}
\usepackage{cmap}
\usepackage[english, russian]{babel}
\usepackage{indentfirst}
\usepackage{amssymb}
\usepackage{amsmath}
\usepackage{mathtools}
\usepackage{tcolorbox}
\usepackage{xfrac}
\usepackage{import}
\usepackage{xifthen}
\usepackage{pdfpages}
\usepackage{transparent}
\usepackage{graphicx}
\graphicspath{ {./figures} }

\newcommand{\incfig}[1]{%
\def\svgwidth{\columnwidth}
\import{./figures/}{#1.pdf_tex}
}

\begin{document}
\section*{\centering Присоединение корня многочлена к полю}

$ \mathbb{R}, \ f = x^2 + 1 $ - неприводим над $ \mathbb{R} $, корни $ \pm i $ \\
$\mathbb{C} = (\mathbb{R}, i) $ - векторное пространство с базисом $ \{ 1, i \} $  

\begin{tcolorbox}
    \underline{Th.1} Пусть К - поле, $ f(x) \in K[x]  $ - неприводим, $ \deg f > 1 $\\
    Тогда $ \exists L \supset K $, в котором $ f(x) $ имеет корень

    \underline{Proof} p(x) - неприводим в $ K[x] $ - ОГИ \\
    $ (p(x)) $ - простой идеал в $ K[x] $ 

    Действительно, пусть $ fg \in (p(x)) \implies f \cdot g = p(x)h(x) $ для 
    некоторого $ h(x) \in K[x] \implies p \mid f \lor p \mid g $ \\
    В ОГИ любой ненулевой простой идеал является максимальным $ \iff 
    \sfrac{K[x]}{(p(x))} $ - поле

    Рассмотрим 
    \[
        \rho: K[x] \to \sfrac{K[x]}{(p(x))} : f(x) \mapsto \overline{f}(x)  
        \text{ - канонический эпиморфизм}
    \]

    Рассмотрим
    \[
        \rho|_{K}: K \to \sfrac{K[x]}{(p(x))} = L \text{ - гомоморфизм полей} 
    \]
    \[
        \implies \rho|_K = 0 \lor \rho|_K - \text{ вложение}
    \]
    \[
        \rho|_K: K \to \rho|_K (x) \text{ - изоморфимз } \implies
        K \cong \text{ подмножество в L, отождествляем: } K \subset L
    \]

    Проверим, что $ \overline{x} \in L $ - корень p(x)
    \[
        p(x) = a_n x^{n} + \dots + a_1x + a_0 \in K[x]
    \]
    \[
        p(\overline{x}) = a_n \overline{x}^{n} + \dots + a_1 \overline{x} + a_0
        = \overline{a_n} \overline{x}^{n} + \dots + \overline{a_1} \overline{x} + 
        \overline{a_0} = \overline{a_nx^{n}+ \dots + a_1 x + a_0} = 
        \overline{p(x)} = 0
    \]
\end{tcolorbox}

\begin{tcolorbox}
\underline{Th.2} В условиях теоремы 1, $ L = <1, \overline{x}, \dots, 
\overline{x}^{n-1}>_K$, где $ n = \deg p(x) $, т.е. $ \{ 1, \overline{x},
\dots , \overline{x}^{n-1} \}$ - базис L над K\\
$ L = \{ a_{n-1} \overline{x}^{n-1} + \dots + a_1 \overline{x} + a_0 \ | \ a_i \in K \} $ \\
В частности $ [L:K] = n $ 

\underline{Proof}\\
1) $ L = <1, \overline{x}, \dots, \overline{x}^{n-1}>_K \quad L = \sfrac{K[x]}{(p(x))} $\\  
\[
    \forall \underbrace{f(x) + (p(x))}_{\overline{f}(x)} \in L
\]
\[
    f(x) = p(x)q(x) + r(x), \ r(x) = 0 \lor \deg r(x) < \deg p(x) = n
\]
\[
    \implies \overline{f}(x) = \underbrace{\overline{p}(x)}_{=\overline{0}}
    \overline{q}(x) + \overline{r}(x) = \overline{r}(x) = 
    \overline{b_{n-1} x^{n-1} + \dots + b_1 x + b_0} = 
    b_{n-1}\overline{x}^{n-1} + \dots + b_1 \overline{x} + b_0
\]
2) \underline{лнз}
\[
    \lambda_0 + \lambda_1 \overline{x} + \dots + \lambda_{n-1} \overline{x}^{n-1}
    = \overline{0}, \ \lambda \in K
\]
\[
    \overline{\lambda_0} + \overline{\lambda_1} \overline{x} + \dots + 
    \overline{\lambda_{n-1}} \overline{x}^{n-1} = \overline{0} \implies
    (\lambda_0 + \lambda_1 x \dots + \lambda_{n-1}x^{n-1}) + (p(x)) = (p(x)) \implies
    \lambda_0 + \lambda_1 x \dots + \lambda_{n-1}x^{n-1} \in (p(x))
\]
\[
    = p(x) \cdot K[x] \implies p(x) \mid \lambda_0 + \lambda_1 x \dots + 
    \lambda_{n-1}x^{n-1} \implies \lambda_0 + \lambda_1 x \dots + 
    \lambda_{n-1}x^{n-1} = 0 \iff \lambda_i = 0, \ i = \overline{1, n-1}
\]
\end{tcolorbox}

\begin{tcolorbox}
\underline{Example} Постороение $ \mathbb{C} $\\
$ \mathbb{R} $  - поле, $ f = x^2 + 1 \in \mathbb{R}[x] \implies \exists $ поле
L $| \  \overline{x} \in L $ - корень $ f = x^2 + 1 $ \\
Th.2 $ \implies [L:\mathbb{R}] = 2 $ с базисом $ \{ 1, \overline{x} \} \implies
\forall z \in L : \ z = a \cdot 1 + b \cdot \overline{x}, \ a,b \in \mathbb{R} $
\end{tcolorbox}

\end{document}
