\documentclass[a4paper]{article}
\usepackage[a4paper,%
    text={180mm, 260mm},%
    left=15mm, top=15mm]{geometry}
\usepackage[utf8]{inputenc}
\usepackage{cmap}
\usepackage[english, russian]{babel}
\usepackage{indentfirst}
\usepackage{amssymb}
\usepackage{amsmath}
\usepackage{mathtools}
\usepackage{tcolorbox}
\usepackage{xfrac}
\usepackage{import}
\usepackage{xifthen}
\usepackage{pdfpages}
\usepackage{transparent}
\usepackage{graphicx}
\graphicspath{ {./figures} }

\newcommand{\incfig}[1]{%
\def\svgwidth{\columnwidth}
\import{./figures/}{#1.pdf_tex}
}

\begin{document}
\title{ОСА. Лекция}
\author{Я}
\maketitle

\begin{tcolorbox}
\underline{Th.6} $ F \subset K \subset L \implies F \subset L $ - алгебраическое

\underline{Proof}
\[
    \forall \alpha \in L - \text{ алгебр над К} \implies \exists f(x) = k_0 + k_1x +
    \dots + k_nx^{n} \in K[x]\ | \ f(\alpha) = 0
\]
\[
    k_0, k_1, \dots , k_n \in K \supset F \implies k_0, k_1, \dots, k_n - \text{ алг над F}
    \implies F \subset F(k_0) \subset f(k_0)(k_1) = f(k_0, k_1) \subset \dots
\]
\[
    \subset F(k_0, k_1, \dots, k_n) \implies F \subset F(k_0, \dots k_n) \text{
    - конечное} \implies F \subset F(k_0, \dots k_n) \text{ - алг} \implies
\]
\[
    F \subset F(k_0, \dots, k_n) \subset K \text{ - алг}
\]
$ \alpha $ - алгебр над $ F(k_0, \dots , k_n) \implies F(k_0, k_1, \dots, k_n)
\subset F(k_0, k_1, \dots, k_n, \alpha) \implies F \subset F(k_0, \dots, k_n)
\subset F(k_0, k_1, \dots, k_n, \alpha) \implies F \subset F(k_0, k_1, \dots, k_n, \alpha)
$ - алг $ \implies \alpha $ алг над F $ \implies F \subset L $ - алг
\end{tcolorbox}

\section*{\centering Поле разложения многочлена}
\begin{tcolorbox}
    \underline{Def} K - поле, $ f(x) \in K[x] $ \\
    Поле $ L \supset K $ - поле разложения f(x), если f(x) раскладывается на линейные
    множители в $ L[x] $ и L - min с этим св-вом, т.е. не существует $ F \ | \ K \subset
    F \subsetneqq L$ и все корни f(x) лежат в F 
\end{tcolorbox}

\begin{tcolorbox}
    \underline{Th.1} $ \forall f(x) \in K[x]\  \exists $ поле разложения

    \underline{Proof} $ K[x] $ - факториально $ \implies f(x) = p_1(x), \dots, p_k(x) $,
    $ p_i(x) $ - непр
    Индукция по $ \deg f = n $\\
    $\underline{n = 1} \quad f(x) = ax + b \implies L = K$\\
    $ \underline{< n \text{ верно}} $\\
    \underline{n} $ p_1(x) $ - непр над К
    \[
        \implies \exists K \subset F \text{ - min} \land \alpha_1 \in F \text{ - корень } p_1(x)
        \implies K \subset \implies K \subset K(\alpha_1)
    \]
    \[
        \implies f(x) = (x - \alpha_1) g(x) \text{, где } \deg g(x) < n \implies
        \exists \text{ поле разложения } \hat{L} \text{ для g(x)} \implies K \subset \hat{L}
    \]
    \[
        \subset \hat{L}(\alpha_1) = L \text{ - поле разложения f(x)}
    \]
\end{tcolorbox}

\begin{tcolorbox}
    \underline{Lemma} $ p(x) \in K[x] $ - неприводим, $ \alpha \in L \supset K $ - 
    корень p(x)\\
    $ \phi: K \to L $ - гомо полей. Тогда гомо $ \phi $ можно продолжить до гомо\\
    $ \psi: K(\alpha) \to L $ ровно столькими способами сколько корней в L 
    имеет многочлен $ \phi(p(x)) $, полученный из p(x) применением $ \phi $ к его
    коэффициентам

    \underline{Proof} $ \phi: K \to L $ - гомо. Нужно построить $ \psi: K(\alpha) \to L $,
    $ \psi |_K = \phi $ 
    \[
        p(x) = k_0 + k_1x + \dots + k_n x^{n} ; k(\alpha) = \{ a_0 + a_1 \alpha +
        \dots + a_{n-1}\alpha^{n-1} \ | \ a_i \in K \}
    \]
    Если $ \psi \ \exists $, то
    \[
        \psi(a_0 + a_1\alpha + \dots a_n \alpha^{n-1}) = \psi(a_0) + \psi(a_1)
        \psi(alpha) + \dots + \psi(a_{n-1})\psi(\alpha)^{n-1} 
    \]
    \[
        = \phi(a_0) + \phi(a_1) \beta + \dots + \phi(a_{n-1})\beta^{n-1}
        \text{, где } \beta = \psi(\alpha) \text{ дост опр-ть}\ \psi(\alpha)
    \]
    \[
        0 = p(\alpha) = \psi(p(\alpha)) = \phi(k_0) + \phi(k_1) \beta + \dots +
        \phi(k_n)\beta^{n} \implies \beta \text{ - корень}\ \phi(p(x))
    \]
\end{tcolorbox}

\begin{tcolorbox}
\underline{Th.2} Поле разложения определенно однозначно с точность до изоморфизма над К

\underline{Proof} $ f(x) = p_1 \dots p_n $ 
\[
    K \subset K(\alpha_1) \subset \dots \subset K(\alpha_1, \dots, \alpha_n) = L
\]
\[
    \alpha_1, \dots, \alpha_n \text{ - корни непр-х многочленов}\ p_1, \dots p_n
\]
Пусть М - другое поле разложения f(x)
\[
    1) \ id = \phi_0: K \to M \text{ - гомо}
\]
\[
    2) \phi_1: K(\alpha_1) \to M \text{ - сущ-ет по лемме}
\]
\[
    3) \phi_2: K(\alpha_1)(\alpha_2) \to M
\]
\[
    \ldots\ldots\ldots\ldots\ldots
\]
\[
    n) \phi_n: K(\alpha_1, \dots, \alpha_{n-1})(\alpha_n) \to M
\]
\[
    \phi_n: L \to M \text{ - гомо}
\]
\[
    Im \phi_n \text{ содержит } \alpha_1, \dots \alpha_n \text{ и поле К и }
    Im \phi_n \subset M \implies Im \phi_n = M \implies \phi_n \text{ - изом-зм}
\]
\end{tcolorbox}

\end{document}
