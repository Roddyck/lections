\documentclass[a4paper]{article}
\usepackage[a4paper,%
    text={180mm, 260mm},%
    left=15mm, top=15mm]{geometry}
\usepackage[utf8]{inputenc}
\usepackage{cmap}
\usepackage[english, russian]{babel}
\usepackage{indentfirst}
\usepackage{amssymb}
\usepackage{amsmath}
\usepackage{mathtools}
\usepackage{tcolorbox}
\usepackage{xfrac}
\usepackage{import}
\usepackage{xifthen}
\usepackage{pdfpages}
\usepackage{transparent}
\usepackage{graphicx}
\graphicspath{ {./figures} }

\newcommand{\incfig}[1]{%
\def\svgwidth{\columnwidth}
\import{./figures/}{#1.pdf_tex}
}

\begin{document}
\title{ОСА. Лекция}
\maketitle

\[
    H \leq Aut \sfrac{F}{K} 
\]
\[
    L^{H} = \{ a \in F \ | \ \phi(a) = a \ \forall \phi \in H\} \text{ - подполе в F}
\]
\[
    \phi(a^{-1}) = a^{-1} \quad a \in L^{H}
\]
\[
    1 = \phi(1) = \phi(a \cdot a^{-1}) = \phi(a) \phi(a^{-1}) = a \phi(a^{-1})
\]

\section*{\centering Конечные поля}

\begin{tcolorbox}
\underline{Note} $ |K| = p^{n} = q \implies x^{q} - x = 0\ \forall x \in K $ \\
$ x = 0 \ + $ \\
$ U(K) = K \setminus \{ 0 \} \implies |U(K)| = q - 1 \implies \forall x_0 \in 
U(K): \ x_0^{q - 1} = 1 \implies x_0^{q} = x_0$ 
\end{tcolorbox}

\begin{tcolorbox}
\underline{Th1} Пусть p - простое. $ \implies \forall n \in \mathbb{N} \ \exists ! $ 
с точность до изоморфизма поле К из $ p^{n} $ элементов

\underline{Proof} Рассмотрим $ f(x) = x^{p^{n}} - x \in \mathbb{Z}_p[x] $ и
$ \mathbb{Z}_p \subset L = \mathbb{Z}_p(x_1, \dots x_{p^{n}}) $ - поле разложения 
f(x)\\
$ x_1, \dots x_{p^{n}} $ - различны ($ f'(x) = -1 $)\\
$ (x_i + x_j)^{p^{n}} = x_i^{p^{n}} + x_j^{p^{n}} = x_i + x_j \implies x_i +
x_j $ - корень f(x), $ i \neq j $ \\
$ \left(\frac{x_i}{x_j}\right)^{p^{n}} = \frac{x_i^{p^{n}}}{x_j^{p^{n}}} = 
\frac{x_i}{x_j}  $ - корень f(x) $ \implies \{ x_1, \dots , x_{p^{n}} \} $ - 
поле из $ p^{n} $ элементов и все эл-ты этого поля - корни f(x) $ \implies $ 
оно $ = L $ - единственное с точностью до изоморфизма 
\end{tcolorbox}

\begin{tcolorbox}
\underline{Th.2} Let L - поле, $ |L| = p^{n}. K \subset L $, где $ |K| = p^{m} $ 
- подполе $ \iff m \mid n $ 

\underline{Proof}$ \implies: $ $ K \subset L \implies $ L - в.п. над K с базисом
$ \{ \alpha_1, \dots, \alpha_s \} $ 
\[
    \implies \forall \alpha \in L: \ \alpha = k_1\alpha_1 + \dots + k_s \alpha_s
    \implies p^{n} = (p^{m})^{s} \implies m \mid n
\]
$ \impliedby: $ $ m \mid n $ 
\[
    p^{n} - 1 = (p^{m})^{s}- 1 = (p^{m} - 1) \cdot t
\]
\[
    x^{p^{n}} - x = x(x^{p^{n} - 1} - 1) = x((x^{(p^{m}-1)})^{t} - 1) =
    x(x^{p^{m} - 1} - 1)h(x) (x^{p^{m}} - x) h(x) \implies (x^{p^{m}} - x)
    \mid (x^{p^{n}} - x)
\]
\[
    \implies \text{ корни } x^{p^{m}} - x \text{ являются корями в } x^{p^{n}} - x
\]
\end{tcolorbox}

\begin{tcolorbox}
\underline{Th.3} Пусть К - поле, G - конечняа подгруппа в U(K). Тогда G - циклическая

\underline{Proof}
\[
    G = G_{p_1} \times \dots \times G_{p_s}, \text{ где } G_{p_i} = \{ g \in G 
    \ | \ O(g) = p_i^{k_i} \}
\]

Выберем элемент $ g^{\star} = (g_1^{\star}, \dots, g_s^{\star}) $, где $ O(g_i^{\star}) $ -
max in $ G_{p_i}, \ i = \overline{1, s}$ т.е $ O(g_i^{\star}) = p_i^{t_i}, \ t_i - \max
\implies O(g^{\star}) = p_1^{t_1} \dots p_s^{t_s} = q$ 
Рассмотрим
\begin{equation}
    x^{q} - 1 = e
    \label{eq:1}
\end{equation}
1) $ \forall g \in G $ - корень ур-я $ (\ref{eq:1}) $: если $ O(g) = p_1^{l_1}
\dots p_s^{l_s}$, то $ l_i \leq t_i \implies g^{q} = 1 $\\
Корней не более q штук, все степени $ g^{\star} $ - корни $ (\ref{eq:1}) $,
их q штук $ \implies \forall g \in G $ - степень $ g^{\star} \implies G = <g> $ 

\underline{Следствие} Если $ |K| = n < \infty $, то $ U(K) $ - циклическая
\end{tcolorbox}

\section*{\centering Алгебры над полями}
\begin{tcolorbox}
\underline{Def} K - поле, множество A наз-ся алгебрлой (ассоциативной), если\\
1) А - кольцо\\
2) А - векторное пр-во над К\\
3) $ \lambda(ab) = (\lambda a)b = a(\lambda b),\ \forall a,b \in A; \ \lambda \in K $ 
\end{tcolorbox}

\underline{Examples} 1) $ K \subset L $ \\
2) $ M_n(K) $\\
3) $ F(X,K), \ X \text{ - мн-во, К - поле} = \{ f: X \to K \}  $ - функции\\
4) $K[x]$

\underline{Подалгебра} - подкольцо + подпр-во

\underline{Идеал} - идеал кольца + подпр-во

\underline{факторалгебра}

\begin{tcolorbox}
\underline{Def} Ассоциативное колько с 1, в котором любой ненулевой элемент обратим
называется телом. Алгебра, являющаяся телом, называется алгеброй с делением 
\[
    Z(A) = \{ a \in A \ | \ ab=ba \ \forall b \in A \} \text{ - центр}
\]
Если А алгебра с делением, то $ Z(A) $ - поле $ \implies $ A - алгебра над центром

А - алгебра с делением над К
\[
    \forall \lambda \in K \mapsto \lambda \cdot 1 \in A
\]
\[
    \{ \lambda \cdot 1 \ | \ \lambda \in K \} \cong K
\]
\end{tcolorbox}

\begin{tcolorbox}
\underline{Утв} А - конечномерная алгебра без делителей нуля $ \implies $ А - 
алгебра с делением

\underline{Proof}
\[
    \forall 0 \neq \alpha \in A; \text{ A кон/мер} \implies 1, \alpha, \alpha^2,
    \dots, \alpha^n \text{ - лз} \implies \exists \text{ не все 0}, \ 
    \lambda_0, \lambda_1, \dots \lambda_k \in K
\]
\[
    \implies \lambda_0 + \lambda_n \alpha^n = 0 \implies \alpha \text{ - корень }
    \lambda_0 + \lambda_n \alpha^n \in K[x] \implies \alpha \text{ - алг над К} \implies
\]
\[
    \text{let } \mu_{\alpha}(x) = a_0 + a_1 x + \dots + x^{s} \in K[x],
    \text{ т.е. } a_0 + a_1 \alpha + \dots + \alpha^s = 0
\]
Если $ a_0 = 0 \implies \alpha(a_1 + a_2 \alpha + \dots + \alpha^{s-1}) = 0 \implies
a_1 + a_2\alpha + \dots \alpha^{s-1} = 0$ 
\[
    \implies a_0 \neq 0 \implies a_1 \lambda + \dots + \lambda^s = 
    - a_0 \implies \alpha(a_1 + \dots + \alpha^s) = -a_0
\]
\[
    \implies \alpha [ (a_1 + \dots + \alpha^{s-1})(-a_0)^{-1} ] = 1
    \implies [ (a_1 + \dots + \alpha^{s-1})(-a_0)^{-1} ] = \alpha^{-1}
\]
\end{tcolorbox}
\end{document}
