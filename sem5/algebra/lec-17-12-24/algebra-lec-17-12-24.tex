\documentclass[a4paper]{article}
\usepackage[a4paper,%
    text={180mm, 260mm},%
    left=15mm, top=15mm]{geometry}
\usepackage[utf8]{inputenc}
\usepackage{cmap}
\usepackage[english, russian]{babel}
\usepackage{indentfirst}
\usepackage{amssymb}
\usepackage{amsmath}
\usepackage{mathtools}
\usepackage{tcolorbox}
\usepackage{xfrac}
\usepackage{import}
\usepackage{xifthen}
\usepackage{pdfpages}
\usepackage{transparent}
\usepackage{graphicx}
\graphicspath{ {./figures} }

\newcommand{\incfig}[1]{%
\def\svgwidth{\columnwidth}
\import{./figures/}{#1.pdf_tex}
}

\begin{document}
\title{ОСА. Лекция}
\maketitle

\begin{tcolorbox}
    \underline{Утв 1} M - max лнз система в А $ \iff <M> \leq_e A $\\
    Если $ B \leq_e A $, то $ \forall \, \max $ система в В - max лнх в А

    \underline{Proof} $ \implies: $ Против: $ <M> \not\leq_e A \implies
    \exists \, a \in A \ | \ <M> \cap <a> = 0 \implies M \cup \{ a \} \text{ - лнз}$\\
    $ (n_1 m_{i_1} + \dots + n_k m_{i_k} + ta = 0 $ если $ 
    \{ m_{i_1}, \dots m_{i_k}, a\} $ - лз, то $ ta = -n_1 m_{i_1} - \dots - n_k
    m_{i_k} \in M $)  

    $ \impliedby: $ против: M - не max лнз, т.е. $ \exists a \in A \ | \ M \cup \{ a\} $ 
    - лнз $ \implies <M> \cap <a> = 0 $ 
\end{tcolorbox}

\begin{tcolorbox}
    \underline{Def} D - делимая группа, если $ nD = D \ \forall \, n \in \mathbb{Z} $.
    те $ \forall \, a \in D \ \forall n \in \mathbb{Z} \ nx = a $ - разрешима
\end{tcolorbox}

\begin{tcolorbox}
    \underline{Факт} 1) Делимые группы tf - это $ \bigoplus Q $ \\
    2) Если D - делимая и $ D \leq A \implies A = D \oplus B $ 
\end{tcolorbox}

\begin{tcolorbox}
    \underline{Утв 2} $ \forall $ абелеву группу можно вложить в делимую группу в 
    качестве подгруппы

    \underline{Proof} $ Z \hookrightarrow Q \implies F = \bigoplus Z \hookrightarrow
    \bigoplus Q$. F - свободная\\
    $ \forall $ группы A $ \exists $ эпи $ F \stackrel{\phi}{\to} A \implies
    A \cong \sfrac{F}{\ker \phi = N}  $  - подгруппа в $ \sfrac{D}{N}  $ \\
    $  \sfrac{D}{N} $ - делимая\\
    $ p: D \to D: d \mapsto d + N $ - эпи\\
    $ nx = d $ - разрешима в D
    \[
        \forall \overline{d} \in \sfrac{D}{N} \implies \overline{d} = p(d)
    \]
    \[
        p(nx) = p(d) \quad np(x) = \overline{d}
    \]

    min делимая группа, содержащая гр. А называется делимая оболочка Е ($ \exists \, ! $ 
    с точностью до изм)
\end{tcolorbox}

\begin{tcolorbox}
    \underline{Утв 3} E - делимая оболочка для А $ \iff A \leq_e E $ 

    \underline{Proof} $ \implies: $ Против $ \exists \, \langle a \rangle \leq E \ | \ 
    A \cap \langle a \rangle = 0, \ \langle a \rangle \hookrightarrow Q \leq E \implies
    E = Q \oplus D$ 

    $ \impliedby: $ Е - не делимая оболочка, то $ \exists \, D \leq E \ | \ 
    A \leq D  \implies E = D \oplus X \quad A \cap X = 0 \implies X = 0$ 
\end{tcolorbox}

\begin{tcolorbox}
    \underline{Утв 4} $ \forall $ группа ранга 1 - подгруппа в Q

    \underline{Proof} 
    \[
        r(A) = 1 \implies A \hookrightarrow E \text{ - делимая оболочка}
        \iff A \leq_e E \implies \forall \text{ max лнз сист в А - max лнз в Е по
        утв 1}
    \]
    \[
        r(E) = 1, \text{ но } r(Q) = 1 \implies E = Q
    \]
\end{tcolorbox}

\begin{tcolorbox}
\underline{Th Бэра (1904-1979)}

Группы А и В ранга 1 изоморфмы $ \iff t(A) = t(B)$ 

\underline{Proof} $ \implies: $ ясно\\
$ \impliedby: $ 
\[
    \forall a \in A , \ B \in B \implies \chi(a) = (k_1, \dots , k_n, \dots)
    \sim (l_1, \dots, l_n, \dots)
\]
\[
    \iff \sum_{i} |k_i - l_i| < \infty.
\]

Пусть отличаются на местах $ k_{i_1}, \dots, k_{i_s} $. Тогда поделим а на $ p_{i_1}^{k_{i_1}},
, \dots, p_{is}^{k_{is}} \to c \in A, \ b \text{ на }p_{i_1}^{l_{i_1}}, , \dots,
p_{is}^{l_{is}} \to d \in B $. и $ \chi(c) = \chi(d) \iff nx = c $ разр в А
$ \iff nx = d $ разрешима в B
\[
    \forall a \in A \ \exists \, m \in \mathbb{Z} \ | \ na = mc
\]
\[
    \forall b \in B \ \exists \, m \in \mathbb{Z} \ | \ nb = md
\]
\[
    \phi: A \to B: a \mapsto b \quad a \to na = mb, \ b \to nb = md
\]

$ \phi $ - гомо
\[
    \phi(a_1 + a_2) = \phi(a_1) + \phi(a_2) \quad \forall a_1, a_2 \in A
\]
\[
    n a_1 = m_1 c \quad n a_2 = m_2 c
\]
\[
    \implies n \phi(a_1 + a_2) = (m_1 + m_2) d
\]

Нужно доказать, что $ n[\phi(a_1) + \phi(a_2)] = (m_1 + m_2) d $ 
\[
    n \phi(a_1) = m_1 d + n \phi(a_2) = m_2 d
\]

\underline{Моно}
\[
    a \in \ker \phi \implies \phi(a) = 0
\]
\[
    na = mc \to n \phi(a) = md \implies d = 0
\]

\underline{Эпи}
\[
    \forall b \in B \implies nb = md \implies \phi(a) = b \text{, где } na = mc
\]
\end{tcolorbox}
\end{document}
