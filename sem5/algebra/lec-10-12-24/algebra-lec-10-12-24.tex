\documentclass[a4paper]{article}
\usepackage[a4paper,%
    text={180mm, 260mm},%
    left=15mm, top=15mm]{geometry}
\usepackage[utf8]{inputenc}
\usepackage{cmap}
\usepackage[english, russian]{babel}
\usepackage{indentfirst}
\usepackage{amssymb}
\usepackage{amsmath}
\usepackage{mathtools}
\usepackage{tcolorbox}
\usepackage{import}
\usepackage{xifthen}
\usepackage{pdfpages}
\usepackage{transparent}
\usepackage{graphicx}
\graphicspath{ {./figures} }

\newcommand{\incfig}[1]{%
\def\svgwidth{\columnwidth}
\import{./figures/}{#1.pdf_tex}
}

\begin{document}
\section*{\centering Элементы теории абелевых групп}

A - аб. группа; $ R = \mathbb{Z} \implies A - \mathbb{Z} $ - модуль

$ O(a) $ - порядок эл-та $ a \in A - \min n \in \mathbb{N} \ | \ na = 0 $ \\
$ \nexists n \implies O(a) = \infty $ 

\begin{tcolorbox}
\underline{Def} p - высота элемента $ a \in A \to \max k \in \mathbb{N} \ | \
p^{k}x = a$ разрешимо в гр A\\
Если нет то p - высота $ = 0 $ \\
Если для любого k разрешимо, то p - высота $ = \infty $ 

\underline{Example} 1) $ A = Z, \ a = 4 \in A $ \\
p - высота 4 $ \to 2^{k}x = 4 $ \\
$ k = 1 \to x = 2 $ \\
$ k = 2 \to x = 1 $ \\
$ k = 3 \to x = 1/2 \notin \mathbb{Z} $ 

\underline{Обозначение} $ h_p(a) $ - p - высота
\end{tcolorbox}

\begin{tcolorbox}
\underline{Def} Характеристика $ a \in A $ - последовательность p - высот
\[
    \chi(a) = (k_1, k_2, \dots , k_n, \dots)
\]
\end{tcolorbox}

\begin{tcolorbox}
\underline{Example} 1) $ A = Z $\\
$ 4 \in Z \implies \chi(4) = (2, 0, \dots , 0) $ 
$ 6 \in Z \implies \chi(6) = (1, 1, 0, \dots , 0) $ 

2) $ A = Q^{(p)} = \{ \frac{m}{p^{s}}  \ | \ m \in \mathbb{Z}, \ s \in \mathbb{Z} \} $ \\
\underline{$p=3$} $ \to 2 \in Q^{(p)} \to \chi(2) = (1, \infty, 0, \dots ) $ 

3) $ A = Q_{p} = \{ \frac{m}{n}  \ | \ (n,p) = 1 \} $ \\
\underline{$p=3$} $ \to 15 \in Q_{p} \to \chi(15) = (\infty, 1, \infty, \dots ) $ 

4) $ A = Q $\\
$ \forall a \in A \to \chi(a) = (\infty, \dots, \infty, \dots) $ 
\end{tcolorbox}

\begin{tcolorbox}
\underline{Св-ва $ \chi $}

1) $ \chi(a) = \chi(-a) $ 

2) $ \chi(a) = (k_1, k_2, \dots k_n, \dots), \ m = p_1^{l_1}\dots p_s^{l_s} \in \mathbb{Z} $ \\
$ m \mid a \iff l_i \leq k_i \ \forall i = \overline{1,s} $ 

\underline{Proof} $ \implies: $ Против $ l_1 > k_1 $, т.е. ур-е $ p_1^{l_1}x = a $ 
неразрешимо
\[
    mx = a \implies p_1^{l_1} \cdot (p_2^{l_2} \cdot \dots \cdot p_s^{l_s}) x
    = a \text{ - разрешимо с реш }x_0 \implies p_1^{l_1}y_0 = a, \ y_0 = p_2^{l_2}
    \dots p_s^{l_s} \cdot x_0 \in A
\]
\[
    p_1^{l_1}x = a \text{ разрешимо}
\]

$ \impliedby: $ Индукция по s\\
$s = 1$ $ p_1^{l_1}x = a $ - разрешимо\\
$ < s $ верно
\[
    \underbrace{p_1^{l_1}}_{n}\underbrace{p_2^{l_2} \dots p_s^{l_s}}_{t} x = a
    \text{ - разрешимо?}
\]
\[
    (n,t) = 1 \iff un + vt = 1 \text{ for some } u,v \in \mathbb{Z}
\]
\[
    nx = a \text{ - разр, решение }a'
\]
\[
    tx = a \text{ - разр, решение }a''
\]
\[
    nt(ua'' + v a') = nu \cdot (ta'') + tv(na') = (un + vt)a = a \implies
    y_0 \text{ решение ур-я } (nt)x = a
\]

3) $ \chi(a) = (k_1, k_2, \dots, k_n, \dots \implies \chi(p_n a) = (k_1, \dots,
k_n + 1, \dots)$ 

\end{tcolorbox}

\[
    \chi_1 = (k_1, \dots, k_n, \dots)
\]
\[
    \chi_2 = (l_1, \dots, l_n, \dots)
\]
\[
    \chi_1 \geq \chi_2 \iff k_i \geq l_i \ \forall i
\]
наименьший $ \chi = (0, \dots, 0, \dots) $ \\
наибольший $ \chi = (\infty, \dots, \infty, \dots) $ 

\begin{tcolorbox}
\underline{Note} любую последовательность $ \chi = (k_1, \dots, k_n, \dots) $ 
неотр-х чисел и символом $ \infty $ будет хар-кой некоторого элемента t.f. группы,
а именно числа 1 в подгруппе группы Q, порожденной эл-ми $ p_i^{-k_i}, \ i = 1,
2, \dots$ 
\end{tcolorbox}

\begin{tcolorbox}
\underline{Def} 2 хар-ки $ \chi_1 \sim \chi_2 $ - эквивалентны, если
\[
    \sum_{i} |k_i - l_i| < \infty \ (\infty - \infty = 0)
\]

$ \sim $ - отношение эквив-ти на мн-ве характеристик

Класс эквтв-ти называется типом $ t(a) $ 
\[
    t(a) = \{ \chi(b) \sim \chi(a) \ | \ b \in A \}, \ \chi(a) \text{ - представитель }
    t(a)
\]
\end{tcolorbox}

\begin{tcolorbox}
\underline{Def} A - t.f.; $ a_1, \dots, a_n $ - лнз (над $ \mathbb{Z} $), 
если из $ n_1 a_1 + \dots + n_k a_k = 0 \implies n_i = 0,\ i = \overline{1,k} $ 

Если M $ - \infty $ сист. эл-тов из А, то M - лнз, если лнз любая её конечная подсистема

Мощность max лнз системы эл-тов в А, наз-ся её рангом, об $ r(A) $ 
\end{tcolorbox}

\begin{tcolorbox}
\[
    r(A) = 1 => \forall a_1, a_2 \in A: \ n_1 a_1 = n_2 a_2 \stackrel{3)}{\implies}
    \chi(a_1) \sim \chi(n_1 a_1) \ \chi(a_2) \sim \chi(n_2 a_2) \implies
\]
\[
    \chi(a_1) \sim \chi(a_2) \implies t(a_1) = t(a_2)
\]
\[
    t(A) \text{ - тип группы ранга 1}
\]
\end{tcolorbox}

\begin{tcolorbox}
\underline{Example} 1) $ t(Z) = (0, 0, \dots, 0, \dots) $\\
2) $ t(Q^{(p)}) = (0, 0, \dots, 0, \infty, 0, \dots) $ \\
3) $ t(Q_{p}) = (\infty, \infty, \dots, \infty, 0, \infty, \dots) $ \\
4) $ t(Q) = (\infty, \dots, \infty, \dots) $ 
\end{tcolorbox}

\begin{tcolorbox}
\underline{Def} $ B \leq A $ - называется существенная, если $ B \cap C \neq 0
\ \forall 0 \neq C \leq A$  

Обозначение $ B \leq_e A $ 
\end{tcolorbox}

\begin{tcolorbox}
\underline{Example} $ Z \leq_e Q $ 
\[
    \forall 0 \neq B \leq Q, \ \forall b = \frac{m}{n}  \in B \implies
    nb = m \in Z \implies nb \in Z \cap B
\]
\end{tcolorbox}
\end{document}
