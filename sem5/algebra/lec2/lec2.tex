\documentclass[12pt]{article}
\usepackage[a4paper,%
    text={180mm, 260mm},%
    left=15mm, top=15mm]{geometry}
\usepackage[utf8]{inputenc}
\usepackage{cmap}
\usepackage[english, russian]{babel}
\usepackage{indentfirst}
\usepackage{amssymb}
\usepackage{amsmath}
\usepackage{mathtools}
\usepackage{tcolorbox}
\usepackage{xfrac}
\usepackage{tikz-cd}

\begin{document}
Нуль в $ \sfrac{R}{I}$ есть I\\
Единица в $ \sfrac{R}{I} $ R - кольцо, $ I \lhd R $ \\
$ \sfrac{R}{I} = \{ r + I \; |\; r\in R\} \text{ - } \overline{r}$- факторколько с оп:
$ \forall r,s \in R (r+I)  + (s + I) = (r+s) + I$ \\
$ (r+I) (s + I) = rs + I $\\


2) F - поле, $F[x]$ - Кги, $ I = f(x)F[x]= (f(x)) $\\
$\sfrac{F[x]}{(f(x))}$ - Поле $\Leftrightarrow f(x) $ неприводим над F

\textbf{\underline{Proof}} $1)\Rightarrow$: Против: $f=gh$, где $\deg g,h < \deg f \Rightarrow
g + (f) \neq (f) \quad (f) = f\cdot F[x], \; h + (f) \neq (f)$ \\
$ (g+(f))(h+(f)) = gh + (f) = f+ (f) = (f) \Rightarrow g + f \text{ и } h + (f) $
- делители нуля в $\sfrac{F[x]}{(f)} $ - поле. Противоречие $\Rightarrow f$ неприводим в
$F[x]$\\
$2) \Leftarrow: \forall g + (f) \neq (f) $ f - неприводим $\Rightarrow g \nmid f
\Rightarrow \exists u,v \in F[x] \; | \; uf+gv = 1 \Rightarrow uf + gv + (f) = 
1 + (f) \Rightarrow (uf + (f)) + (gv + (f)) = 1 + (f) \Rightarrow (u + (f))(f+(f))
+ (g + (f)) (v + (f)) = 1 + (f) \Rightarrow (g + (f))(v+(f)) = 1 + (f) \Rightarrow
v + (f) = (g + (f))^{-1}$ \\
$ uf + gv = 1 \Rightarrow \overline{u}\overline{f} + \overline{g}\overline{v} = 
\overline{1} $ \\
Например,
\[
    \sfrac{\mathbb{R}[x]}{(x^2+1)} \cong \mathbb{C}
\]
\\
\textbf{\underline{$ F = \mathbb{Z}_2 $}} $ = \{\overline{0}, \overline{1}\} $

$ x^2 + x + \overline{1} $ - неприводим над $\mathbb{Z}_2$
$ \mathbb{Z}[x] / (x^2 + x + 1) $ - поле из 4 элементов: $\{ \overline{0}, 
\overline{1}, \overline{x}, \overline{x+1} \} \quad \overline{x} = x+(x^2+x+1)$ \\
$ \overline{x^5 + x^3 + 1} = (\overline{x^2 + x + 1})(\overline{x^3 - x^2 +x}) +
\overline{1 - x} $

\section*{\centering Гомоморфизмы колец}
R, S - кольца

\textbf{\underline{Def.}} $ f: R \to S$ - гомо-зм, если \\
1) $f(a+b) = f(a) + f(b) $\\
2) $ f(ab) = f(a)f(b) $\\
3) $ f(1) = 1 $\\
$ \ker f = \{ r\in R \; | \; f(r) = 0 \} \lhd R \\
Imf = \{ s \in S \; |\; s = f(r) \text{ для нек-го } r \in R \} - \text{подколько в S}\\
$
Если f - биекция $\Rightarrow $ f - изоморфизм
    
$\rho : R \to \sfrac{R}{I} $ -каноническйи гомоморфизм с $\ker \rho = I$

\textbf{\underline{Th.}}(об изоморфизме) \\
Пусть $f : R \to S $ - сюрьективный гомоморфизм(эпиморфизм) $\Rightarrow S \cong
\sfrac{R}{\ker f} $

\textbf{\underline{Proof}} 
\[
\begin{tikzcd}
    R \arrow{rr}{f} \arrow[swap]{dr}{\rho} & & S \\[10pt]
    & \sfrac{R}{\ker f} \arrow[swap]{ur}{\psi}
\end{tikzcd}
\]
 $ f =\psi\rho $\\
Строим $\psi : R/ \ker f \to S, \psi(r+\ker f) = f(r) $ \\
\textbf{\underline{Корр-ть}} Если $r + \ker f = r' + \ker f, \text{then } \psi(r' + \ker f) =
f(r') \stackrel{?}{=} f(r) $
$ r' = r+x $, где $x\in \ker f$
\[
    \psi(r' + \ker f) = f(r')  =f(r+x) = f(r) + f(x) = f(r)
\]

$\psi \text{ гомо-зм } $:
\[
    \psi( (r+ \ker f)+(r' + \ker f)) = \psi(r + r' + \ker f) = f(r + r') =
    f(r) + f(r') = \psi(r+ \ker f) + (r' + \ker f)
\]
\textbf{\underline{Иньективность}} $\psi \text{ - иньективен } \Leftrightarrow \ker \psi = 0$ \\
$ r + \ker \psi \in \ker\psi \Rightarrow \psi(r + \ker f) = f(r) = 0 \Rightarrow 
r\in\ker f \Rightarrow r + \ker f = \ker f = \overline{0}$ \\
\textbf{\underline{Сюрьективность}}  \\
$ \forall s \in S \stackrel{\text{f - эпи}}{\Rightarrow} s = f(r) = \psi(r + \ker f)
\Rightarrow \psi \text{-эпи}$

\textbf{\underline{Th.}}(о соответствии) \\
Пусть $f : R \to S$ - эпи $\Rightarrow \exists $ вз/одноз соотв-е между идеалами
в S и идеалами в R, содержащими kerf

\textbf{\underline{Proof}} Th об изо $\Rightarrow S \cong \sfrac{R}{\ker f}$, поэтому рассмотрим
$ \rho: R \to \sfrac{R}{I}, \text{ где } I = \ker f$ \\ 
$ \Omega(R, I) $ - все идеалы в R, содержащие I\\
$ \Omega(\sfrac{R}{I}) $ - все идеалы в \sfrac{R}{I}\\
$ \psi: \Omega(R,I) \to \Omega(\sfrac{R}{I})$ \\
$ I \subseteq J \mapsto \rho(J) = \{ j + I \; |\; j\in J \} \lhd \sfrac{R}{I} $ \\
1) $ j_1 + I, j_2 + I \in \rho(J) \\
(j_1 + I) + (j_2 + I) = (j_1 + j_2 ) + I \in \rho(J) $ \\
2) $ j + I \in \rho(J), r + I \in \sfrac{R}{I} \Rightarrow (j + I)(r + I) = jr + I \in \rho(J) $\\

\textbf{\underline{Иньективность}} $ \psi(J_1) = \psi(J_2) \Rightarrow J_1 + I = J_2 + I
\Rightarrow \Rightarrow J_1 = J_2 + I = J_2$ \\

\textbf{\underline{Сюрьективость}} $\forall \overline{K} \in \Omega(\sfrac{R}{I})$, где $\overline{K}
= \{ k + I \; |\; k\in K \}$\\
Then $\psi(K) = \overline{K} $ \\
$ \forall k_1, k_2 \in K: (k_1 + I) + (k_2 + I) = (k_1 + k_2) + I \in \overline{K} 
\Rightarrow k_1 + k_2 \in K$ \\
$ \forall k\in K, r \in R: (k+I)(r+I) = kr + I \in \overline{K} \Rightarrow
kr \in K $ \\
Then $K \lhd R$


\end{document}

