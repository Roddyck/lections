\documentclass[a4paper]{article}
\usepackage[a4paper,%
    text={180mm, 260mm},%
    left=15mm, top=15mm]{geometry}
\usepackage[utf8]{inputenc}
\usepackage{cmap}
\usepackage[english, russian]{babel}
\usepackage{indentfirst}
\usepackage{amssymb}
\usepackage{amsmath}
\usepackage{mathtools}
\usepackage{tcolorbox}
\usepackage{xfrac}
\usepackage{import}
\usepackage{xifthen}
\usepackage{pdfpages}
\usepackage{transparent}
\usepackage{graphicx}
\graphicspath{ {./figures} }

\newcommand{\incfig}[1]{%
\def\svgwidth{\columnwidth}
\import{./figures/}{#1.pdf_tex}
}

\begin{document}
\title{ОСА. Лекция}
\author{he's an idiot}
\maketitle

\begin{tcolorbox}
\underline{Proof} (continuation)

Рассмотрим $ \mathcal{A}' = \{ u \in \mathcal{A} \ | \ u^2 \leq 0 \} $ - подпространство
в $ \mathcal{A} $ 
\[
    W \leq V_k:
\]
\[
    1)\  w_1 + w_2 \in W \quad \forall w_1, \ w_2 \in W
\]
\[
    2) \ \forall \lambda \in K \ \forall w \in W: \ \lambda w \in W
\]
\[
    2) \ \forall \lambda \mathbb{R}, \ \forall u \in \mathcal{A}' : \quad
    (\lambda u)^2 = \lambda^2 \cdot u^2 \leq 0 \implies \lambda u \in \mathcal{A}
\]
\[
    1) \ \forall \ u,v \in \mathcal{A}' \implies u + v \in \mathcal{A}' \, ?
\]

a) Пусть $ u = \lambda v $ for some $ \lambda \in \mathbb{R} $ 
\[
    (u + v)^2 = (\lambda v + v)^2 = (1 + \lambda)^2 v^2 \leq 0 \implies u + v \in
    \mathcal{A}'
\]

б) $ u,v $ - лз $ \implies u = \alpha v + \beta $ 
\[
    u^2 = (\alpha v + \beta)^2 = \alpha v^2 + 2 \alpha \beta v + \beta^2 \in \mathbb{R}
    \implies \alpha = 0 \lor \beta = 0
\]

в) $ u, v $ - лнз $ \rightarrow \{ 1, u, v \} $ - лнз; $ u + v \notin \mathbb{R},
u - v \notin \mathbb{R}$\\
$ \mathbb{R} \subset \mathcal{A} $ - конечное расширение $ \implies \mathbb{R} \subset 
\mathcal{A} $ - алгебраическое $ \implies u + v, \ u - v $ - алгебраичны над $ \mathbb{R} $ 
$ \implies \deg \mu_{u+v} = \deg \mu_{u-v} = 2 \implies (u+v)^2 = p(u+v) + q, \ 
(u-v)^2 = r(u-v)+s$ 
\[
    (u + v)^2 = u^2 + uv + vu + v^2 = p(u+v) + q
\]
\[
    (u - v)^2 = u^2 - uv - vu + v^2 = r(u-v) + s
\]
\[
    (p+r)u + (p-r)v + 2u^2 + 2v^2 - q - s) = 0 \text{ - лин комбинация } \{ 1, u, v \}
\]
\[
    \begin{cases}
        p + r = 0\\
        p - r = 0
    \end{cases}
    \implies p = r = 0 \implies (u+v)^2 = q \in \mathbb{R} \implies q \leq 0 \implies
    u + v \in \mathcal{A}
\]
\[
    \forall u \in \mathcal{A}': \ u^2 = -q(u) \leq 0, \text{ where } q(u) \geq 0
\]

Рассмотрим $ \forall u,v \in \mathcal{A}': f(u,v) = q(u+v) - q(u) - q(v) = 
-(u+v)^2 + u^2 + v^2 = -u^2 - uv -vu - v^2 + u^2 + v^2 $ 
\[
    f(u,v) = -(uv + vu)
\]

$ f(u,v) $ - скалярное произведение на $ \mathcal{A}': $ \\
1) $ f(\lambda u, v) = - (\lambda uv + v \lambda u) = \lambda \cdot (-(vu+uv)) =
\lambda f(u,v)$\\
2) $ f(u+w, v) = -((u+w)v + v(u+w)) = -(uv + wv + vu +vw) = -(uv+vu) -(wv + vw) =
f(u,v) + f(w,v)$ \\
3) $ f(u,v) = f(v,u) $ \\
4) $ f(u,u) = -(u^2 + u^2) = -2 u^2 \geq 0 $ 

Первый случай $ [\mathbb{R}_{\mathbb{R}}: \mathbb{R}_{\mathbb{R}}] = 1 \implies
 \mathcal{A} = \mathbb{R} $ 

Второй случай $ \mathcal{A}' \neq \{ 0 \} \implies \exists u \in \mathcal{A}': 
\ u^2 = -r < 0, \text{ где } r > 0 \implies \frac{u^2}{r} = -1 \implies 
\left( \frac{u}{\sqrt{r} } \right)^2 = -1 \implies i^2 = -1$ 

Если $ \mathcal{A}' = \mathbb{R} \cdot i \implies \mathcal{A} = \mathbb{C} $ 

Третий случай $ \mathbb{R} \cdot i \subsetneqq \mathcal{A}' \implies \mathcal{A}' = \mathbb{R} i
\oplus \mathbb{R}_{i}^{\perp}$, пусть $ j \in \mathbb{R}_{i}^{\perp} \land j^2 = -1$  
\[
    0 = f(i,j) = -(ij + ji) \implies ij = -ij
\]
Обозначим $ ij = k $ 
\end{tcolorbox}

\begin{tcolorbox}
\[
    f(k,i) = -(ki + ik) = -(iji + iij) = -(-i^2 j + i^2 j) = 0 \implies
    k \perp i
\]
\[
    f(k,j) = -(kj + jk) = - (ijj + jij) = -(ij^2 - ij^2) = 0 \implies k \perp j
\]
\[
    \implies \{ 1, i, j, k \} \text{ - базис } \mathcal{A} \text{ над }\mathbb{R}
\]
\[
    k^2 = ijij = -i^2j^2 = -1 \implies \mathcal{A} = \mathbb{H}
\]

Четвертый случай $ \mathbb{H} \subsetneqq \mathcal{A} $ 
\[
    \implies \mathcal{A}' = (\mathbb{R}i + \mathbb{R}j  + \mathbb{R}k) \oplus
    (\mathbb{R}i + \mathbb{R}j  + \mathbb{R}k)^{\perp}
\]
\[
    \implies \exists l \in (\mathbb{R}i + \mathbb{R}j  + \mathbb{R}k)^{\perp}
    \ | \ l^2 = -1 \land l \perp i, \ l \perp j, \ l \perp k \implies 
    li = -il, \ lj = -jl, \ lj = -jl, \ lk = -kl
\]
\[
    lk = l(ij) = (li)j = (-il)j = -i(lj) = -i(-jl)  = (ij)l = kl
\]
\end{tcolorbox}
\end{document}
