\documentclass[a4paper]{article}
\usepackage[a4paper,%
    text={180mm, 260mm},%
    left=15mm, top=15mm]{geometry}
\usepackage[utf8]{inputenc}
\usepackage{cmap}
\usepackage[english, russian]{babel}
\usepackage{indentfirst}
\usepackage{amssymb}
\usepackage{amsmath}
\usepackage{mathtools}
\usepackage{tcolorbox}
\usepackage{import}
\usepackage{xifthen}
\usepackage{pdfpages}
\usepackage{transparent}
\usepackage{graphicx}
\graphicspath{ {./figures} }

\newcommand{\incfig}[1]{%
\def\svgwidth{\columnwidth}
\import{./figures/}{#1.pdf_tex}
}

\begin{document}
\title{Алгебра. Лекция}
\author{AAAAAAAA}
\maketitle

$ \mathcal{A} $ - алгебра над полем К

\underline{Example} 1) $ \mathbb{R}_{\mathbb{Q}} $ - алгебра с делением\\
2) $ \mathbb{C}_{\mathbb{R}} $ - двумерная алгебра с делением над $ \mathbb{R} $ \\
3) $ K \subset L \implies L $ - алгебра с делением над К

\section*{\centering Алгебра кватернионов}
\[
    \mathbb{H} = <i, j \ | \ i^2 = j^2 = -1, \ ij = -ji >
\]
Как векторное пр-во над $ \mathbb{R} $ с базисом $ \{ 1, i, j, k \}, \ k = ij \implies
\forall \, q \in \mathbb{H}: q = a + bi + cj + dk, \ a,b,c,d \in \mathbb{R}$ 

\begin{equation*}
    \begin{aligned}
        1 &\quad i &\quad j &\quad k\\
        1 &\quad i &\quad j &\quad k \\
        i &\quad -1 &\quad k &\quad -j \\
        j &\quad -k &\quad -1 &\quad i \\
        k &\quad j &\quad -i &\quad -1
    \end{aligned}
\end{equation*}
\[
    \overline{q} = a - bi - cj - dk \text{ - сопряженный}
\]
\[
    \overline{q_1 + q_2} = \overline{q_1} + \overline{q_2}
\]
\[
    \overline{q_1q_2} = \overline{q_2} \overline{q_1} \text{ - проверим на базисе}
\]
\[
    \overline{ij} = \overline{k} = -k
\]
\[
    \overline{j}\cdot \overline{i} = (-j)(-i) = ji = -k
\]
\[
    \overline{ik} = \overline{-j} = j
\]
\[
    \overline{k}\cdot\overline{i} = (-k)(-i) = ki = j
\]
\[
    \overline{jk} = \overline{i} = -i
\]
\[
    \overline{k} \cdot \overline{j} = kj = -i
\]
\[
    q = a + bi + cj+ dk \neq 0 \iff (a,b,c,d) \neq (0,0,0,0)
\]
\[
    q \overline{q} = (a + bi + cj + dk)(a - bi - cj - dk) = a^2 - abi - acj -
    adk + abi + b^2 - bck + bdj + acj + bck + c^2 - cdi + adk - dbj + cdi + d^2
\]
\[
    = a^2 + b^2 + c^2 + d^2 = N(q) -\text{ норма}
\]
\[
    q \overline{q} = N(q) \ | \ : N(q) \implies q \left(\frac{\overline{q}}{N(q)} 
    \right) = 1 \implies q^{-1} = \frac{\overline{q}}{N(q)} \implies
    \mathbb{H} \text{ - алгебра с делением }
\]
\[
    N(q_1 q_2) = N(q_1) N(q_2)
\]
\[
    N(q_1 q_2) = q_1 q_2 \overline{q_1q_2} = q_1 q_2 \overline{q_2} \cdot \overline{q_1}
    = q_1 N
\]

\begin{tcolorbox}
    \underline{Th} $ \mathcal{A} $ - к/мер алгебра над К\\
    1) $ \forall a \in \mathcal{A} $ алгебраичен над К. В частности, существует
    $ \mu_a(x) in K[x] $ - min мн-н\\
    2) $ a \in \mathcal{A} \text{ обратим } \iff \mu_a(0) \neq 0 $ \\
    3) если А - без делителей нуля $ \implies A $ - алгебра с делением

    \underline{Proof} 1) $ \dim_k A = n < \infty \implies 1, a, a^2, \dots a^{n} $ 
    - лз над К $ \implies \exists $ не все равные нулю $ k_0, \dots k_n \in K | \
    \ k_0 + \dots k_n a^{n} = 0 \implies a $ - корень $ f(x) = k_0 + \dots + k_n x^{n}
    \in K[x] \implies a $ алгеб-н над К

    2) $ \implies: $ 
    \[
        \mu_a(x) = k_1 + k_2 x + \dots + k_m x^{m} \in K[x].
    \]
    Допустим противное: $ \mu_a(0) = 0 \iff k_1 = 0 $ 
    \[
        \mu_a(a) = k_2 a + \dots + k_m a^{m} = 0 \implies a(k_2 + \dots + 
        k_m a^{m-1}) = 0 \ | \ \cdot a^{-1}
    \]
    \[
        \implies k_2 + \dots + k_m a^{m-1} = 0 \implies g(a) = 0, \text{ где }
        g(x) = k_2 + \dots k_m x^{m-1} \in K[x], \ \deg g(x) < \deg \mu_a (x)
    \]

    $ \impliedby \ \mu_a \neq 0 \iff k_1 \neq 0$ 
    \[
        k_1 + k_2a + \dots + k_m a^{m} = 0 \implies
        k_2 a + \dots k_m a^{m} = - k_1 \ | \ \cdot(-k_1)^{-1}
    \]
    \[
        a \cdot [(k_2 + \dots k_m a^{m-1}) \cdot (-k_1)^{-1}] = 1
    \]

    3) доказано 
\end{tcolorbox}

\begin{tcolorbox}
\underline{Th} (Фробениус, 1877)\\
Существует точно 3 ассоциативные конечномерные алгебры с делением над $ \mathbb{R}:
\mathbb{R}, \ \mathbb{C} $ и $ \mathbb{H} $ 

\underline{Proof} Let A - конечномерная ассоциативная алгебра с делением над 
$ \mathbb{R} $. Определим вид её элементов
\[
    \forall a \in A \text{ - алгебраичен над } \mathbb{R}, \text{ let }
    \mu_a(x) \in \mathbb{R}[x] \text{ - мин мн-н - неприводим над } \mathbb{R}
\]
\[
    \implies \mu_a(x) = x - a \lor \mu_a(x) = x^2 - 2 \alpha x + \beta, \ 
    D = 4 \alpha^2 - 4 \beta = 4(\alpha^2 - \beta) < 0
\]
\[
    b \stackrel{\text{df}}{=} a - \alpha
\]
\[
    \mu_a(a) = a^2 - 2 \alpha a + \alpha^2 - \alpha^2 + \beta = (a - \alpha)^2
    + (\beta - \alpha^2) = 0 \implies b^2 = \alpha^2 - \beta < 0
\]
Две возможности для $ a \in A $\\
1) $ a \in \mathbb{R} $ \\
2) $ a = \alpha + b $, где $ \alpha \in \mathbb{R}, \ b^2 < 0 $ 
\end{tcolorbox}
\end{document}
