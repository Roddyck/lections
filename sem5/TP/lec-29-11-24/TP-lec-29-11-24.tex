\documentclass[a4paper]{article}
\usepackage[a4paper,%
    text={180mm, 260mm},%
    left=15mm, top=15mm]{geometry}
\usepackage[utf8]{inputenc}
\usepackage{cmap}
\usepackage[english, russian]{babel}
\usepackage{indentfirst}
\usepackage{amssymb}
\usepackage{amsmath}
\usepackage{mathtools}
\usepackage{tcolorbox}
\usepackage{import}
\usepackage{xifthen}
\usepackage{pdfpages}
\usepackage{transparent}
\usepackage{graphicx}
\graphicspath{ {./figures} }

\newcommand{\incfig}[1]{%
\def\svgwidth{\columnwidth}
\import{./figures/}{#1.pdf_tex}
}

\begin{document}
\title{ТВиМС. Лекция}
\author{aaaaa fuck}
\maketitle

\section*{\centering Применение ТВ}
Гуденко БВ, КТВ, 2005 с. 152-153

\underline{Задача 1}

n - станков\\
$ L = (n-1)a $ \\
Средняя длина пути?
\[
    \lambda_{i}^{(k)}  =
    \begin{cases}
        (k-i)a, \quad k \geq i\\
        (i-k)a, \quad k < i
    \end{cases}
\]
\[
    E(\lambda | B_k) = \frac{a}{n} \left( \sum_{k=1}^{k} (k-i) + 
    \sum_{j = k + 1}^{n} (j-k) \right) = \frac{a}{n} \left( \frac{(k-1)k}{2} 
    + \frac{(n-k)(n-k+1)}{2} \right)
\]
\[
    E(\lambda) = \sum_{k=1}^{n} E(\lambda | B_k) \cdot P(B_k) = 
    \frac{a}{n^2} \left(\frac{1}{2} \sum_{k=1}^{n} (k^2 - k) + \frac{1}{2} \sum_{k=1}^{n}
        (n - k)^2 + \frac{1}{2} \sum_{k=1}^{n} (n-k) \right)
\]
\[
    \frac{1}{2} \sum_{k=1}^{n} k^2 = \frac{n(n+1)(2n+1)}{6} 
\]
\[
    E(\lambda) - \frac{a (n^2 - 1)}{3n} = \frac{L(n+1)}{3n} = \frac{2}{3} 
    \left(1 + \frac{1}{n}\right) \approx \frac{2}{3} 
\]

\underline{Задача 2}

\[
    d_2 > d_1
\]
\[
    X \in N\left(\frac{d_1 + d_2}{2}, \alpha^2(d_2 - d_1)^2\right)
\]
Каким надо выбрать $ \alpha $, чтобы вероятность события $ P(d_1 < X < d_2) = 0.99 $ 

\[
     P(d_1 < X < d_2) = P\left(d_1  - \frac{d_1 + d_2}{2} < X - \frac{d_1 + d_2}{2}
    d_2  - \frac{d_1 + d_2}{2}\right)
\]
\[
     = P\left( -\frac{d_2 - d_1}{2} < X - \frac{d_1+d_2}{2}  
    < \frac{d_2 - d_1}{2}\right) = P \left( \frac{-1}{2 \alpha} < 
    \frac{X - \frac{d_1+d_2}{2} }{\alpha(d_2 - d_1} < \frac{1}{2 \alpha} \right)
\]
\[
    \Phi(\frac{1}{2 \alpha} ) - 1 + \Phi(\frac{1}{2 \alpha} ) = 
    2 \Phi(\frac{1}{2 \alpha} ) - 1 = p
\]
Вер. Брако (Драко втф)?????
\[
    1 - p = q = 2 - 2 \Phi\left(\frac{1}{2 \alpha} \right)
\]

б) Вер брака $ q = 0,02 $ 
\[
    2 - 2 \Phi \left( \frac{1}{2 \alpha} \right) = 0.02
\]
\[
    \text{ И тут он стёр (ХУЙ, кто хуй? Ты?)}
\]

\end{document}
