\documentclass[a4paper]{article}
\usepackage[a4paper,%
    text={180mm, 260mm},%
    left=15mm, top=15mm]{geometry}
\usepackage[utf8]{inputenc}
\usepackage{cmap}
\usepackage[english, russian]{babel}
\usepackage{indentfirst}
\usepackage{amssymb}
\usepackage{amsmath}
\usepackage{mathtools}
\usepackage{tcolorbox}
\usepackage{import}
\usepackage{xifthen}
\usepackage{pdfpages}
\usepackage{transparent}
\usepackage{graphicx}
\graphicspath{ {./figures} }

\newcommand{\incfig}[1]{%
\def\svgwidth{\columnwidth}
\import{./figures/}{#1.pdf_tex}
}

\begin{document}
\title{ТВиМС. Лекция}
\maketitle

Монета $ A = \{ 1 \} $

$ (\Omega, \mathcal{F}, P) $ $ A \subset \mathcal{F} $  
\[
    \Omega \to (R^1, \mathcal{B}, P^x)
\]
\[
    \Omega \in \mathcal{A} \quad A \in \mathcal{A} \to \overline{A} \in \mathcal{A}
\]
\[
    A_1, A_2, \dots A_n \in \mathcal{A}
\]
\[
    \bigcup_{i=1}^{\infty} A_i \in \mathcal{A}
\]

\textbf{Вероятность по Колмагорову}
\[
    P: A \in \mathcal{F} \to P(A)  
\]
\[
    1) \ P(A) \geq 0
\]
\[
    2) \ P(\Omega) = 1
\]
\[
    3) \ A, B \quad A \cdot B = \varnothing
\]
\[
    P(A \cup B) = P(A) + P(B)
\]

Случайная величина
\[
    X: \ \omega \to X(\omega)
\]
\[
    \{ \omega: \ X(\omega) \in B \} \subset \mathcal{F} \quad B \in \mathcal{B}
\]
\[
    \{ \omega: \ X(\omega) < x \} \in \mathcal{F}
\]
\[
    P^x(B) = P\{ \omega: \ X(\omega) \in B)
\]
\[
    P^x( (-\infty, x)) = F(x)
\]
\[
    \forall x \in R^1
\]
\[
    P(X < x) = F(x)
\]

Дискретная случайная величина
\[
    x_1, x_2, \dots x_n
\]
\[
    p_i = P(X = x_i)
\]
\[
    1) \ p_i \geq 0
\]
\[
    2) \ \sum_{i=1}^{n} p_i = 1
\]

Условная вероятность:
\[
    P(A|B) = \frac{P(A \cdot B}{P(B)}, \ P(B) > 0 
\]

$ X $ - непр. с.в.
\[
    E(X) = \int_{-\infty}^{\infty} xf(x) dx
\]
\[
    E(X^2) = \int_{-\infty}^{\infty} x^2 f(x) dx
\]

\textbf{Закон больших чисел:}\\
Частота ...

В теории вероятности: \\
Функия распределения известна $ F(x) $ 
\[
    P(a \leq X < b)
\]

В мат стате $ (x_1, x_2, \dots, x_n) \to F(x)? $ 
\[
    P(X < x) = F(x) \text{ ТВ}
\]
\[
    h_N(X<x) = F_N(x) \text{ - эмперическая функция распределения}
\]
\[
    \frac{F_n(x + \delta) - F_N(x)}{\delta} = f_N(x)
\]

\textbf{Оценка:}
\[
    x_1, x_2, \dots, x_n
\]

Оценка - измеримая ф-ия, которая каждой выборке ставит в соот-е число

\end{document}
