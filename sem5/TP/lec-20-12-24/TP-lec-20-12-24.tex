\documentclass[a4paper]{article}
\usepackage[a4paper,%
    text={180mm, 260mm},%
    left=15mm, top=15mm]{geometry}
\usepackage[utf8]{inputenc}
\usepackage{cmap}
\usepackage[english, russian]{babel}
\usepackage{indentfirst}
\usepackage{amssymb}
\usepackage{amsmath}
\usepackage{mathtools}
\usepackage{tcolorbox}
\usepackage{import}
\usepackage{xifthen}
\usepackage{pdfpages}
\usepackage{transparent}
\usepackage{graphicx}
\usepackage{romannum}
\graphicspath{ {./figures} }

\newcommand{\incfig}[1]{%
\def\svgwidth{\columnwidth}
\import{./figures/}{#1.pdf_tex}
}

\begin{document}
\section*{\centering Случайные процессы}

\[
    (\Omega, A, P) \text{ - вер. прос-во}
\]
\[
    X: \omega \to X(\omega) \in R^{1} 
\]
\[
    A = X^{-1}(B) - \{ \omega \in B: \ X(\omega \in B\} \in A \to P(A)
\]

СлоП (Случайный процесс?)
\[
    X(t, \omega) \text{ - ф-ия двух аргументов}
\]
a) t - фикс. $ X_t(\omega) $ - с.в.\\
b) $ \omega $ - фикс. $ X_\omega(t) $ - траектория

Взгляд: мн-во траекторий (?)
\[
    \{ X(t,\omega) \}_{t \in T}
\]
\romannumeral 1) $ X(t,\omega) $ принимает значения $ \{ 0, 1, 2, \dots \} = \mathbb{N} \cup \{ 0 \}$ 
- считающий процесс\\
\romannumeral 2) $ T = \{ 0, 1, 2, \dots \} $ - послед-ть случайных величин\\
$ T = [0, 1] \quad T = [0, +\infty] \ni t $ - временной ряд
\[
    \forall \, t_1 <t_2 < \dots < t_n \quad \forall \, n
\]
\[
    (X_{t_1}, X_{t_2}, \dots, X_{t_n})^{T} \quad F(x_1, x_2, \dots, x_n)
\]

1)
\[
     \ F(x_1, x_2, \dots, x_n, x_{n+1}) |_{x_{n+1} = \infty} = F(x_1, x_2, \dots, x_n)
    \text{ - функция распределения}
\]

2)
\[
    F_{X_1, \dots, X_n}(x_1, \dots, x_n) = F_{\pi(X_1, \dots X_n)}(\pi(x_1, \dots x_n))
\]

\begin{tcolorbox}
    \underline{Th (Колмогоров А. Н.)}
    \[
        (\mathbb{R}^{n}, \mathcal{A}(\mathbb{R}^{n}), P^{n})_{n \geq 1}
    \]
    \[
        x_{\omega}(t) \text{ - траектория}  \to X_{t_1}, X_{t_2}, \dots, X_{t_n}
    \]
    \[
        (\mathbb{R}^{\infty}, \mathcal{A}( \mathbb{R}^{\infty}))
    \]

    Существует мера ..............................................
\end{tcolorbox}

\begin{tcolorbox}
    Случайный процесс - согласованное семейство случайных величин
\end{tcolorbox}

\begin{tcolorbox}
\underline{Example}
\[
    \{ X_t \}_{t \geq 0} \quad t_1 < t_2< \dots <  t_n
\]
\[
    f(x_1, \dots, x_n) = C \exp(-\frac{1}{2} \sum_{i,j = 1}^{n} (x_i - \mu_i) 
\]

Модификация процесса
\[
    \{ X_t \}_{0 \leq t \leq 1}
\]

Случайные величины $ X, \widetilde{X} $ эквив-ны
\[
    F_{\widetilde{X}}(x) = F_X(x), \ \forall \, x \in \mathbb{R}^{1}
\]

$ (X_{t_1}, \dots, X_{t_n}) $ совпадает с распределением\\
$ (\widetilde{X}_{t_1}, \dots, \widetilde{X}_{t_n}) $, то $ \widetilde{X} $ 
модификаия $ X = \{ X_t \}_{t \geq 0} $ 
\end{tcolorbox}

\subsection*{\centering Типы случайных процессов}

\textbf{Процесс с независимыми значениями}
\[
    X_1, X_2, \dots X_n \text{ - с.в. независимы}
\]
\[
    \{ X_t \}_{t \geq 0} \quad \underbrace{X_t \land X_s}_{\text{независимы?}}, \ t \neq s
\]

\textbf{Марковские (или Мартовские, или хз) процессы}
\[
    \{ X_t \}_{t \geq 0} \quad X_t \text{ принимая усл зн-я}
\]
\[
    t_1 < t_2 < \dots < t_n < t_{n+1}
\]
\[
    P(X_{t_{n+1}} = x_{n+1} \ | \ X_{t_n} = x_n, X_{t_{n-1}}= x_{n-1}, \dots,
    X_{t_1} = x_1) = P(X_{t_{n+1}} = x_{n+1} \ | \ X_{t_n}= x_n)
\]
\[
    P(a \leq X_{t_{n+1}} \leq b | X_{t_n} = x_n) = P([a,b), t_{n+1}; t_n, x_n)
    \text{ - переходная вероятность}
\]

\textbf{Стационарные процессы}

\subsection*{Математическое ожидание}

\begin{tcolorbox}
    \underline{Def} t - фикс $ E(X_t) = m(t) $ 
    \[
        \forall t \quad \{ m(t) \}, \ t \geq 0 \text{ - мат ожидание}
    \]

    \underline{Def} Дисперсия
    \[
        D(X_t) = \sigma^2(t), \ \forall t - \text{ дисперсия }
    \]
\end{tcolorbox}
\[
    K(t,s) = E((X_t - m(t))(X_s - m(s)))
\]
\[
    K(t,s) = K(s,t)
\]
\[
    \{ X_t\}_{t \geq 0} \text{ - стационарный процесс}
\]
\[
    E(X_t) = const 
\]
\[
    D(X_t) = const
\]
\[
    K(t,s) = R(|t-s|) = R_1(t-s)
\]

\begin{tcolorbox}
    \underline{Def} Процесс $ \{X_t\}_{t \geq 0} $ - стационарный в широком смысле,
    если $ m(t) = const, \ K(t,s) = R(t-s) $ 
\end{tcolorbox}

\begin{tcolorbox}
    \underline{Lemma} $ \{ X_t \}_{t \geq 0} $ - гауссовский процесс стационарен
    в узком смысле $ \iff $ он стационарен в широком смысле
\end{tcolorbox}

\begin{tcolorbox}
\underline{Example} $ \xi_1, \xi_2 $ - незавсимые с.a.
\[
    P(\xi_i = \pm 1) = \frac{1}{2} 
\]
\[
    E(\xi_i) = \sum x_k p_k = 1 \cdot \frac{1}{2} - 1 \cdot \frac{1}{2}  = 0
\]
\[
    D(\xi_i) = E(\xi_i^2) - E^2(\xi_i) = 1 \cdot \frac{1}{2} + 1 \cdot \frac{1}{2} 
    - 0^2 = 1
\]
\[
    cov(\xi_1, \xi_2) = 0 \text{ т.к. независимы}
\]

\[
    \xi_t = \xi_1 \cos t + \xi_[ \sin t
\]
\[
    E(\xi_t) = \cos t E(\xi_1) + \sin t E(\xi_2) = 0
\]
\[
    D(\xi_1 \cos t + \xi_2 \sin t) = \cos^2 t + \sin^2 t = 1
\]
\[
    K(t,s) = E((\xi_1 \cos t + \xi_2 \sin t) (\xi_1 \cos s + \xi_2 \sin s)) = 
    \cos t \cos s + \sin t \cos t = \cos (t-s)
\]
\end{tcolorbox}
\end{document}
