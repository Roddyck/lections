\documentclass[a4paper]{article}
\usepackage[a4paper,%
    text={180mm, 260mm},%
    left=15mm, top=15mm]{geometry}
\usepackage[utf8]{inputenc}
\usepackage{cmap}
\usepackage[english, russian]{babel}
\usepackage{indentfirst}
\usepackage{amssymb}
\usepackage{amsmath}
\usepackage{mathtools}
\usepackage{tcolorbox}
\usepackage{graphicx}
\graphicspath{ {./figures} }

\begin{document}
\begin{center}
    \textbf{ \underline{Топология. Лекция(21.09.24)}}
\end{center}

\begin{tcolorbox}
\underline{Th. 6} $ (X, \tau) $ - т.п. $ A \subset X $ $ F \subset A $ \\
F - замкнуто в $ \tau_A \iff \exists G \text{ - замкнутое подмн-во в } \tau \; |
\; F  = G \cap A$ \\
\underline{Proof} \\
\begin{equation}
    A \setminus ( A \cap B) = A \cap C(A \cap B) = A \cap CB
\end{equation}

$1) \stackrel{?}{\implies}$ 
\[
    \exists\, U \in \tau_A \; | \; F = A \setminus U
\]
\[
    \exists U_1 \in \tau \; | \; U = U_1 \cap A
\]
\begin{equation*}
    \begin{aligned}
        F = A \setminus U = A \setminus (U_1 \cap A) \stackrel{(1)}{=} A \cap
        CU_1 \text{ - замкнуто в A}
    \end{aligned}
\end{equation*}

$2) \stackrel{?}{\impliedby}$ $ F = G \cap A \; G -\text{ замкнуто в } \tau $  
\[
    A \setminus F = A \setminus (G \cap A) = A \cap CG \in \tau \implies
    A \cap CG \in \tau_A
\]
\end{tcolorbox} 

\underline{Def.} Точка x наз-ся точкой прикосновения множества A, если $ 
\forall U \implies U_x \cap A \neq \varnothing $ \\
$ \overline{A} = ClA $ 

\begin{tcolorbox}
    \underline{Th.7} Замыкание любого множества замкнуто \\

    \underline{Proof} $ A \subset X $ $ \overline{A} $ - замкнуто?\\
    $ C \overline{A} \in \tau $ \\
    \begin{equation*}
        \begin{aligned}
            &\forall x \in CA \implies x \notin \overline{A} \implies \exists U_x \;
            | \; U_x \cap A = \varnothing \\
            &\forall y \in U_x \implies U_x \text{ является иакде окр-тью т.y}\\
            &U_x \cap A = \varnothing \implies y \notin \overline{A} \implies
            y \in C \overline{A} \implies U_x \subset C \overline{A}, \text{т.е} \;
            C \overline{A} - \text{ - замкнуто}
        \end{aligned}
    \end{equation*}
\end{tcolorbox}

\begin{tcolorbox}
\underline{Th.8} замыкание мн-ва А является пересечением всех замкнутых мн-в,
содержащих множество А, т.е. замыкание А является самым маленьким замкнутым мн-ом
содержащим А \\
$ \overline{A} = \bigcap\limits_{i \in I} A_i $ \\
$ A_i \text{ - замкнуто } A_i \supset A  \; \forall i \in I $ 

\underline{Proof} 1. $ \overline{A} \supset \bigcap\limits_{i \in I} A_i$ 
\[
    \exists A_{i_0} = \overline{A} \implies \text{ 1. выполнено из св-й }
\]
2. $ \overline{A} \subset \bigcap\limits_{i \in I} A_i$ \\
\[
    \forall x \in \overline{A} \implies \forall U_x \cap A \neq \varnothing 
    \implies U_x \cap A_i \neq \varnothing \; \forall i \implies
    X \in \overline{A}_i \; \forall i \stackrel{Th.7.5}{\implies} \overline{A}_i = 
    A_i \implies X \in A_i \; \forall i \implies 
\]
\end{tcolorbox}

\begin{tcolorbox}
\underline{Th.7.5} Замыкание замкнуто мн-во совпадает с этим мн-вом \\

\underline{Proof} F замнуто $ \implies \overline{F} = F $  \\
1. $ \overline{F} \supset F $ из опр-я\\
2. $ \overline{F} \subset F $ для замкнутого F.\\
От противного: \\
\[
    \text{ Let } \exists x \in \overline{F} \; \text{ и } x \notin F
\]
\[
    CF \in \tau \text{ и } x\in CF \implies \exists U_x \; | \; U_x \subset CF
    \implies x \notin \overline{F} \text{ противоречие}
\]
\end{tcolorbox}

\underline{Следствие} $ \overline{A} = \overline{\overline{A}} $ \\
\underline{Proof} $ \overline{A} $ замкнуто $ \implies $ по Теореме 7.5

\begin{tcolorbox}
    \underline{Th.9} (Св-ва замыкания)\\
    1. $ A \subset B \implies \overline{A} \subset \overline{B} $ \\
    2. $ \overline{A\cup B} = \overline{A} \cup \overline{B} $ \\
    \underline{Proof} 1. из определения\\
    2. $  \overline{A\cup B} \subset \overline{A} \cup \overline{B} $ \\
    От противного:\\
    \begin{equation*}
        \begin{aligned}
        \text{Пусть } x\in \overline{A\cap B} \text{ и } x \notin \overline{A}
        \cup \overline{B} \implies \begin{cases}
            x \notin \overline{A} \implies \exists U_x \; | \; U_x \cap A = \varnothing \\
            x \notin \overline{B}\implies \exists V_x \; | \; V_x \cap A = \varnothing 
        \end{cases}\\
        \text{ Пусть } W_x = U_x \cap V_x \implies W_x \cap (A \cup B) = \varnothing
        \implies x\notin \overline{A \cup B}
        \end{aligned}
    \end{equation*}
    $ \overline{A \cup B} \stackrel{?}{\supset} \overline{A} \cup \overline{B} $ 
    \[
        \forall x \in \overline{A} \cup \overline{B} \implies \begin{cases}
            x \in \overline{A} \implies \forall U-x \cap A \neq \varnothing \\
            x \in \overline{B} \implies \forall U_x \cap B \neq \varnothing
        \end{cases}
    \]
\end{tcolorbox}

\underline{Пр} $ \overline{A\cap B} \stackrel{?}{=} \overline{A} \cap \overline{B} $ \\
\[
    A= (0, 1) \subset (\mathbb{R}, \tau_0)
\]
\[
    B = (1, 2) \subset (\mathbb{R}, \tau_0)
\]
\[
    A\cap B = \varnothing \implies \overline{A\cap B} = \varnothing
\]
\begin{equation*}
    \begin{aligned}
        \overline{A} &= [0,1]\\
        \overline{B} &= [1,2]
    \end{aligned}
    \implies \overline{A} \cap \overline{B} = \varnothing
\end{equation*}

\underline{Def} точка а наз-ся граничной точкой А, если её любая окр-ть пересекается
с самим мно-ом, а также с его дополнением $ FrA $ 

\begin{tcolorbox}
    \underline{Th.10} $ FrA = \overline{A} \setminus IntA $  \\

    \underline{Proof} 1. $ FrA \stackrel{?}{\subset} A \setminus IntA $  \\
    \[
        \forall x \in FrA \implies \forall U_x \cap A \neq \varnothing \implies
        x\in \overline{A}
    \]
    Докажем, что $ x \notin IntA $ \\
    От противного:
    \[
        x\in IntA \implies \exists U_x \; | \; U_x \subset A \implies
        \text{ противоречие с определением граничной точки}
    \]
1. $ FrA \stackrel{?}{\supset} A \setminus IntA $
\[
    \forall x \in \overline{A} \setminus IntA
\]
Те же рассуждения в обратном порядке
\end{tcolorbox}

\textbf{\underline{Def}} x называется предельной точкой мн-ва А, если любая
проколотая окр-ть имеет непустое пересечение со мн-вом А, т.е.  
$ \forall(U_x \setminus \{x\}) \cap A \neq \varnothing $. обозн.$ A' $ \\

\textbf{\underline{Def}} $ a: \mathbb{N} \to X $ наз последовательностью в X\\

\textbf{\underline{Def}}  Точка b называется пределом посл-ти $ a_n $, если
$ \forall U_b \; \exists \, n_0 \; | \;  \forall n > n_0 \implies 
a_n \in U_b $  $ b = \lim\limits_{n \to \infty}a_n  $ 

\textbf{\underline{Def}} Точка a наз-ся изолировнной точкой множества А, если
$ \exists U_a \; | \; U_a \cap A = \{a\} $  $ Iso A $

\section*{\centering\S2. Непрерывные отображения в топологических пространствах}
\textbf{\underline{Def}} Отображение $ f: (X, \tau) \to (Y, \omega)  $ называется
непрерывным в т. $ x_0 $, если $ \forall U_{f(x_0)} \, \exists U_{x_0} \; | \;
f(U_{x_0}) \subset U_{f(x_0)}$ 

\begin{tcolorbox}
    \underline{Th.1} Отображение $ f: (X, \tau) \to (Y, \omega) \text{ непрерывно }
    \iff $ Прообраз каждого открытого мн-ва открыт
    
    \underline{Th.2} Отображение $ f: (X, \tau) \to (Y, \omega)\text{ непрерывно }
    \iff $ Прообраз
    каждого замкнутого мн-ва замкнут
\end{tcolorbox}

\underline{Ex.} $ c: (X, \tau) \to (Y, \omega : c(x) = c \in T $ 

\end{document}
