\documentclass[12pt]{article}
\usepackage[a4paper,%
    text={180mm, 260mm},%
    left=15mm, top=15mm]{geometry}
\usepackage[utf8]{inputenc}
\usepackage{cmap}
\usepackage[english, russian]{babel}
\usepackage{indentfirst}
\usepackage{amssymb}
\usepackage{amsmath}
\usepackage{mathtools}

\begin{document}
\section*{\S 1. Топологические пространства}
\underline{\textbf{Опр.}.} Пусть на мн-ве X задано семейство $\tau$, удовлетворяющие
условиям: \\
\textbf{1.} X, $\varnothing \in \tau$ \\
\textbf{2.} $\forall \text{ U}_\alpha \text{, } \alpha \in \text{A} \quad \text{U}_\alpha
\in \tau \quad \forall \alpha \Rightarrow \cup_{\alpha\in\text{A}} 
\text{U}_\alpha \in \tau$  \\
\textbf{3.} $ \forall \text{U}_i \text{, } i=\overline{1,n} \quad \text{U}_i \in \tau \Rightarrow
\cap_{i = 1}^n \text{U}_i \in \tau $\\
Тогда (X, $\tau$) - топологическое простванство. $\tau$ - топология\\
\underline{\textbf{Опр.}.} U $\in \tau \Rightarrow$ U - открытое. CU - замкнутое \\
\underline{\textbf{Опр.}.} $\forall$ открытое мн-во, содеражащие точку x, называется
его окресностью\\
\underline{\textbf{Теорема}} Мн-во U $\in \tau \Leftrightarrow \forall \text{x} \in$ U
входит в него с нек-рой окр-тью \\
\underline{\textbf{Опр.}} Пусть на мн-ве X задано отображение\\
d: X$\times$X $\to \mathbb{R}^+$, удовлетворяющие условиям: \\
\textbf{1.} d(A,B) $\geq$ 0, $\forall$A,B \\
d(A,B) $=$ 0 $\Leftrightarrow$ A$=$B\\
\textbf{2.} d(A,B) $=$ d(B,A) $\forall \text{A,B}$ \\
\textbf{3.} $ \text{d}(\text{A,B}) + \text{d}(\text{B,C}) \geq \text{d}(\text{A,C})
\quad \forall \text{A,B,C} \in \text{X}$ \\
Тогда (X, d) - метрическое пространство \\

\underline{\textbf{Опр.}} Пусть (X, d) - метр. пр-во. \\
Тогда объединение произвольных наборов открытых шаров является топологией\\
$ \text{d(a,b)}= \sqrt{\sum_i^n(a_i - b_i)^2} $ \\
$ (\mathbb{R}^n, \text{d})$ - метр. пр-во\\ 
$\tau_0 = \tau_d $\\
\underline{\textbf{Опр.}} Пусть (X, $\tau$) - топология. $ A \subset X $ \\

Тогда $\tau_A = \{U \cap A | U \in \tau \}$ называется топологией индуцированной
их X на A\\
$ (A, \tau_A) - \text{подпространство пространсва } (X, \tau) $ \\

\underline{\textbf{Теорема}} (Транзитовность инд. топологии)
Let $(X, \tau)$, $ B \subset A \subset X $ \\
Тогда $\tau_B = (\tau_A)_B $ 

\underline{\textbf{Д-во.}}\\
\textbf{\textbf{1.}} $ \tau_B \subset (\tau_A)_B $
\begin{equation*}
    \forall U \in \tau_B \Rightarrow \exists V \in \tau \quad | \quad
    U = V \cap B \Rightarrow
    \tilde{V} = V \cap A \in \tau \Rightarrow \tilde{V} \cap B = (V \cap A)
    \cap B = V \cap B \Rightarrow U \in (\tau_A)_B
\end{equation*}
\textbf{2.} $ (\tau_A)_B \subset \tau_B $\\
\begin{equation*}
    \forall U \in (\tau_A)_B \Rightarrow \exists V \in \tau_A \quad | \quad
    U = V \cap B \Rightarrow \exists \tilde{V} \in \tau | \\
    V = \tilde{V} \cap A \Rightarrow U = \tilde{V} \cap B \in \tau_B
\end{equation*}

\underline{\textbf{Опр.}} $ (X, \tau) \Sigma \subset \tau $ наз-ся базой $\tau$, если
каждое открытое множество является объединением подмножеств из $\Sigma$

\underline{Пр} $(X, \tau_d)$ $\Sigma$ - база $ \tau_d $ состоит из открытых шаров\\ 

\underline{\textbf{Теорема}} (Критерий базы в топологическом пространстве) \\ 
Let $ (X, \tau) $ \\
$\Sigma$ является базой $\Leftrightarrow$ \\
$ 
    \textbf{1. } \Sigma \subset \tau \\ 
    \text{2. } \forall U \in \tau \text{, } \forall x \in U \, \exists V \in \Sigma
    \, | \, x \in V \subset U
$

\underline{\textbf{Теорема}} (Критерий базы в на мн-ве) \\ 
Пусть X множество без топологии \\
$ \Sigma $ - база некоторой топологии $\Leftrightarrow$ \\
$
    \textbf{1. } X = \bigcup\limits_\alpha U_\alpha, \, U_\alpha \in \Sigma\\
    \textbf{2.} \, \forall \, U, V \in \Sigma \quad \forall \, x \in U \cap V \:
    \exists W \in \Sigma \; | \; x \in W \subset U \cap V
$

\underline{\textbf{Опр.}} Let $(X, \tau) $ - т.п. $\forall A \subset X$ \\ 
т. a называется внутренней, если $\exists U_a \; | \; U_a \subset A$ \\
мн-во внутренних точек $\text{Int}A = \mathring{A}$

\underline{\textbf{Теорема}} $\forall A \Rightarrow \text{Int}A \in \tau$

\underline{\textbf{Теорема}} $ A \in \tau \Leftrightarrow A = \text{Int}A $

\underline{\textbf{Теорема}} $ \forall A, B \Rightarrow \\
    \indent \textbf{\textbf{1.} } A \subset B \Rightarrow \text{Int}A \subset \text{Int}B \\
    \indent \text{2. } \text{Int}(A \cap B) = \text{Int}A \cap \text{Int}B\\
$
\underline{\textbf{Д-во.}} \\
\textbf{1.} 
\begin{equation*}
    \forall \, x \in \text{Int}A \Rightarrow \exists \, U_x \; | \; U_x \subset A
    \xRightarrow{A \subset B} U_x \subset B \Rightarrow x \in \text{Int}B
\end{equation*}
\textbf{2.} a) $ \text{Int}(A \cap B) \stackrel{?}{\subset} \text{Int}A \cap \text{Int}B $\\
\begin{equation*}
    \forall x \in \text{Int}(A \cap B) \Rightarrow \exists U_x \subset A \cap B
    \Rightarrow U_x \subset A \; \land \; U_x \subset B \Rightarrow
    x \in \text{Int}A \; \land \; x \in intB \Rightarrow x \in \text{Int}A \cap \text{Int}B
\end{equation*}
b) $ \stackrel{?}{\supset} $
\begin{equation*}
    \forall x \in \text{Int}A \cap \text{Int}B \Rightarrow \begin{cases}
        x &\in \text{Int}A \Rightarrow \exists U_x \; | \; U_x \subset A\\
        x &\in \text{Int}B \Rightarrow \exists V_x \; | \; V_x \subset B\\
    \end{cases}
    \Rightarrow W_x = U_x \cap V_x \Rightarrow W_x \subset A \cap B
\end{equation*}

\underline{\textbf{Теорема}.} Внутренность множества A является объединением всех открытых
подмножеств, содержащихся во множестве A \\
\[ \text{Int}A = \bigcup_{i \in I}A_i \quad A_i \in \tau \; A_i \subset A \]

\underline{\textbf{Д-во.}} \\
\textbf{1. } $ \text{Int}A \stackrel{?}{\subset} \bigcup\limits_{i \in I}A_i 
\quad A_i \in \tau, \; A_i \subset A$ \\
\[ \text{Int}A \in \tau, \; \text{Int}A \subset A \Rightarrow \text{Int}A = A_{i_0} \]
\textbf{2. } $ \text{Int}A \stackrel{?}{\supset} \bigcup\limits_{i\in I}A_i $\\

\begin{center}
    \begin{aligned}
        & \forall x \in \bigcup_{i \in I} A_i \quad x \in A_{i_0}, \; A_{i_0} \in \tau
        \quad A_{i_0} \subset A \\
        & A_{i_0} \text{окрестность т.x, которая включена в A} \Rightarrow
        x \in \text{Int}A
    \end{aligned}
\end{center}

\underline{\textbf{Опр.}} точка a называется точкой прикосновения множества A, 
если любая окресность точки a пересекается со множеством A.\\
Множество точек прикосновения называется замыканием $\bar{A}$

\underline{\textbf{Пр.}} $ (\mathbf{R}, \tau_0) $ $ A = [0, 1) \cup \{2\} $ \\
\indent $ \bar{A} = [0,1] \cup \{2\} $




\end{document}
